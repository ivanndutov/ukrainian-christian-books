\documentclass[main.tex]{subfiles}

\begin{document}
\chapter{Закон Божий і заповіді}

\subsection{Що таке закон?}

Закон — це правила, дані Богом людям, яких вони повинні триматися в житті, щоб жити боговгодно, відповідно своєму від Бога призначенню.

\subsection{Що таке заповіді?}

Це окремі статті закону, в яких коротко, але точно, вказано, як повинні жити люди: що виконувати, а від чого ухилятися.

Отже, Господь дав закон для того, щоб люди могли розпізнавати, які діла добрі, а які лукаві. Перше закон внутрішній (сумління), потім зовнішній (заповіді).

Внутрішній закон існував завжди, у всіх людей. Про це так говорить Святе Письмо: Погани, які не знають Божого закону, «виявляють, що діло закону у них написано на серцях, їхні думки то звинувачують, то виправдують одна одну» (Рим. 2, 15).

Але після гріхопадіння люди стали легко підпадати впливам лукавого, забували Бога, стали провадити лише тілесне життя і керуватися лише хотінням тіла. Тим вони заглушили голос духовного внутрішнього закону — голос сумління. Щоб пробуджувати їхню свідомість і утримувати їх на шляху праведного життя, Господь дав зовнішній закон — закон заповідей.
 
«Бо для чого ж закон? — каже апостол. — Він даний потім з причин переступів... через ангелів, рукою посередника (Мойсея)» (Гал. З, 19).

\subsection{Кола і як даний зовнішній закон?}

Закон заповідей даний був людям на горі Синаї через пророка Мойсея у п`ятдесятий день після виходу Ізраїльтян з Єгипту. В пустині на горі Синаї Бог явив Свою присутність у вогні блискавок та в хмарах і голосом, подібним до грому, висловив Свій закон. Потім дав його через Мойсея, написаний на двох скрижалях — кам`яних дошках-таблицях.

Головніші заповіді даного Богом Закону такі:

\subsubsection{ПЕРША СКРИЖАЛЬ}
\begin{enumerate}
    \item Я є Господь Бог твій; нехай не буде в тебе інших
    богів, крім Мене.
    \item Не твори собі кумирів і всякої подоби того, що
    (ти бачиш) на небі вгорі, на землі внизу, і в водах під зем
    лею; не поклоняйся їм і не служи їм.
    \item Не призивай імені Господа Бога твого всує (марно,
    кощунно, без пошани).
    \item Пам`ятай день суботній, щоб святити його: шість днів працюй і роби в них усі діла твої; день же сьомий, субота — Господу Богу твоєму. (В Новому Заповіті Господь постановив святкувати день воскресний (неділю), тобто перший день тижня.
\end{enumerate}

\subsubsection{ДРУГА СКРИЖАЛЬ}
\begin{enumerate}
    \setcounter{enumi}{4}
    \item Шануй батька й матір: добре тобі буде і довго будеш жити на землі.
    \item Не вбивай.
    \item Не перелюбствуй.
    \item Не кради.
    \item Не будь неправдивим свідком на ближнього твого.
    \item Не побажай жони приятеля твого, не побажай дому ближнього твого, ні поля його, ні слуги його, ні служниці його, ні вола його, ні осла його, ні всього, що є у ближнього твого.
\end{enumerate}

Ці заповіді дані народові ізраїльському. Але й ми повинні їх виконувати, бо вони дані для всіх людей і на всі часи людського життя. Вони є тим самим законом, який, по слову апостола, написаний у серцях їх з початку (Рим. 2, 15).

Господь Ісус Христос не відмінив заповіді, а навпаки велів дотримуватися їх і виконувати пильніше, ніж їх виконували до Його пришестя (Мт. 5, 17-48).

\subsection{Чому заповіді написані на двох скрижалях і не порівно, а на одній чотири, а на другій шість?}

Тому, що в перших чотирьох заповідях заповідуеться любов до Бога, а в останніх шести — любов до ближніх.

Цей розподіл повторив і Сам Господь Ісус Христос. Один законник спитав Його: «Яка заповідь найбільша в законі?» Він відповів: «Полюби Господа Бога твого всім серцем твоїм, і всією душею твоєю, і всім розумінням твоїм. Це перша і найбільша заповідь. А друга подібна до неї: «Полюби ближнього твого, як самого себе!» На цих двох заповідях увесь закон і пророки основуються (Мт. 22, 36-40).

\subsection{Хто ж наші ближні?}

Всі люди, бо всі вони — творіння Єдиного Бога і народилися від одного подружжя — Адама і Єви. Значить всі вони нам брати по крові. А ще більше — близькі нам по вірі, як діти Єдиного Небесного Отця і матері — Православної Христової Церкви.

\subsection{Чому нема заповіді про любов до себе?}

Тому, що по природі ми любимо себе. Одначе любов до себе не потрібно доводити до самообожнювання. Любити себе треба тільки як творіння Боже і для Бога. Оберігати себе треба для ближніх, не допускати себе до скотоподобія і тим зневажати гідність людини. Любити ж ближніх треба для Бога як Боже творіння, бо так заповідав Бог.

А любити Бога - це природній обов`язок людини, як любов дитини до свого батька. Любов прив`язує нас до Бога, і Бога прихиляє до нас.

Любов до себе ми мусимо приносити в жертву любові до ближніх, як і каже Господь: «Немає більшої любові над ту, коли хто душу свою покладе за друзів своїх (за ближніх)» (їв. 15, 13).

А любов до себе й до ближніх треба приносити в жертву любові до Бога.

«Хто любить батька або матір більше, ніж Мене, той не достойний Мене, і хто любить сина або дочку більше, ніж Мене, той не достойний Мене», — каже Христос (Мт. 10, 37).

\section{ПЕРША ЗАПОВІДЬ}

Я є Господь Бог твій; нехай не буде в тебе інших богів, крім Мене.

Ця заповідь є найвище Богооб`явлення людям. Нею Сам Господь у громі й блискавці проголошує: «Я є! Я — Господь Бог твій!»

Це абсолютна Істина, її приймаємо без ніяких умов. Це перший догмат, даний всім людям Самим Богом. Тому ніхто з людей не має права сказати: «Я не знаю, чи є Бог».

Одначе знати чи вірити, що Бог є, не досить. Цією заповіддю Господь велить нам пізнавати Господа Бога, бо коли люди вже визнавали «інших богів», то треба пізнати який же Істинний Бог.

«Шукайте Господа, коли можна знайти (пізнати) Його», — каже Господь через пророка Ісайю, — «призивайте Його, коли Він близько..» (Іс. 55, 6). «Нахиліть вухо ваше і послухайте, і жива буде душа ваша» (Іс. 55, 3).

\subsection{Звідкіля ж навчитися Богознання?}

Джерелом Богознання є об`явлення Боже, записане у святих книгах Біблії і святих піснях та молитвах богослу-ження. Треба уважно слухати в церкві науку про Бога і розмовляти з більш досвідченими правовірними людьми.

Треба вивчати Святе Письмо і не тлумачити його по своєму власному розумінню, а сприймати його так, як приймає його Свята Православна Вселенська Церква. Для того треба читати вчення Отців і Учителів Церкви, як свв. Васи-лій Великий, Григорій Богослов, Іван Золотоустий та ін.

\subsection{Чи можна пізнати саме Божество?}

Ні. Божество вище людського, навіть ангельського розуміння, як і каже апостол: «Божого (життя) ніхто не знає, тільки Дух Божий» (1 Кор. 2, 11). Ні Божества, ні Божого розуму розум людський збагнути не може, як не може він своїм знанням охопити всесвіту, бо малим великого охопити не можна. Тому ніхто не в силах обчислити розум Божий або силу Божу. А коли б хтось насмілився таке робити, то був би подібний до того, що хотів у пляшку вмістити океан. Тому не треба читати таких книжок, які критикують діла Божі, тому, що то безумство і богопротивне нахабство, адже перший сатана хотів порівнятися з Богом і знецінював діла Його.

\subsection{Що означають слова: «Нехай не буде в тебе інших, богів, крім Мене?»}

Цими словами Господь забороняє кого б то не було вважати за Бога і віддавати йому пошану, як Богові.

«Я — Господь, і іншого немає, крім Мене», — каже Господь через пророка Ісайю (Іс. 45, 5).

\subsection{Що ж наказує нам перша заповідь?}

Перша заповідь наказує нам визнавати й шанувати всім серцем лише Єдиного Бога Істинного:
\begin{enumerate}
    \item Вірувати в Бога.
    \item Ходити перед Богом, цебто завжди пам`ятати про Бога і у всьому поводити себе боговгодно, знаючи, що Він кожну хвилину бачить і знає не тільки діла й слова наші, а й найтаємніші помисли наші.
    \item Боятися Бога, або благоговіти перед Ним, і гнів Отця Небесного вважати для себе найбільшим нещастям, а тому старатися не гнівити Бога гріхами, а благоугождати Богові покорою та добрими ділами.
    \item Любити Бога всім серцем, усім розумінням і всією істотою своєю.
    \item Надіятися на Бога, знаючи Його всемогутність і любов до нас.
    \item Коритися Богові і бути завжди готовими виконувати повеління Його; не ремствувати, якщо доля наша іноді буває не такою, як ми б хотіли.
    \item Поклонятися Богові, як найвищій і найсвятішій Істоті.
    \item Прославляти Бога, як всенайдосконалішого, всемогутнього, всепремудрого, всеблагого і всеправедного.
    \item Дякувати Богові, що створив нас, піклується нами і спасає нас.
    \item Призивати Бога в молитві при кожному ділі й на кожному місці, як всеблагого і всемилосердного Помічника й Порадника.
\end{enumerate}

Шанувати Бога ми повинні не тільки самі в собі, а пошану і служіння Йому виявляти й назовні, а саме:

\begin{enumerate}
    \item Проповідувати всім і всюди, що Бог є, що Він Отець наш Небесний, а ми творіння Його. Всюди оберігати славу Його. Не відхилятися від Нього ні при яких обставинах, хоч би довелося й постраждати та вмерти за ім`я Його.
    \item Завжди брати участь у громадському богослуженні, що встановлено від Бога і заповіджено Святою Православною Церквою.
\end{enumerate}

Щоб точніше виконувати цю заповідь Божу, треба знати, чого кожна людина повинна оберігатися:

\begin{enumerate}
    \item Безбожності. Розпутні й жорстокі люди, бажаючи заспокоїти свою зіпсовану совість, що ніби вони за свої злочини не будуть відповідати перед Богом, переконують себе, що ніби Бога немає. Як говориться у псалмі: «Сказав безумний у серці своїм: «немає Бога». І повіривши в те, ще більше починає грішити: «розтлилися, омерзенилися в ділах своїх» (Пс. 13, 1).
    \item Багатобожжя. Деякі люди впадають в іншу крайність і думають, що Бог не Один, а що їх багато, або природу вважають всебогом, або душу людську — часткою Божества. Така єресь зветься «пантеїзмом». Вона рівнозначна й безбожництву, бо відкидає Єдиного Живого Бога і противна Богові, Який сказав: «Я Перший, Я й Останній, і крім Мене немає Бога» (Іс. 44, 6). «Я Господь, це Моє ім`я, слави Моєї іншому не дам і хвали Моєї ідолам (видуманим богам)» (Іс. 42, 8).
    \item Невірства та полувірства чи маловірства. Це буває тоді, коли люди або зовсім не вірують у Бога, або начебто вірують і визнають, що Бог є, але не вірять в Його об`явлення, що Бог знає все, турбується про створений Ним світ і, особисто, про кожну людину. Сюди ж стосується й байдужість до Бога і своєї душі.
    \item Єресі, коли люди до богооб`явлених істин, визнаних Всесвітньою Церквою, додають щось від себе, суперечне правді, перекручують догмати по своєму, або ж тлумачать Святе Писання суперечно церковному розумінню чи відкидають щось, наприклад: Страшний Суд, загробне життя, святий хрест і ін.
    \item Розколу, тобто своєвільного ухилення від єдності Богошанування й від єдності з Вселенською Церквою через які-небудь дрібниці, по впертості або з честолюбства.
    \item Боговідступництва, коли люди із страху або заради вигод світу відступають від Бога і святої істинної віри.
    \item Відчаю, коли грішник втрачає надію, що може бути помилуваний Богом і тоді вже цілком віддається своїм беззаконням, ставши у ворожнечу проти Бога.
    \item Чаклунства, коли люди, нехтуючи віру в Божу силу, шукають допомоги в різних чарах та прикметах, вірять у злу силу різних тварин, прикликають злих духів і намагаються діяти ними.
    \item Суєвірства, коли люди починають вірити в силу якоїсь звичайної речі, наприклад вузликів на чомусь, голки в полі чи чогось подібного, надіються на це, навіть бояться.
    \item Лінощів до молитви, лінощів піти до церкви на богослуження.
    \item Любові до чогось більшої, ніж до Бога, наприклад до окремої людини, до окремої науки або мистецтва не заради Бога.
    \item Людиноугодництва, коли хтось силкується догодити тій чи іншій людині на шкоду Божій славі, забувши християнські обов`язки й заповіді Божі.
    \item Надії на людей, коли замість Бога люди починають надіятись на сильних віку цього або лише на людську силу, зброю та мудрість.
    \item Надії на себе, коли люди свій хист ставлять вище Божої мудрості й помочі.
\end{enumerate}

У всіх цих гріхах люди шукають іншого «бога» і цим тяжко грішать проти першої заповіді Божої.

Так каже Господь: «Проклятий чоловік, який надіється на чоловіка... а від Господа відступить серце його» (Єр. 17, 5).
 
\subsection{Що ж треба людині, щоб виконати, першу заповідь і уникнути названих гріхів?}

Те, що сказав Христос:

Любити Господа всім серцем, всією душею і всім розумінням. Відчувати Бога своїм Отцем, а себе — Його сином чи дочкою. Знати, що Він любить нас і боятися втратити Його любов. Тоді ми, як діти Божі, любитимемо лише те, що Він любить і будемо противитися тому, чому Він противиться.

«Мені любі друзі Твої, Боже», — каже пророк Давид. — «Чи не зненавидів я тих, що Тебе, Господи, ненавидять? Повною ненавистю я ненавиджу їх. Вони вороги мені» (Пс. 138, 17, 21-22).

\subsection{Якщо перша заповідь наказує шанувати Єдиного Бога, то чи не грішимо ми, шануючи ангелів і святих людей?}

Ні, не грішимо, коли не ставимо їх вище Бога або нарівні з Ним.

Шануючи армію, шанують володаря. Так і тут. Шануючи ангелів, ми шануємо їх, як слуг Божих. Шануючи святих людей, шануємо в них друзів Божих і Божу благодать, яка освячує їх.

«Дивний Бог у святих Своїх» (Пс. 67, 36).

«Святими, що на землі Його, виявив Господь всі хотіння Свої в них» (Пс. 15, 3).

Якщо ж молимося до святих, то лише просимо, щоб вони помолилися за нас Богові.

\section{ДРУГА ЗАПОВІДЬ}

Не твори собі кумирів і всякої подоби того, що (ти бачиш) на небі вгорі, на землі внизу, і в водах під землею; не поклоняйся їм і не служи їм.

\subsection{Що таке кумир?}

То зображення наче «бога» будь-якого творіння. На небі — сонце, місяць, зорі, хмари, блакить, грім, блискавка й ін„ на землі — всі речі, які бачимо: вогонь, море, ліс, звірі і все, що викликає страх. У водах — риби, гади, шумні хвилі і т. п. Під землею — землетруси, вулкани.

Отже, цією заповіддю наказується нічого з того не вважати за «бога», не робити з нього подоб для пошани, не кланятися, не служити їм, наче Богові.

\subsection{А чи не забороняє ця заповідь робити й святих зображеннь — святих ікон — і шанувати їх?}

Ні. Це видно з того, що Господь тому ж Мойсееві, через якого дав ці заповіді, велів поставити у Скінії золоті святі зображення херувимів на покришці ковчега. (Скінія — похідний храм, ковчег — озолочена скринька, що в ній зберігалась манна, скрижалі та жезл, що розцвів). Також були виткані два херувими на завісі перед «Святим Святих», де стояв жертівник кадильний і туди народ кланявся (Вих. 36, 8; 37, 7-9; Євр. 9, 5).

На ці місця треба звернути увагу, бо вони стверджують правильність шанування ікон у Православній Церкві.

\subsection{Що значить слово «ікона»?}

Слово це грецьке і значить «образ», або зображення Бога в тому образі, в якому Він являвся людям: Бога Отця в образі «ветхого днями» дідуся, як Його бачили пророки (Іс. 6, 1; Єз. 1, 26) або Авраамові у виді трьох ангелів та ін.; Ісуса Христа в тому образі, в якому Він був на землі; Духа Святого у виді голуба або вогненних язиків, як Він зійшов на апостолів; святих ангелів, як вони являлися людям; Пречистої Богоматері і святих Божих.

\subsection{Чи друга заповідь не заперечує шанування святих?}

Ніскільки. Кумир — то образ видуманого, то ідол-бовван, ніщо. Хто шанує ідола, той шанує демона і служить йому (1 Кор. 8, 4; 10, 19-21).

А ікона — то образ Істинного Бога Живого або друзів Його. Ніхто не думає, що ікона є якесь божество і що діє вона сама від себе. Таке розуміння було б богопротивним.

Дивлячись на ікону, ми не думаємо про ікону, а про того живого, образ котрого написаний на іконі.

«Ікони — то книги написані не літерами, а лицями», як вчить святий Григорій Богослов, учитель Церкви IV віку.

Ікона вже тим святиня, що на ній написаний образ Бога або святих Його. Але святість ікон збільшується тим, що через них діє Божа благодать. Ми знаємо тисячі ікон, од яких видимо проявлялась Божа благодать до людей. Такі ікони відомі під ім`ям чудотворних.

\subsection{Які ж гріхи проти другої заповіді?}

Це — ідолопоклонство в різних видах.

Крім грубого ідолопоклонства, як поклоніння і служіння ідолам (Перун, Зевс, Діана, Юпітер, Бел), яких шанували погани, є ще «тонке ідолопоклонство», суть якого та сама. Сюди належать:

\begin{enumerate}
    \item Скупарство, коли людина в скупості збиває собі земні скарби і так серцем прив`язується до них, що, забувши Бога, на них тільки й надіється, про них тільки й думає. Тоді багатство стає для такої людини «богом».
    \item Себелюбство до обожнення, бо тоді людина починає сама собі поклонятися, вважає себе найкращою, наймудрішою за всіх. Вона горда, Богу не покірна, і тому богопротивна.
    \item Ненаситливість або обжерете о, п`янство, ласунство. Опанована цим гріхом людина тільки й думає про своє черево, все приносить йому в жертву. Кривдить, обманює, краде і все то для п`янства, для обжирання. Такі люди забувають Бога, бо, як каже апостол, «богом їх черево їх» (Фил. З, 19).
    \item Гордощі та марнославство. Гордий не здатний ні до якого доброго діла. В ньому немає любові до ближніх, бо він вимагає, щоб усі йому служили. Тому то сказано: «Господь гордим противиться, а смиренним дає благодать» (Як. 4, 6; Прит. З, 34).
    \item Користолюбство. Апостол називає його ідолослужінням (Кол. З, 5). Користолюбець не знає милості. Він тільки й думає про те, щоб з кожної людини зискати для себе матеріальну користь. Він не пожаліє нещасного, ні вдови, ні сироти. Матеріальна користь для нього «бог» (ідол).
\end{enumerate}

Вищеназваним гріхам по цій заповіді треба протиставити християнські чесноти:
\begin{enumerate}
    \item Щедрість.
    \item Любов до ближніх.
    \item Повздержливість.
    \item Смирення.
    \item Некорисливість.
\end{enumerate}

\section{ТРЕТЯ ЗАПОВІДЬ}

Не призивай імені Господа Бога твого всує (марно, кощунно, без пошани).

Бог — то Єство вище всякого розуміння. Бог незбагненний. Бог — Святиня. Перед Ним увесь світ ніщо. Ім`я Його святе і страшне Ще. ПО, 9; Лк. 1, 49).

\subsection{З якою ж пошаною повинна людина призивати ім`я Боже?}

Святі праведники й великі мужі справжньої науки ніколи не називали імені Божого, не знявши шапки (наприклад, Ньютон).

\subsection{Коли, ж. прозивається ім`я Боже всує?}

При пустій божбі, у пустих розмовах, у всякій грі і тому подібному.

\begin{enumerate}
    \item Найтяжчий гріх проти третьої заповіді — це блюзнірство-богохульство-кощунство, лайка в Бога, зневага до Бога, до святих, до всього святого. Хто на це пустився, той став на бік сатани, став ворогом Божим, як і диявол. Безумствує той, хто на це пускається, бо свідома хула на Духа Святого не прощається (Мт. 12, 31-32).
    \item Нарікання на Бога та волю Божу.
    \item Неповага до Бога та святощів і кепкування з них.
    \item Неувага в молитві, коли людина ніби й молиться, але з обов`язку, і в душі незадоволена з того.
    \item Неправдива присяга та божба, коли призивають ім`я Боже для того, щоб ствердити неправду.
    \item Порушення присяги, даної правдиво.
    \item Порушення обітниць, даних Богові.
    \item Божба або вживання клятьби з призиванням імені Божого у пустих розмовах.
\end{enumerate}

\subsection{Чи, не суперечить третій заповіді приймання присяги?}

Господь каже: «Не кляніться (ім`ям Божим) ніяк: досить з вас слова: так, так, ні, ні. Що ж більше від цього, те від лукавого» (Мт. 5, 34, 37). Це у щоденному житті кожної людини зокрема.

Однак в ділах громадських друга заповідь не забороняє присяги, як каже апостол: «Люди клянуться Вищим і клятьба, як запевнення, закінчує суперечки» (Євр. 6, 16).
 
Сам Бог для запевнення незмінності Своїх обітувань, вжив клятьби, сказавши Авраамові: «Клянуся Мною Самим» (Бут. 22, 16).

Тому й нам дозволено у важливих випадках вживати присягу перед хрестом та Євангелією і перед Богом, щоб ствердити, що свідчення дано правдиво (наприклад на суді), або що будемо чесно виконувати взяті на себе обов`язки.

\section{ЧЕТВЕРТА ЗАПОВІДЬ}

Пам`ятай день суботній, щоб святити його: шість днів працюй, і роби в них усі діла твої: день же сьомий, субота — Господу Богу твоєму.

Господь наказує святкувати сьомий, а не якийсь інший день, бо Він Сам за шість «днів» створив світ, а в сьомий день спочив від діл Своїх (див. 1-й член Символа Віри). Благословив Бог сьомий день і освятив його (Бут. 2, 2-3). Сьомий день Господь відділив для Себе, щоб у цей день люди прославляли Бога, віддавали Йому славу і чинили діла милосердя.

\subsection{Чому в християнській Церкві святкуеться не сьомий день тижня (субота), а перший (неділя)? І чи не порушується цим закон?}

Старий заповіт був тільки тінню Нового Заповіту (Євр. 10, 1). Його значення закінчилося в момент смерті Спасителя нашого на хресті, коли завіса церковна роздерлася (Мт. 27, 51; Дан. 9, 27). Христос, як Господь суботи (Мт. 12, 8), відмінив суботу згідно з пророцтвом:

«Ось дні надходять, — каже Господь, — і укладу з домом Ізраїля і з домом Юди Заповіт Новий. Не такий, як Я уклав був з батьками їх, коли вивів їх з землі єгипетської... Я вкладу закон Мій у мислі їх і напишу його на серцях їх, і буду Богом їх, а вони будуть Моїм народом» (Єр. 31, 31-33; Євр. 8, 8-12).

Апостольський Собор у 51 році в Єрусалимі відмінив обрядовий закон Мойсея (Діян. 15, 1-32).

Уже перші християни святкували перший день тижня, що звався «день Господній» і «день Воскресний» (Об. 1, 10). В цей день апостоли й перші християни збиралися для богослуження — переламання Хліба (Святого Причастя) на Літургії (Діян. 20, 7).
 
Отже коли в Старому Заповіті приносили Богові сьомий день, тобто шість днів працювали для себе, а сьомий святили для Господа, то в Новому Заповіті по заповіді Божій стали святити для Господа день перший, день воскресіння Господнього, а наступні дні працювали для себе.

\subsection{СВЯТА ПРАВОСЛАВНОЇ ЦЕРКВИ}

Але крім суботи у Старому Заповіті і неділі у Новому Заповіті, Господь повелів святкувати ще й інші спеціально встановлені дні. У Старому Заповіті святкували Пасху, П`ятидесятницю, дні очищення, новомісяччя, ювілейні роки й ін. У Новому Заповіті Свята Церква встановила святкувати крім неділі ще й дні, коли споминаються події, які Господь звершив для нашого спасіння. Також і на честь Самого Господа та Пречистої і Преблагословенної Богоматері, ангелів і святих угодників Божих. Ці дні звуться святами або празника-ми. Головні з них такі (дати відзначення подаються за новим стилем):
\begin{enumerate}
    \item Народження (Рідзво) Пресвятої Богородиці — 21 вересня.
    \item Введення в храм Пречистої Діви Марії — 4- грудня.
    \item Благовіщення Пресвятій Діві про народження від Неї Спасителя світу — 7 квітня.
    \item Успіння Богоматері — 28 серпня.
    \item Різдво Христове — 7 січня, і другий день Різдва, Собор Богоматері — 8 січня.
    \item Стрітення Господнє —15 лютого.
    \item Хрещення Господнє (Богоявлення) — 19 січня.
    \item Преображення Господнє — 19 серпня.
    \item Вхід Господній в Єрусалим — за тиждень до Пасхи.
    \item Пасха. Свято Пресвітлого Христового Воскресіння (Великдень) — в першу неділю після весняної повні місяця.
    \item Вознесіння Господнє — на 40-й день після Пасхи.
    \item Зшестя Святого Духа на апостолів, Свята Тройця (Зелені свята) — на 50-й день після Пасхи. На другий день — свято на честь Святого Духа.
    \item Воздвиження Хреста Господнього — 27 вересня.
\end{enumerate}

День Воскресіння Христового, Свята Пасха Господня — то найголовніше християнське свято, свято із свят, празників празник. Інші ж перелічені тут свята звуться «великими роковими» або «дванадесятими», бо їх є 12.

Православна Церква святкує ще й інші свята Богоматері і великих святих.

\subsection{ПОСТИ ПРАВОСЛАВНОЇ ЦЕРКВИ}

Крім свят, як окреме високе служіння Богові і похвалене Господом, Свята Церква установила пости (Мт. 17, 21). їх є чотири:

\begin{enumerate}
    \item Великий піст. Він триває 40 днів, як і Господь постив у пустині 40 днів. До нього ще приєднуються Лазарева
    субота, Вербна неділя і 6 днів Страсного Тижня.
    \item Пилипівка або Різдвяний піст перед Різдвом Христовим. Триває він 40 днів, починаючи з 27 листопада. Цей піст встановлено Церквою для того, щоб вірні постом достойно приготували себе до зустрічі і святкування свята Різдва Христвого.
    \item Петрівка. Від понеділка після неділі Всіх Святих (тиждень після Тройці) до дня святих апостолів Петра й Павла 12 липня. Піст встановлений на спомин посту святих апостолів.
    \item Спасівка. Піст перед Успінням Богоматері, на спогад того, як Пречиста Діва Марія постом готувала Себе до блаженної кончини життя на землі.
\end{enumerate}

Є ще одноденні пости кожного тижня: середа на спомин Юдиного гріха — зради Спасителя, п`ятниця на спомин страждань і смерті Господа Ісуса Христа.
Також є піст 18 січня, перед днем Хрещення Господнього, у день Воздвиження Хреста Господнього на спомин страждань Христових і в день Усічення голови святого Івана Предтечі, 11 вересня.

Піст не є самоціль. Це засіб приборкати тіло.

Коли людина сита, вона помишляє про земне, тілесне, мирське. А мирське відомо яке: «похоть тіла, похоть очей та гордість житейська». В ситому стані трудно й каятися. Краще каються люди, коли вони не ситі. Як каже апостол, хто страждає, той не грішить...

Сите тіло володіє над духом. Коли ж тіло страждає, то дух починає володіти над тілом. Піст пригнічує тіло. Коли ми навмисно боремося з похотями тіла і приборкуємо його постом для покаяння, то дух наш звільняється, мислі скупчуються, і молитва наша стає чистою й щирою.

Хто для Бога постить, той щиро молиться, кається в гріхах і готовий до всякого доброго діла.

Хто не постить в установлені Церквою дні, той тяжко согрішає, наприклад у Великий піст, а особливо на Страсному тижні. Цим він зневажає страждання Христові і стає невдячним за виявлену любов Христову до нас принесенням за нас Себе в жертву. Якби не було пісної їжі, то й тоді треба зменшити їжу або постити до півдня на вдячність Богові.

\subsection{Як же треба проводити дні неділь та свят?}

Не робити в ці дні робіт житейських. Дні неділі і свят — то Господні дні. Хто в ці дні працює для себе, краде у Господа. Такої роботи Господь не благословить і те, що зробиш, не піде тобі на користь. Сьогодні ніби заробиш, завтра втратиш або прохворієш. Не можна в Бога взяти силою. Краще робити в ці дні те, що заповідав Господь: присвячувати їх на добрі діла:

\begin{enumerate}
    \item Зранку йти до церкви на Службу Божу, бо, як каже святий Іван Золотоустий, «одна молитва в церкві більше значить, ніж сотні поклонів поза церквою під час Служби Божої».
    \item Коли немає можливості бути в церкві, треба молитися дома, прочитати акафіст або кафізму псалтиря чи інші молитви.
    \item Вносити свою посильну датку на потреби Церкви Христової, на поміч убогим, на оздоблення храму, на оливута ін.
    \item Подавати милостиню убогим, відвідувати хворих та засмучених, допомагати й тим, що з якихось обставин опинилися у в`язниці.
\end{enumerate}

Це не значить, що ніби діла милосердя треба робити тільки у свята, їх треба робити щохвилини, на кожному місці, а у свята особливо.

Також і молитися треба щодня.

Що ж треба сказати про тих, які в свята замість церкви йдуть до театру або на інші ігрища, чи починають обідатися та напиватися? Вони зневажають святість днів, які Сам Бог освятив і цим прогнівляють Бога.
 
Не добре й таке, коли люди йдуть до церкви наївшись. Тоді тяжко молитися, нападає дрімота і т. ін. Краще йти не ївши, бо тоді молитва з`єднана з постом, вона щиріша і вгодніша Богові.

Заповідаючи працю на протязі шести днів, четверта заповідь рішуче засуджує лінощі. Ніхто, крім хворих та немічних, не має права їсти, коли не хоче працювати згідно свого стану. Про таких і апостол каже: «Хто не працює, той і не їсть» (2 Сол. З, 10).

\section{П`ЯТА ЗАПОВІДЬ}

Шануй батька й матір: добре тобі буде і довго будеш жити на землі.

Заповідаючи любов до ближніх, Господь наказує нам ту любов найперше виявляти до наших батьків (Вих. 20, 12; Мт. 19, 19;

Любити батьків значить любити себе, бо ми від їхньої крові, тіла й душі. Це любов природна. Навіть тварини з любов`ю ставляться до своїх батьків. Тому непошана до батьків — це явище протиприродне, то порушення законів єства, встановлених Богом.

Кожна людина — великий боржник своїм батькам. Від батьків ми одержали життя. Наша мати в болях породила нас, груддю годувала нас, вболівала з нами від самого малечку і готова віддати за нас своє життя, або врятувати життя наше. Вона не спала ночей над нами, не доїдала, не допивала, не зодягалася як треба, аби ми були ситі і здоровенькі виростали. Батько захищав нас від усього поганого, вчив нас працювати, може голодував, щоб дати нам можливість учитися в школі.

Ніхто не любить так, як батьки. Ніхто так не молиться за дітей, як ненька. Не дарма сказано: «Материна молитва зо дна моря витягає».

Отже любити батьків і шанувати їх не тільки не тяжко, а дуже приємно.

А непошана до батьків дуже тяжить на душі, бо то тяжкий гріх. Хто не любить і не шанує своїх батьків, а, навпаки, зневажає та кривдить, той повстає на своє єство.

Тому то в законі Мойсея за зневагу батьків визначена страшна кара: «Хто злословить батька й матір, хай буде покараний смертю» (Вих. 21, 17).
 
\subsection{Як же шанувати батьків?}

\begin{enumerate}
    \item Любити їх, як себе. Поводитися з ними з пошаною, з любов`ю та повагою. Слухати їх настанов та порад, як каже премудрий: «Сину мій, шануй настанови батька твого і не відкидай поради матері твоєї, бо то найкращий вінець для голови твоєї і добра прикраса для душі твоєї» (Прит. 1, 8-9).
    \item Коритися батькам не за страх, а за совість.
    \item Оберігати їх честь і поводити себе так, щоб не
    засмучувати їх та не осоромлювати іх перед людьми.
    \item Молитися за них.
    \item Оберігати їхнє здоров`я, допомагати їм у праці і в усьому, а на старості полегшувати їхні немочі, годувати та давати притулок.
    \item Не допускати, щоб батьки померли без християнської підготовки — без причастя Святих Таїн.
    \item Поховати їх з належною честю та пошаною.
    \item Молитися за спокій душ їх та подавати за них милостиню.
    \item Згадувати їх з великою пошаною перед своїми дітьми та внуками, з побажанням їм Царства Небесного. Наприклад: «Мій покійний батько,Царство його душі...» Тримати їхні могилки в порядку, прикрашувати квітами, а в належні дні служити за них панахиди або подавати їхні імена для поминання на Літургії або коли інші люди служать панахиди. Взагалі ж молитися за них, згадуючи їх імена у кожній своїй молитві, вранці і увечорі. Це треба робити й тому, що така молитва дуже вгодна Богові, та й батьки моляться тоді за нас.
    \item Виконувати добрі, не протизаконні заповіти їхні.
\end{enumerate}

Найкращий приклад пошани до батьків показав нам Господь наш Ісус Христос. Бувши всесильним Богом, Він з самого малечку, як людина, корився Своєму опікунові Йосифу і Пречистій Своїй Матері, був слухняним, допомагав їм у праці і ніколи нічим не засмучував.

Але найразючіший, навіть потрясаючий приклад синовньої любові до Своєї Пресвятої Матері виявив Він на хресті. Забувши Свої страшні муки, Він зглянув на Матір Свою, що Вона без Нього залишається круглою сиротою самотньою, і усиновив її улюбленому ученикові Своєму Іванові, сказавши так, щоб вороги не почули; «Жено, це син Твій», а йому: «Це Мати твоя». І тільки після того віддав Свій дух Богові» (їв. 19, 26-27).

\subsection{Чому каже Господь, що за потану до батьків тобі добре буде і будеш ти довго жити на землі?}

Бо Господь дає батькам велику силу над дітьми (див. як благословили патріархи Ісаак та Яків своїх синів: Буття 27, 27-29; гл. 49).

«Благословенння батькове зміцняє доми дітей, а прокляття матері руйнує дощенту» (Сірах 3, 9).

Апостол Павло також заповідє пошану до батьків:

«Діти, коріться своїм батькам для Господа, бо цього вимагає справедливість» (Еф. б, 1).

«Діти, будьте слухняні своїм батькам у всьому, бо те вгодне Богові» (Кол. З, 20).

І не тільки на землі Господь обіцяє нагороду за виконання п`ятої заповіді. Коли юнак запитав Христа: «Що мені робити, щоб унаслідувати життя вічне?», Спаситель, між іншим, сказав: «Шануй батька й матір» (Мт. 19, 19).

Але п`ята заповідь заповідає нам пошану не тільки до батьків, а й до всіх, хто так або інакше робить нам добро:

\begin{enumerate}
    \item До Церкви й духовних пастирів, як каже апостол: «Слухайте наставників ваших і коріться їм, бо вони піклуються про душі ваші і за них мають дати відповідь перед Богом, щоб вони з радістю це чинили, а не зітхаючи, бо не корисно вам таке» (Євр. 13, 17).
    \item До батьківщини і її влади, як сказано: «Бога бійтеся, а владу шануйте» (1 Петр. 2, 17). «Віддайте кесареве
    кесареві, а Боже Богові» (Мт. 22, 21). «Бо влада від Бога» (Рим. 13, 1-2).
    \item До учителів та наставників шкільних.
    \item До старших віком. «Старшого не докоряй, а благай, як батька, рівних собі як братів, а старих бабусь як матерів» (1 Тим. 5. 1-2). «Перед сивим лицем устань і вшануй старе обличчя для страху перед Господом Богом твоїм» (Лев. 19, 32).
\end{enumerate}

Велика окраса для юнака, коли він ввічливо поводиться з старшими за себе, і дуже втрачає він в честі, коли зневажливо говорить про старших.
 
Святий апостол заповідає: «Віддавайте всім належну пошану» (Рим. 13, 7).

Одначе, заповідаючи таку пошану до старших, п`ята заповідь ніяк не припускає того, щоб старші зловживали станом старшинства. Так Святе Письмо каже до батьків: «Батьки, не дратуйте дітей ваших, а виховуйте їх в науці та в настановах Господніх» (Еф. 6, 4).

Також і до пастирів: «Пасіть доручене вам стадо Боже не з примусу, а з доброї волі, не для поганої користі, а з доброго серця» (1 Петр. 5, 2).

До начальників: «Начальники, виявляйте справедливість до підвладних, знаючи, що й над вами є Господь на небесах» (Кол. 4, 1).

\subsection{Чи треба коритися батькам чи начальникам, коли б вони навертали нас проти Бога, проти віри або на щось гріховне, недобре, протизаконне?}

Ні. їм треба казати так, як казали апостоли начальникам юдейським: «Не можна слухати людей більше, ніж Бога» (Діян. 4, 19).

Краще зазнати зневаги або навіть муки й смерті, ніж щось зробити проти Бога і Його наказів.

Чеснота по п`ятій заповіті — це слухняність і пошана.

\section{ШОСТА ЗАПОВІДЬ}

Не вбивай.

Життя людині дає Бог, і тільки Бог має право його відібрати. Тому шоста заповідь рішуче забороняє вбивати людей або відбирати їм життя яким би то не було чином. Бо вбити можна не тільки зброєю, але чимось іншим фізично. Можна вбити людину й словом образливим або наклепом чи поводженням з нею так, що вона сама собі заподіє смерть.

Убивство — то смертельний гріх. За вбивство месником стає Господь:

«Мені (належить) помста, Я віддам, каже Господь» (Рим. 12, 19; Втор. 32, 35).

Коли Каїн убив брата свого Авеля, то Господь сказав йому: «Кров брата твого голосить до Мене від землі: ти тепер проклятий на всій землі, що пройняла кров брата твого» (Бут. 4, 10-11).
 
«Хто пролив кров людини, того кров також проллється рукою людини, бо людина створена на образ Божий» (Бут. 9,6).

Отже заповідь Божа наказує нам оберігати життя кожної людини, як себе.

\subsection{Чи можна вбивати на війні?}

Кожне вбивство рівно противне шостій заповіді, і за кров убитих відповідають ті, що своїми протихристиянськи-ми діями доводять до війни.

Буває необхідність захищати батьківщину і свій народ, і при цьому мусять бути і вбивства. Одначе, не можна вбивати тих, що кинули зброю і здалися.

Треба по-людському поводитися з полоненими, не мучити непосильною працею або холодом та голодом, а ставити їх у такі людські умови, при яких вони не могли б шкодити. Не тримати в полоні без необхідності.

Належить лікувати ворожих поранених, як і своїх.

\subsection{Чи можна карати на смерть убивць?}

І це суперечить шостій заповіді Божій. Але хто хапається за меч, той від меча гине (Мт. 26, 52).

Кожний убивця знає, що вбивати він не має права і що вбиваючи, він сам собі виносить смертний вирок. Коли він не виявляє каяття, то правосуддя кожної держави, маючи обов`язок оберігати життя своїх громадян, примушене нищити кожну небезпеку засобами, які вважає найдоцільнішими.

Одначе Господь рішуче вимагає праведного і безстороннього суду над кожним звинуваченим.

\subsection{Як дивитися на невільне вбивство? Чи винен той, хто його вчинив?}

Винен, якщо він не вживав зусиль, щоб того не допустити. За невільне вбивство по правилах церковних на винних накладається покута і забороняється на цілі роки допускати їх до Святого Причастя.

У вбивстві винен не тільки той, хто безпосередньо вчинив убивство, а й той, хто спричинився до нього:

\begin{enumerate}
    \item Суддя, що свідомо засуджує на смерть невинного.
    \item Свідки, які неправдивим свідченням спричиняються до засудження невинного на смерть.
    \item Той, хто міг би так чи інакше врятувати від смерті, наприклад, коли він має достаток і допускає нещасного вмерти з голоду.
    \item Той, хто в корисних цілях виснажує підвладних йому робітників непосильною роботою, обмежує їм їжу та відпочинок і цим спричиняє хвороби та смерть.
    \item Той, хто ховає та звільняє душогубів і тим дає їм можливість чинити нові убивства.
\end{enumerate}

\subsection{Гріх самогубства}

Самогубство — найтяжчий гріх, бо воно робиться наперекір волі Божій, як відчай або протест. Тому Церква Христова відмовляє самогубцям у християнському похороні і дозволяє ховати лише тих, які вчинили це в стані втрати розуму або доведені до того іншими.

До самогубців зараховуються й ті, які нестриманістю, наприклад п`янством та розпустою чи іншими пороками, руйнують своє здоров`я та вкорочують собі віку.

До вбивства належить і те, коли мати вбиває в собі плід, або не годує дитини, щоб вона вмерла.

\subsection{Вбивство духовне}

Крім тілесного вбивства може ще бути вбивство душевне, духовне, а саме:

Коли хто збиває людину з доброї дороги життя на беззаконство, на розпусту, безвірство та безбожництво і тим спричиняє її душі духовну смерть. Про таких каже Господь: «Хто спокусить хоч одного з малих цих, що вірують у Мене, то краще йому було б, якби йому почепити на шию млиновий камінь і він потонув у безодні морській» (Мт. 18, 6).

До гріха за вбивство належать усі діла й слова наші, які суперечать християнській любові до ближніх: заздрість, наклепи, ненависть і подібне. «Бо всякий, хто ненавидить брата свого, той чоловіковбивець» (1 їв. З, 15).

Забороняючи чинити будь-яку шкоду життю наших ближніх, Шоста заповідь одночасно наказує нас всіма силами оберігати їхнє здоров`я та добробут. Вона заповідає:

\begin{enumerate}
    \item Помагати бідним.
    \item Утішати зажурених.
    \item Служити немічним.
    \item Полегшувати стан знедолених.
    \item Зо всіма поводитися лагідно та з любов`ю повчати їх на добре.
    \item Миритися й мирити тих, що ворогують.
    \item Прощати образи й робити добро ворогам, як і каже апостол Павло: «Нехай не зайде сонце у гніві вашому» (Еф. 4, 26).
\end{enumerate}

\section{СЬОМА ЗАПОВІДЬ}

Не перелюбствуй.

Господь спочатку із землі створив одного чоловіка на образ Свій і подобу. А тому, що одному чоловікові було б недобре самому серед інших тварин, було б і нудно, та й розмножуватися він би не міг, то Господь із костей його створив йому дружину. Так би мовити, розділив одного чоловіка надвоє так, що хоч їх тепер стало двоє, але вони — одне тіло. Тому могли породжувати дітей своїх (Бут. 1, 27; 2, 18, 21-25).

Ту ж істину підтвердив Господь Ісус Христос (Мр. 10, 6-9).

Отож по волі Божій подружжя — чоловік і жінка — складають одне тіло.

«Жінка не власна над своїм тілом, а чоловік; і чоловік не власний над своїм тілом, а жінка» (1 Кор. 7, 4). Жінка належить чоловікові, а чоловік жінці.

Святий апостол заповідає: «Так повинні чоловіки любити своїх жінок, як свої тіла, бо хто любить свою жінку, той себе самого любить» (Еф. 5, 28).

«Жінки коріться своїм чоловікам, як Господу» (Еф. 5, 22).

Коли ж чоловік віддає своє тіло іншій, або жінка своє тіло іншому, то вони розривають свою єдність, обкрадають одне одного, сквернять чистоту й святість свого шлюбу, поганять свою родину й руйнують її.

Через те перелюбство є тяжкий гріх перед Богом. Як каже апостол, перелюбники й розпусники Царства Божого не можуть наслідувати (Еф. 5, 5). Чому? Тому, що вони страшно оскверняються.

Страшно тяжкий гріх лучитися з розпутними жінками, бо вони осквернені, а той, хто лучиться з такою, стає з нею одне тіло, і гріх бере на себе (1 Кор. 6, 16).
 
Тяжкий гріх проти сьомої заповіді — це розпуста молоді. Вона вбиває їхній молодий розум і притуплює так, що вони стають нездатні вчитися. Надриває сили, впливає на зріст і красу, дочасно робить старими. Від таких виснажених дітей діти родяться каліками. Наслідком розпусти є виродження нації. Люди стають дрібними, похилими, некрасивими, нерозумними.

Страшний гріх кровозмішання, коли близькі родичі наперекір законові сходяться для подружнього життя. Від таких шлюбів діти часто родяться ідіотами, і батьки стають виною довічного страждання дітей. Тому такі шлюби Церквою заборонено.

Забороняючи сьомою заповіддю перелюбство і всяку розпусту, Господь високо цінить вірність подружжя й чистоту подружнього життя (книга Даниїла, гл. 13).

А ще вище в очах Божих чистота й невинність юнацтва (Буття, гл. 39). Вона дорівнюється чистоті ангелів (Об. 14, 4).

Той, хто обирає для себе монашество (чернецтво) і дотримується його в чистоті, той істинно зодягнувся в ангельський чин.
Велика заслуга у Бога, коли вдівець або вдова зостаються ціломудреними після смерті дружини. То подвиг дуже вгодний Богові.

Сьома заповідь Божа вказує нам оберігати себе не тільки від фактичної розпусти, а й думки свої оберігати від поганого, бо кожний гріх починається в думках: «Всякий, хто погляне на жінку, бажаючи її, той уже любодійствує в серці своєму», — каже Христос (Мт. 5, 28).

Також навчає оберігати й уста свої від поганих слів і вуха свої, щоб не чути, як каже апостол:

«Нехай не виходить із уст ваших ніяке гниле слово» (Еф. 4, 29).

«Розпуста і всяка нечистота не повинні й згадуватися між вами» (Ефес. 5, 3-4).

Взагалі треба уникати всього, що наводить на гріх розпусти: поганих ігор, поганих танців, цинічних вистав, книжок і подібного, щоб не сквернити себе.

Пам`ятаймо слова апостола Павла: «Хіба ви не знаєте, що тіла ваші — то члени Христові?.. Хіба ви не знаєте що тіло ваше — храм Духа Святого, Який живе у вас і Якого ви маєте від Бога? Прославляйте Бога в душах і тілах ваших, бо вони Божі» (1 Кор. 6, 15, 19-20).

\section{ВОСЬМА ЗАПОВІДЬ}

Не кради.

Ще в раю Господь встановив Адамові, а в ньому і всім нам, працю, сказавши: «В поті лиця свого будеш добувати собі хліб» (Бут. З, 19). Тому кожна людина повинна сама на себе працювати.

Господь дуже прихильно ставиться до трудящих. Він благословляє їхню працю успіхом, а їх самих — здоров`ям. А ще більше любить таких, які від трудів своїх уділяють убогим, немічним, сиротам, калікам. Про це каже премудрий:

«Шануй Господа всім серцем твоїм і давай (ради Господа) найкраще з усіх прибутків твоїх, тоді наповняться засіки твої доверху, і кадоби твої переповнятимуться новим вином» (Прит. З, 9-10).

Разом з тим дуже противні Господу ледачі, які не хо-тять на себе працювати, а простягають руку, щоб украсти чи якимось незаконним чином присвоїти собі чуже. Восьма заповідь Божа рішуче забороняє всяку крадіж або незаконне присвоювання собі чужого.

Особливо тяжкий гріх обманом або насильством ошукувати нещасного, сироту, вдову. Цей гріх прирівнюється до вбивства. Також і невиплата, затримка в свою користь, заробленої платні робітникам, бо ця неправда взиває до Бога (Як. 5, 4). Те ж саме — неправдиво відсуджене у вдови, сироти або в убогого (Мт. 23, 14).

Восьмою заповіддю забороняється:

\begin{enumerate}
    \item Грабіжництво насильством.
    \item Крадіжка потайки.
    \item Обман у торгівлі, як продаж поганого товару, видаючи його за добрий, неправильна вага і подібне: «Невірна вага — мерзенність перед Господом, вага ж вірна приємна Йому» (Прит. 11, 1).
    \item Святокрадство — особливо тяжкий гріх проти цієї заповіді. Це крадіжка або присвоювання того, що посвячене Богові і належить Церкві.
    \item Духовне Святокрадство, коли одні продають, а інші купують і захоплюють священничі посади, висвячують і висвячуються за гроші не по достоїнству, не для Бога і спасіння людей, а для поганої для себе користі.
    \item Хабарництво, коли беруть підкупи в суді або хабарі з підлеглих: виправдують винних і засуджують або принижують невинних.
    \item Дармоїдство, коли отримують платню, а нічого не роблять. Вони разом крадуть і незаслужену заробітну плату і ту користь, яку могли б принести своєю працею. Або коли хто здібний працювати і має на те можливість, а замість того простягає руку за подаянням.
    \item Лихварство, коли, користуючись з тяжкого стану нещасних, позичають їм за такі відсотки, які вони не в силах сплатити і потрапляють у ще більшу біду. Також підвищення ціни на хліб та продукти під час голоду і подібне.
\end{enumerate}

Забороняючи названі вище гріхи і подібні їм, восьма заповідь наказує нам:

\begin{enumerate}
    \item Бути безкорисними, цебто не визискувати користі з інших.
    \item Бути справедливими й чесними до людей.
    \item Бути правосудними до себе.
    \item Бути милосердними і, замість того, щоб красти та віднімати у ближнього, поспішати на кожному кроці допомагати йому, бо Бог нам дає для того, щоб ми давали: «Хто дає бідному, той позичає Господу. За нього Господь віддасть» (Прит. 19, 17).
\end{enumerate}

Найвища чеснота по цій заповіді — некорисливість: «Коли хочеш досконалим бути, піди продай майно твоє, роздай убогим, і будеш мати скарби на небесах» (Мт. 19, 21).

\section{ДЕВ`ЯТА ЗАПОВІДЬ}

Не будь неправдивим свідком на ближнього твого.

Неправда противна й нам, грішним людям, але яка ж вона огидна перед лицем всеправедного Бога! Не Господь створив неправду, її появив на світ сатана, як і каже Христос:

«... він був убивцею людей спочатку. Він не встояв у правді, бо в ньому немає правди. Коли ж говорить неправду, то говорить від свого, бо він сам неправда і отець неправди» (їв. 8, 44).
 
Отже всяка неправда богопротивна, і хто нею користується, той стає братом сатані. Тому дев`ята заповідь забороняє всяку неправду проти ближнього. Особливо тяжкі гріхи такі:

\begin{enumerate}
    \item 1. Неправдива присяга на суді. Це гріх смертельний. Таких кривосвідків Господь часто карав тут же на суді. (Наприклад, Сусану — Дан., гл. 13; Ананія й Сапфіру — Діян. 5, 1-11). Бо клястися іменем Господнім на неправду те саме, що обманювати Самого Бога.
    \item Неправдиве свідчення на ближнього і поза судом. Дев`ята заповідь наказує не тільки не свідчити неправдиво, а навіть не слухати наклепів на людей. Навіть тоді, коли справді на комусь є порок, ми не повинні всім про те говорити. Краще, коли можна, виправити.
\end{enumerate}

«Не судіть, — каже Христос, — щоб і ви не були осуджені» (Мт. 7, 1). Суд наш часто буває пристрасний і тому неправдивий. Краще допомогти ближньому звільнитися від пороку, але порадити так мудро, щоб не понизити його в очах людей і не образити.

Дев`ята заповідь забороняє й таку неправду, яка ніби не має наміру комусь пошкодити, бо всяка неправда вводить людей в облуду.

«Відкиньте всяку неправду, — каже апостол, — та говоріть один одному тільки правду» (Еф. 4, 25).

\subsection{Як уникати гріхів проти дев`ятої заповіді?}

Треба загнуздувати свій язик:

«Хто любить життя і хоче бачити щасливі дні, — каже апостол Петро, — нехай стримує язика свого від зла і уста свої, щоб не говорити наклепів» (1 Петр. З, 10).

«Бо хто думає, що він є вірний, а не загнуздує язика свого і обманює серце своє, в того марна віра» (Як. 1, 26).

\section{ДЕСЯТА ЗАПОВІДЬ}

Не побажай жони приятеля твого, не побажай дому ближнього твого, ні поля його, ні слуги його, ні служниці його, ні вола його, ні осла його, ні всього, що є у ближнього твого.

Господь Своїми заповідями забороняє не тільки шкодити ближнім, а навіть бажати шкодити.
 
Десята заповідь забороняє бажати чужого, бо й таке бажання супротивне любові до ближнього.

\subsection{Чому не можна бажати чужого?}

Тому, що коли в душі недобрі бажання, то вона не чиста перед Богом, як і каже премудрий: «Мерзотний перед Богом помисел неправедний» (Прит. 15, 26).

Отже треба оберігати й очищати серце своє від отих внутрішніх нечистот:

«Очистьмо себе від усякої скверни тіла й духа, творячи святиню в страху Божому» (2 Кор. 7, 1).

Бо всякий гріх зароджується в думках, там визріває і примушує нас здійснювати його на ділі. Каже Господь:

«Із серця виходять помисли злі, убивства, перелюбства, розпуста, крадіж, лжесвідкування, хули» (Мт. 15, 19).

«Кожний спокушається від своєї похоті, притягається й приваблюється. Потім похоть, зачавши, породжує гріх, а зроблений гріх породжує смерть» (Як. 1, 14-15).

Гріх проти десятої заповіді зветься заздрощами.

Коли говориться: «не побажай жони приятеля твого», то значить: не дивися на неї з гріховним бажанням. Не бажай також і майна твого ближнього, бо коли будеш бажати, то або готовий будеш украсти або відсудити і впадеш в інші тяжкі гріхи.

\subsection{Що наказує нам десята заповідь?}

\begin{enumerate}
    \item Оберігати чистоту серця від бажань чужого.
    \item Бути задоволеним тим, що дав нам Господь, а коли й бажати щось мати, то здобувати його шляхом праведним, чесною працею на добре.
\end{enumerate}

Найкращий засіб боротися проти гріховних бажань — це часта молитва до Господа Ісуса Христа, коли виникає таке бажання.

\subsection{Заздрість — тяжкий гріх}

Сатана був у Бога найпершим ангелом, але він позаздрив славі Божій і насмілився стати на місце Бога. Це погубило його: він з ангела зробився демоном, із добра перетворився на страшне зло. Через заздрість від погубив перших людей. Єва позаздрила Богові, зневажила його заповідь і цим погубила увесь рід людський (Бут. З, 1-19).
 
Заздрість, мов отрута, стискає серце і примушує або вкрасти, пограбувати або неправдою відсудити.

Заздрість — то початок усіх людських страждань, сварок та війн.

Не бажай чужого: Бог дає все кожному по його силі й заслузі, дає й тобі. Тому не чужого бажай, а проси у Бога того, що тобі потрібно. Чуже, здобуде неправдою, йде не на користь, а на погибіль коли не тобі, то твоїм дітям аж до 14-го роду.

Оберігай чуже, то Господь дасть тобі твоє.

\section{ЯКА КОРИСТЬ ЗНАТИ ЗАПОВІДІ БОЖІ?}

Вони те саме, що віхи по дорозі зимою в хуртовину.

Хто знає заповіді й пам`ятає про них, той завжди знає куди йому йти і як поводитися. Коли навіть він грішить, то заповіді показують йому його гріхи і примушують покаятися.

Каже Давид: «Закон Твій — то світильник для ніг моїх і світло на стежках моїх» Ще. 118, 105).

Юнак запитав Христа: «Що робити мені, щоб унаслі-дувати життя вічне?» Христос відповів: «Додержуй заповіді: не вбивай, не перелюбствуй, не кради, не будь лжесвідком, шануй батька й матір» (Мт. 19, 16, 18-19).

Отож і ви, юнаки й діти, коли ви хочете знати правдиву дорогу життя, то навчіться розуміти заповіді Божі, пам`ятайте їх, тоді не заблудите.

«Бо хто порушує й малу заповідь і так навчає людей, той не осягне Царства Небесного, а хто й сам виконує й інших навчає, той великим буде в Царстві Небесному» (Мт. 5, 19), — каже Спаситель Ісус Христос.

\section{ОСТАННІ НАСТАНОВИ}

\subsection{Як треба користуватися вченням про віру й побожність?}

Треба виконувати те, чого навчає нас Православна віра, пам`ятаючи, що Господь обіцяв нам за те блаженство в Його Небесному Царстві, а за невиконання ми підлягаємо Божому осуду. Як і каже Господь:

«Коли знаєте (закон), блаженні будете, коли буде виконувати його» (їв. 13, 17).

«Той раб, який знав волю господаря свого, але не готувався і не чинив по волі Його, буде багато битий» (Лк. 12, 47).

\subsection{Що робити, коли помітимо в собі гріх або нахил до гріха?}

Зразу ж вирвати його з коренем, як отруту, щоб не вкорінився, і покаятися перед Богом. Дати обіцянку уникати його, а зроблений уже гріх загладити добрими ділами. Так зробив Закхей, коли сказав: «Ось я пів майна мого віддам убогим, і якщо я когось чим скривдив, то поверну йому вчетверо» (Лк. 19, 8).

Якщо ж нам Господь допомагає виконувати ту чи іншу заповідь, то не треба ставити собі в заслугу, бо ми й повинні виконувати їх для себе. Господь у таких випадках навчає нас казати:

«Ми раби нікчемні, ми виконали тільки те, що повинні були виконати» (Лк. 17, 10).
\end{document}