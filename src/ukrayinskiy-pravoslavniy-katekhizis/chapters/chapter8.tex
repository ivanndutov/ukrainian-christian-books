\documentclass[main.tex]{subfiles}

\begin{document}
\chapter{Таїнства Православної церкви}

\subsection{Що зветься таїнством?}

Таїнство — це священнодій, встановлена Самим Ісу-сом Христом, Головою Церкви. При звершенні цієї священно-дії невидимо подається від Бога благодать Пресвятого Духа тому, над ким священнодія таїнства звершується.

Таїнств у Православній Церкві є сім: Хрещення, Миропомазання, Причастя, або Свята Євхаристія (Благодатна трапеза), Покаяння (Сповідь), Священство (хіротонія, висвята в єпископа, священника або диякона), Шлюб (вінчання, подружжя), Маслособорування (Єлеосвячення).

\section{ХРЕЩЕННЯ}

Найперше таїнство, яке звершується над людиною — це святе Хрещення.

Про його значення Христос Спаситель сказав Никодимові: «Істинно кажу тобі: хто не народиться від води й Духа, той не може ввійти в Царство Боже. Бо народжене від тіла є (лише) тіло, а народжене від Духа є дух» (їв. З, 5-6).

З цих слів видно, що святе Хрещення є духовне народження (звище від Бога) від води й Духа.

Таїнство святого Хрещення установив Сам Христос Господь. Він Сам хрестився від Івана в Йордані. А після воскресіння Свого, посилаючи апостолів на проповідь, сказав: «Ідіть, навчайте всі народи і хрестіть їх в ім`я Отця і Сина і Святого Духа» (Мт. 28, 19).

Таїнство святого Хрещення здійснюється так: того, що хрестять, тричі занурюють у воду з словами:

В ім`я Отця. Амінь. (Перше занурення).

І Сина. Амінь. (Друге занурення).

І Святого Духа. Амінь. (Третє занурення). Це — формула святого Хрещення. Без неї охрещення не здійснюється.

\subsection{Що означає трикратне занурення в воду?}

Занурення в воду означає духовне умирання для життя гріховного і зняття вини первородного гріха Адамового.

А виринання з води означає духовне відродження від води й Духа в нове християнське життя.

\subsection{Що потрібно для того, щоб достойно прийняти таїнство святого Хрещення?}

Для тих, що досягли віку 7 років і більше — покаяння і глибока віра. «Покайтеся, — каже апостол, — і нехай хреститься кожний з вас в ім`я Ісуса Христа для охрещення від гріхів, і приймете дар Святого Духа» (Діян. 2, 38).

І Христос Господь те ж говорить: «Хто увірує й охреститься, той спасенний буде» (Мр. 16, 16).

Хрещення дорослих не робиться без підготовки. Спочатку готують людину до святого Хрещення і навчають її віри, молитов, навчають покаятися в гріхах. Це зветься катехизація. А перед самим Хрещенням звершується оголошення хрещеника.

Хрещеник мусить бути вимитий, переодягнений у чисту одежу. До нього ставлять хрещених батьків (кума й куму), які, як духовні батьки, приймуть його від купелі Хрещення і будуть навчати його та керувати ним в доброму християнському житті. Куми самі повинні бути добрими православними християнами. Іновірних або безвірних не можна брати кумами. Не можна також брати в куми людей порочних та розпутних.

\subsection{Як звершується таїнство Хрещення?}

Ставиться купіль з водою, запалюються свічки. На аналой кладуть хрест і Євангеліє. священник в єпитрахилі (іноді в ризі) починає оголошення відповідними молитвами за хрещеника і заклинанням, тобто призиванням імені Божого для прощання від хрещеника всіх підступів диявольських.

Тоді хрещеник тричі відрікається від сатани, від демонів його й від усіх діл його, тричі ж висловлює приєднання до Христа. Урочисто проголошує свою віру в Христа, як Царя й Бога, і тричі промовляє Символ Віри.

Коли хрестять малих дітей, то хрестять їх по вірі їхніх батьків та кумів, а все вище наведене за дитину промовляють куми.

Після того хрещеника помазують святою оливою на знак того, що він прищеплюється до дерева Церкви Христової.

Далі освячується вода, і хрещеника занурюють, як сказано вище, тричі у воду.

Людина після виходу з купелі стає очищеною від гріхів, святою. На знак того на новоохрещеного одягається чиста біла одежина, а на шию одягається хрест, на знак того, що новоохрещений є віднині християнин.

Таїнство святого Хрещення, якщо воно виконано правильно, тобто в ім`я Отця і Сина і Святого Духа, з трьома зануреннями, не може повторюватися над тією ж людиною, бо духовне народження буває раз, як і фізичне (тілесне).

Для того й говориться у Символі Віри: «Визнаю одне Хрещення на відпущення гріхів».

\subsection{Чи, можна хрестити немовлят, коли вони не можуть вірувати і виявляти свою віру?}

В часи Старого Заповіту у восьмий день після народження над немовлятами звершувалося обрізання (Лк. 2, 21), а святе Хрещення у Новому Заповіті заступило собою обрізання, як і каже апостол:

«Ви обрізані обрізанням нерукотворним у звільненні від гріховного тіла в обрізанні Христовому, умерши з Ним у Хрещенні» (Кол. 2, 11-12).

Перед святим Хрещенням, як сказано вже, виголошується заклинання на диявола, який після гріхопадіння прародителів здобув собі деяку владу над людьми.

Святий апостол Павло пише: всі люди, які під благодаттю, «ходять по звичаях віку цього, за князем влади повітряної, за духом, який нині діє в синах супротивлення» (Еф. 2, 2).

Сила заклинання полягає в призиванні імені Христового по обітуванню Його: «Іменем Моїм бісів виганятимуть» (Мр. 16, 17).

Для звершення таїнства святого Хрещення освячується вода. В молитвах освячення випрошується у Бога благодать Святого Духа, яка учинила б ту воду купіллю, що породжує в нове благодатне життя. Тайнозвершитель тричі знаменує воду хресним знаменням і каже: «Нехай під знаменням хреста Твого знищені будуть усі супротивні сили».

\subsection{Яку ж силу має святий хрест?}

Хрест — це знамення Ісуса Христа Сина Божого, на якому Він звершив наше спасіння. То зброя, якою Христос переміг диявола.

Святий Кирил, патріарх Єрусалимський, пише: «Не стидаймося виявляти нашу віру в Розп`ятого і сміливо зображаймо рукою знамення хреста на нашому чолі і на всьому: на хлібі, що їмо, на воді, що п`ємо, при виході з хати й при вході; коли лягаємо спати й коли встаємо, коли йдемо в дорогу, чи в дорозі, чи відпочиваємо. Він є велика оборона, дана нам бідним на дар. Бо це є благодать Божа, ознака вірних і страх для злих духів» («Оглаш. Повчен.» 13, 36).

Вживання хресного знамення почалося з часів апостольських, як про це свідчать учителі Церкви перших віків християнства (Діонисій Ареопагіт, Тертуліан та ін.)

Після помазання охрещеного святим миром, священник і куми з хрещеником з запаленими свічками тричі урочисто обходять навкруги купелі на знак духовної радості і співають: «Усі, що в Христа хрестилися, у Христа одягнулися. Алілуя».

Таїнство святого Хрещення — це двері до Царства Божого. Хто не охрестився, той не є членом Церкви Христової. Це треба пам`ятати всім батькам, які байдуже ставляться до святого Хрещення. Нехрещена дитина загине з невірними, а душі її Господь від батьків пошукає.

Треба пам`ятати й те, що ми після святого Хрещення були святими і на знак того нас одягали в чисту білу одежу (крижму). Блаженний той, хто збереже ту білу одіж аж до смерті незаплямованою! А той, хто, зневаживши святість святого Хрещення, безоглядно грішить, стає більш винуватий, ніж поганин.

«Коли ми, відбігши від скверни світу цього до розуміння Господа нашого Ісуса Христа, почнемо знову сплутуватися з ними, бувши ними переможені, тоді для нас останнє стає гірше першого» (2 Петр. 2, 20).

Для тих нещасних Господь з милосердя Свого дав покаяння для очищення нових гріхів.

\section{МИРОПОМАЗАННЯ}

Це таїнство, в якому при помазанні частин тіла в ім`я Духа Святого, віруючому подаються дари Святого Духа, що дають зростання в благодаті Божій та укріпляють у житті духовному.

Початок помазання св. єлеєм (оливою) ми бачимо ще в Старому Заповіті, коли Господь повелів Мойсееві помазати Аарона й його синів на священство (Вих. 29, 7). Святим єлеєм помазували царів на царство, наприклад пророків Самуїла, Савла й Давида, після чого на них зійшов Дух Святий.

Сам Господь Ісус Христос не мав потреби вживати помазання. Він як Бог дихнув на апостолів і сказав: «Прийміть Духа Святого» (їв. 20, 22).
Святі апостоли після зшестя на них Духа Святого, покладали на новоохрещених свої руки, і на тих сходив Святий Дух (Діян. 19, 6). Але апостоли вживали і помазання св. єлеєм:

«Ви маєте помазання від Святого і знаєте все... і як те Помазання (тобто Дух Святий, даний через помазання) навчає вас усьому, і як воно є правдиве й необманне, то чого воно навчило вас, в тому й перебувайте» (1 їв. 2, 20, 27), — навчає Іван Богослов. Так і апостол Павло каже:

«Той, що утвердив нас у Христі і помазав нас (Той) єсть Бог. Він і вибрав нас і дав запоруку Духа (Святого) в серця наші» (2 Кор. 1, 21-22).
Тому переємники апостолів — архипастирі, а від них і пастирі Церкви Христової для зведення Святого Духа на новоохрещених стали вживати помазання святим миром.

\subsection{Коли і як звершується це таїнство?}

Таїнство Миропомазання звершується над новоохрещеним зразу після Хрещення. Його мають право звершувати як архиєрей, так і священники. Але освячувати миро дозволено тільки архиєреям, як пересмішкам апостольським.

Святим миром помазують чоло, груди, очі, вуха, ніс, уста, руки, ноги й спину, щоб освятити всі почування і всі діяння нового християнина. На кожне помазання частин тіла промовляється формула Миропомазання: «Печать дару Духа Святого».

Таїнство Миропомазання не повторюється

\section{ПРИЧАСТЯ (ЄВХАРИСТІЯ)}

Це таїнство, в якому вірні під видом хліба й вина причащаються (споживають) Тіла й Крові Господа нашого Ісуса Христа на прощення гріхів і життя вічне.

Таїнство святої Євхаристії встановив Господь Ісус Христос на Тайній Вечері, перед Своїми стражданнями, так:

Споживши старозаповітню пасху, як годилося по закону, Христос Господь взяв хліб, звів очі до неба, подякував Богові й Отцю, поблагословив, розламав і, подаючи ученикам, сказав:

«Прийміть і споживайте, це є Тіло Моє, що за вас ламається на відпущення гріхів» (Мт. 26, 26; Мр. 14, 22; Лк. 22, 19).

Так само й чашу після вечері подав і сказав: «Пийте з неї всі. Це Кров Моя Нового Заповіту, що за вас і за мно-гих проливається на відпущення гріхів» (Мт. 26, 27-28; Мр. 14, 23-24; Лк. 22, 20).

\subsection{Що дав нам Господь у таїнстві святого Причастя?}

Христос Господь Сам пояснює: «Хто їсть Моє Тіло і п`є Мою Кров, той має (в собі) життя вічне, і Я воскрешу його в останній день» (їв. 6, 54).

«Хто їсть Моє Тіло і п`є Мою Кров, той у Мені перебуває, і Я в ньому» (їв. 6, 56).

»Якщо не будете їсти Тіла Сина Чоловічеського і пити Крові Його, життя не матимете в собі» (їв. 6, 53).

\subsection{Про що доводить нам це величне й найсвятіше Таїнство?}

Воно доводить нам про безмежну любов Божу до нас. Христос Спаситель так полюбив нас, що не тільки викупив нас Своєю безцінною Кров`ю, а ще й побажав, щоб Його Дорогоцінне Тіло перебувало в нашому тілі, і Пречиста Кров текла в нашій крові. Іншими словами, Христос Господь захотів, щоб вірні були Йому кревними братами й сестрами.

Звідси зрозуміло, яку ціну має для нас Божественна Євхаристія і як нерозумно й шкідливо для себе роблять ті, що ухиляються від Святого Причастя.

\subsection{Як і коли звершується тайна Святої Євхаристії?}

Вона звершується єпископами й священниками на Божественній Літургії. Звершується лише в храмі на престолі, на освяченому архиєреєм антимінсі. (Антимінс — це шовкова хустка, на якій зображено покладення Христа Господа в гробі, а вище зашита часточка святих мощів).

Літургія складається з трьох частин: Проскомидії, Літургії оголошених і Літургії вірних.

\subsubsection{Проскомидія}

Па Проскомидії особливим чином приготовляють хліб і вино для святої Євхаристії. Звершує Проскомидію лише один священник, на жертівнику, ліворуч престолу.

Для Проскомидії потрібно п`ять проскур (просфор) з чисто пшеничного білого борошна (на архиєрейській Службі вживається сім проскур). Проскура складається з двох частин — верхньої і нижньої, на знак двох природ в Ісусі Христі — Божественної й людської. На верхній частині хрест з написом ІС ХС НІ-КА, що означає ІСУС ХРИСТОС ПЕРЕМОЖЕЦЬ.

Промовляючи слова пророка Ісайї (з 53-ї глави) про страждання Христові, священник виймає з першої проскури чотирикутну частку і кладе її на дискос (срібне блюдце на ніжці). Ця частка зветься Агнець, бо на Літургії вірних він має принестися в жертву, як старозаповітнє пасхальне ягнятко (Вих. 12;, 3-14 Іс. 53, 7).

Тоді ж вливається й вино в чашу і трошки води на спомин того, що з проколотого ребра Христового витекла кров і вода.
Після того священник споминає всю Церкву небесну. Виймає з другої проскури частку на честь Пресвятої Богородиці і ставить на дискосі праворуч Агнця. З третьої виймає дев`ять часток на честь дев`яти чинів ангельських, на спомин Івана Предтечі, пророків, апостолів, мучеників, преподобних, безсрібників, богоотців Якима й Анни і того святого, чия правиться Літургія — Івана Золотоустого чи Василія Великого. Частки їх ставляться ліворуч Агнця.

З четвертої проскури виймаються частки за Церкву, державу й за здоров`я живих, а з п`ятої — за померлих. Частки їх кладуться в ногах Агнця.
Для святого Причастя потрібний лише один хліб (проскура), як і каже апостол: «Один хліб, одне тіло, хоч і багато нас, бо всі від єдиного Хліба причащаємося» (1 Кор. 10, 17).

\subsubsection{Літургія оголошених}

Так зветься друга частина Літургії, бо на ній мають право бути й ті, що ще готуються до святого Хрещення. На Літургії оголошених слуханням слова Божого та молитвами вірні готують себе до звершення святої Євхаристії.

\subsubsection{Літургія вірних}

Третя й найголовніша частина Літургії — це Літургія вірних, бо на ній мають право бути присутні тільки вірні.

Після двох коротких єктеній і тайних молитов архиєрея або священника співається Херувимська пісня: «Ми таємно херувимів з себе уявляємо». Цей момент важливий тим, що присутні у храмі, злившись у спільній щирій молитві, правдиво духовно стають херувимами.

Після першої половини Херувимської пісні відбувається Великий вхід. На ньому приготовлені під час Проско-мидії святі Дари — хліб і вино — урочисто переносять з жертівника на престол. Цей вхід нагадує нам вхід Христа Спасителя в Єрусалим на добровільні страждання.

Далі священнослужителі й вірні в молитвах та єктеніях просять Бога очистити їх від гріхів і зробити достойними великого Таїнства, а на предложені Дари послати Пресвятого Духа. Всі вірні перед Богом ісповідують свою віру, співаючи Символ Віри: «Вірую в Єдиного Бога...»

Після того починається Євхаристичний канон, під час якого звершується сама Тайна перетворення хліба на Тіло Христове, а вина на Кров Христову. В молитвах Євхаристичного канону архиєрей або священник споминають усі благодіяння Божі родові людському і особливо зіслання у світ Сина Божого на страшні муки. Споминають Його страждання, смерть і погребіння, воскресіння на третій день, во-знесіння на небо і друге Страшне Пришестя.

Далі священнослужитель молиться про зіслання Святого Духа і просить Господа, щоб перетворив хліб на Святе Тіло Христове, а вино — на Святу Кров Христову. Він благословляє святий хліб і каже: «І сотвори хліб цей Пречистим Тілом Христа Твого. Амінь». А потім чашу, кажучи: «А те, що в чаші -- самою Чесною Кров`ю Христа Твого». Втретє благословляє обох словами: «Перетворивши їх Духом Твоїм Святим. Амінь. Амінь. Амінь».

Всі впадають і вклоняються до землі Господу, Який тепер уже перебуває у Святих Дарах.

В наступних молитвах усі готуються до святого Причастя. Причащаються спочатку священнослужителі, а потім вірні.

Сама Тайна перетворення хліба на Тіло Христове, а вина на Кров Христову є незбагненна для людського розуму. Святий Іван Дамаскин так пише про Божественні Тайни:

«То є Істинне Тіло, прийняте Господом від Святої Діви, з`єднане з Божеством. Але не так, ніби вознесене тіло (на небо) знов на землю сходить, а так, що самий хліб і саме вино перетворюються на Тіло й Кров Господні. Коли ж шукаєш пояснення: як те стається, то досить з тебе того, що чуєш, як Духом Святим Собі Самому і в Собі Самому Господь тіло Собі зіткав. Більше нічого не можемо сказати; знаємо тільки, що Слово Боже істинне, діюче і всесильне, але дія Його недослідима» (Кн. 4, розд. 13).

\subsection{Що потрібно для того, щоб достойно приступати до святого Причастя?}

Треба перевірити перед Богом своє сумління, покаятися, молитвою, постом і сповіддю очистити себе від гріхів.

«Нехай перевірить себе людина, — каже апостол, — і тоді від Хліба нехай їсть і від Чаші нехай п`є, бо хто їсть і п`є недостойно, той осуд собі їсть і п`є, не розважаючи про тіло Господнє» (1 Кор. 11, 28-29).

\subsection{Яка користь від Святого Причастя?}
Кожний, хто причащається, найтісніше з`єднується з Христом Господом, і через Нього стає спільником життя вічного.

«Хто їсть Моє Тіло і п`є Мою Кров, той має життя вічне» (їв. 6, 54).

«Хто їсть Моє Тіло і п`є Мою Кров, той у Мені перебуває, і Я в Ньому» (їв. 6, 54, 56).

\subsection{Наскільки часто можна причащатися?}

Хто чистий душею, той може і щонеділі причащатися, як це робили древні християни. А нам свята Церква велить причащатися принаймні чотири рази на рік, коли не кожного місяця.

\subsection{Як повинні тримати себе на святій Літургії ті, що не причащаються?}

На Божественній Літургії повинні бути присутні не тільки ті, що причащаються, а кожний вірний і кожна вірна. Обов`язок кожного християнина — брати участь в Божественній Літургії вірою й молитвою та спогадами про Господа Ісуса Христа, Який сказав: «Чиніть це на спогад Мені» (Лк. 22, 19).

\subsection{Спогади про Ісуса Христа в Божественній Літургії}

Такими спогадами повна вся Літургія:

Проскомидія знаменує народження Христове.

Малий вхід з Євангелієм — зшестя Христа на проповідь. Читання Євангелія та Апостола — проповідь Самого Господа і святих апостолів. Великий вхід на Херувимській — зшестя Господа на страждання.

Увесь Євхаристійний канон — це спомин про Тайну Вечерю, суд, страждання, розп`яття, смерть, погребіння Спа-сителя нашого.

Після молитви «Отче наш» — спомин про воскресіння Христа Господа і явлення Його апостолам (виніс чаші для причастя). Віднесення чаші з престолу на жертівник нагадує вознесіння Господа на небо.

Переживаючи молитовне всі ті спогади, серце наповнюється тихою неземною радістю, втіхою та спокоєм. Таке чуття стається з нами навіть якщо ми не причащаємося, бо Дух Святий Своєю благодаттю обвіває наші серця.

Так що на Божественній Літургії всі присутні в Церкві наповняються Духом Святим і стають причасниками Божої благодаті духовно, як і співають: «Ми бачило Світло Істинне, прийняли Духа Небесного, знайшли Віру Істинну...»

Божественна Літургія і Тайна святої Євхаристії будуть звершуватися аж до другого Пришестя Христового.

Каже апостол: «Скільки будете Хліб цей їсти й Чашу цю пити, будете про смерть Господню сповіщати, аж доки Він прийде» (1 Кор. 11, 26).

\section{ПОКАЯННЯ (СПОВІДЬ)}

Це таїнство, в якому той, хто щиро кається перед Богом, сповідає (виказує) свої гріхи перед священником і через видиме прощення гріхів від священника, невидимо одержує прощення від Самого Ісуса Христа.

Покаяння дане людям від Бога з самого початку, як засіб виправляти свої помилки й провини пред Богом. Так Господь кликав уже Адама до покаяння, коли питав: «Адаме, де ти?» І коли б той був покаявся, Господь би простив йому. Так Ной з наказу Божого проповідував допотопним людям, щоб вони покаялися (Бут., гл. 6).

У Старому Заповіті ми маємо багато величних прикладів покаяння. Наприклад, каялися царі Давид, Єзекія, Манасія й ін. Каялися ніневітяни (книга пророка Йони). До покаяння кликали всі пророки. Особливим же проповідником покаяння був святий Іван Предтеча. Як каже евангелист, «приходили люди й сповідали перед ним гріхи свої» (Мк. 1, 4-5).

Коли Господь послав учеників Своїх на проповідь, то вони пішли і проповідували, кажучи: «Покайтеся, наблизилось бо Царство Боже» (Мр. 1,15; Діян. 2, 38).

\subsection{Що значить покаятися?}

Ворог Бога, лукавий сатана, завжди спокушає людей на гріх, на погані протизаконні вчинки, то розпалюючи їх чуття до пристрасті, то приваблюючи оманною насолодою гріха. Тому люди легко впадають у тяжкі гріхи, які віддаляють їх від Бога, бо всякий гріх не від Бога, а від диявола (1 їв. З, 8). Тому то вчинки свої завжди треба перевіряти і, якщо вони суперечать заповідям Божим, треба щиро покаятися, тобто осудити себе перед своєю совістю, поставити в серці твердий намір не повторювати того гріха, а просити в Бога допомоги й прощення.

Одначе таке покаяння, хоч і щире, не є закінченим, бо хоч ми й покаялися, відповідальність за вже вчинений гріх все ж таки тяжіє на нас.

Тому, щоб зовсім очиститися від гріха, тобто одержати від Бога прощення, треба не тільки розкаятися, а ще й зробити так, як заповідав Господь: сповідати (виказати) гріхи свої перед Церквою в особі її священства — перед священником — і одержати через нього відпущення гріхів.

Бо Господь дав апостолам, а через них і їх наступникам — архиєреям, а далі священникам, в`язати й розв`язувати гріхи людям: «Що зв`яжете на землі, буде зв`язано й на небі, і що розв`яжете на землі, буде розв`язано й на небі» (Мт. 18, 18).

А після воскресіння Свого у всій повноті дав апостолам на те владу, коли дихнув на них і сказав: «Прийміть Духа Святого. Кому відпустите гріхи — відпустяться, а на кому зоставите — зостануться» (їв. 20, 22-23).

Таким чином, покаяння складається з двох частин:
\begin{enumerate}
    \item Із внутрішнього визнання своєї провини перед Богом, жалкування, що так зробив, самоосудження і твердого наміру виправити своє життя.
    \item Із щирої сповіді гріхів своїх перед священником і одержання від нього в імені Божому прощення гріхів.
\end{enumerate}

Самий акт прощення гріхів священником складає таїнство Покаяння.

Одначе й після щирої сповіді не треба забувати своєї вини перед Богом, бо ми гріхами образили Бога. Треба правдиво жалкувати, що ми так зробили, просити у Бога пробачення і силкуватися добрими ділами загладити вину свою, як той Закхей, що казав: «Якщо я когось чим скривдив, то поверну йому вчетверо» (Лк. 19, 8).

«Бо скорбота перед Богом незмінно приводить до покаяння на спасіння» (2 Кор. 7, 10).

«І коли грішник повернеться від беззаконства свого і стане чинити правосуддя й правду, він зостанеться живий через них» (Єз. 33, 19).

«Про Нього (Ісуса Христа) всі пророки свідчать, що кожний, хто увірує в Нього, одержить прощення гріхів іменням Його» (Діян. 10, 43).
 
\subsection{Які є засоби, що готують до покаяння?}

Це — піст і молитва, бо вони наближають до Бога, а Господь розм`якшує серце і нахиляє до покаяння.

Іноді буває, що внутрішнє розкаяння й самоосуд не заспокоюють сумління грішника, і він бажає зовнішньої кари або подвигу, щоб загладити гріх і заспокоїти совість. Тоді духовник накладає на нього покуту або епітимію на певний реченець. Наприклад, нічні молитви, поклони, пости, обов`язок творити милостиню або щось інше, як знайде краще духовник. Іноді духовник забороняє на певний реченець приступати до святого Причастя.

Каятися треба кожний день і час:

«Гнівайтесь, та не грішіть. Нехай не зайде сонце у гніві вашому» (Еф. 4, 26).

«На ліжках ваших розкайтеся» (Пс. 4, 5).

А сповідатися треба перед кожним Причастям, бо таїнство Покаяння очищає від гріхів і робить достойними прийняти Тіло і Кров Господні.

\subsection{Чи можна сповідатися у священника, який сам согрішає?}

За свої гріхи він відповідає перед Богом. Коли ж недостойно сповідає, то гріхи бере на душу свою. Одначе той, хто сповідається, не терпить шкоди у сповіді своїй, якщо щиро кається.

\section{ТАЇНСТВО СВЯЩЕНСТВА}

Це — таїнство, в якому Дух Святий настановляє правильно обраного достойного кандидата через святительське рукоположення священнодіяти таїнства й пасти отару Христову. Священство установлене Богом ще у Старому Заповіті. Господь повелів Мойсееві посвятити Аарона на первосвященника, синів його на священників, а все покоління Левія на левитів, тобто нижчих служителів при Скінії (Лев. 8. 1-36).

Христос Господь, установляючи Новий Заповіт, не відмінив священства, як інституту священнослужителів Церкви, лише наповнив його новим змістом, як і сказав: «Я не прийшов зруйнувати Закон, а виповнити» (Мт. 5, 17).

У Старому Заповіті Господь повелів Мойсееві помити Аарона й його синів та посвятити в священство, помазавши їх святою оливою (Лев. 8, 5-13). В Новому Заповіті Сам Син Божий, як вічний Первосвященник-Архиєрей (Пс. 109, 4; Євр. 5, 6; 8, 1-2; 9, 11-14) на Тайній Вечері помив ноги ученикам, освятив їх Причастям Тіла й Крові Своєї під видами хліба й вина і заповів їм священнодіяти на спомин Йому (Лк. 22, 19).

А коли воскрес, то того ж дня явився ученикам і сказав: «Мир вам! Як послав Мене Отець, так і Я посилаю вас» (тобто з повнотою духовної влади), дихнув на них і сказав: «Прийміть Духа Святого. Кому відпустите гріхи — відпустяться, а на кому зоставите — зостануться» (їв. 20, 21-23).

«Дана Мені всяка влада на небі й на землі. Ідіть, навчайте всі народи і хрестіть їх в ім`я Отця і Сина і Святого Духа. Навчайте їх додержуватися всього, що Я заповідав вам» (Мт. 28, 18-20).

Цими словами Христос Господь передав апостолам повноту архиєрейської влади у Церкві Своїй.

У п`ятдесятий день після Воскресіння Христос послав їм від Отця Духа Святого, як помазання силою звище (Діян. 1, 8; 2, 1-4), і цим довершив їхнє освячення.

Так установив Господь найвищий ступінь священства — апостольсько-архиєрейську, і повелів строїти тайни Божі.

«Так нехай кожний розуміє нас, як слуг Божих і строїтелів таїн Божих» (1 Кор. 4, 1).

Святі апостоли, як архиєреї, відразу після зшестя на них Святого Духа, почали строїти ті тайни Божі. Проповідували, хрестили, а потім рукоположили на дияконське служіння сімох мужів, звівши на них Духа Святого через покладення на них рук (Діян. 6, 5-6). Цим вони поклали початок першого, найнижчого ступеня священства.

Далі святі апостоли стали настановляти для окремих Церков пресвітерів-священників, які діяли під наглядом апостолів («пресвітер» значить старший) Ці пресвітери іноді звалися єпископами, бо вони наглядали за дорученоюю їм церквою (грецьке слово «єписко-пео» означає наглядаю). Але вони не були архиєреями. Коли ж церкви намножилися і апостоли вже не мали змоги скрізь самі управляти, вони стали настановляти собі наступників з повною архиєрейською владою (1 Тим. 4, 14), щоб вони керували домом Божим (1 Тим. З, 15), настановляли (рукополагали) пресвітерів (1 Тим, 5, 22; Тит. 1, 5), достойних нагороджували (1 Тим. 5, 17), винних судили (1 Тим. 5, 19-20).

Так були наставлені апостолами Тимофій, Тит, Полікарп, Кіпріян, Лін, Анаклет, Климент та інші через покладення на них апостольських рук (1 Тим. 4, 14) і освячення їх Духом Святим.

Після апостолів стали рукополагати їхні наступники одні за одними, і таким чином установилося апостольське переємство священства через архиєреїв від одного до другого, яке продовжується у Православній Церкві аж до наших днів.

В перших трьох віках християнства кожний єпископ у своїй області в міру потреби мав право рукоположити собі наступника. Пізніше Вселенські Собори постановили дотримуватися 1-го Апостольського правила: «Єпископа нехай настановляють два або три єпископи, а інші єпископи тієї області нехай дадуть на те свою писемну згоду».

\section{ПРАВА ЄПИСКОПІВ, священниКІВ 1 ДИЯКОНІВ}

\subsection{Єпископ}

Єпископ-архиєрей, як апостольський наступник, є головним керівником Церкви своєї області як духовно, так і адміністративно. Він є відповідальним за моральний стан вірних його області як перед Богом, так і перед вищою церковною владою, теж і за адміністративний у ній порядок. Він рукополагає пресвітерів і дияконів та відповідає за них. Йому підлягають як духовно, так і адміністративно, всі парафії його області. Він призначає настоятелів та інших духовних, в разі потреби знімає їх. Йому прислуговує право нагороджувати за ретельну службу, йому ж належить і суд за проти-християнські, протиканонічні і протиморальні вчинки як підлеглого йому духовенства, так і вірних. Він благословляє шлюби і розв`язує їх.

Єпископ звершує всі таїнства і всі богослужбові відправи, також освячує миро й антимінси.

Єпископ, як наступник апостолів, є носієм прав священства. В Службах Божих він являє собою Христа Спасителя і тому йому треба віддавати найвищу пошану. Єпископа має право судити лише собор єпископів. Як зовнішню ознаку, єпископ, крім хреста, носить на грудях панагію (медальйон з образом Спасителя або Богоматері), а на богослуженні поверх священничих одеж носить саккос замість риз, омофор і митру.

\subsection{священник}

священник (пресвітер) з благословення свого обласного єпископа проводить своє пастирське служіння в дорученій йому парафії. Звершує всі богослужбові відправи і всі таїнства, крім таїнства священства.

священник у своїй парафії є пастирем, наставником і першим порадником своєї пастви. Він є духовним отцем і вчителем віри Христової та доброго християнського життя. В богослуженні він, як і архиєрей, часто зображає собою Христа Господа, наприклад на Божественній Літургії, особливо тоді, коли звершує Божественну Євхаристію та подає Святі Христові Тайни.

Тому священникові треба віддавати велику духовну пошану заради Христа Господа. При зустрічі брати благословення, при чому не про священника тоді думати, а про Господа Ісуса Христа, бо рукою священника Сам Господь благословляє.

священник є ангел (вісник) Господа Вседержителя, як каже Господь через пророка Малахію (Мал. 2, 7). Бо він посередник між Богом і людьми, як і єпископ, та відповідає за доручену йому паству. Він є устами Божими до народу й устами народу до Бога. Звання його дуже високе. Це ж саме накладає на нього обов`язок бути зразком для вірних у всьому (1 Тим. 4, 12).

священник як пастир керує своєю паствою за дорученням єпископа і за його настановами. Без волі єпископа він не має права залишити доручену йому паству і переходити до іншої.

\subsection{Диякон}

Диякон є найближчим помічником як єпископа (протодиякон), так і священника на богослуженні. Диякон також є помічником настоятеля в навчанні віри членів парафії.

Диякони виголошують єктенії на богослужбових відправах і тим самим стають устами народу в молитві до Бога. Дияконське служіння є перехідне, бо кожний диякон повинен готувати себе до вищого ступеня — священства. Тому й його моральні обов`язки не менші, ніж обов`язки священника бути у всьому зразком для вірних. Диякон є освячений служитель Божий і тому заслуговує на добру пошану з боку вірних (1 Тим. З, 8-13).

\subsection{Як відбувається хіротонія (висвята) єпископа, священника, диякона?}

Єпископа висвячують два або три єпископи на Божественній Літургії після співу «Святий Боже». Тоді архиє-реї кладуть на нього святе Євангеліє словами вниз, як знак руки Христової і свої руки. Старший єпископ виголошує відповідні молитви, новопоставленого облачають в архиєрейські одежі, і він бере участь у Божественній Літургії, як архиєрей.

священника висвячує один єпископ на Божественній Літургії після Херувимської пісні. Кандидат є дияконом і до часу висвяти приймає участь у Службі Божій, як диякон. Після Херувимської пісні його обводять тричі навкруги престолу із співами відповідних тропарів. Далі він стає праворуч престолу на коліна, поклавши руки на престол. Єпископ кладе на нього краї омофора й обидві руки та читає відповідні молитви. Новопоставленого облачають у священничі одежі, і він бере участь у Євхаристійному каноні, як ієрей.

Диякона висвячує архиєрей після Євхаристійного канону так само, як ієрея, але молитви читає інші. Після молитов новопоставленого одягають у дияконські одежі. Він причащається як диякон і виголошує останню єктенію.

За 68-м Апостольським правилом ні одна з цих трьох хіротоній не може бути повторена над однією й тією ж особою, а коли б хто це зробив, то правило вимагає позбавлення сану як того, хто святив, так і висвяченого.

\section{ШЛЮБ}

Шлюб — це таїнство, в якому жених і наречена при добровільній обіцянці вірності одне одному перед священником і Церквою, благословляються на подружнє співжиття на зразок спілки (союзу) Христа з Церквою, і їм випрошується благодать чистої однодушності для благословенного породження й християнського виховання дітей.

Таїнство Шлюбу установлене Богом при самому створенні мужа й жони. Господь благословив перших людей і сказав: «Плодіться, розмножуйтесь» (Бут. 1, 28). Христос Спаситель підтвердив святість шлюбу, коли сказав фарисеям: «Покине чоловік батька свого й матір, пристане до жони своєї і будуть удвох одне тіло, і що Бог з`єднав, люди нехай не розлучають» (Мт. 19, 5-6).

Шлюб тоді є законним, коли він освячений Церквою. Без того немає шлюбу, а лише беззаконне співжиття. Як і Христос сказав самарянці: «Ти п`ять мужів мала, але й той, що тепер маєш, не є тобі муж» (їв. 4, 18).

Подружнє життя — то велика святиня перед Богом. Чесне подружжя й ложе нескверне. За велику вірність Господь нагороджує щастям у дітях. І навпаки, зрада подружжя дуже гнівить Бога. Найтяжчий смертельний гріх — це перелюбство (зрада дружині). Апостоли твердо зазначають, що перелюбники, зрадники, розпусники Царства Божого осягнути не можуть (Еф. 5, 5). Чоловік і жінка удвох є одне тіло, а тому й повинні любити й оберігати одне одного, як себе, до самої смерті. Сьома заповідь просто каже: «Не перелюбствуй». А в Старому Заповіті за подружню зраду побивали камінням.

\section{МАСЛОСОБОРУВАННЯ (ЄЛЕОСВЯЧЕННЯ)}

Це таїнство, в якому при помазанні хворого святою оливою (єлеєм) призивається на нього благодать Божа, що зціляє немочі душевні й тілесні.

Маслособорування має свій початок від апостолів, які, бувши послані Ісусом Христом, «мазали оливою багатьох недужих, і (ті) зцілялися» (Мр. б, 13).

Те ж таїнство святі апостоли передали й Церкві. «Коли хто хворіє у вас, — каже апостол Яків, — нехай покличе пресвітерів церковних і нехай помоляться над ним і помажуть його оливою в ім`я Господнє. Молитва віри спасе хворого, і підійме його Господь, і якщо гріхи содіяв, простяться йому» (Як. 5, 14-15).

Це таїнство зветься ще «соборуванням», бо його повинен, за словами апостола, звершувати собор священників у кількості сім, хоч на практиці його виконує один священник.

Таїнство єлеосвячення може повторюватися.
\end{document}