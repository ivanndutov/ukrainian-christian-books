\documentclass[main.tex]{subfiles}

\begin{document}
\chapter{Бог і найвище знання у світі}

Найвище знання у світі — це знання Бога.

\emph{{\color{red}«Життя вічне в тому, щоб знали Тебе, Єдиного й Істинного Бога і Того, що Ти послав - Ісуса Христа»}}, — так сказав Ісус Христос, звівши очі Свої до неба у молитві до Свого Отця Небесного \emph{(Ів. 17, 3)}. Продовжуючи Свою молитву, каже Христос:

\begin{FlushRight}
    \emph{{\color{red}«Я виявив Ім`я Твоє людям... Тепер зрозуміли вони, що все, що Ти дав Мені, є від Тебе»} (їв. 17, 6, 8). }
\end{FlushRight}

Зрозуміли істинно і увірували.

Святий апостол Павло у своєму посланні сказав:

\begin{FlushRight}
    \emph{«... той, хто до Бога приходить, мусить вірувати, що Він є, а тим, хто шукає Його, Він дає нагороду». (Євр. 11, 6).}
\end{FlushRight}

Далі Христос у тій же молитві каже:

\begin{FlushRight}
    \emph{{\color{red}«Не за них (апостолів) тільки благаю, а й за тих, що увірують у Мене через слово їх. Щоб усі були одно, як Ти, Отче, в Мені, і я в Тобі, щоб і вони в Нас були одно. Ти полюбив Мене перше создання світу... Щоб любов, що нею Ти полюбив Мене, в них була, і я в Них» } (їв. 17, 20-21, 24, 26).}
\end{FlushRight}

\section{Чому богопізнання так потрібне?}

Тому що від нього залежить вся наша поведінка. Якщо Бог є і Він створив нас, то людина має безмежно високе покликання і призначення. Вона не менша від ангелів. Вона — цар землі, призначена і настановлена від Бога:

\begin{FlushRight}
    \emph{«Коли погляну на небо — діло рук Твоїх, на місяць і зорі, що Ти їх поставив, то що є чоловік, що Ти пам`ятаєш про нього, і син чоловіка, що Ти опікуєшся ним? Ти створив його мало чим меншим від ангелів, славою й честю увінчав його. І поставив над творінням рук Твоїх; все підкорив під ноги Його», — каже пророк Давид (Пс. 8, 4-7).}
\end{FlushRight}

Коли Бог є, то людина має безсмертну душу, яка призначена до вічного щастя, вічної радості. Тоді життя людини — щастя.

Коли ж Бога нема і коли насправді людина походить з тварин, тоді вона ніщо. Тоді вона нічим не відрізняється від тварин, хіба тільки тим, що підліша. Тоді життя її поспільне нещастя, бо страждань у її житті більше, ніж утіх. Тоді немає на що надіятися у майбутньому, бо там нічого немає. Тоді й буття світу якесь непорозуміння. Життя втрачає смисл: бо коли попереду нічого немає, то для чого мучитися?

\section{Що ми розуміємо під іменем Бог?}

Під іменем Бог ми розуміємо живу, свідому себе, розумно діючу Істоту, Яка створила всесвіт і управляє ним.
Приступаючи до вивчення основ богознання, візьмім до уваги слова Премудрого:

\begin{FlushRight}
    \emph{«Сину мій, здобувай мудрість, здобувай розуміння. Не забувай цього і не віддаляйся від слів уст моїх. Не цурайся мудрості, вона буде охороняти тебе. Полюби її, вона буде берегти тебе» (Прит. 4, 5-6).} Бо:
\end{FlushRight}

\begin{FlushRight}
    \emph{«Господь премудрістю збудував землю і небеса утвердив розумом» (Прит. З, 19).}
\end{FlushRight}

\begin{FlushRight}
    \emph{«Початок (же) премудрості — страх Божий» (Прит. 1,7).}
\end{FlushRight}

\begin{FlushRight}
    \emph{«Господь дає премудрість, із уст Його знання й розуміння» (Прит. 2, 6).}
\end{FlushRight}

\section{Пізнання Бога через вивчення явищ всесвіту та видимої природи}

Про це пізнання св. апостол Павло каже: \emph{«Що можна знати про Бога, те відкрите для людей. Бо Його невидима вічна сила й Божество від початку світу через розглядання творінь стають видимі» (Рим. 1, 19-20).}
 
\begin{FlushRight}
    \emph{«Від величності краси природи пізнається Виновник її» (Прем. 13, 5).}
\end{FlushRight}

Це шлях правдивої науки. По ньому багато навіть поган прийшли до богознання. Видима природа — то жива книга, по якій кожний має можливість довідатися про її Творця:

\begin{FlushRight}
    \emph{«Небеса проповідують про славу Божу, і про діла рук Його сповіщає твердь. День дневі передає слово, і ніч ночі відкриває розуміння» (Пс. 18, 2-3).}
\end{FlushRight}

\begin{FlushRight}
    \emph{«Господь вичерпав води жменею Своєю, і п`яддю виміряв небеса, вмістив мірою порох землі, зважив гори» (Іс. 40, 12).}
\end{FlushRight}

\section{Премудрість Божу видно на всіх її творіннях}

На кожній тварині і рослині ми спостерігаємо турботливу премудрість Творця. Краса, симетрична будова, скерована на користь доцільність — все це яскраво свідчить, що над ним хтось думав і турбувався.

Тому правдива наука ніколи не суперечить богооб`явленим істинам і завжди з ними сходиться. Наприклад, геологія сходиться з оповіданнями Мойсеевої книги Буття про походження землі, рослин, тварин.

Скільки чудес виявляє видима природа. Приклади:
\begin{itemize}
\item Вода
\end{itemize}
Усі фізичні тіла від холоду стискаються, а вода розширюється. Чи видно тут якийсь намір? Подумаймо, яке б то було нещастя, коли б і вода від холоду стискалася! Тоді б лід тонув і всі води вимерзали б, а все, що у водах, гинуло б. Вода, замерзаючи, розширюється і, потрапляючи в каміння й породи землі, рихлить їх, робить здатними вирощувати рослини. Вода має властивість переходити в пару. Це дає хмари, дощі, сніг, росу. Це дає й нам можливість висушувати все, що потрібно. Яскраво видно призначення води. А хто призначив?
\begin{itemize}
    \item Земля
\end{itemize}
Одна й та ж грядка, той же грунт, ті самі соки. Сама земля мертва, а на ній виростають разом різні рослини: морква, буряки, редька, хрін, кавуни, дині, перець, цибуля, виноград, груші, яблуні і тисячі інших видів. Ростуть одне біля другого. Коріння їхні переплітаються, але те солодке, те кисле, те гірке. У кожного свій смак, свій колір, свої властивості. Що за дивна лабораторія? Як те відбувається? Хто вклав у неї такий розумний та мудрий закон?

\begin{itemize}
    \item Рослини
\end{itemize}

Кожна з них являє собою мудрість та красу. Чудова симетрія, гармонія кольорів, дивний рух соків, дихання листя. Погляньте на кожну квіточку та придивіться. Чи не прославите ви її Творця, ту премудру руку, що її так гарно змайструвала?

На грядці ростуть кавуни, дині, гарбузи. Погляньте на їх бадилля: воно, виростаючи, випускає з себе вусики-батіжки. А для чого? Вони ними прив`язуються до всього, що трапиться на дорозі, щоб вітер не скрутив. Та хіба бадилля знає, що буде вітер? Кажуть: «То природа». А хіба природа свідомо діє? Пі, розум не в природі. Шукайте його вище.

Або плоди: кавуни, дині, помідори, груші, яблука, сливи, персики, різні ягоди. Чи не написано на них: «То для тебе, бери та споживай!» А хто написав?

На світі немає нічого зайвого. Все створене на користь людини і все на своєму місці. Людина повинна тільки пізнавати та винаходити користь, яку вклав у все Господь.

\begin{itemize}
    \item Тварини
\end{itemize}

Кожна тварина має своєрідну красу й доцільність. Тіло кожної побудоване відповідно до її потреб. В організмі її відбуваються тисячі різноманітних дій: дія серця, біг крові, травлення їжі, ріст, інстинкти, чуття й т. і. Тварини мають свої закони життя, любов до дітей і інше. Хто все те їм дав?

\begin{itemize}
    \item Людина
\end{itemize}

Але найбільше чудо в природі — це сама людина.
Колись учні грецького мудреця Сократа попросили його, щоб він сказав їм суть мудрості. Він відповів: «Пізнай себе». В цьому справді велика мудрість: хто пізнає себе, той пізнає світ, бо сама людина є малий світ у порівнянні до великого всесвіту. Тому необхідно вивчати самим себе.

Розум наш у мозку. Мозок родить думку, але як? Розум наш породжує слово, але як? Не знаємо. Розум не матерія, і думка не матерія, бо про них не скажеш, що вони там або тут, спереду чи ззаду, або які вони завдовжки чи завширшки. Це й не світло і не електрика. Як же матеріальний мозок породжує духовну думку? Розум має волю. Подумає, побажає — і ноги йдуть, руки щось роблять, уста промовляють. Знов же, як дух примушує рухатись тіло?

Розум має пам`ять. То дивна комора — в неї ми складаємо все, що бачимо, чуємо, і до чого доторкаємося. В неї складаємо все, чого протягом життя навчаємося. Там немає поличок, і ми не складаємо там рядками і не нумеруємо. Але коли ми захочемо, можемо коли завгодно взяти звідти все, що нам потрібно (згадати). Пам`ять — то скарбниця нашого знання.

Ми можемо з пам`яті викликати образи давно померлих людей, уявити їх голос, ходу, рухи і т. д. Ми пригадуємо пісні й інші музичні речі. Ми думкою говоримо, думкою співаємо, думкою будуємо, малюємо. А між тим у думці немає ні рухів, ні звуків, ні фарб.
Правдиво: дивні діла Твої, Господи!

\begin{itemize}
 \item[--]{Людське тіло.}   
\end{itemize}
Серце — то дивний апарат! Ще Ноєві Бог сказав: \emph{«Життя людини в крові її» (Бут. 9, 5).}

Серце — це вічнорухлива машина-помпа, яка на протязі
усього нашого життя гонить по тілу кров. Без нас воно діє, і ми не маємо сили йому наказувати.

Кров нас гріє, годує і оберігає. Червоні тільця крові розносять кисень по тілу, гріють, годують кожну клітину, забирають з тіла азот та все непотрібне і виносять його геть у легені, щоб видихнути. Білі тільця — це армія тіла, що оберігає його від усього чужого. Як тільки щось чуже, постороннє потрапить у тіло, вони його обліплюють з усіх боків, і хоч самі гинуть, його викинуть. Без них ми усі швидко померли б. Отже, бачимо, що наш організм діє планово й розумно у той час, коли наш розум у тому участі не приймає.

\begin{itemize}
    \item[--]{Травлення}   
\end{itemize}

Людина їсть їжу. Шлунок виділяє соки, розщеплює її, робить молочко. Кров і лімфа розносять його по тілу. Кишки женуть їжу по стравоходу, всмоктують з неї усе, корисне для тіла, а непотрібне виганяють геть. Язик чує смак, ніс чує запах, і все те робиться без нас. Не ми тим завідуємо.

\begin{itemize}
    \item[--]{Зір}   
\end{itemize}

Очі — дивний апарат! Вони фіксують тисячі
вражень: кольори, форми, величини і дають нам правдиве уявлення про все, що нас оточує. Цей орган — один з найпотрібніших. Але погляньмо, скільки прикладено уваги, щоб охоронити око від ушкодження! Очі приміщені у міцних кістяних коробках. Прикриті повіками, над ними вії й брови, щоб захистити від пороху. До очей проведені сльозові рурочки, щоб завжди ополіскувати їх. Вії мигають, щоб змочувати й очищати очі. Хто ж так потурбувався про них, якщо не розумний Творець?

\begin{itemize}
    \item[--]{Слух}   
\end{itemize}

Вуха розпізнають тисячі й тисячі найрізноматнітніших звуків, шумів, гуркотів. Вони чують музику, фіксують слова. Вуха — то найдивніший у світі музичний інструмент з сотнями тисяч «струн»-вузликів. На кожний звук — своя струна. Доторкнеться її звук, вона зазвучить і дасть знати нашій свідомості.

\section{Властивості Божої природи}

Розглядаючи творіння Божі, ми довідуємося по них про ті самі властивості Божі, про які свідчить і Св. Письмо.

Христос Господь каже: \emph{{\color{red}«Бог є Дух»} (їв. 4, 24)}, і ми це бачимо, бо тільки необмежений Дух може діяти в усьому світі в ту ж саму мить. Його діяння видно всюди, а Його Самого не видно, бо Він невидимий.

Єдність законів і гармонія життя природи свідчать, що Бог лише Один. Так і пророк Ісайя передає слова Самого Бога: \emph{{\color{red} «Я Господь і іншого немає крім Мене»} (Іс. 45, 5)}.

Бог живий. Ми це бачимо з того, що творіння живуть, бо тільки життя творить життя.

Бог створив світ, значить, був раніше світу.

Бог вічний.

Бог розумний і премудрий. Про це свідчить мудрість законів, якими живе всесвіт. Написано: \emph{«Господь премудрістю збудував землю і небеса утвердив розумом» (Прит. З, 19).}

Бог — вічна краса. Про це свідчать міріади творінь, які вражають нас всебічною красою. Творець їх Бог, початок і кінець всякої краси.

Бог добрий і благий, Він любить Своє творіння. Про це свідчить доцільність будови усіх творінь, від рослин до людини. Доцільність же скерована на добру користь їм. Так і написано: \emph{«Добрий Господь до всього, і ласка Його на всіх ділах Його» (Пс. 144, 9).}

\textbf{\emph{Бог є любов (1 їв. 4, 8, 16).}}

Бог необмежений і всюди сущий. Він проймає Собою все, як проміння сонця кришталь, і діє всюди. Поза Богом немає нічого.
 
Бог всезнаючий. Він проникає все. Йому все відоме і відкрите перед очима Його. Він не має минулого або майбутнього: усе перед ним сучасне. Йому відомі всі людські помисли.

\emph{Бог знає все (1 їв. З, 20).}

\begin{FlushRight}
    \emph{«Дух Господній наповняє всесвіт. Він обіймає все, знає кожне слово» (Прем. 1, 7).}
\end{FlushRight}


Бог є вічне світло і творець світла у всесвіті:

\begin{FlushRight}
    \emph{«Бог є світло, і в Ньому немає ніякої тіні» (1 їв. 1, 5).}
\end{FlushRight}

\begin{FlushRight}
    \emph{«Бог живе у світлі неприступному» (1 Тим. 6, 16).}
\end{FlushRight}


Бог незмінний. Він завжди той самий вічно, і яким був тоді, таким є тепер, і таким буде вічно:

\begin{FlushRight}
    \emph{«У Отця світів немає переміни або хоч тіні зміни» (Як. 1, 17).}
\end{FlushRight}

\begin{FlushRight}
    \emph{{\color{red}«З початку Я той самий»} (Іс. 43, 13).}
\end{FlushRight}


З об`явлень Божих і з досліджень над Божими творіннями ми упевняємося, що:

Бог Один.

Бог Дух, Живий, Оживляючий, Вічний, Розумний, Премудрий, Всеблагий, Вседобрий, Всеправедний, Незмінний, Неосяжний, Необмежений, Самосущий, Самобутній, Всесильний, Всемогутній, Вседержитель, Всеуправитель, Вседосконалий, Всезадоволений, Всещасливий, Всеблажнний. Бог — Вічна Істина, Вічна Правда, Світло, Любов.

Властивостей Божих перелічити неможливо:

\begin{FlushRight}
    \emph{«Великий Господь, і велика сила Його, і розуму Його немає числа» (Пс. 146, 5).}
\end{FlushRight}


Кожна властивість Божа має своє ім`я. Одначе одне ім`я обіймає все; це ім`я — Бог. В ньому ми знаходимо все, що є найвище, найкраще, найчистіше, найсвятіше — Істину, Любов, Правду, Красу, Радість, Утіху, Щастя, Спокій, Блаженство. Немає більшого щастя, як знати, що нас створив Вседосконалий Бог, а звідси і наша любов до Бога і наше благоговіння перед Ним, перед Його величністю.

\subsection{Троїстість Єдиного Пресвятого Бога}

Бог Істотою Один, але потрійний Особами (Лицями або Іпостасями): Отець, Син (Слово) і Святий Дух. Бог є Пресвята Тройця Одноістотна (Єдиносущна), Нероздільна і Незлитна.

Про Троїстість Єдиного Бога свідчить Сам Син Божий Ісус Христос, як свідчить Він і про рівність усіх трьох Осіб Пресвятої Тройці такими словами:

\begin{FlushRight}
    \emph{{\color{red} «Я і Отець — Одно»} (їв. 10, 30).}
    
    \emph{{\color{red} «Хто бачив Мене, той бачив Отця»} (їв. 14, 9).}
\end{FlushRight}

\begin{FlushRight}
    \emph{«Великий Господь, і велика сила Його, і розуму Його немає числа» (Пс. 146, 5).}
\end{FlushRight}

Після Свого славного воскресіння, посилаючи учнів на проповідь, Ісус Христос заповідав їм:

\begin{FlushRight}
    \emph{{\color{red} «Ідіть, навчайте всі народи і хрестіть їх в ім`я Отця і Сина і Святого Духа. Навчайте їх додержуватися всього, що Я заповідав вам»} (Мт. 28, 19-20).}
\end{FlushRight}

Всі три Особи (Іпостасі) Святої Тройці рівні між Собою, Рівносовічні, Рівносопрестольні. Ніхто не більший, ніхто не старший, ніхто не менший. Єдине Божество, Єдине Царство, Єдина Сила, бо Єдиний завжди Сам Собі рівний. Початком же Божественних Осіб (Лиць) є Отець.

У творінні світу брали участь усі три Особи Пресвятої Тройці. Отець помислив (побажав) і сказав (Слово — Син): \emph{«Нехай буде»}. А Дух Божий носився над водою (хаосом). \emph{(Бут. 1, 2-3)}.
Перед тим, як створити людину — чоловіка, Бог сказав (у Святій Тройці): \emph{{\color{red} «Сотворімо чоловіка (а не сотворю) по образу Нашому і подобі»} (Бут. 1, 26)}.

Після гріхопадіння перших людей, Адама і Єви, Бог сказав: \emph{{\color{red} «Ось Адам став, як один із Нас»} (Бут. З, 22)}.

Перед розселенням народів після потопу у Вавилоні, Господь сказав (у Святій Тройці):

\emph{{\color{red} «Зійдімо, змішаймо мову їх»} (а не зійду, змішаю). (Бут. 11,7)}.

Тому й пророк Давид у псалмі говорить:

\begin{FlushRight}
    \emph{«Словом Господнім небеса утворені і Духом уст Його вся сила їх» (Пс. 32, 6).}
\end{FlushRight}

\begin{FlushRight}
  \emph{«Він (Отець) сказав (Слово — Син), і сталося, повелів\footnote{Повелів означає виявив волю, послав Духа Святого} і створилося» (Пс. 32, 9). }  
\end{FlushRight}

\subsubsection{Властивості Осіб Пресвятої Тройці}

Отець не народжується й не сходить від іншої Особи Пресвятої Тройці. Син превічно народжується від Отця. Дух Святий превічно сходить від Отця.
 
Одначе внутрішнього Божого життя ніхто не знає: ні ангели, ні люди, як і каже святий апостол:

\begin{FlushRight}
    \emph{«Божого ніхто не знає, тільки Дух Божий» (1 Кор. 2, 11).}
\end{FlushRight}

\begin{FlushRight}
    \emph{{\color{red} «Ніхто не знає Сина, тільки Отець, і Отця ніхто не знає, тільки Син»} (Мт. 11, 27).}
\end{FlushRight}

\subsection{Образ Божий на людині}

Деякі розуміння великої тайни Святої Тройці дано нам у нас самих.

Людина створена по образу Божому і подобі. Чим же вона подібна до Бога? Зрозуміло, що не тілом, бо Бог безтілесний. На Бога ми подібні нашою душею. Душа наша розумна, вільна і безсмертна. Вона — дух. Виявом душі є розум.

Бог є розум усього світу, Він і Владика світу. Розум людини є розум її тіла (організму) і є владика всієї людської істоти. Коли б розум людини не виявляв себе назовні, то це все рівно, що його й не було б. Розум може виявити себе лише двома способами: словом і волею (вольове діяння). Інших способів немає.

Слово безліч разів народжується на протязі всього життя розуму, і ніколи не буває так (в нормальному стані), щоб слово, народившись раз із розуму, уже не могло б більше народитися, тобто, що ми не могли б сказати його вдруге. Бо хоч ми й сказали його, але воно від розуму не відділилось. Назовні воно втілилося в звук, а в розумі лишається те ж саме.

У той же час слово, бувши сказаним (народившись) уже не може повернутися в розум, наче і не було сказане. Отже, слово наше від розуму невіддільне і незлитне з ним.

Подібно і наше вольове діяння. Розум помислив, і воля виходить в діях. Такі дії ми можемо повторювати безліч разів, бо воля від розуму ніколи не відділяється. Воля (як діяння), яка виявила себе в діях, не може повернутися в розум і злитися з ним. Значить і діяння (вольове) невіддільне від розуму і незлитне з ним.

Звідси бачимо, що й наш розум (а значить і наша душа) є тройця одноістотна, нероздільна і незлитна (слово — то втілений розум, а воля - розум в діянні). Коли б розум був позбавлений слова або діяння, то він не був би довершений і досконалий. Тому троїстість розуму є необхідністю. Троїстість — то повнота життя (буття): перше в Богові, а потім і в нас, як в образі Божому. Одначе наша троїстість дуже далека від Божої, оскільки Бог безмежно більший від нас. На таке розуміння наводить нас і Святе Письмо: \emph{«Словом Господнім небеса створені і Духом уст Його вся сила їх... Він сказав, і сталося, повелів, і впорядкувалося» (Пс. 32, 6, 9).}

Про Ісуса Христа, Сина Божого, апостол Іван каже: \emph{«Споконвіку було Слово, і Слово було у Бога, і Слово було Бог. Усе Ним сталося, і без Нього ніщо не сталося, що сталося» (їв. 1, 1, 3).}

Дух Святий у Св. Письмі завжди зветься: Животворчий, Вседіючий, Оживляючий \emph{(їв. 6, 63)}.

\subsection{Чим людина подібна до Бога?}

Подібність людини до Бога у тому, що дух наш прагне до добра, правди, краси, досконалості, любові, правосуддя та інших подібних доброт, які належать лише Богові.

Ми отримали ці прекрасні якості від Самого Бога тоді, коли Він вдихнув нам від Себе нашу душу (але не частину Себе Самого). І з того часу й дотепер душа наша шукає тих небесних красот і насолоджується, коли знаходить їх (або бодай подібні до них). Чим ближче людина наближається до Бога через віру, любов, праведне життя, тим більше зростає бажання й необхідність тих Божих чеснот.

Св. апостол Павло каже: \emph{«Ми від Нього, Ним живемо і до Нього повертаємось» (Рим. 11, 36).}

\emph{«Відзначилося на нас світло лиця Твого, Господи», — каже пророк Давид (Пс. 4, 7).}

Найвищий образ і подобу Бога явив нам Собою Син Божий Ісус Христос, щоб ми мали зразок для себе.
\end{document}