\documentclass[main.tex]{subfiles}

\begin{document}
\chapter{Молитва}

\subsection{Що таке молитва?}

Молитва — то душевна розмова людини з Богом з піднесенням до Бога розуму й серця.

Ми діти Божі через Ісуса Христа, отже якою повинна бути наша молитва?

Щиріша, аніж розмова найлюбіших сина чи дочки зі своїм татом чи мамою, не лукава, не хитра, а пройнята гарячою любов`ю та острахом перед Богом, з повною надією, що Господь в силах дати нам усе, чого просимо. В молитвах наших ми, як діти Божі:

\begin{enumerate}
    \item Просимо у Бога всього необхідного до життя.
    \item Дякуємо Богові за Його до нас милості.
    \item Прославляємо Бога за Його всемогутність, мудрість, любов до нас і ласку.
\end{enumerate}

Відповідно до цього й молитви наші поділяються на прохальні, вдячні та хвальні (прославлення).

\subsection{Чи треба молитися?}

Господь наш Ісус Христос Сам безперестанно молився (Мр. 1, 35; Лк. З, 21).

Він пробув у пустині сорок діб у пості й молитві (Мт. 4, 2). Після Своїх цілоденних проповідей до народу ночі проводив у молитві (Мт. 14, 23). Молився перед споживанням їжі й після нього (Мр. 14, 26). Молився перед Своїми стражданнями (їв., гл. 17; Мт. 26, 39-43; Лк. 22, 39-45), перед розп`яттям і на хресті (Мт. 27, 46). Він нам заповідав молитися. І не тільки заповідав, але й навчав, про що і як треба молитися (Мт. 6, 5-15).

Молитися не тяжко. Бо хіба тяжко добрим дітям розмовляти зі своїм батьком? Розповідати йому про свої жалі або успіхи і т. ін.?

Хто всім серцем любить Бога, той день і ніч з Ним розмовляє, бо та розмова солодша від меду.

Зрозумій собі Бога, як свого Отця, а себе дитиною Божою почуй, тоді ніколи не наситишся молитвою.

\subsection{Для чого потрібно молитися?}

Коли ми щиро молимося, то через віру й любов до Бога об`єднуємося з Ним, і через те Його сила та Його святість передається й нам, подібно до того, як ми гріємося на сонці, то від його світла й тепла і ми стаємо світлі й теплі. Для того й Христос як людина завжди молився. Молилися й апостоли і нам заповідали: «Безперестанно моліться» (1 Сол. 5, 17).

\subsection{Які є молитви?}

Молитва поділяється на два види: внутрішня і зовнішня.

Внутрішня — це молитва серцем і думкою без слів. Молитва душі й серця інакше зветься молитвою духовною або розумовою. Так молився Мойсей перед переходом через Червоне море. Так молиться кожний християнин там, де неможливо словами молитися, наприклад серед невірних.

Зовнішня молитва — це та, коли ми шепочемо слова, хрестимося, навіть кладемо поклони, але думаємо про щось житейське. Про таку молитву Господь каже: «Наближаються до Мене люди, устами своїми шанують Мене, а серце їхнє далеко від Мене. Даремно вони шанують Мене» (Мт. 15, 8-9). З цих слів Христових видно, яка ціна такої молитви.

Навпаки, внутрішня молитва дуже вгодна Богові. Одначе для людини, яка складається з тіла й душі, одної внутрішньої молитви не досить.

Апостол каже: «Прославляйте Бога в тілі вашому і в дусі вашому, бо вони Божі» (1 Кор. 6, 20). Тому зовнішня молитва завжди повинна сполучатися з внутрішньою. Зовнішня молитва повинна виявляти внутрішнє переживання людини. Так, поклін має значення тоді, коли стається внаслідок благоговіння перед Богом або вияву прохання, подяки чи прославлення.

Хресне знамення ми завжди повинні сполучати з глибокою вірою у страждання Христові. Без того зовнішні дії будуть порожніми рухами.

Кожне слово молитви повинно виливатися з серця, як каже Христос:

«Від повноти серця уста промовляють» (Мт. 12, 34).

Одначе й побожні рухи в молитві не менше важливі. Бо й Христос Господь коли молився, то виявляв молитву словами на устах, схиляв голову й коліна, зводив очі до небес і т. п. (Мт. 26, 39; Лк. 22, 41; їв. 17, 1).

Апостоли також схиляли коліна на молитві (Діян. 20, 36) і підносили руки (1 Тим. 2, 8).

\section{ПРО МОЛИТВУ ВЗАГАЛІ}

Своєю Господньою молитвою — «Отче наш» — Христос Спаситель навчає нас, про що і як треба молитися (про цю молитву, що її дав нам Сам Христос, див. нижче). Але це не означає, що тільки цією молитвою можна молитися. Вона дана нам як зразок, і до неї можна додавати й інші молитви.

Сам Христос молився безперестанно на кожному кроці Свого спасительного життя на землі. Наприклад: 17-й розділ Євангелія від Івана — то первосвященницька архиєрейська молитва Христа перед принесенням Себе Самого в жертву. Особливо часто Господь молився ночами на горі Єлеонській після Своїх цілоденних проповідей до народу (Мт. 14, 23 й інші місця; Лк. 5, 16; 6, 12). Молився на ТайІІій Вечері (Лк. 22, 19). Молився в саду Гефсиманському перед Своїми стражданнями.

Молитва — то синовня розмова віруючого серця з Богом. Моляться не тільки люди, моляться й ангели, прославляючи Бога.

Так, люди навчені ангелами співати Богові Трисвяту пісню: «Святий Боже, Святий Кріпкий, Святий Безсмертний...»

«Свят, свят, свят Господь Саваоф, повні небо і земля слави Його». Це пісня серафимів, її чув у видінні пророк Ісайя (Іс. 6, 3). «Слава на висоті Богові, і на землі мир, між людьми благовоління (Лк. 2, 14). Цю пісню чули пастухи під час народження Господа нашого Ісуса Христа.

Інші молитви складені святими пророками по натхненню Духа Святого, наприклад псалми й пісні псалтиря, написані пророком Давидом (Діян. 1, 16; Мт. 22, 43-44).

Ще інші молитви вилилися з уст великих праведників, наприклад пісня Пресвятої Діви Богородиці: «Величає душа Моя Господа...» (Лк. 1, 46-55). Також молитви святих апостолів та інших святих.

Особливо угодні Господу святі пісні псалтиря. Це видно з того, що Сам Христос їх співав (Мт. 26, ЗО). Тому кожний християнин, який часто читає й співає перед Господом псалми з псалтиря, має у Господа велику ласку.

Молитися до Господа можна й своїми словами: просити, що потрібно, прославляти й вихваляти за величність. Але молитися треба чистим, негордим, люблячим серцем, як добрий син перед своїм батьком, зі страхом, покорою та любов`ю. І не потрібно говорити зайвого: виправдувати себе або хвалити, щоб не уподобитися фарисеєві (Лк. 18, 10-14).

Молитися можна й думкою, як сказано раніше, але людина складається з душі й тіла, тому й молитися треба душею й тілом, як каже Господь: «духом і істиною» (їв. 4, 24), цебто всією своєю істотою. Тому молитва сильніша, коли ми молимося розумом і серцем, а душевні почуття свої виявляємо зовнішніми знаками: осіняємо себе святим хрестом, схиляємо перед Господом коліна й голову, молитовно складаємо або підносимо руки, зводимо очі до небес, вклоняємося до землі і т. ін„ як і Господь наш молився.

Найприємніше молитися насамоті або вночі, коли всі сплять. Тоді ніхто не заважає, думки всі легко збираються докупи і скеровуються на молитву.

Тайну молитву Господь дуже високо цінить: «Молися Отцю своєму в тайні, і Він віддасть тобі явно» (Мт. 6, 6). Христос Спаситель високо оцінив нічну молитву Нафанаїла під смоковницею (їв. 1, 48).

Але молитися спільно ще угодніше Богові: «Де двоє або троє зберуться в ім`я Моє, там і Я посеред них», — каже Господь (Мт. 18, 20).

Чому так? Очевидно тому, що у спільній молитві недостача одних поповнюється праведністю других, і горіння віри одних запалює других, а Господь бажає, щоб усі спаслися.

Тому в дні недільні та святкові кожний християнин повинен бути в храмі, брати участь у відправах Служб Божих, особливо Божественної Літургії, як то робили святі апостоли і всі праведні християни (Лк. 24, 53).

За вченням Святої Христової Церкви в молитвах до Бога ми призиваємо на поміч святих Божих угодників: найперше Пресвяту і Пречисту Богоматір, святих апостолів та інших святих. Призиваємо також і святих ангелів і просимо, щоб і вони молилися за нас, бо їхні молитви більше вгодні Богові, аніж наші. В молитвах ми призиваємо й силу хреста Господнього. Властиво просимо Бога послати нам божественну силу, яку Він передав і цьому своєму знаменню.

Коли кажемо: «Радуйся, хресте Господній» або «Помагай нам», то тим самим прославляємо ту Господню силу і просимо, щоб вона оселилася в нас.

\subsection{Коли треба молитися?}

Молитися треба ранком, умившися після сну; вдень, перед початком кожного діла і після закінчення; перед споживанням їжі і після неї; перед тим, як іти спати і лягаючи в постіль. А найкраще молитися завжди — і за роботою, і в дорозі, і прокинувшись у постелі. І на кожному місці, як каже апостол:

«Безперестанно моліться» (1 Сол. 5, 17).

\section{ПРО ХРЕСНЕ ЗНАМЕННЯ Й ІНШІ ПОБОЖНІ РУХИ В МОЛИТВІ}

В молитвах ми осіняємо себе хресним знаменням. Для того три більші пальці правої руки складаємо докупи на знак єдності й рівності Пресвятої Тройці Триєдиного Бога, а два менші пальці пригинаємо до долоні на честь Ісуса Христа, Який має в Собі дві природи (два єства) — Божу й людську, і є Богочоловіком.

Отже склавши так пальці руки, ми виявляємо нашу віру в Святу Тройцю й Господа нашого Ісуса Христа. І так складену руку кладемо на чоло, щоб освятити розум і помисли; на груди (нижче ложечки), щоб освятити серце й чуття; на праве рамено й на ліве, щоб освятити наші сили й діяння. Самим хрестом ми виявляємо нашу віру у розп`ятого Господа нашого за нас і Його хресні заслуги в Бога. Отже хресне знамення діє на нас тоді, коли ми вживаємо його з вірою та благоволінням.

В молитві ми вклоняємося Богові. Але поклони ті не повинні бути якимсь механічним, мотанням головою. Поклони повинні бути благоговійні перед величчю Божою, виявом пошани, благання, сердечної щирої подяки.

\section{МОЛИТВА ГОСПОДНЯ}

Одного разу, коли Христос в одному місці молився й перестав, один з учеників Його сказав: «Господи, навчи нас молитися, як Іван (Хреститель) навчив своїх учеників».

Христос відповів йому так: «Коли молитесь, кажіть: Отче наш, що єси на небесах! Нехай святиться ім`я Твоє. Нехай прийде Царство Твоє. Нехай буде воля Твоя, як на небі, так і на землі. Хліб наш щоденний дай нам сьогодні. І прости нам провини наші, як і ми прощаємо винуватцям нашим. І не введи нас у спокусу. Але визволи нас від лукавого. Бо Твоє є Царство, і сила, і слава на віки». Амінь. (Мт. 6, 9-13; Лк. 11, 1-4). Ця молитва дана нам Самим Господом Ісусом Христом і тому зветься «Молитвою Господньою».

Молитва Господня є зразком для молитов, бо в ній сказано як і про що молитися. Тому кожний християнин мусить вивчити її напам`ять, починаючи з дитинства. При тому треба не лише пам`ятати слова, але й зміст кожного слова і пояснювати дітям.

Правдиві християни висловлюють молитву Господню з найвищим чуттям благоговіння, бо в ній Сам Господь дав нам право звати Бога Отцем, а тим самим і нам, як каже євангелист Іван, дав право дітьми Божими бути (їв. 1, 12).

Для зручнішого вивчення, молитву Господню поділяють на дев`ять частин: призивання, сім прохань і славословлення.
 
\subsection{ПРИЗИВАННЯ}

Отче наш, що єси на небесах!

Тут Господь навчає нас, що ми є всиновлені Йому через віру в Єдиного Сина Його — Господа нашого Ісуса Христа. Що Отець наш Небесний істинно перебуває на небесах і чує наші молитви, а ми у своїх молитвах переносимось духовно до небесного й божественного, лишивши все тлінне земне.

Святий апостол Іван каже: «Тим, що прийняли Його (Ісуса Христа) й повірили в ім`я Його, Він дав силу (і право, владу) дітьми Божими бути» (не здаватися, а бути уже по духу своєму) (їв. 1, 12).

«Дивіться, яку любов виявив до нас Отець, щоб нам зватися й бути дітьми Божими... Улюблені! Ми тепер діти Божі» (1 їв. З, 1-2).

А святий апостол Павло каже:

«Всі ви сини Божі через віру в Ісуса Христа» (Гал. З, 26).

«Ти вже не раб, а син, а коли син, то й наслідник Божий через Христа» (Гал. 4, 7).

Сказав Господь-Вседержитель у Старому Заповіті:

«Оселюся в них... і буду їм Отцем, а вони будуть Мені синами й дочками» (Єр. З, 19; Ос. 1, 10).

Треба розуміти, що увірувати в Христа — значить переродитися і жити у Христі, тобто в повній згоді з волею Божою, як син з Отцем. Для того й дає нам Христос право взивати до Бога у молитвах: «Отче наш». Переродження починається вірою, а довершується від Духа у святому Хрещенні Апостол Павло навчає: «Хто в Христі, (ті) розп`яли тіло (своє) з його пристрастями й прихотями» (Гал. 5, 24).

Народження від Бога є народження душі для того, щоб людина стала храмом Духа Святого.

Святий апостол Павло, звертаючись до коринтян, промовляє: «Хіба ви не знаєте, що тіло ваше — храм Духа Святого... і ви не свої, бо ви куплені ціною. Тому прославляйте Бога в тілі вашому і дусі вашому, бо вони Божі» (1 Кор. 6, 19-20).

Важливо звернути увагу на те, що Господь навчає нас казати: «Отче наш», а не «Отче мій». Цим він навчає, щоб завжди всі молилися за всіх.
 
\subsection{ПЕРШЕ ПРОХАННЯ}

Нехай святиться ім`я Твоє.

Цими словами Господь навчає нас просити для себе святості, бо тільки святі достойні шанувати ім`я Боже. Господь сказав:

«Святі будьте, бо Я Святий» (1 Петр. 1, 16).

«Будьте звершені (в праведності), як Отець Ваш небесний звершений» (Мт. 5, 48).

«Так нехай світить світло ваше (вашого життя) перед людьми, щоб вони бачили ваші добрі діла і прославляли Отця вашого Небесного» (Мт. 5, 16).

Немає іншого імені під небом такого, як ім`я Бога нашого. Перед ним у благоговінні тремтять небесні сили. Від імені Його в жаху тікають демони. Ім`я Господнє Страшне й Святе по віки.

Свята й Нренепорочна Діва Богомати промовляла: «Створив Мені величність Сильний, і Святе ім`я Його... (Лк. 1, 49). Отже, ім`я Господнє Святе само в собі.

Ім`я Господнє святе Само в Собі, а нам необхідно відчувати і виявляти Його нашим життям, щоб воно святилося в нас самих, щоб ми завжди тримали Його в серці й думках, як найбільшу святиню, з найвищою пошаною.

Ми повинні й уста наші оберігати від усього скверного, від поганих слів, пісень, від неправди, щоб чистими устами ми могли прославляти Бога, як каже апостол:

«Нехай не виходить із уст ваших ніяке гниле слово, а тільки добре, щоб воно приносило благодать тим, що слухають» (Еф. 4, 29).

«Бо хто сам так робитиме й інших навчить, — каже Господь, — той великим буде в Царстві Небесному» (Мт. 5,19).

Тяжкий гріх бере на себе той, хто зневажливо лається та нехтує ім`я Боже. Бережіться від такого, навіть не слухайте тих страшних і ганебних слів, щоб не стати спільником того, хто Тх каже.

Хула на Бога не прощається, каже Христос (Мт. 12, 31).

А святий апостол Яків радить нам:

«Хто наверне грішника з поганої дороги його, той спасе душу від смерті і покриє багато своїх гріхів» (Як. 5, 20).
 
\subsection{ДРУГЕ ПРОХАННЯ}

Нехай прийде Царство Твоє.

Тут Господь навчає нас молитися про царство благодаті Божої, шо виявляється «в праведності, у згоді та радості у Святому Дусі» (Рим. 14, 17). Щоб над нами у всьому царював Господь, а не ворог наш — диявол. Тому коли кажемо «Нехай прийде Царство Твоє», то просимо Бога, щоб скерував серця наші до заповітів Його, зміцнив любов до них, очистив від поганих звичок і допоміг осягнути добре, праведне і богоугодне життя.

Господь Ісус так каже: «Царство Боже не приходить помітним (видимим) чином, і не скажуть, що воно тут або там. Бо ось Царство Боже внутрі вас є» (Лк. 17, 20-21).

Отож Царство Боже починається в серці кожної людини тоді, коли вона починає жити по заповідях Божих, коли з серця її починає виходити тільки добро, правда, милосердя, любов до ближнього.

Для праведників воно вже прийшло, бо вони всюди сіють добро, як сонце ясне промінця, а грішники не знають його, бо ними керує воля гріха (Рим. 6, 12).

Разом з тим ми просимо Бога, щоб сподобив нас осягнути й вічне Царство Боже, яке починається для кожного праведника після тілесної смерті.

Апостол Павло пише у посланні до Тимофія:

«Час мого відходу настав. Подвигом добрим я подвизався, життя прожив, віру зберіг, а тепер готується мені вінець правди, якого дасть мені Господь, Праведний Суддя в той день, і не тільки мені, а й усім, хто полюбив явління Його» (2 Тим. 4, 6-8).

А потім, звертаючись до филіпійців, він каже:

«Маю бажання розв`язатися (від тіла) і з Христом перебувати» Фил. 1, 23).

\subsection{ТРЕТЄ ПРОХАННЯ}

Нехай буде воля Твоя, як на небі, так і на землі.

Цим проханням Господь навчає нас просити Отця Небесного, щоб не наша непевна, а може й грішна, воля була, а Його свята Воля, яка завжди скеровує нас на добре.

Сам Христос Господь, Превічний Син Божий, перед стражданнями Своїми навіть не помислив поставити Свою волю вище волі Отця Свого Небесного, хоч і жахався і смертельно тужив, але казав: «Не Моя, а Твоя воля нехай буде» (Лк. 22, 42).

Ми ж тим більше не повинні покладатися на свою недосвідчену волю. Наш розум дуже обмежений, ми часто не знаємо, що добре для нас і бажаємо того, що буде нам шкідливе. Бог же, як рідний батько, бажає нам найліпшого. Він напевне знає, що чекає нас і що нам буде потрібне.

Про це й Спаситель каже нам: «Знає Отець ваш Небесний, що все те потрібне вам» (Мт. 6, 32).

Святий апостол Павло у посланні до ефесян пише: «Йому, що діючою в нас силою може багато більше зробити ніж те, чого ми просимо або про що помишляємо, Йому слава в Церкві через Христа Ісуса у всі роди на віки вічні. Амінь» (Еф. З, 20-21).

Словами «як на небі, так і на землі» Господь навчає нас, щоб ми на землі виконували Божу волю з такою ж любов`ю та ретельністю, як її виконують на небесах святі ангели та духи святих людей.

\subsection{ЧЕТВЕРТЕ ПРОХАННЯ}

Хліб наш щоденний дай нам сьогодні.

Тут ми просимо у Господа про необхідну поживу для душі й тіла. «Хліб» означає поживу для життя взагалі. «Сьогодні» — період нашого життя.

Людина складається з тіла й нематеріальної (неречовинної) душі. Для тіла потрібен хліб та інша (фізична) пожива, також одежа й житло. Звертаючись до Господа, ми просимо Його помочі й благословення на нашу працю: садити, сіяти, будувати, учитись і т. п. Але:

«Не одним тільки хлібом буде жити людина, а всяким словом, що виходить з уст Божих», — сказав Господь (Мт. 4, 4).

Для душі потрібна духовна пожива: любов до Слова Божого, Святого Письма, Євангелій з правдивим їх розумінням. У цьому проханні молитви Господньої ми просимо про благодать Божу, яка б підживляла, підкріпляла та скеровувала життя наше по Законах Божих.

Одночасно, кажучи «Хліб наш щоденний», ми просимо у Господа, щоб сподобив нас причащатися Вічного Хліба, Який зійшов з неба, тобто Тіла і Крові Господа нашого Ісуса Христа:
 
«Тіло Моє — то істинна їжа, і Кров Моя — то істинне питво. Як послав Мене Живий Отець, і я живу Отцем, так і той, хто їстиме Мене (причащатиметься), буде жити Мною» (їв. 6, 55, 57).

«Тіло Христове прийміть, і джерела безсмертного вкусіть», — співається в церкві.

Святе Письмо навчає нас, що душа по природі своїй є від Бога і до Бога повернеться. Душа любить все, що від Бога. Дух або духовна душа в людині — це те, що Бог вдихнув у неї при творінні від Самого Себе безпосередньо (але не частину Себе Самого). Саме в цьому духові є образ і подоба наша до Бога.

Цьому духові дано розуміння високих моральних чеснот: краси, справедливості, прагнення до добра, до Бога, свободи волі. Святий апостол Павло каже: «Тіло ваше — храм Духа Святого... тому прославляйте Бога в тілі вашому і в дусі вашому, бо вони Божі» (1 Кор. 6, 19-20).

Цього духа богословіє називає душею людини. За наукою Христа і Святого Письма, душа-дух є безсмертна і по смерті людини повертається до Бога: «І повернеться порох (тіло) в землю, чим він і був, а душа повернеться до Бога, Який і дав йому« (Екл. 12, 7).

Пожадливість людського тіла дуже сильно впливає на душу-духа. З ним людина мусть постійно боротися. Впливи диявола розпалюють природні потреби тіла до похоті й пожадливості, і тоді настає боротьба похоті-тіла з бажанням душі-духа:

«Тіло бажає противного духові, а дух — противного тілові. Вони одне одному противляться так, що ви (часто) не те робите, чого бажаєте» (Гал. 5, 17).

Коли людина має свою волю, то вона може вибирати те, що добре, що від Бога й веде до Бога. В таких обставинах людині допомагає благодать (сила) Божа, і тіло цілком підкоряється духові.

Якщо ж людина схиляється без спротиву до пожадливості тіла, то тіло може так заволодіти її душею-духом, що вже про божественне в її думках не знаходиться місця. Тоді така людина перестає розуміти віруючого в Бога, переслідує його. Перетворюється в бездушну тварину з усіма її властивостями, бо викинула з себе духа, а значить і Бога; стала бездушною, безсовісною, егоїстичною.
 
Святий апостол Павло у своєму посланні пише:

«Бо ви, браття, покликані до свободи, тільки щоб свобода не була засобом для тіла... Відомі діла тіла — це перелюб, блуд, розпуста, ворогування, заздрість, гнів, незгода... А плоди духа — це любов, радість, мир, милосердя, віра, лагідність, стриманість» (Гал. 5, 13, 19, 22-23).

«Ніхто не може двом панам служити: або одного зненавидить, а другого полюбить, або одного триматиметься, а другого занедбає. Не можна служити Богові й мамоні (багатству). Заради цього кажу вам: не журіться душею вашою, що вам їсти й що пити, ані тілом вашим, у що одягнутися. Душа ж чи не більша, як харч, і тіло, як одежа?.. Бо про все те (майно) пильно клопочуться невірні; адже ж знає Отець ваш Небесний, що все те потрібне вам», — каже Спаситель (Мт. 6, 24-25, 32).

Пророк Давид навчає нас: «Не вдавайтеся до насильства, а коли багатство намножується, не віддавайте йому серця» (Пс. 61, 11). Тобто дивіться на багатство, як на тимчасовий дар Божий, який дається вам не для гордині чи розпусти, а для того, щоб допомагати іншим, кому потрібно. І не побивайтесь за багатством аж так, щоб міняти на нього душу свою.

\subsection{П`ЯТЕ ПРОХАННЯ}

І прости нам провини наші, як і ми прощаємо винуватцям нашим.

Тут ми просимо у Бога прощення наших гріхів. В це прохання вкладена велика Божа премудрість: коли ти сам не прощаєш, то як смієш просити прощення для себе?

«Якщо ви будете прощати людям провини їх, то простяться і вам Отцем вашим Небесним, а як не будете прощати людям провини їх, то й Отець ваш Небесний не простить вам провини ваші» (Мт. 6, 14-15).

Той, хто не прощає — повний зла. Він недобрий і не заслуговує на прощення від Бога. Тому, стаючи на молитву, перше простім в душі своїй тим, хто покривдив нас:

«Коли стаєте і молитесь, прощайте коли що на кого маєте, щоб і Отець ваш Небесний простив вам гріхи ваші» (Мр. 11,25).

Святий апостол Петро запитав Спасителя: «Скільки разів можна прощати братові моєму? Чи можна до семи разів?» Христос відповів: «Не до семи разів, а сімдесят разів по сім» (Мт. 18, 21-22).

Тобто — завжди прощати. Бо буває людина несвідомо, мимоволі, ображає, а потім жаліє, що так зробила. Простивши, ми вже не ображаємо взаємно, не мстимось, не збільшуємо ворожнечі. Тому трапляється, що той ніби ворог стає приятелем.

Господь не мстився за Себе. Коли злословили Його, не злословив взаємно. Страждав, але не загрожував, а все передавав Судді Праведному (1 Петр. 2, 23). Він молився за розпинаючих Його: «Отче, прости їм, бо вони не знають, що роблять» (Лк. 23, 34).

Учив Він і нас: «Моліться за тих, хто вас переслідує» (Мт. 5, 44).

«Не мстіть самі, а все передавайте на суд Божий, бо написано: Мені (належить) помста. Я віддам (Рим. 12, 19; Втор. 32, 35).

«Гнівайтесь, та не грішіть. Нехай не зайде сонце у гніві вашому, не давайте місця дияволові» (Еф. 4, 26).

«Перше помирись з братом твоїм, тоді приходь і принеси дар твій до вівтаря» (Мт. 5, 23-24).

Якщо ж брат твій десь далеко, то прости йому в душі своїй перед Всевишнім Богом. Брат — то кожна людина, а особливо віруюча.

«Коли то залежить від вас, зо всіма мир майте», — каже апостол (Рим. 12, 18).

\subsubsection{Чи не суперечить всепрощення Боже Його правосуддю, згідно з яким кожне порушення Божої Правди повинно бути покаране?}

Господь воїстину правосудний, і ніщо не сховається від суду Його, але, як каже пророк Давид, Він щедрий, довготерпеливий і многомилостивий» (11с. 102, 8), вільний по Своєму правосуддю прощати або карати. А за словами апостола, «де збільшився гріх, там перевищила благодать» (Рим. 5, 20).

\subsection{ШОСТЕ ПРОХАННЯ}
І не введи нас у спокусу.
Цим проханням ми звертаємось до Отця Небесного, щоб не вводив нас у спокусу (випробування) і не допустив диявола спокушати на гріх. А коли б таке сталося, то щоб дав силу терпеливо перенести випробування і встояти проти спокус диявола.

Спокуси, за їхнім характером, можна поділити на дві групи:

\begin{enumerate}
    \item Спокуси (випробування) від Бога, які посилаються на добро.
    \item Спокуси від диявола, що робляться на зло.
\end{enumerate}

Святий апостол Яків каже: «Бог не спокушається злом і Сам не спокушає (злом) нікого (Як. 1, 13).

\subsubsection{Випробування від Бога}

Вони посилаються тоді, коли Богові угодно:

\begin{enumerate}
    \item Випробувати віру й покірливість волі Божій (в самій людині).
    \item Навернути чи обновити віру людей, що занехтували свою віру.
    \item Прикладом віри й надії однієї людини зміцнити віру в інших.
\end{enumerate} 

Свідчення про такі спокуси (випробування) знаходимо у Святому Письмі і переказах. Наприклад, спинимось на подружжі Авраама та його дружини Сарри.

Вони жили в місцевості, де запанувало ідолопоклонство. Але Авраам зберіг віру в Істинного Бога. Бог з`являвся йому у видіннях та снах і скріпляв віру. По велінню Божому Авраам і Сарра переселилися в Палестину. Там народився їх син Ісаак. Бог мав свої «плани» і захотів випробувати Авраа-мову віру та відданість Йому. Звелів Авраамові принести в жертву (по тодішніх звичаях) свого сина Ісаака. Дуже шкода було Авраамові свого сина, але покорився Божій волі. Зготовив жертівник і був готовий... Раптом почув голос ангела: «Не піднімай руку свою на юнака, бо Бог побачив твою віру і відданість Йому». Бог благословив Авраама і обіцяв, що в нього буде багато нащадків (Буття, глави 12, 13, 21, 22).

За наших часів жила побожна і благочестива родина. Молитва й церква були для них необхідністю. Працювали й багатіли. Згодом багатство забрало всю їхню увагу й час. Не стало часу на молитву й церкву: стали щасливими й без них. Одначе Богові було угодно спасти душі їх від спокуси, але не наказом чи силою, а їх власним бажанням. Захворіла дружина. Гроші були. Лікували як могли, але одного дня лікар порадив їм звернутися до Бога. З того часу вони задумались. Пригадали, якими вони були і якими стали. Почали молитись, звернулись до церкви, до священників, давали милостиню убогим. На їх велике здивування і радість, дружині поліпшало, і вона видужала. Сила Божа через непохитну віру й надію скріпила її власні сили.

Святий апостол Павло навчає: «Улюблені, вогненної спокуси, що на випробування вам посилається, не стороніться, ніби чогось для вас чужого» (1 Петр. 4, 12).

Одначе, Господь навчає нас просити, щоб Бог випробовував по мірі наших сил і можливостей та допоміг терпеливо переносити випробування.

\subsubsection{Спокуси від диявола}

Вони скеровані на зло і гріх. Цих спокус треба остерігатися й ніколи не йти на приманку, бо гріх завжди привабливий, солодкий назовні і гіркий та отруйний всередині.

Гріх небезпечний. Він може зруйнувати віру і схилити на беззаконство.

Святий апостол Петро застерігає нас: «Диявол, як лев, роззявляє пащу, завжди шукає кого проглинути» (1 Петр. 5, 8).

Диявол — ворог Бога і людей. Він діє через наше тіло, розпалюючи в ньому різні пристрасті. Затемняє розум і свідомість. Він діє через недобрих людей, які легко піддаються йому. Іноді діє й сам безпосередньо, скеровуючи наші помисли на лукаве або зневагу Бога.

Диявол намовив ангелів не коритись волі Божій. Він спокусив Єву, щоб вона з їла заборонений овоч, потім Каїна (сина Адама і Єви), щоб він убив свого брата Авеля.

Диявол намагався спокушати Самого Ісуса Христа, Він знав, що Христос Син Божий, знав для чого Він прийшов на землю, тому переслідував Його. Святий євангелист Матвій описує хитрі диявольські спокуси, скеровані проти Спасителя:

«Ісус був одведений Духом у пустиню, щоб диявол Його випробовував. І постив Він сорок днів і сорок ночей, і потім схотів їсти. Прийшов до нього спокусник і сказав: „Скажи, щоб це каміння стало хлібом". Ісус сказав йому: «Написано: не одним тільки хлібом буде жити людина, а всяким словом, що виходить з уст Божих" (Втор. 8, 3). Тоді диявол привів Його на високе місце і сказав: «Кинься вниз, бо написано: ангели понесуть Тебе і нога Твоя не спіткнеться об камінь" (Пс. 90, 11-12). Ісус сказав йому: «Написано: не спокушай Господа Бога твого* (Втор. 6,
Іб). Тоді диявол показав Йому всі царства світу і славу їх і сказав Йому: «Все те я дам Тобі, коли поклонишся мені". Ісус відповів йому: "Іди геть від мене, сатано! Бо написано: Господу Богу твоєму поклоняйся і Йому Єдиному служи" (Втор. 6, 13). Тоді покинув Його диявол, і ангели приступили і служили Ісусові» (Мт. 4, 1-11).

Подібні спокуси від диявола терпів праведний багатостраждальний Іов. Він був богобоязний, справедливий, побожний, уникав зла. Був багатий і славний. Сатана позаздрив Іову. Наслав на нього нещастя, щоб примусити його зректися Бога. У нього забрали все майно, загинули його діти. Але Іов не ганьбив Бога, а розірвавши одежу на собі, сказав: «Нагим я прийшов на світ, нагим і відійду. Господь дав, Господь узяв. Нехай буде благословенне ім`я Господнє. Тоді сатана уразив Іова хворобою прокази. Він тяжко страждав. Його дружина бачила це і питала: «Чи ти ще й досі твердий у своїй вірі, переносячи такі болі і страждання? Скажи недобрі слова на Бога і помри». Іов відповів їй: «Ти говориш як одна з безумних. Чи лише добре приймати від Бога, а лихого ні?«

За його непохитну віру Господь повернув Іову здоров`я і в два рази більше майна ніж те, що він згубив.

Спокуси диявола продовжуються і триватимуть до часу другого пришестя Господа Ісуса Христа.

\subsection{СЬОМЕ ПРОХАННЯ}
Але визволи нас від лукавого.

Тут ми просимо Отця нашого Небесного, щоб визволив нас від впливу диявола (лукавого) і всього того, що приходить від нього, тобто від лукавих діл його і гріха.

«Хто чинить гріх, той від диявола, тому що з початку згрішив диявол» (1 їв. З, 8).

Апостол Павло навчає як боротися з дияволом:

«Наша боротьба не проти тіла і крові, а проти начальства, проти власті і світоправителів темряви віку цього, проти піднебесних духів злоби. Станьте (проти диявола), опоясавши бедра свої правдою, одягши на себе броню праведності, взувши ноги в готовість Євангелія миру і взявши щит віри, який погасить всі розпалені стріли лукавого. Візьміть і шолом спасіння та духовний меч — Слово Боже. Моліться кожного часу духом невсипуще та з терпеливістю» (Еф. 6, 12, 14-18).

\subsection{СЛАВОСЛОВЛЕННЯ}

Бо Твоє е Царство, і сила, і слава на віка. Амінь.

Цими словами, що ними закінчується молитва Господня, ми благаємо собі в Отця Небесного милостей в надії, що Він в силі дати нам те, чого ми просимо в усіх семи проханнях, бо це в Його владі і проситься для Його слави. В той же час ми віддаємо Йому достойну пошану, думаючи про Його вічне Царство, силу і славу.

Також ми стверджуємо, що просимо з вірою без сумніву, як і навчає нас апостол Яків: «Коли щось потрібно кому, нехай просить у Бога, що всім дає без докору, і Він дасть йому. Нехай просить з вірою, не сумніваючись, бо хто сумнівається, той подібний до морської хвилі, яку піднімає вітер і роздуває» (Як. 1, 5-6).

А святий євангелист Матвій передає нам науку Ісуса Христа: «Просіть і дасться вам, шукайте і знайдете, стукайте і відчинять вам. Бо кожний, хто просить, отримує; хто шукає, знаходить; хто стукає, тому відчиняють. Бо чи є між вами такий чоловік, який, коли син його просить хліба, дасть йому камінь, а коли просить рибу, дасть йому змія? То ж коли ви, будучи злі, даєте блага дітям вашим, то тим більше Отець ваш Небесний дасть блага тим, що просять у Нього» (Мт. 7, 7-11).

«Амінь» означає істинно так.
\end{document}