\documentclass[main.tex]{subfiles}

\begin{document}
\chapter{До вірних}
\begin{FlushRight}
    \emph{{\color{red}«Збудую Церкву Мою, і пекельні сили не подолають її»} (Мт. 16, 18).}
\end{FlushRight}

 
Ще вчора мільйони людей підіймали руки проти Бога й кричали: «Нема Бога!» А сьогодні здіймають руки до Того ж, Котрого прокололи \emph{(Ів. 19, 34, 37; Зах. 12, 10)}. Скільки раз в історії людства так було і, певно, й ще буде? А це тисячі раз підкреслює правдивість наведених слів пророка.

Ми ж, християни, не повинні хитатися кожним вітром людських міркувань. Вивчаючи події, ми з кожним разом повинні поглиблювати нашу віру в Господа, в яких би обставинах ми не були і де б не знаходилися.

Через безвірство й відступлення від Бога безмірно збільшилися беззаконства, злоба та неправда. Бо де Бог, там і любов, бо Бог є Любов \emph{(1 Ів. 4, 8, 16)}. Де Бог, там згода, там милосердя, там чесноти й братерство, там спокій і радість. А де немає Бога, там неправда, там насильство, розбрат, страждання. Правдиве слово: \emph{«Хто не служить Богові, той служить сатані»}. Середини нема.

Які ж наслідки безбожного життя? Кров, руїна, смерть, голод, нужда, страждання...

Життя лише в Богові, бо Він Сам — життя, а смерть у сатані, бо він сам неправда і смерть, виновник смерті.

Ось більша частина світу стала руїною, ось мільйони калік та безпритульних, ось мільйони позбавлені Батьківщини. Чому так сталося? Де початок цього лиха?

У боговідступництві. В тому, що, забувши Бога, люди стали жорстокі, егоїстичні, ненаситливі. Ще давно пророк про них сказав: «Не змилосердиться людина до людини» \emph{(З Езд. 15, 19)}. Отже саме життя навчає нас, як небезпечно відступати від Бога, відкидати віру в Нього, зневажати Його святі заповіді.

Тому будемо не тільки самі навчатися істин Святої Православної Віри, а навчати й інших та спасати душі їх від безвірства, як від чуми.
В цій нашій книжці сказано, по можливості коротко, про все щодо віри й служіння Богові.

\begin{FlushRight}
    \textbf{Мюнхен, Німеччина, 1946}
\end{FlushRight}
\end{document}