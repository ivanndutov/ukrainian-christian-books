\documentclass[main.tex]{subfiles}

\begin{document}
\chapter{Заповіді блаженства}
Розділ 8. Заповіді блаженства
Батьківська категорія: Бібліотека - Основи віри
 
 
Господь Ісус Христос одного разу так сказав: «Що ви звете Мене «Господи, Господи", а не робите так, як Я кажу Вам?» (Лк. 6, 46).

«Не всякий, хто каже Мені: «Господи, Господи", увійде в Царство Небесне, а той, хто виконує волю Отця Мого Небесного» (Мт. 7, 21).

З цих слів ясно стає, що самої молитви не досить, щоб стати угодним Богові і спільником Царства Небесного. До молитви треба ще додати добрі діла — виконування заповідей Божих. Серце наше повинне бути сповнене готовністю завжди чинити так, як заповідав Господь, бо інакше й молитва і віра наші будуть марними.

«Наближаються до Мене люди, — каже Господь, — устами своїми шанують Мене, а серце їхнє далеко від Мене. Даремно вони шанують Мене» (Мт. 15, 8-9).

І ось Господь показує нам, так би мовити, установчі засади, з яких виходять оті добрі діла. Він дає нам дев`ять заповідей блаженства, а саме:
\begin{enumerate}
    \item Блаженні убогі духом, бо таких є Царство Небес
    не.
    \item Блаженні ті, що плачуть, бо вони втішаться.
    \item Блаженні тихі, бо вони осягнуть землю.
    \item Блаженні голодні та жадні на правду, бо вони на
    ситяться.
    \item Блаженні милостиві, бо вони помилувані будуть.
    \item Блаженні чисті серцем, бо вони Бога побачать.
    \item Блаженні миротворці, бо вони синами Божими на
    звуться.
    \item Блаженні вигнані за правду, бо таких є Царство
    Небесне.
    \item Блаженні ви, коли вас ганьбитимуть і гнатимутьта щиритимуть про вас усяку лиху славу та наклепи ради Мене. Радійте і веселіться, бо велика нагорода вам на небесах.
\end{enumerate}

Необхідно звернути увагу на таке:

У Старому Заповіті Господь наказував з примусом, наприклад у 5-й заповіді: «Шануй батька й матір...» а далі: «хто злословить батька або матір, смертю нехай буде покараний» і інше (див. книгу Второзаконня).

А в Новому Заповіті Господь ніколи не примушує. Як кроткий і смиренний серцем (Мт. 11, 29), Він радить, закликає, щоб охоче прийняли Його вчення і зазначає, які будуть наслідки для тих, що будуть його виконувати. Тому в кожній заповіді блаженства треба бачити дві частини: у першій Господь дає заповідь-пораду, а в другій визначає нагороду.

Розглянемо тепер усі дев`ять заповідей окремо.

\section{ПЕРША ЗАПОВІДЬ БЛАЖЕНСТВА}

Блаженні убогі духом, бо таких є Царство Небесне.

У слові «блаженні» розуміємо дві сторони. Перша: блаженні щасливі в собі тим, що осягнуть Царство Небесне з вічною радістю та втіхою. Друга: їх ублажить Господь, похвалить, приласкає, обдарує славою; їх ублажать і ангели та святі, їх ублажатиме й Церква Христова на землі.

\subsection{Що означає бути убогим духом?}

Гордощі — найтяжчий гріх перед Богом. Через гордощі упав один з найперших ангелів і став сатаною. Упали й ті, що послухали його. Через гордощі упали Єва й Адам, втратили своє райське блаженство і накликали прокляття та страждання на всі покоління свої.

Горді ніколи не осягнуть Царства Небесного, бо вони лише собі служать, собі поклоняються. Вони не можуть любити інших і комусь робити добро. Тому Господь і відзначає, що Царство Небесне можуть осягнути тільки убогі духом.

Бути убогим духом значить бути завжди переконаним, що наш розум, наша мудрість, наші сили перед Богом нічого не варті. Що ми нічого свого не маємо, а маємо лише те, що нам дає Господь. Також, що своїми силами ми нічого доброго не можемо зробити, якщо Господь не допоможе нам, пославши благодать Духа Свого. Отже треба не покладатися на себе, а у всьому вдаватися до милосердя Божого. Убогість духовну святий Іван Золотоустий коротко зве смиренномудрієм (бесіда 15-а на Євангеліє від Матвія).

Убогість духа не залежить від маєтків, від матеріального багатства. Не залежить вона також від високої освіти людини або доброї земної слави.

Багатство, навіть якщо воно нажите правдою, не є нашою власністю. То дар Божий, який дається нам тимчасово, бо Господь зубожує і збагачує, понижує й підносить (1 Цар. 2, 7). Тому на матеріальні блага треба дивитися як на тлінні, нетривкі й скороминущі. На них не можна покладатися й віддавати їм свого серця. Вони не замінять благ духовних.

«Коли багатство намножується, не віддавайте йому серця» (Пс. 61, 11).

«Бо яка користь людині, коли вона і ввесь світ пригорне, а душу свою занапастить? Або який викуп дасть людина за душу свою?» — каже Господь (Мт. 16, 26).

Не гріх бути багатим праведно, але тяжкий гріх гордитися своїм багатством і бути скупим. Праведний Іов також був багатий, але це йому ніскільки не зашкодило бути убогим духом (див. книгу Іова).

Але матеріальне убозство більше сприяє осягненню Царства Небесного, ніж багатство, бо воно паралізує гордощі.

«Коли хочеш досконалим бути, піди продай майно твоє, роздай убогим і будеш мати скарби на небесах. Приходь і йди за мною» (Мт. 19, 21). Так сказав Христос багатому юнакові, який хотів унаслідувати Царство Небесне й життя вічне. І до апостолів каже:

«Істинно кажу вам: трудно багатому ввійти в Царство Небесне» (Мт. 19, 23).
 
Але й матеріальне убозство тільки тоді сприяє нам осягнути Царство Небесне, коли ми або добровільно його собі вибираємо, як це робили багато святих, або, якщо воно не від нас залежить, ми несемо його з покорою, дякуємо Богові за те, що Він нам дає і не заздримо чужому.

Також можна бути й дуже освіченою людиною, а через те і славною, а в той же час пам`ятати, що знання дає Бог, як сказано:

«Господь дає мудрість; із уст Його знання й розум» (Прит. 2, 6).

І треба казати, як кажуть багато вчених: «Я тільки те знаю, що нічого не знаю». Бо справді, людське знання ніщо в порівнянні із знанням Божим.

Свої знання такі люди скеровують на пізнання Бога. Бо найвище знання — то Богознання (їв. 17, 3).

\subsection{Убогим належить Царство Небесне. Коли воно прийде?}

Для них воно починаєтиься вже в цьому житті. Бо смиренна і покірна волі Божій людина вже тепер знаходить у своїй душі блаженний мир. Вона лагідна до всіх, весела і спокійна.

Повноту ж Царства Небесного убогі духом осягнуть у житті вічному.

Духовне убозство або смиренність — то ґрунт до добродійств; з нього вони виростають. То перший щабель на драбині до Царства Небесного, бо з нього починаються всі інші чесноти й добродійства.

\section{ДРУГА ЗАПОВІДЬ БЛАЖЕНСТВА}

Блаженні ті, що плачуть, бо вони втішаться.

Говориться тут не про всякий плач. Люди часто плачуть із-за примх, або з образи, або з інших житейських причин. Не про такий плач тут говорить Господь. Коли людина уяснить собі свої провини перед Богом, свою нікчемність, убогість та недосконалість, безперечно вона почне сумувати, плакати й смирятися та благати у Бога помочі. Людські ж недосконалості та помилки вона почне пробачати, а не осуджувати. Більше того, буде поспішати допомогти кожному нещасному в його помилках та нерозважностях. Тому й сказано:

«Скорбота перед Богом незмінне приводить до покаяння на спасіння, а журба світу цього приводить до смерті» (2 Кор. 7, 10).

Святий Іван Золотоустий у своїй молитві каже: «Господи, дай мені сльози і пам`ять про смерть та розчулення» (Молитви вечірні).

\subsection{Чому великий святитель просить у Господи сліз?}

Сльози в покаянні — то найкращі ліки для гріховних недуг. Покаянні сльози — то дар Божий, і щасливий той, хто його має.

Нерозкаяний і гордий грішник не заплаче, бо скам`яніле його серце не зм`ягчається. То страшний стан душі. То гріховна смерть, бо закам`янілий грішник не може покаятися.

Для того, щоб заплакати перед Богом, треба осудити себе перед своєю совістю, як блудний син (Лк. 15, 11-32). Тоді починається сумування над собою, поклик до молитви, а в щирій молитві мимоволі починається святий покаянний плач. Сльози в самотній молитві — то вірна ознака щирого сумування за свою гріховність і ознака щирого покаяння.

\subsection{Що обіцяє Господь тим, що плачуть?}

Вони втішаться доброю Христовою втіхою. Господь заспокоїть їх, обгорне ласкою Своєю, простить їм провини їх, окриє Своєю благодаттю і вже тепер наповнить тихою мирною радістю серця їх, Але істинно висока втіха чекає їх на небесах, де світла оселя всіх праведників.

Отже цією заповіддю Господь ублажає тих, що плачуть, молячись за гріхи свої, і в той же час навчає нас, щоб скорбота за наші гріхи не доходила до відчаю. Хоч би який тяжкий був гріх, у Господа милосердя більше. Треба тільки каятися та не втрачати надії.

\section{ТРЕТЯ ЗАПОВІДЬ БЛАЖЕНСТВА}

Блаженні тихі (смиренні, незлобиві), бо вони осягнуть землю.

У цій заповіді Господь ублажає тихість духа людини. По суті, тихість духа — це лагідна духовна рівновага, пройнята обережністю: нікого не ображати й самому не ображатися ні за яких оставин. Коротко — мирний дух. Блаженні ті, хто зусиллям над собою здобув собі оту величну чесноту — християнську тихість! У них в серці завжди перебуває лагідний радісний спокій. Вони завжди задоволені з своєї долі і дякують Богові. Вони ні на кого не нарікають і нікому не заздрять. Такі люди правдиво щасливі: їх любить Господь і завжди радує Своєю божественнною радістю. Вони терпеливо й спокійно переносять усяке лихо і навіть серед великого горя знаходять в душі своїй утіху.

\subsection{Звідки походить ота свята тихість?}

Той, хто зрозумів свою недосконалість, хто усвідомив, що без Божої помочі він ніщо, той не має підстав гордитися і ставити себе вище інших. Цим він здобув смирення, убогість духа. А від цієї першої основної чесноти починають виявлятися інші.

\subsection{Яку землю осягнуть тихі?}

Найвиразніше це справдилось на християнах. Вони більше трьох століть були гнані, без жалю нищені, а врешті вони опанували всі країни землі, якими так міцно володіли поганські народи.

Але істотніше говорить Господь про «Землю живих», про тих, які вже не вмирають, про землю вічного життя, що про неї апостол каже:

«Нового неба й нової землі ми по Його обітувашію чекаємо, на яких правда живе» (2 Петр. З, 13).

«Вірую, що побачу блага Господні на Землі живих» (Пс. 26, 13).

\section{ЧЕТВЕРТА ЗАПОВІДЬ БЛАЖЕНСТВА}

Блаженні голодні та жадні на правду, бо вони наситяться.

У цій заповіді говориться про правду, що про неї каже пророк Давид:

«Праведний Господь, любить правду, до праведних звернені очі Його» (Пс. 10, 7).

Хто живе серцем у Богові, тому кривда противна: вона для нього чужа і мучить його. Праведний всюди бажає правди й справедливості, а також праведності, що вміщає в собі всі добродійства. Такі люди любі Богові, тому Господь їх ублажає. Але говориться тут і про іншу правду, про яку сказано у пророцтві Даниїла: «Приведеться правда вічна» (Дан. 9, 24). Тут мова йде про виправдання винної перед Богом людини через благодать і віру в Господа Ісуса Христа.

Про ту ж правду каже й апостол:

«Виправдання Боже через віру в Ісуса Христа у всіх і на всіх віруючих, бо немає різниці, тому що всі згрішили і позбавлені слави Божої. (Ми) одержуємо виправдання даром по благодаті Його, викупленням у Христі Ісусі» (Рим. З, 22-24).

\subsection{Голодні і жадні на правду — хто вони?}

Це ті, що люблять добро і роблять його, але не вважають себе за праведників і не покладаються на свої добрі діла, а визнають себе грішниками і винуватцями перед Богом. Тому молитвою віри з надією вони бажають благодатного виправдання перед Богом через Ісуса Христа.

\subsection{Як розуміти слово «наситяться»?}

Господь обіцяє голодним і жадним на правду, що вони наситяться, подібно до того, як голодні і спраглі задовольняють їжею й питтям свою спрагу і тим підкріпляють свої сили. Так і вони одержать від Господа задоволення своїх бажань. Він пошле їм також благодатні сили для дальших добродійств ще тут на землі.

Але повне задоволення Господь обіцяє їм у житті вічному, як каже пророк: «Пасищуся, коли стану перед славою Твоєю» (Пс. 16, 15).

\section{П`ЯТА ЗАПОВІДЬ БЛАЖЕНСТВА}

Блаженні милостиві, бо вони помилувані будуть.

Ніщо так не умилостивляє Бога, як милостиня, як милосердя.

«Милості хочу, а не жертви», — сказав Господь через пророка (Ос. 6, 6). Те саме повторив Христос (Мт. 12, 7).

На Своєму останньому суді, Господь скаже до праведних:

«Прийдіть, благословенні Отця Мого, унаслідуйте Царство, приготоване вам від початку світу. Я голодний був, ви нагодували Мене; хотів пити, ви напоїли Мене. Роздягнений був, ви зодягнули Мене; подорожнім був, ви прийняли Мене; хворий був, ви відвідали Мене; у в`язниці був, ви прийшли до Мене» (Мт. 25, 35-36).

«Милосердя перевищує суд» (Як. 2, 13).

Милостиня на протязі всієї історії людства високо оцінюється в очах Божих.

Так Авраам, бувши милостивим, прийняв Бога у Святій Тройці, що явився йому під виглядом подорожніх (Бут. 18, 1-22).

Праведний Іов так каже до друзів: »Я спас страждальця, що плакав, і сироту безпорадного... Я був очима сліпому і ногами кривому. Я був батьком для убогого» (Іов. 29, 12, 15-16).

Товіт так навчає сина свого Товію:

«Від майна твого подавай милостиню, і нехай око твоє не шкодує, коли будеш творити милостиню. Ні від якого жебрака не відвертай лиця твого, тоді й від тебе не відвернеться» (Тов. 4, 7).

«Праведний милує й позичає» (Пс. 36, 21).

«Пе відмовляй у допомозі, коли рука твоя в силах це зробити» (Прит. З, 27).

З особливою силою милостиню похваляє Христос: «Всякому, хто просить у тебе, дай і від того, хто хоче позичити у тебе, не відвертайся... Добро творіть тим, що ненавидять вас» (Мт. 5, 42, 44).

Іван Золотоуст про милосердя каже так: «Різноманітні образи милосердя, широка ця заповідь» (Бесіда 15-а на Євангеліє від Матвія).

І справді: милосердя можна чинити на кожному кроці. Допоможи немічному й старому, поможи підняти, уступи місце старшому, покажи дорогу тому, хто не знає куди йти, порадь людині у скрутному стані, допоможи нещасному матеріально, скільки можеш.

Цих образів милосердя безліч. Ось головніші:

Діла милості для тіла людини:

\begin{enumerate}
    \item Нагодуй голодного.
    \item Напій спраглого.
    \item Зодягни нагого або такого, що не має пристойної одежі.
    \item Відвідай хворого, розваж, порадь, допоможи видужати. Прохай за нього лікарів. А якщо має вмирати, то постарайся зменшити йому страждання їжею, питтям, ліками, розвагою. Коли ж умре, допоможи достойно поховати.
    \item Прийми подорожнього в хату, нагодуй і дай йому відпочити.
    \item Поховай або допоможи поховати померлого в убожестві.
\end{enumerate}

Діла милосердя для душі:
\begin{enumerate}
    \item Умов грішника й наверни його з загибельної дороги на добру (Як. 5, 20).
    \item Навчи віри й добра того, хто не знає.
    \item Подай ближньому добру вчасну пораду під час його труднощів або тоді, коли він не помічає небезпеки.
    \item Молися за нього Богові, щоб допоміг йому.
    \item Утіш і розваж засмученого.
    \item Не віддавай злом тим, що спричиняють тобі прикрість.
    \item Від серця прощай тим, хто тебе скривдив.
    \item Допоможи й порадь хворому поговіти, щоб йому не вмерти без Святого Причастя.
\end{enumerate}

\subsection{Чи не суперечить кара ділам милосердя?}

Не суперечить лише тоді, коли кого карають по суду за справжні злочини, при чому або для того, щоб злочинця виправити, або за правосуддям припинити його злочини та уберегти інших від небезпеки.

\subsection{Яка ж нагорода милостивим?}

Вони помилувані будуть. За їх милості до них буде Милостивим Господь на Страшному Суді Своєму. Але й тепер їх милосердя покриє багато гріхів.

\subsection{А що чекає немилостивих?}

Те, що каже апостол Яків: «Суд без милості тим, що не були милостиві» (Як. 2, 13; див. Мт. 25, 41-46).

\section{ШОСТА ЗАПОВІДЬ БЛАЖЕНСТВА}

Блаженні чисті серцем, бо вони Бога побачать.

Щоб відповісти на питання, що значить бути чистим серцем, треба насамперед взяти до уваги, що не можна змішувати поняття «чистосердечність» і «чистота серця».
 
Чистосердечність — це щирість, якою людина нелицемірно висловлює те, що в неї на душі, а коли щось добре робить, то не на показ, а з доброго наміру. Але це лише нижчий ступінь чистоти серця.

Повної ж чистоти серця людина досягає постійним зусиллям над собою у виконанні Божих заповідей та щирою молитвою. Постійним думанням про Бога та про все божественне вона проганяє з свого серця все гріховне — і помисли, і бажання, і тілесні пристрасті, бо все те гине від божественного, як темрява від світла.

Тоді сердце людини правдиво стає храмом Духа Святого (1 Кор. 6, 19), бо в ньому оселяється Бог, як і сказано через пророка: «Оселюся в них і ходитиму (в них)» (2 Кор. 6, 16). Те ж і в книгах пророків Єремії (3, 19) та Осії (1, 10).

Так і Христос каже: «Хто любить Мене і дотримується слів Моїх, того й Отець Мій полюбить, і Ми прийдемо до нього й оселю у нього (в серці) створимо» (їв. 14, 23).

«Ось Я стою коло дверей (серця) і стукаю. Хто почує голос Мій і відчинить двері, Я ввійду до нього і буду вечеряти з ним, а він зо Мною (буде в серці в нього)» (Об. З, 20).

Тоді серце такої людини наповнюється тихою святою радістю й безмежною любов`ю до Бога і всього Божого. Молитва стає для неї немов би диханням, вона повторює ім`я Боже безперестанно, як і каже пророк Давид:

«Які солодкі для гортані моєї слова Твої і солодші від меду устам моїм» Ще. 118, 103).

Не тяжко зрозуміти, що буде виливати з себе таке серце (Мт. 7, 17-18).

За свідченням праведних, молитва тоді для людини стає великою радістю, бо вона немов би поринає в Богові і ніби сама стає часткою божественного, її серце наповняє якийсь невимовний радісний мир, лагідність і любов до всіх.

«Хочеться всіх обняти, кожному зробити щось добре і поділитися тією святою небесною солодкістю. Які то велично прекрасні переживання!.. Господи, дай кожному християнинові хоч на хвилиночку спробувати тієї радості, тоді він назавжди буде Твій» (з життя святих).

\subsection{Як і коли чисті серцем побачать Бога?}

Не очима тіла, а очима просвіченого серця (Еф. 1, 18). Уже тепер на кожному місці.
 
«Я бачу Господа перед собою завжди, — каже пророк Давид, — бо Він праворуч мене» (Пс. 15, 8).

Чисте від гріховності серце також здатне бачити (відчувати) Бога, як чисте око спостерігає й бачить світло.

Але споглядаючи чистим серцем Бога тепер, праведники сподобляться побачити Його, як Він є, в житті вічному. Святий апостол Іван Богослов пише так:

«Улюблені! Ми тепер діти Божі; але ще не відкрилося, що ми будемо. Знаємо тільки, що коли відкриється, будемо подібні до Нього, тому що побачимо Його, як Він є. І кожний, хто має цю надію на Нього, очищає себе через те, що Він чистий» (1 їв. З, 2-3).

Отже нагорода така, якої тепер ми собі уявити не можемо, і блаженство вище нашого розуміння.

\section{СЬОМА ЗАПОВІДЬ БЛАЖЕНСТВА}

Блаженні миротворці, бо вони синами Божими назвуться.

Ця заповідь навчає нас, що Христос Господь, Єдинородний Син Божий, прийшов на землю не для того тільки, щоб примирити людей з Богом, взявши на Себе провини їхні, а ще й для того, щоб і їх примирити між собою. На протязі усього Свого земного життя Він навчав людей жити у згоді й однодушності:

«Мир зоставляю вам, мир Мій даю вам, не такий, як люди дають вам», — каже Він апостолам (їв. 14, 27).

«Щоб усі були одно, як Ти, Отче, в Мені і Я в Тобі, щоб і вони в Нас були одно», — каже Христос у Перво-священницькій молитві до Отця Небесного (їв. 17, 21).

«Мирися з суперником своїм скоріше, поки ти ще в дорозі з ним (поки ще живеш)» (Мт. 5, 25).

«Бог не є (Бог) безладдя, а Бог миру (згоди)», — вказує апостол Павло (1 Кор. 14, 33).

Христос навчає не мститися, а прощати братові провину хоч до 70 раз по сім (Мт. 18, 22), і то - для згоди.

Так і апостол благає: «Браття, коли то залежить від вас, зо всіма мир (згоду) майте» (Рим. 12, 18).

\subsection{Як же практично виконувати цю заповідь?}

Треба бути лагідним з кожною людиною, себе не вивищувати, а її не понижувати. Не давати приводу до незгод, а коли хтось інший такий привід дає, то поспішитися затерти той привід. Якщо ж сталася незгода, то силкуватися припинити її, навіть поступившись дечим своїм, якщо те не суперечить обов`язкові й не спричинить комусь прикрості. Старатися примирити тих, що ворогують, а як не вдасться, то молитися за них Богові, щоб зм`ягшив їх серця і привів до згоди.

«Миротворці синами Божими назвуться», — каже Господь. Що означає таке обітування?

Воно відзначає високість подвигу митротворців, а також і велич їх нагороди. Оскільки вони подвигом своїм стали подібні до Єдинородного Сина Божого, то й самі вони стають благодатними синами Божими і, як сини, будуть співнаслідниками Божими через Ісуса Христа (Гал. 4, 7).

Височезний ступінь нагороди!

\section{ВОСЬМА ЗАПОВІДЬ БЛАЖЕНСТВА}

Блаженні вигнані за правду, бо таких є Царство Небесне.

«Праведний Господь любить правду, на праведних дивиться лице Його» (Пс. 10, 7).

«Невірна вага — мерзенність перед Господом, вага ж вірна приємна Йому» (Прит. 11, 1).

\subsection{Від кого неправда?}

Від диявола.

«Диявол... не встояв у правді, бо в ньому немає правди. Коли ж говорить неправду, то говорить від свого, бо він сам неправда, лжець і отець неправди», — каже Христос (їв. 8,44).

Христос є «Путь і Істина й Життя» (їв. 14, 6), і «всяка неправда не від Істини» (1 їв. 2, 21), а від лукавого.

Тому Христос так похваляє правду і обіцяє таку високу нагороду гнаним за правду.

\subsection{Чого навчає ця заповідь?}

Нею Господь навчає нас бути правдивими й справедливими. Навчає любити правду і мужньо її відстоювати, хоч би довелося за неї й потерпіти зневагу, гоніння і навіть смерть.

Проти цієї заповіді страшний гріх — свідчити проти правди або давати неправдиву присягу іменем Господнім.

\subsection{Що ж обіцяє Господь гнаним за правду?}

Царство Небесне. Бо вони, борючись за правду, уподобилися Христу Господу, Який Сам перетерпів велике гоніння за правду. Господь вважає справедливим винагородити їх за страждання, що їх вони зазнали за правду.

\section{ДЕВ`ЯТА ЗАПОВІДЬ БЛАЖЕНСТВА}

Блаженні ви, коли вас ганьбитимуть і гнатимуть та щиритимуть про вас усяку лиху славу та наклепи ради Мене. Радійте і веселіться, бо велика нагорода вам на небесах.

Цією заповіддю Господь навчає бути готовими перетерпіти за Його ім`я зневагу, гоніння і навіть саму смерть. В іншому місці Він каже:

«Якщо світ вас ненавидить, то знайте, що Мене раніше, ніж вас, він зненавидів. Бо якби ви були від світу (цього), то світ своїх любив би, а як ви не від світу цього, бо Я вибрав вас від світу, через те ненавидить вас світ. Коли Мене гнали, то й вас гнатимуть» (їв. 15, 18-20).

«На світі (цьому) скорбні будете, але будьте мужні: Я переміг світ» (їв. 16, 33).

Так ще до страждань Своїх у цій заповіді блаженства Христос провістив сумну долю Своїх послідовників, але утішив їх, щоб не сумували під час гонінь, а раділи й веселилися, бо нагорода їм велика на небесах.

Тому то кожний, хто називає себе християнином, повинен бути готовим з радістю прийняти зневагу, вигнання, страждання й саму смерть за ім`я Христове, за святу Православну віру, за Церкву Христову. Цей подвиг є подвиг муче-ничий. Його з великою честю виконали мільйони мучеників і воїстину одержали велику нагороду на небесах.
\end{document}