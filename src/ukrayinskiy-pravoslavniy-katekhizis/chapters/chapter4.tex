\documentclass[main.tex]{subfiles}

\begin{document}
\chapter{Чесноти Божі: віра, надія, любов}
Вся наша християнська релігія основується на трьох головних чеснотах, як пише св. апостол Павло: \emph{«Нині діють оці три: віра, надія, любов» (1 Кор. 13, 13)}. Ці три чесноти настільки тісно пов`язані між собою, що стають нерозлучними. Вони робляться досконалими і угодними Богові тоді, коли діють разом.
\section{Про віру}
\begin{FlushRight}
    \emph{«Без віри не можна догодити Богові, і кожному, хто приходить до Бога, треба вірувати, що Він є і тим, що шукають Його, дає нагороду» (Євр. 11, 6).}
\end{FlushRight}
\begin{FlushRight}
    \emph{«Віра — це основа того, на що надіємось. Повність того, чого не бачимо» (Євр. 11, 1).}
\end{FlushRight}
\begin{FlushRight}
    \emph{«Бо як тіло без душі мертве, так і віра без діл (любові) мертва» (Як. 2, 26).}
\end{FlushRight}

\subsection{Що значить вірувати?}
То значить: мати повну упевненість в існуванні Бога і в правдивості богооб`явлених істин, записаних у книгах Святого Письма.

Ми віруємо в Бога тому, що Бог є Абсолютна Істина.
 
В Ньому немає неправди \emph{(Пс. 91, 16)}. Він праведний \emph{(1 Ів. 2, 29; 3, 7)}. Дух (Божий) - Істина \emph{(1 Ів. 5, 6)}.

Тому то все, що об`явлено нам Богом через пророків і що об`явлено Сином Божим та записано в Євангелії — це безумовна Істина, і ми приймаємо її безумовно.

\subsection{Чому необхідно вірувати в Бога?}

Тому, що без віри неможливо догодити Богові \emph{(Євр. 11, 6)}. Сам Господь наш Ісус Христос заповідає нам віру, як єдину умову, при якій можливе здійснення того, чого просимо в молитві:

\begin{FlushRight}
    \emph{«Коли стоїте й молитесь, віруйте, що одержите, і буде вам... майте віру Божу» (Мр. 11, 22, 24-25).} 
\end{FlushRight}

При кожному уздоровленні хворих і калік Господь вимагав віри. Наприклад:

\noindent Каже Він капернаумському сотникові: {\color{red} \emph{«... іди, і як увірував, так нехай буде тобі» (Мт. 8, 13).}}

\noindent Кровоточивій: {\color{red} \emph{«будь бадьора, дочко, віра твоя спасла тебе» (Мт. 9, 22).}}

\noindent Двом сліпцям: \emph{{\color{red} «Чи віруєте ви, що Я можу зцілити вас?»} Вони кажуть: «Так, Господи». Христос доторкнувся до очей їх і каже: {\color{red} «По вірі вашій нехай буде вам»} (Мт. 9, 29).}

\subsection{Чому так? Що за сила віри?}
Віра з любов`ю нерозлучні. Коли ми віримо кому, то стаємо з ним одно. Діти вірять своїм батькам і стають з ними одно. Вірні друзі вірять один одному, і тому вони — одно: те саме люблять, те саме діють. Отже, віра нас об`єднує в одно, наче зливає докупи. А тим більше — віра в Бога.

Чим сильніше й глибше віруємо, тим більше любимо, тим більше наближаємося до Бога, а тим самим робимося кращими. Через віру ми стаємо одне з Богом, як рідні діти з батьками.

Без віри ж неможливо ні служити Богові, ні молитися, бо молитва без віри — то порожні слова.

\section{Про надію християнську}

Надія християнська — то заспокоєння серця в Бозі з певністю, що Він безперестанно турбується про нас і наше спасіння, дасть нам усе необхідне до життя і обітоване блаженство, якщо ми будемо виконувати заповіді Його.

\begin{FlushRight}
    \emph{«Господь Ісус Христос — надія наша» (основа нашої надії). (1 Тим. 1,1).}
\end{FlushRight}

\begin{FlushRight}
    \emph{«Цілком покладайтеся на подану нам благодать явленням Ісуса Христа» (1 Петр. 1, 13).}
\end{FlushRight}

Надія християнська полягає в тому, що істинний християнин не надіється на свою силу, або мудрість чи багатство, і не на сильних віку цього, а на Бога, що підводить мертвих \emph{(2 Кор. 1, 9)}.
\subsection{На чому основується християнська надія?}
На Божих обітуваннях. Як сказав Господь:

\begin{FlushRight}
    \emph{\color{red}«Чи забуде мати грудне дитя своє, щоб не жаліти його? Але коли б і забула, то Я людей Своїх не забуду»} (Іс. 49, 15).
\end{FlushRight}

Бог є творець і Отець наш, ми ж Його творіння й діти, хоч і прогнівили Його. Але Христос Господь, прийнявши на Себе наше тіло, взяв і наші гріхи і викупив нас Своєю Кров`ю, загладив нашу вину перед Богом і знову дав нам право дітьми Божими бути \emph{(Ів. 1, 12)} і, як синам і дочкам, дав право кликати до Бога: «Авва, Отче» \emph{(Гал. 4, 6).}

Отже, ми не раби у Бога, а сини й дочки, наслідники ласки Божої через Ісуса Христа \emph{(Гал. 4, 7).}

Христос виразно навчає нас надіятися на Бога. Наприклад:

\begin{FlushRight}
   \emph{{\color{red} «Не турбуйтеся душею вашою, що вам їсти, що пити або в що одягатися... Бо знає Отець ваш Небесний} (раніше, ніж ви попросите){\color{red}, що все те потрібне вам»} (Мт. 6, 25, 32).}
   \footnote{Цих слів не треба розуміти так, що можна скласти руки і чекати поки Бог дасть. Така надія гріховна і невгодна Богові. Господь Сам сказав: {\color{red} «Всякий, хто трудиться, заслуговує на нагороду свою»} (Лк. 10, 7).}
\end{FlushRight}


\subsection{Що основується на християнській надії?}
\begin{enumerate}
\item Молитва. Бо нащо б і молитися, коли б не було надії, що Господь задовольнить те, чого просимо? А Він сказав: \emph{{\color{red}«Коли чого попросите у Отця в ім`я Моє, Я за довольню, щоб прославився Отець у Сині»} (Ів. 14, 13)}.
\item Учення про небесне блаженство і слідування за тим ученням.
\end{enumerate}
 
\section{Про любов християнську}

Християнська любов — це прив`язаність нашого серця до Бога. Це чуття, яке невидимими нитками зв`язує нас з Богом невимовним чуттям душі і серця до ревності. Через любов ми стаємо одно з Богом, так що коли хтось зневажає Бога, то він страшними болями проймає наше серце і найніжніші почуття душі.

Така любов буває лише тоді, коли ми всією душею віруємо в Бога, бо без віри любити не можна. Віра тоді корисна, коли вона об`єднується з любов`ю до Бога та ближніх і виявлена добрими ділами.

Кажуть апостоли:

\begin{FlushRight}
    \emph{«Яка користь, брати мої, коли хто каже, що має віру, а діл (добрих) не має? Чи ж може (така) віра спасти його?.. Бо як тіло без душі мертве, так і віра без (добрих) діл мертва» (Як. 2, 14, 26, 20).}
\end{FlushRight}

\begin{FlushRight}
    \emph{«Ти віруєш, що Бог один. Добре робиш. Біси також вірують і тремтять» (Як. 2, 19).}
\end{FlushRight}

\begin{FlushRight}
    \emph{«Хто не любить брата свого (ближнього), той перебуває у смерті» (1 Ів. З, 14).}
\end{FlushRight}

\begin{FlushRight}
   \emph{«Без віри неможливо догодити Богові, і кожному, хто приходить до Бога, треба вірувати, що Він є і тим, що шукають Його, дає нагороду» (Євр. 11, 6).} 
\end{FlushRight}

Так і любов: якщо вона не виявляє себе добрими ділами, вона не є правдива:
\begin{FlushRight}
    \emph{}
\end{FlushRight}

\begin{FlushRight}
    \emph{{\color{red} «... бо хто любить Мене, той заповіді Мої виконує»} (Ів. 14, 21).}
\end{FlushRight}

\begin{FlushRight}
    \emph{«Це є любов Божа, щоб ми заповіді Його виконували» (1 Ів. 5, 3).}
\end{FlushRight}

\begin{FlushRight}
   \emph{«Не любім словом або язиком, а ділом та Істиною» (1 Ів. З, 18).} 
\end{FlushRight}

Звідси бачимо, що неможливо спастися лише одними добрими ділами, не маючи віри в Бога. Бо людина, яка не вірить, не може любити Бога і не може нічого доброго зробити для Нього. Всі її «добрі діла» робляться з розрахунком на людську славу або задля власних вигід.

Люди мають природну любов до своїх батьків, дітей, рідних, до свого народу. Така любов побуджує їх піклуватися за них.

Але є й вища любов до людей — ради Господа. Вона базується на тому розумінні, що кожна людина є творіння Боже, що вона викуплена Дорогоцінною Кров`ю Христа Спасителя, що людина має душу, призначену до вічного життя, що кожна людина — це наш брат чи сестра, особливо єдиновірні.

Кожний і кожна з нас повинні любити і шанувати також і себе самих, бо тіло наше — це храм Духа Святого. Ми сини і дочки Божі. Ми маємо в собі душу, яку вдихнув у нас Сам Бог. В цьому Духові є образ і подоба Бога, розуміння високих моральних чеснот: краси, справедливості, любові до Бога, ближнього, свободи волі.
\end{document}