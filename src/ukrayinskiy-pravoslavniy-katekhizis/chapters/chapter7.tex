\documentclass[main.tex]{subfiles}

\begin{document}
\chapter{Догмати Православної віри}
Слово «догма» або «догмат» значить учення. Але таке вчення, що відкрите Самим Богом через Господа Ісуса Христа та святих пророків. Воно не може бути ніким змінене, зменшене чи доповнене.

Наприклад — Бог Один. Бог Триєдиний. Бог вічний. Догмат про Осіб Пресвятої Тройці. Про втілення Сина Божого. Про зачаття Його від Духа Святого. Про безсмертя душі. Про загробне життя. Про Страшний Суд і ін.

Догмати знаходяться у Святому Письмі у різних книгах і в різних місцях. Для того ж, щоб вони всі стали відомі всім людям, святі Отці і Вчителі Церкви виявили їх та зібрали в один список, що зветься Символом Віри.
\section{Символ віри}

Символом Віри зветься коротке, але точно викладене вчення (догмати) віри, як повинен вірувати кожний християнин, щоб стати членом Істинної Церкви Христової і унаслідувати спасіння. Для легшого розуміння і вивчення Символа Віри, його поділяють на дванадцять частин, які звуться членами Символа Віри. Перші сім з них були прийняті Отцями Першого Вселенського Собору (325 рік), решту прийняли Отці Другого Вселенського Собору (381 рік). Члени Символа Віри ідуть у такому порядку:
\begin{FlushRight}
    1. Вірую в Єдиного Бога Отця, Вседержителя, Творця неба і землі, видимого всього і невидимого.

    2. І в Єдиного Господа Ісуса Христа, Сина Божого, Єдинородного, від Отця рожденого перше всіх віків: Світло від Світла, Бога Істинного від Бога Істинного, рожденого, несотвореного, Єдиносущного з Отцем, через Якого все сталося.
    
    3. Що заради нас людей і для нашого спасіння зійшов з небес і тіло прийняв від Духа Святого і Марії Діви істав чоловіком.
    
    4. І розп`ятий був за нас при ПонтіІ Пілаті, і страждав і був похований.

    5. І воскрес на третій день, як було написано.

    6. І вознісся на небеса і сидить праворуч Отця.

    7. І знову прийде зі славою судити живих і мертвих, і Царству Його не буде кінця.

    8. І в Духа Святого, Господа, Животворчого, що від Отця сходить, що з Отцем і Сином рівнопоклоняємий і рівнославимий, що говорив через пророків.

    9. В єдину, Святу, Соборну і Апостольську Церкву.
     
    10. Визнаю одне Хрещення на відпущення гріхів.

    11. Чекаю воскресіння мертвих.

    12. І життя будучого віку. Амінь.
\end{FlushRight}

\section{ПЕРШИЙ ЧЛЕН СИМВОЛА ВІРИ}

\begin{FlushRight}
    Вірую в Єдиного Бога Отця, Вседержителя, Творця неба і землі, видимого всього і невидимого.
\end{FlushRight}

У цьому члені подається вчення (догмат) про Першу Особу (Лице, Іпостась) Пресвятої Тройці — Бога Отця.

Слово «вірую» означає: всім серцем визнаю, приймаю і відкрито перед усіма людьми ісповідую (виявляю) мою віру.

\subsection{Чи потрібно виявляти свою віру перед людьми?}

Христос Господь так каже: «Хто визнаватиме Мене перед людьми, того визнаю і Я перед Отцем Моїм небесним, а хто одцурається Мене перед людьми, того відцураюся і Я перед Отцем Моїм Небесним» \emph{(Мт. 10. 32-33)}.

Відкрито виявляти свою віру в Бога, значить проповідувати віру. Тому й сказано: «Серцем вірується на праведність, а устами визнається на спасіння» \emph{(Рим. 10, 10)}.
 
\subsection{... в Єдиного Бога ...}

Цими словами відзначається, що Бог є тільки один. Сам Господь через пророка Ісайю свідчить: \emph{{\color{red} «Я — Господь, і іншого немає, крім Мене»} (Іс. 45, 5)}. \emph{{\color{red} «Я Перший, Я й Останній, і крім Мене немає Бога»} (Іс. 44, 6)}.

Так же свідчить і апостол Павло: \emph{«Немає іншого Бога, крім Одного... у нас один Бог Отець, з Котрого все і ми для Нього, і Один Господь Ісус Христос, через Котрого все і ми Ним» (1 Кор. 8, 4, 6)}.

Самого Єства Божого ніхто не знає: ні ангели, ні люди, бо Воно вище їх розуміння, як і написано: \emph{«Божого ніхто не знає, тільки Дух Божий (1 Кор. 2, 11). «Бог живе у світлі неприступнім, Його ніхто з людей не бачив і бачити не може» (1 Тим. 6, 16)}.

Бог Дух \emph{(їв. 4, 24, 2 Кор. З, 17).}

Бог Вічний \emph{(Іс. 41, 4).}

Бог Добрий, Благий \emph{(Мт. 19, 17).}

Бог — Любов \emph{(1 їв. 4, 16).} \emph{«Щедрий і Милостивий Господь, Довготерпеливий і Многомилостивий. Добрий Господь до всього, і ласка Його на всіх ділах Його» (Пс. 144, 8-9).}

Бог знає все \emph{(1 їв. З, 20).} \emph{«Дух Господній наповняє всесвіт. Він обіймає все, знає кожне слово» (Прем. 1, 7).}

Бог Правосудний. \emph{«Праведний Господь, любить правду, на праведних дивиться Лице Його» (Пс. 10, 7).}

Бог Всемогутній. \emph{«Він сказав, і сталося, повелів, і створилося» (Пс. 32, 9).}

Бог Всюдисущий, Повсюдний. \emph{«Дух Господній наповняє всесвіт. Він обіймає все» (Прем. 1, 7).}

Бог Незмінний. \emph{«У Отця світу немає переміни або хоч тіні зміни» (Як. 1, 17).}

Бог Вседовільний, Самодостатній. \emph{«Нічого ні від кого не потребує, і Сам усім дає життя і все» (Діян. 17, 25).}

Бог Всеблаженний, Найщасливіший, Джерело найвищого щастя. \emph{«Цар царів і Господь володарів» (1 Тим. 6, 15).}

Бог — Дух, то значить не має тіла. Коли ж Святе Письмо приписує Йому серце, руки, очі, слух, то все те треба розуміти духовно. Наприклад: серце — Доброта Божа, очі — Всебачення Боже і т. п. Бог, як Дух, перебуває всюди, але місцем Його вічної слави є Небеса, як і сказав: \emph{{\color{red}«Небо — престол Мій»} (Іс. 66, 1)}.
 
Бог також перебуває Своєю Благодаттю і у святих храмах на землі, через те святі храми називаються домами Божими. Присутність Бога у храмах віруючі відчувають серцями у своїх благоговійних молитвах. Іноді Господь виявляє Свою присутність особливими ознаками. Сам Христос Спаситель сказав: \emph{{\color{red} «Де двоє або троє зберуться в ім`я Моє, там і Я посеред них»} (Мт. 18, 20).}
Дух Божий перебуває і в серцях тих людей, які люблять Бога, тому святий апостол Павло каже: \emph{«Хіба ви не знаєте, що тіло ваше — храм Духа Святого, Який живе у вас і Якого ви маєте від Бога?» (1 Кор, 6, 19).}

... в Єдиного Бога Отця ...

Бог — Єдиний Істотою, але потрійний Особами (Лицями): Отець, Син і Святий Дух.

Бог — Тройця Одноістотна, Незлитна і Нероздільна.

Бог Отець є Перша Особа (Іпостась) Пресвятої Тройці. Від Нього превічно народжується Бог Син і від Нього ж превічно сходить Дух Святий.

Сам же Бог Отець не народжується і не сходить від іншої Особи Пресвятої Тройці і тим одрізняється від інших Осіб Пресвятої Тройці.

... Вседержителя...

Бог є Вседержитель тому, що Він сам все содержить (утримує) Своєю силою і Своєю владою. \emph{«Зведіть очі ваші на небеса й погляньте: Хто створив їх? Хто виводить воїнство їх числом? Він усіх їх знає на ім`я і по множеству могутності Його в Нього ніщо не вибуває» (Іс. 40, 26).}

... Творця неба й землі, всього видимого й невидимого.

Бог Отець усе створив Словом (Сином) Своїм і впорядкував Духом Своїм. Ніхто інший Йому не помагав. \emph{«Ним створено все, що на небесах і що на землі, видиме й невидиме» (Кол. 1, 16)}. \emph{{\color{red} «Я створив землю й... чоловіка»} (Іс. 45, 12)}. \emph{{\color{red} «Я — Бог, Який усе створив один»} (Іс. 44, 24).}

\subsection{Що ми розуміємо під словом «небо»?}

Перше — світ ангельський (Іов 38, 7; Пс. 32, 6; Кол. 1, 6), а потім — небесні оселі, серед яких перебувають ангели і духи святих праведних Божих людей \emph{(Мт. 25, 34)}. Це ті оселі, про які сказав Господь до апостолів: \emph{{\color{red} «В домі Отця Мого осель багато»} (їв. 14, 2)}. Про них же говорить і апостол Павло, називаючи ті оселі «раєм»: \emph{«Знаю такого чоловіка... що був взятий у рай і чув невимовні слова, яких не можна людині висловити» (2 Кор. 12, 3-4)}.

\section{АНГЕЛЬСЬКИЙ СВІТ}

Ангели — це розумні безтілесні духи, створені Богом раніше видимого світу, як і каже пророк Давид: \emph{«Ти твориш ангелами Своїми духів...» (Пс. 103, 4)}. Слово «ангел» грецьке і означає вісник. Так ангели звуться тому, що Господь посилає їх сповіщати (вістити) людям Свою волю. Ангели створені раніше видимого світу. Це ми бачимо із слів, сказаних Богом праведному Іову: \emph{{\color{red} «Хто поклав наріжний камінь на ній (основу землі)? Коли були створені зорі, (тоді) великим голосом похвалили Мене всі ангели Мої»} (Іов 38, 6-7).}

Ангели обдаровані великою мудрістю й силою, так що найменший з ангелів вищий від найвищого чоловіка, яким є святий Іван Предтеча \emph{(Мт. 11, 11).} Тільки Пренепорочну Богоматір, Пречисту Діву Марію, Господь превозніс вище херувимів і серафимів.

Святі ангели є посередниками між Богом і людьми та послуговують нашому спасінню, як каже апостол Павло: \emph{«Всі вони — служебні духи, які посилаються на служіння тим, що мають унаслідувати спасіння» (Євр. 1, 14).}

Ангельське служіння для нашого спасіння нам відоме із Святого Письма (наприклад служіння святого архангела Гавриїла \emph{(Лк. 1, 26).}

Всі ангели створені Богом добрими. Всі вони однодушно прославляли Бога і раділи великою радістю, коли Він творив світ \emph{(Іов, 38, 7).}

Але один з найвищих ангелів, що звався Денниця (Найперший) позавидував славі Божій, забажав сам діяти, як Бог, незалежно від Бога. Він загордився, став противитися Богові, зненавидів Бога, став непокірний і злий, став говорити неправду на Бога, і тому набув назву «диявол», що значить обмовник, наклепник. Є така церковна думка, що цей ангел мав спочатку ім`я Сатанаель, що значить «перший Божий», бо на гебрейській мові Бог зветься Еллогім. Звідси старші ангели на кінці своїх імен мають «ель» (Бог), напр. Миха-ель, Гаври-ель (по-нашому Михаїл, Гавриїл). Коли ж той недобрий ангел став ворогом Богові, тобто став безбожником, то й ім`я Боже «ель» у нього відібрано і він став зватися лише «сатана».

Як дух сильної волі, він вплинув і на підкорених йому ангелів, обмовляв перед ними Бога. Вони повірили йому, зненавиділи Бога, стали противитися Йому, стали злими й неправдивими. Таким чином сатана перший согрішив \emph{(1 їв. З, 8)}, учинив беззаконство \emph{(1 їв. З, 4)}, порушив закон Єства, бо повстав проти Творця свого і за добро відплатив злом. Отже зло вніс у світ сатана. Зло від диявола, бо Бог зла не створив. Від диявола ж і смерть \emph{(Прем. 1, 13-14).}

Так і всі ті ангели, що пішли за ним, втратили святість, сяйво покинуло їх, обличчя їх потемніли. Вони стали злі і хулили Ім`я Боже.

Тоді всі ангели, на чолі з архистратигом Михаїлом, піднялися проти них за славу Бога свого і з криком Імені Божого: «Хто як Бог!» устремилися на злих духів і скинули їх з осель небесних у безодню. Демони утворили тоді свою державу з князем своїм сатаною і в страшній злобі та ненависті до Бога поклялись робити все наперекір Йому, руйнувати й поганити все, що будь-коли творитиме Бог. Всі ті злі ангели стали тепер злими демонами, ворогами Бога й людей, і всього того, що від Бога - правди, згоди, любові...

Перебуваючи увесь час у злобі, лжі, ненависті та лукавстві, вони робляться щодалі гіршими, тому що не можуть покаятися, і їх жде страшна погибіль \emph{(Мт. 25, 41).}

Свою руйнуючу діяльність демони виявляють у тому, що, як духи, впливають на людей і примушують їх діяти проти добра, проти любові й згоди між людьми. Демонське діяння завжди веде до руїни, до смерті. Тому й каже святий апостол: \emph{«Хто чинить гріх, той від диявола, бо спочатку диявол грішить» (1 їв. З, 8)}.

Всі ж інші ангели, що зосталися вірні Богові, перебуваючи завжди у славі Божій, у покорі та благоговінні перед Величчю Бога, навіки утвердилися в святості, у правді, в добрі, так що стали неприступними ні для якого зла.

\subsection{Чи багато є ангелів?}
їх \emph{«тьма тьмуща і тисячі тисяч» (Об. 5, 11)}, тобто безліч.
 
Про життя ангелів Святе Письмо нам мало каже. Але святий апостол Павло був Духом піднесений «до третього неба» (2 Кор. 12, 2), тобто в оселі Божі між ангелів, і там бачив і чув таке, що його неможливо висловити. Він не відкрив того людям з невідомих нам причин, можливо через нездібність грішними людьми те собі уявити. Але дещо сказав улюбленому ученикові своєму Діонисію Ареопагіту. І ось святий Діонисій свідчить, що ангели поділяються на дев`ять чинів у такому порядку:
\begin{enumerate}
    \item Серафими, херувими, престоли.
    \item Господства, сили, власті.
    \item Начала, архангели, ангели.
\end{enumerate}

Господь доручає ангелам різні служіння, як це бачимо із Святого Письма Старого й Нового Заповіту, їм доручено охороняти окремі народи, як архистратигу Михаїло-ві народ ізраїльський. Крім того ангели охоронителі даються й людям, як свідчив сам Ісус Христос, коли говорив про дітей: \emph{{\color{red} «Не зневажайте ні одного з малих цих, бо кажу вам, що ангели їх на небесах повсякчас бачать лице Отця Мого Небесного»} (Мт. 18, 10).}

Ангели завжди оточують на небесах престол Божий і невпинно прославляють Величність Божу \emph{(Об. 5, 11-12).}

Ангели Господні є друзі наші, вони завжди моляться за нас перед Богом \emph{(Об. 8, 3-4).}

\section{ТВОРЕННЯ ВИДИМОГО СВІТУ}

У словах Символа Віри: \emph{«Творця неба і землі»} під словом «земля» розуміється видимий всесвіт. Творення всесвіту тривало шість днів (очевидко це не були дні в нашому розумінні, а довгі періоди часу).
\subsection{День перший}
Пр початок створення всесвіту пророк Мойсей так повіствує: \emph{«В початку створив Господь небо і землю...» (про небо вже сказано вище). Земля ж була невидима і невпорядкована. Темрява оповивала її, і Дух Божий носився над водою» (Бут. 1, 3-5).}

Як бачимо, то була не наша планета-Земля. То був якийсь хаос газопаровидної речовини-матерії, з якої Господь створив всесвіт за шість днів творіння.
\begin{FlushRight}
    \emph{«І сказав Бог: нехай буде світло)! І сталося так. І бачив Бог, яке воно гарне, і відділив Бог світло від темряви. І назвав Бог світло днем, а темряву назвав ніччю».}
\end{FlushRight} 

Що то було за світло? Припускають, що Дух Святий зогрів ту матерію, і вона почала горіти й виділяти світло. Так думають тому, що зорі, які є часткою тієї горіючої матерії, й досі горять. Інші думають, що з того хаосу перше виділилася енергія світла і освітила небесні простори. Це був перший день творіння (очевидно, що не день, як період часу в нашому розумінні).

\subsection{День другий}

\begin{FlushRight}
    \emph{І сказав Бог: {\color{red} «Нехай буде твердь! І нехай буде розділ між «водою» і «водою».} І сталось так (Бут. 1, 6-7).}
\end{FlushRight}

З цього бачимо, що той початковий клубень хаосу з наказу Божого розірвався на окремі різні великі й малі частини. Ті частини зайняли визначене їм місце в небесних просторах і понеслися по орбітах своїх по закону установленого Богом тяжіння. Так утворив Господь оту дивну небесну механіку небесних світил у «день другий».

Під словом «твердь» треба розуміти небесні простори зо всіма світилами небесними — небозвід або видиме небо.

\subsection{День третій}

І сказав Бог: \emph{«{\color{red}Нехай збереться вода у зборища свої, і нехай явиться суша.} І сталось так...» (Бут. 1, 9)}. Як то зробилося, не знаємо. Припускаємо, що з наказу Божого планета Земля охолола, і пара, що обгортала її густими хмарами, охолола і зібралася океанами в місцях своїх. Інші ж речовини, охолонувши, укрили земну кулю земною корою.
\begin{FlushRight}
    \emph{... І сказав Бог: {\color{red} «Нехай вирощує земля билини, трав по роду їх так, щоб було в них насіння по роду їх»}... (Бут. 1, 11).}
\end{FlushRight}

\subsection{День четвертий}

І сказав Бог: «Нехай будуть світила на тверді небесній, щоб освітлювали землю і одділяли день від ночі та відзначували час днів і років»... і сталося так, і бачив Бог, як то гарно \emph{(Бут. 1, 14, 19).}

Події другого дня дають підставу думати, що сонце й зорі почали існувати від другого дня творіння, але може в іншому стані, ніж у четвертий день. Крім того, сама планета Земля була тоді вогняною, і сонце для неї не мало значення. У третій день земля була оповита хмарами, і світил небесних не було видно. Тільки в четвертий день, з наказу Божого всі небесні світила зайняли своє становище щодо землі з визначеними їм Богом функціями. Це — день четвертий.

\subsection{День п`ятий}
\begin{FlushRight}
    \emph{І сказав Бог: {\color{red} «Нехай виведе вода лазячих істот живих, і птахи нехай полетять понад землею у просторах небесних»}. І сталось так. І створив Бог риб великих і всяких лазячих, що їх вивела вода по роду їх, і всяких птахів пернатих по роду їх. І бачив Бог, яке все гарне. І благословив їх Бог і сказав: {\color{red} «Плодіться, розмножуйтеся, наповняйте води в морях; і птахи нехай розмножуються на землі»} (Бут. 1, 20-23). }
\end{FlushRight}

Це — п`ятий день творіння.

\subsection{День шостий}

\emph{І сказав Бог: {\color{red} «Нехай виведе земля істот живих по роду їх: скотів і гадів і звірів земних»}. І сталося так... І бачив Бог, які вони гарні. Це — день шостий. (Бут. 1, 24-25).}

Так створив Господь всесвіт і всю природу на землі. Нарешті, Господь створив перших людей і тим закінчив творити спочатку \emph{(Бут. 1, 26-31)}.

У сьомий день Господь «відпочив», тобто перестав творити спочатку, благословив сьомий день і освятив його для Себе \emph{(Бут. 2, 1-3)}.

Необхідно звернути увагу на те, що все живе з наказу Божого вивели вода й земля, а людину Бог створив інакше. Перед творінням людини Бог у Святій Тройці сказав: \emph{{\color{red} «Сотворімо чоловіка по образу нашому й по подобі»} (Бут. 1-26)}. І створив Бог першого чоловіка, тіло йому Він створив із землі (грязива землі), а потім удихнув у нього (створену Ним) душу розумну, вільну й безсмертну \emph{(Бут. 2, 7)}, і назвав його Адам, що значить земляний.

Адам був створений найкращим по красі з великим світлим розумом, і безгрішним, як ангел, тільки в тілі.

Розум свій Адам виявив у пізнаванні Бога та в синовній до Нього любові. Коли ж Господь привів до нього й показав йому всіх тварин, то Адам швидко розпізнав їх характери й якості й від того дав їм назви \emph{(Бут. 2, 19-20)}.
 
\emph{Тоді сказав Господь: {\color{red} «Недобре бути чоловікові одному; створімо йому помічника, відповідного йому»}. І навів Господь на чоловіка міцний сон, а коли він заснув, узяв одне ребро його і закрив те місце тілом. І створив Господь Бог із ребра, взятого у чоловіка, жону і привів її до чоловіка. Коли Адам прокинувся і побачив її, то сказав: «Це кість від костей моїх, і тіло від тіла мого. Вона буде зватися „жона", бо від чоловіка вона взята» (Бут. 2, 18-23).}

Як Адам, так і жона його Єва були обоє непорочні, як діти. Вони обоє були нагі, але не помічали того й не соромились \emph{(Бут. 2, 25)}. Перше подружжя Господь створив з одної крові \emph{(Діян. 17, 26)}, щоб могли породжувати дітей, подібних до себе.

Для їх життя Господь насадив рай, тобто сад. Дав їм на їжу всі плоди райські, також і плоди з Дерева життя, що росло посеред раю. Плоди ж Дерева життя давали їм безсмертя, завжди оновляли їх.

Але щоб укріпити волю їх і навчити слухняності, Господь заборонив їм їсти плоди від одного лише дерева, яке теж росло посеред раю і сказав так: \emph{{\color{red} «Від усіх дерев будете їсти плоди, а від того дерева не їжте, бо як тільки з`їсте, то зараз же помрете»} (Бут. 2, 16-17).}

Після того Бог благословив перших людей і сказав: \emph{{\color{red} «Плодіться, розмножуйтесь, наповняйте землю й володійте нею»} (Бут. 1, 28)}, тобто володійте над усіма творіннями, навіть над самою природою землі та щоб через творіння все більше пізнавали Бога і разом з Ним блаженствували.

Бог вдихнув у людину душу. Що таке душа? Душа життя є в кожній тварині, навіть у рослин. То — саме життя. Але те життя умирає, воно несвідоме. Людина ж має інакшу душу, ніж тварини. Тваринам життя з наказу Божого дали земля та вода, а душу людині дав Бог. Душа людини — це істота духовна (дух), самосвідома, розумна й безсмертна. Вона має волю й може чинити так, як хоче. Апостол Павло відзначає в людині трьохчастковість: дух (від Бога), душу (життя тіла) і тіло \emph{(1 Сол. 5, 23)}. Але ми душею називаємо оту духовну істоту, дану нам від Бога. Та душа безсмертна.

\subsection{Що таке смерть?}

То — руйнування тіла, розпад на початкові елементи (речовини). Дух же не складний, він простий (одиночний), тому не може розпастися, а значить і вмерти. Дух від Бога й до Бога повернеться. Так навчає нас Святе Письмо. Але ми самі в собі маємо доказ, що дух наш від Бога. Наприклад, ми розуміємо правду й прагнемо справедливості. Ми любимо красу, прагнемо до неї, скрізь її шукаємо: в кольорах, у звуках, у симетріях. Розуміємо добро, ласку, любов, радість, мораль. Дух наш прагне до доброго, високого, гарного. Все це не властиве тваринам. Чому? Тому, що їх з наказу Божого, створила земля, яка не могла їм дати тих властивостей, бо сама не має.

Людину ж безпосередньо створив Бог, який Сам є вічна Правда, Краса, Любов, Добро, і все те ми маємо від Бога, воно й веде нас до Бога.

\subsection{Боже піклування про людей}

У сьомий день Бог спочив від діл Своїх, тобто перестав творити з нічого.

Одначе Бог не залишив створеного Ним світу. Він ним управляє й піклується, особливо долею людини. Ця Божа турбота за світ зветься Божим Промислом або Божим піклуванням, тобто це безперестанне діяння всемогутності, премудрості й ласки Божої, якими Бог оберігає життя й сили творінь та все скеровує до доброї мети на користь їм, перешкоджає діянню зла й повертає його на добро. Про Боже піклування Господь так каже: \emph{{\color{red} «Погляньте на пташок небесних: вони не сіють, не жнуть і не збирають у засіки, а Отець Небесний годує їх, а чи ж ви не кращі від них?»} (Мт. 6, 26).}

Псалом 90 з особливою силою відзначає Боже піклування про людей.

\section{ДРУГИЙ ЧЛЕН СИМВОЛА ВІРИ}
\begin{FlushRight}
    І в Єдиного Господа Ісуса Христа, Сина Божого, Єдинородного, від Отця рожденого перше всіх віків, Світло від світла, Бога Істинного від Бога Істинного, рожденого, несотвореного, Єдиносущного з Отцем, через Котрого все сталося.
\end{FlushRight}

В цьому члені Символа Віри викладено вчення (догмат) про другу Особу Пресвятої Тройці — Сина Божого.

Сказано: в Єдиного Господа Ісуса Христа. Це значить, що Син Божий є Єдиний Господь, як і Бог Отець.
 
Син Божий по благовістю архангела прийняв на землі ім`я Ісус, що означає Спасипгель. Слово Христос значить Помазаник (Духом Святим), як і сказано через пророка Ісайю: \emph{{\color{red} «Дух Господа Бога на Мені, бо Господь помазав Мене благовіствувати убогим»} (Іс. 61, 1)}.

... Сина Божого, Єдинородного ...

Цими словами відзначено, що тільки Боже Слово — Ісус Христос є істинним Сином Божим, що предвічно народжується від Бога Отця з Його Єства й Божества. Святе Письмо про Сина Божого — Бога Слово так говорить: \emph{«В початку було Слово, і Слово було у Бога, і Слово було Бог» (їв. 1, 1).}
Бог Слово — Син Божий є Єдинородний, бо тільки Він один предвічно народжується з Єства Бога Отця, тому Він такий же Істинний Бог, як і Бог Отець. Він без усякого порівняння є превищий усіх ангелів і святих людей, які також звуться синами Божими, але не по єству, а по благодаті \emph{(їв. 1, 12).}

Сам Христос називає Себе Сином Божим Єдинородним: \emph{{\color{red} «Так Бог полюбив світ, що й Сина Свого Єдинородного віддав, щоб усякий, хто вірує в Нього, не загинув, а мав би життя вічне»} (їв. З, 16).}

... від Отця рожденого перше всіх віків ...

Цими словами відзначається превічне народження Сина Божого від Отця, чим Він відрізняється від інших Осіб Пресвятої Тройці, та що Син Божий такий же вічний Бог, як і Бог Отець.

Про Свою вічність Господь Ісус Христос у молитві до Отця так каже: \emph{{\color{red} «Отче, хочу, щоб де Я, там і вони (апостоли й вірні) були, щоб бачили Славу Мою, яку Я мав у Тебе раніше, ніж настав світ»} (їв. 17, 24).}

... Світло від Світла ...

Цими словами відзначається незбагненна таємниця народжування Сина Божого від Отця. Як світло народжується від світла, так і Син Божий народжується від Отця. Бо Бог є Світло \emph{(1 їв. 1, 5)}. Від Нього народжується Син Божий, Який є те ж Світло, невідділиме від Божественного Єства Бога Отця.
 
... Бога Істинного від Бога Істинного...
Цим відзначається, що Син Божий є такий же Істинний Бог, як і Бог Отець. Каже святий апостол Іван Богослов:
\begin{FlushRight}
    \emph{«Знаємо, що Син Божий на землю прийшов і дав світло й розуміння, щоб ми пізнали Бога Істинного і щоб перебували в Істинному Сині Його Ісусі Христі, Він є Істинний Бог і життя вічне» (1 їв. 5, 20).}
\end{FlushRight}

... рожденого, несотвореного...

Ці слова додані проти учення Арія, який учив, що ніби Син Божий є створений, а не рождений, не рівновічний з Отцем і не рівний Йому.

... Єдиносущного з Отцем ...

Цими словами показано, що Син Божий є того ж самого Божественного Єства, що й Бог Отець.

Сам Бог через пророка до Сина Свого каже: \emph{{\color{red} «Із Себе Самого раніше світу я породив Тебе»} (Пс. 109, 3)}.
Сам Христос про Себе й Отця так свідчить: \emph{{\color{red} «Я й Отець — Одно»} (їв. 10, ЗО)}. \emph{{\color{red} «Хто бачив Мене, той бачив Отця»} (їв. 14, 9)}.

... через Якого все сталося.

Ці слова показують, що Бог Отець усе створив Сином Своїм, Який є Предвічна Божа Премудрість, Вічне Боже Слово.

Святий Іван Богослов так каже: \emph{«Все через Нього сталося, і без Нього ніщо не сталося, що сталося» (їв. 1, 3).}

Сам Син Божий про Себе через премудрого так каже: \emph{{\color{red} «Господь поставив Мене на початок дій Своїх... раніше всіх створінь споконвіку. Від вічності Я помазаний раніше, ніж з`явилася земля. Я народився, коли ще не було безодні. Коли Він готував небеса. Я був там, коли Він проводив кругову лінію по безодні»} (Прит. 8, 22-27).}
\begin{FlushRight}
   \emph{«Словом Господнім небеса створені» (Пс. 32, 6).} 
\end{FlushRight}

Це проголошує і апостол Павло:

\begin{FlushRight}
    \emph{«Ним створено все, що на небесах і що на землі, видиме й невидиме, чи то престоли (це ангельські чини), чи то господства, чи начала, чи власті, все Ним і для Нього створене — Він є раніше всього, все Ним стоїть» (Кол. 1, 16-17).} 
\end{FlushRight}

\section{ТРЕТІЙ ЧЛЕН СИМВОЛА ВІРИ}
\begin{FlushRight}
    «Що заради нас людей і для нашого спасіння зійшов з небес і тіло прийняв від Духа Святого і Марії Діви і став чоловіком».
\end{FlushRight}

В цьому члені говориться про зшестя з небес Сина Божого, про Його зачаття від Духа Святого і про втілення та вчоловічення.

\subsection{Як же міг сходити, з небес Син Божий, коли Він, як Бог, перебуває всюди?}

То правда, що Син Божий як Бог завжди перебував і перебуває на небі й на землі, але Єством Він перебуває в Отці, як Особа Пресвятої Тройці. Його то й послав тепер Отець, як Свого Сина, на землю для спасіння людей. Раніше, як Дух, Він був невидимий, тепер же, одягшися в наше тіло, Він став видимий, як і каже апостол: \emph{«Велична побожності тайна: Бог з`явився в тілі...» (1 Тим. З, 16).}
Про зшестя Своє Христос каже Никодимові: \emph{{\color{red} «Ніхто не виходив на небеса, тільки Той, що з небес зійшов — Син Чоловічеський, сущий на небесах»} (їв. З, 13).}

Син Божий зійшов на землю, як сказано, ради нас людей і ради нашого спасіння, щоб спасти нас від гріха, прокляття й смерті.

\subsection{Що таке гріх?}

То — беззаконство \emph{(1 їв. З, 4)}, порушення закону. Гріх — то діло диявола: \emph{{\color{red} «Хто чинить гріх, той є від диявола, бо спочатку диявол згрішив»} (1 їв. З, 8).}

Отже диявол навчив людей грішити й далі навчає. Про це Святе Писання так оповідає:

Адам і Єва жили в раю, як ангели, і блаженствували. Сатана їм позаздрив і вирішив погубити їх. Він увійшов у змія і почав клеветати на Бога, що ніби Бог тому заборонив їм їсти плоди від Дерева пізнання добра і зла, щоб вони не стали такими, як Бог. Єва, до якої диявол зі змія говорив, повірила йому, а Богові перестала вірити, перестала Його любити, зрадила Бога, загордилася і, на зло Богові, наперекір Його заповіді, стала їсти заборонені плоди, щоб самій стати, як Бог. Потім пішла і Адама намовила зробити те ж саме. Він їй повірив і також став їсти заборонені плоди. Так вони обоє, по намові сатани, згрішили, переступили Божу заповідь, зрадили Бога й схилилися до ворога Божого, до сатани.

І ось з ними сталася велика зміна: їм стало соромно й страшно. Вони побачили, що вони голі, й стали соромитися. Тепер вони стали боятися Бога. Природа їхня зіпсувалася і підпала впливам зла й тління. Раніше вони їли плоди від Дерева життя, і тіло їхнє не старіло. Тепер же, бувши заражені гріхом гордині, вони вже не могли доторкнутися до Дерева життя, і над їх тілом запанувала смерть, як над тілом тварин, а душі їх, заражені гріховністю, стали чужі Богові.

Гріх свій перед Богом вони вчинили свідомо, тому провина їхня перед Богом була повною. Вони свідомо стали ворогами Богові, а тому вже не могли лишатися в раю. їхній гріх так образив Бога, що вже ніякі заслуги людські не могли задовольнити Боже Правосуддя. Тому Господь прокляв спокусника на гріх змія-диявола. Але людей Господь пожалів і не знищив їх, а визначив їм кару. Адамові сказав: \emph{{\color{red} «Проклята земля в ділах твоїх: колючки та бур`ян буде вона вирощувати тобі замість хліба. А ти в поті чола свого будеш добувати собі хліб, аж доки не повернешся в землю, з якої ти взятий, бо ти земля, і в землю повернешся»} (Бут. З, 17-19).}

А Єві сказав: \emph{{\color{red} «Ти будеш залежати від чоловіка твого, матимеш багато скорбот і в муках будеш породжувати дітей»} (Бут. З, 16)}. Одначе, проклинаючи сатану, Господь тут же дав людям обітування Спасителя, Який поверне людям утрачене блаженство \emph{(Бут. З, 15)}. Після того Господь зробив Адамові й Єві «одежини» (опоясання) із шкіри тварин і вислав їх з раю, щоб вони тепер не доторкнулися до Дерева життя, не з`їли його плоду та не жили б отакими скверними вічно \emph{(Бут. З, 21-23)}.

Так осквернилася гріхом природа перших людей. Так повисло на них Боже прокляття й засудження на смерть. Все те перейшло й на їх дітей, і на внуків, і на все потомство, бо від грішних родителів стали родитися зіпсовані гріхом діти, як з поганого джерела витікає погана вода, каже апостол:
\begin{FlushRight}
    \emph{«Через одного чоловіка гріх увійшов у світ і з гріхом смерть, так смерть увійшла у всіх людей, бо в ньому (першому чоловікові) всі згрішили» (Рим. 5, 12).}
\end{FlushRight}

Обітування Спасителя Господь дав у таких словах (до сатани): \emph{{\color{red} «Ворожнечу покладу між тобою і жоною, між нащадками твоїми й нащадком її. Ти будеш жалити Його в п`яту, а Він зітре тобі голову»} (Бут. З, 15)}.

І ось: «Коли прийшов призначений час, Бог послав Сина Свого Єдинородного, Який народився від жони, підкорився законові, щоб викупити підзаконних і повернути їм знову синівство Боже» (Гал. 4, 4-5).

Тому й говориться: «... заради нас людей і для нашого спасіння зійшов з небес... і тіло прийняв від Духа Святого і Марії Діви і став чоловіком».
Господь не тільки дав обітування Спасителя першим людям. Він багато разів повторив його на протязі історії Старого Заповіту, щоб люди не втратили в нього віри.

Так він повторив його Авраамові (Бут. 22, 18), Ісаа-кові (Бут. 26, 4), Якову (Бут, 28, 14), Давиду і через Давида (див. псалми 2, 6-9; 44, 3-9; 71, 6-17; увесь псалом 109 -- взагалі, більшість псалмів Давидових повні пророкувань про Спасителя). Пророк Ісайя передсповістив, що Син Божий народиться від Діви (Іс. 7, 14), про Його смирення (розділ 42), про страждання (розділ 53), і ін. Пророк Михей — що Він народиться у Вифлеємі (Мих. 5,1-2). Пророк Захарія — про вхід Спасителя в Єрусалим (Зах. 9, 9), про затьмарення сонця в час смерті Господа Ісуса Христа (Зах. 14, 6-7). Пророк Осія — про воскресіння Христове і з Ним воскресіння мертвих (Ос. 6, 1-2). Господь же через ангела відкрив пророкові Даниїлу, що Спаситель явиться через сімдесят сідмин після початку відбудови Єрусалиму (Дан. 9, 24-27).

Син Божий зійшов на землю, як «роса на руно» (Пс. 71, 6), тобто непомітно, невидимо.

Перед самим зшестям Його посланий був архангел Гавриїл від Бога в місто Галілейське, що звалося Назарет, до Діви, зарученої мужеві на ім`я Йосиф, з дому Давидового, ім`я Діві Марія. І, ввійшовши до неї, сказав: «Радуйся, Благодатна, Господь з Тобою, Благословенна Ти між жонами». Вона ж, побачивши ангела, схвилювалася від слів Його й думала: що за таке дивне привітання?

Але ангел сказав: «Не бійся, Маріє, Ти знайшла благодать у Бога. Ось зачнеш в утробі й породиш Сина і даси Йому ім`я Ісус. Він буде Великий і назветься Сином Вишнього. І дасть йому Господь престол Давида Отця Його, і буде царювати над домом Якова повіки, і Царству Його не буде кінця».
 
Сказала Марія: «Як же те може статися, коли Я мужа не знаю?»

Ангел відповів їй; «Дух Святий найде на Тебе, і Сила Вишнього осінить Тебе, тому й Те, що родиться, Святе і назоветься Сином Божим».

Марія сказала: «Я раба Господня, нехай буде зо мною по слову твоєму».

І відійшов від неї ангел (Лк. 1, 26-38).

Так сталося Благовіщення Святій Діві, і разом із словами архангела, Дух Святий вселився в Неї. Вона зачала в утробі Сина Божого. Тому й сказано: тіло прийняв від Духа Святого і Марії Діви (Іс. 7, 14).

Сказано «Діви», а не «жони» тому, що Вона не знала мужа, зачала Сина по нашестю Духа Святого, породила безболізно і після родин зісталася такою, як і була, Дівою. Свята Церква правдиво називає її «Завжди Діва» або «Вседіва», бо вона до родин була непорочною й чистою Дівою і тілом, і душею, так віддавши Себе Богові, що ніколи не помишляла про земного мужа. Такою ж непорочною й чистою Дівою лишилася й після родин.

В Неї вселилося Слово Боже і «стало тілом» (їв. 1, 14), тобто прийняло на Себе тіло й кров від Пресвятої Діви Марії, або, як у церковних піснях говориться; «Зіткалося з святого тіла й крові її і стало Дитятком, як людина. Одначе народилося дивно й безболізно, не так, як усі люди. Тому й Свята Діва ні в чому не змінилася й зосталася після родин такою ж, якою була до родин.

... і став чоловіком.

Цими словами відзначається, що Син Божий — Слово Боже — прийняв від Діви Марії не тільки тіло, а й душу людську, тільки без прародительського Адамового гріха, бо зачався не від гріховного мужа, а від Бога. Син Божий став правдиво повною людиною з тілом і душею, але не перестав бути й Істинним Богом, цебто Божество Його не змінилося й не змішалося з людством. Він є Богочоловік (Боголюдина). А Свята Діва Марія, хоч породила Сина Божого не по Божому Єству, а по людському, одначе через Неї пройшов Бог і в образі людини народився на землі, Тому Вона є істинно й правдиво Богородиця і Мати Божа, бо в Ній перебувало Божество Єством Своїм.
 
Цим Господь поставив Пресвяту Діву Марію превище всіх людей і всіх ангелів, як і співається в церковній пісні: «Чеснішу від херувимів і незрівняно славнішу від серафимів...»

\section{ЧЕТВЕРТИЙ ЧЛЕН СИМВОЛА ВІРИ}

І розп`ятий був за нас при Понтії Пілаті, і страждав, і був пожований.

\subsection{Як можна розп`ясти Бога?}

Ісус Христос, Син Божий, виростав на землі як і всі люди. Божество Його не проявлялося дуже помітно. Він до часу Свого не творив чудес, хоч благодать Божа перебувала на Ньому невідлучно (Лк. 2, 40). Євангелія споминають лише один випадок, коли та Божа благодать проявилася на Ньому особливо ясно.

Коли Ісусу Христу було 12 років і Він з Матір`ю Своєю та праведним Йосифом, як звичай велів, був на свята в Єрусалимі в храмі, то давав такі відповіді й запитання книжникам та законнникам, що ті дуже дивувалися й казали: «Звідкіля така мудрість у цієї Дитини?» (Лк. 2, 42-52).

Але Христос завжди знав Себе, як Сина Божого й Своє призначення. Це видно із слів Його, які Він сказав Матері Своїй, що шукала Його: «Нащо вам було шукати Мене? Хіба ви не знали, що Мені належить бути в тому (місці), яке належить Отцеві Моєму?» (Лк. 2, 49).

Коли ж Він досяг повноліття — 30 років, прийняв хрещення і тоді виступив на проповідь одкрито, як Месія — Син Божий. Проповідь Його тривала три з половиною роки, як передповістив ангел через пророка Даниїла (Дан. 9, 26). Зціляв недужих, воскрешав з мертвих, виганяв бісів з людей, навчав людей притчами і просто. Сталося все так, як про Нього передповістили пророки.

Народ пізнав у Ньому небесного Посланця — Месію-Христа. Але юдейський синедріон (найвища керівна установа), первосвященники й законники зненавиділи Його за те, що не йшов поруч з ними в беззаконствах і докоряв їм за їх лицемірство. Чудеса Його вони стали приписувати нечистій силі (їв. 8, 52), Його ж Самого за те, що звав Себе Сином Божим, звинуватили в богохульстві, а перед прокуратором Понтієм Пілатом, римським управителем, оклеветали як бунтаря, що підбурює народ проти імператора, називаючи Себе Царем Юдейським. Пілат бачив, що юдеї клеветали на
 
Христа, і сам у Ньому не знаходив ніякої вини, але, бувши легкодухим, побоявся відпустити Його і, щоб заспокоїти народ, віддав Його на знущання воякам, а потім засудив на смерть.

Суд над Ісусом Христом був цілком беззаконний, бо Він провини не мав (Іс. 53, 9), і Його судили на показах лжесвідків (Мт. 26, 59). Беззаконний був суд і по самій формі. По юдейських законах для того, щоб засудити людину на смерть, треба було судити 9 днів в такому порядку: один день слухали справу звинуваченого, а два дні ходили сурмачі по місту й сурмили, може знайдеться хтось, хто скаже щось на захист звинуваченого. І так до трьох разів по три дні. Тільки тоді, коли вже ніхто не ставав на захист звинуваченого, віддавали його на смерть. У найгостріших випадках суд тягся три дні. Христа ж засудили на смерть менше, ніж за один день (див. «Життя Ісуса Христа» Барсова).

Так і пророк каже: «У приниженні Його на суд Його взято» (Іс. 53, 8).

Суд не мав права мучити підсудного. Христа ж Господа в синедріоні нещадно били й мучили (Мт. 26, 67). Пілат не мав права засуджувати Христа на хресну смерть через недоведеність Його вини (Лк. 23, 14г 16), але засудив Його, щоб догодити юдеям (Мр. 15, 15), ще й перед тим велів мучити.

Смерть на хресті вважалася за найганебнішу. На неї засуджували тільки безправних рабів за найтяжчі провини. Ще в Старому Заповіті сказано: «Проклятий всякий, хто висить на дереві» (Втор. 21, 23). Цю муку й смерть присудили Христові.

\subsection{Чи не міг би, Христос, як Бог, знищити ворогів Своїх і не страждати?}

Міг би. Але Він на муки пішов добровільно.

Сам Він сказав: «Я душу (життя) Мою віддаю, щоб знову прийняти її. Ніхто не відбирає її в Мене, Я Сам віддаю її. Я маю владу віддати її, і маю владу знову взяти її. Цю заповідь я одержав від Отця Мого» (їв. 10, 17-18).

Ще давно Син Божий сказав до Отця: «Жертв і приношень Ти не захотів, а звершив тіло Мені... тоді Я сказав: ось іду... виконати волю Твою, Боже. Закон Твій у серці Моїм (Пс. 39, 7-9).
 
Христос добровільно підкорив Себе волі Отця Свого, як каже апостол: «смирив себе, слухняним був аж до смерті хресної» (Фил. 2, 8). Бо Отець — Бог так «полюбив світ, що й Сина Свого Єдинородного віддав, щоб усякий, хто вірує в Нього, не загинув, а мав би життя вічне» (їв. З, 16).

Отже Христос не тільки не відмовлявся від страждань, а навмисно Сам не хотів полегшити собі їх. Коли перед розп`яттям Йому дали перекислого оцту з жовчю, від якого терпло тіло й менше відчувався біль, Він не захотів його пити (Мт. 27, 34).

... і страждав ...

Це сказано для того, щоб не подумав хто, що Христос як Бог тільки показував вид, що ніби страждає. Так думали єретики «докети». Ні, страждання Його були такі тяжкі, що Він умлівав і просив у Бога помочі словами:

«Боже Мій, Боже Мій, нащо Ти Мене покинув?» (Мт. 27, 46).

Смерть Його сталася від страшних мук і виснаження. Подія то була така страшна, що навіть сама природа здригнулася. Коли Христос востаннє скрикнув, зітхнув і вмер, земля потряслася, каміння розпалося, мертві ожили, завіса церковна роздерлася надвоє зверху донизу (Мт. 27, 50-53).

Коли розп`яли Христа, то від полудня почався якийсь сумерк по всій землі, бо сонце потьмарилося (Мт. 27, 45). Дехто з невірних хоче вбачати в тому звичайне затемнення сонця. Але то неправда: затемнення сонця ніколи не триває три години. Крім того, те затьмарення сонця передповіщене пророком: «І буде в той день — не стане світла, світила затьмаряться... день той буде єдиний (безподібний), відомий тільки Господу: ні день, ні ніч, тільки під вечір явиться світло» (Зах. 14, 6-7).

\subsection{Чи не міг би, Господь спасти світ без страждань Христових?}

Міг би. Але тоді порушилося б правосуддя Боже. Закон моральний діє так же необхідно, як і фізичний. Коли нас хтось образив, ми вимагаємо відплати, кари чи пробачення, або прощаємо, передавши справу на суд Божий. Без цього ми не заспокоюємося, хоч фізичної рани нам не зроблено.
 
Отже, й правосуддя Боже було порушене людьми. Люди за любов і ласку Божу відплатили Йому не тільки чорною невдячністю: вони зрадили Бога, не повірили Богові, а повірили сатані — страшному ворогові Бога — й стали на його бік проти Бога, осквернили своє чисте єство гордощами та злобою, затьмарили в собі образ Божий. Падіння страшенне, і образа правди Божої така велика, що тільки Сам Бог зміг її урівноважити. І ось, Господь по милосердю Своєму вибирає такий засіб, щоб разом і правосуддя Боже задовольнити, і людей спасти.

Він посилає Сина Свого Єдинородного на землю, щоб узяв на Себе провину всього світу. І Син Божий підкоряється волі Отця, бере на Себе вину всіх, терпить страшні муки, проливає Свою святу кров і нею змиває з людей ту вину. Він віддає Себе на смерть і за Себе викупляє нас від гріха, прокляття й смерті. Умирає на хресті, сходить Духом в пекло, знесилює сатану, розриває кайдани смерті і виводить усіх, починаючи від Адама. А тим, що живі або ще мають народитися, відкриває новий шлях до неба, шлях Євангельський, давши засоби спасіння — Свої Божественні Тайни.

Так і апостол свідчить:

«Ми маємо викуплення кров`ю Його, і прощення гріхів по багатству благості Його» (Еф. 1, 7).

«Бог... нас, мертвих гріхами, оживив з Христом, воскресив із Ним і посадив на небесах у Христі Ісусі» (Еф. 2, 5-6).

«Христос викупив нас з-під прокляття закону, взявши на Себе прокляття» (Гал. З, 13).

«Як в Адамі всі вмерли, так у Христі всі ожили» (1 Кор. 15, 22).

... і був похований.

Ці слова додано для того, щоб не подумав хто, що Ісуса Христа зняли з хреста ще живого.

Він дійсно вмер. Для того, щоб не було сумніву в Його смерті, вояк навіть проколов Йому списом ребра так глибоко, що з рани полилися кров і вода (їв. 19, 34). Його мертвого Йосиф з Никодимом зняли з хреста, оповили покривалами й пахощами, по тодішньому звичаю, поклали в новому гробі, в якому ніхто не був покладений, привалили до дверей тяжкий камінь. А первосвященники запечатали гріб і поставили сторожу, щоб ніхто не виніс тіла Ісусового (Мт. 27, 60-66; Мр. 15, 46; Лк. 23, 53; їв. 19, 40-41).

\subsection{Як же страждання і смерть Христові визволяють нас від гріха, прокляття і смерті?}

Через глибоку віру. Щоб скористати з них, треба від усього серця вірити у Христа як Спасителя світу і через віру й таїнства стати членом зібраної Ним Його кров`ю Церкви — громади Христової. Треба дотримуватися всього, що Він заповідав нам. Іншими словами, треба жити у Христі через віру й таїнства, і не по прихотях свого серця, а по заповідях Христових.

«Хто Мені служить, нехай іде слідом Моїм; де Я, там і слуга Мій буде» (їв. 12, 26).

«Хто в Христі, ті розп`яли тіло (своє) з пристрастями та прихотями» (Гал. 5, 24), тобто відмовилися від гріховних бажань своїх і всеціло передали себе Богові.

Син Божий звершив наше спасіння на землі троїстим чином: як Бог, як Цар, як Первосвященник.

Зійшов на землю, як Бог — Син Божий, таким перебував, і таким до Отця повернувся.

Як Бог, Він творив чудеса, воскрешав мертвих, наказував природі, і як Бог, зійшов до пекла, знесилив смерть, знесилив губителя-сатану, вивів з пекла Адама й увесь рід його, а потім Сам воскрес із мертвих силою Божества.

Як Цар і Богочоловік, Христос Сам постраждав людським (а не Божественним) єством і викупив нас від гріха, прокляття й смерті Своєю безцінною Кров`ю. Заснував Своє Царство — Церкву Свою — і дав їй закони Правди й Істини.

Як Первосвященник Вічний, по чину Мелхиседека (Євр. 5, 5-10), Він звершив страшну жертву — Сам Себе в жертву приніс на страшному жертівнику — на хресті. Його молитва записана в Євангелії від Івана. То первосвященницька молитва над жертвою — Самим Собою, молитва освячення, як і казав: «... за них (апостолів і людей) Я освячую Себе, щоб і вони були освячені Істиною» (їв. 17, 19-20).

Голгофська жертва — то незбагненна таємниця Божої любові до людей. Перед цією страшною таємницею усі народи в усі віки в німому благоговінні схиляються, і те благоговіння ніколи не зменшиться, бо людство ніколи не вичерпає глибини її, як і каже святий апостол, що перед «ім`ям Ісуса преклониться всяке коліно небесних, і земних, і підземних, і всякий народ визнаватиме, що Господь Ісус Христос (є) слава Бога Отця» (Фил. 2, 10-11).

Йому слава на віки вічні. Амінь.

\section{П`ЯТИЙ ЧЛЕН СИМВОЛ А ВІРИ} 

І воскрес на третій день, як було написано.

В цьому члені Символа Віри сказано про воскресіння Господа нашого Ісуса Христа.

Смерть Його сталася в п`ятницю о 9-й годині (за нашим обрахунком часу це була 3-я година після полудня). В суботу тіло Господнє перебувало в гробі, а душею (людською) й Божеством Христос зійшов до пекла і звідти вивів душі померлих, як свідчить апостол Петро:

«Щоб привести нас до Бога, був умертвлений по тілу, ожив духом, котрим (духом) зійшов до темниці (пекла) і духам (померлих), що там були, проповідав (воскресив)» (1 Петр. З, 18-19).

Яскраво змальовує пасхальна пісня стан душі Його в час смерті:

«У гробі тілом, у пеклі душею, як Бог, в раю з розбійником і на престолі був єси, Христе, з Отцем і Духом, все наповняючи, як Необмежений».
Воскрес же Господь на третій день.

В перший день тижня дуже рано, на світанку здригнулася земля. Ангел Господній, зійшовши з небес, одвалив камінь від дверей гробу й сів на ньому. Вид його був, як блискавка, а одежа біла, як сніг. Вояки, що стерегли гріб, від страху попадали як мертві, а потім, отямившись, розбіглися (Мт. 28, 1-3).

Як сталося саме воскресіння, євангелисти не повіствують. Може тому, що воно сталося без свідків, а може й тому, що немає слів, щоб його передати. То невимовне й незбагненне чудо з чудес: як Господь устав тілом з гробу, не порушивши печаті?!

Церковні пісні оповідають, що першу звістку про воскресіння одержала Пренепорочна Мати Господня: «Ангел привітав Благодатну: Діво Чистая, радуйся, твій Син воскрес!»

Свята Діва не відходила від гробу Христового, як чуємо в каноні «Плач Богоматері» у Велику П`ятницю:
 
«Не відійду від гробу, Сину Мій...»

Пречиста Діва Марія бачила сильне світло з гробу в момент воскресіння. Жінки бачили велике сяйво ангела. Ангел сказав до них: «Не бійтеся ви, знаю, що шукаєте Ісуса розп`ятого. Нема Його тут, Він воскрес. Ось місце, де лежав Господь. Біжіть скоріше і скажіть ученикам Його, що Він воскрес із мертвих і буде ждати вас у Галілеї, там Його побачите. Це я сказав вам». Як вони йшли, ось Сам Ісус зустрів їх і каже їм: «Радуйтеся». Вони ж упали до ніг Його і поклонилися Йому (Мт. 28, 5-7, 9).

А вояки побігли й розказали про все первосвященникам. Тоді ті дали їм грошей і наказали їм розпускати чутку, що ніби ученики вкрали тіло Христове, коли вони спали (Мт. 28, 11-15).

Після воскресіння Господь явився окремо Марії Маг-далині (їв. 20. 11-18), апостолу Петрові (1 Кор. 5, 12), апостолам Луці і Клеопі (Лк. 24, 13-34), жінкам мироносицям (Мт. 28, 1-8), всім апостолам (Лк. 24, 36-53), більш ніж 500 вірних на галілейській горі (1 Кор. 15, 6).

Господь являвся їм протягом 40 днів, з ними їв і навчав їх про Царство Боже. В сороковий день востаннє з`явився їм, благословив їх і вознісся на небо (Діян. 1, 4-10).

... як було написано.

Господь воскрес так, як написано у пророків:

«Він (Бог) поранив нас, Він і зцілить нас... оживить нас через два дні, на третій день воскресить нас і будемо жити перед лицем Його» (Ос. 6, 1-2).

«Оживуть мертві Твої і встануть мертві тіла» (Іс. 26, 19).

Але ті пророцтва є прикровенні. Яскравий прообраз воскресіння Христового Господь показав на пророкові Іоні. На нього й Сам Христос вказує:

«Як Іона пробув в утробі китовій три дні, так і Христос у гробі» (Мт. 12, 40; кн. пророка Іони).

Сам Господь Ісус Христос чудо з Іоною називає прообразом Свого воскресіння.

\subsection{Яке значення має воскресіння Христове?}

Воскресіння Христове є завершальним актом в ділі нашого спасіння.
 
Коли б Христос не воскрес, то Його наука і Його чудеса й страждання не мали б тієї ціни, яку мають тепер. Його б розцінювали тільки як великого учителя й мученика за ідею, не більше. Самий же факт воскресіння Христа має вирішальне значення для нашої віри. Апостол Павло каже:
«Якщо Христос не воскрес, то суєтна віра наша, бо ми ще під гріхом» (1 Кор. 15, 17).

Воскресіння — то найяскравіший доказ Божественної місії Ісуса Христа. Своїм воскресінням Христос, наче печаттю, затвердив істинність усього Свого Вчення. Коли б Христос не воскрес, то смерть і пекло ще б і досі панували над родом людським.

\subsection{Що ж дало нам воскресіння Христове?}

Христос воскрес із мертвих і став першим між мертвими (1 Кор. 15, 20). Він разом з Собою воскресив усіх померлих від віку і всім життя дарував.
Як каже святий Іван Золотоустий:

Воскрес Христос, і зруйновано смерть. Воскрес Христос, і впали демони. Воскрес Христос, і радуються ангели. Воскрес Христос, і життя живе. Воскрес Христос, і спустошено гроби... Христос Своїм Воскресінням потвердив істину, що всі люди воскреснуть в останній день.

«Як в Адамі всі вмерли, так у Христі всі оживуть, кожний у своєму порядку. Перший Христос, а потім усі Христові в пришестя Його», — каже апостол (1 Кор. 15, 22-23).

«Нині Христос устав з мертвих, ставши первенцем з мертвих» (1 Кор. 15, 20).

«Христос один раз за гріхи наші постраждав. Праведник за неправедних, щоб привести нас до Бога, був умертвлений по тілу, ожив духом, котрим (духом) зійшов до темниці (пекла) і духам (померлих), що там були, проповідав (воскресив) (1 Петра 3, 18-19).

\section{ШОСТИЙ ЧЛЕН СИМВОЛА ВІРИ}

І вознісся на небеса і сидить праворуч Отця.

В цьому члені Символа Віри говориться про вознесіння Господнє і сидіння праворуч Бога.
 
Христос Господь після воскресіння Свого 40 днів являвся ученикам Своїм, їв з ними і навчав про Царство Боже (Діян 1, 3). Заповів їм іти навчати й хрестити всі народи (Мт. 28, 19-20), а поки що наказав не відлучатися з Єрусалиму і ждати обітованого від Отця Святого Духа (Діян. 1, 4).

Після того вивів їх до Віфанії на гору Єлеон (Оливна гора), підняв руки і благословив їх. І сталося, коли благословляв їх, то відділився від них, вознісся на небо і сів праворуч Бога Отця (Лк. 24, 50-52; Діян. 1, 9).

Так завершив місію нашого спасіння Син Божий і знову повернувся до Отця. Але тепер він повернувся до Отця в нашому тілі, яке прийняв від Пресвятої Діви Марії, в якому страждав і в якому воскрес з мертвих. Одначе у воскресінні тіло Його стало інакшим, ніж було до воскресіння. Воно одухотворилося так, що стало не речовинним. Тому Христос вийшов з гробу, не порушивши печаті: з цим тілом Він входив до апостолів, як двері були замкнені (їв. 20, 19).

... і сидить праворуч Отця.

Ці слова показують, що Син Божий знову одержав від Отця рівну з Ним честь і славу (1 Петр. З, 22).

Христос вознісся до Отця разом із Своєю людською природою і наше людське єство в Собі посадив поруч з Отцем. Цим Він підніс нас на недосяжну височінь і показав нам, де наша правдива Батьківщина.

«Бог, багатий в милості, по Своїй великій любові нас, мертвих гріхами, оживив з Христом... воскресив із Ним, і посадив на небесах у Христі Ісусі» (Еф. 2, 4г6). «Той, що зійшов (з неба), Той і вийшов превище всіх небес, щоб наповнити (Собою) все (Еф. 4, 10). «Такого маємо Первосвященника, Який сів праворуч Престолу величності на небесах» (Євр. 8, 1). Так свідчить нам Святе Письмо.

Господь наш Ісус Христос вознісся на небо тільки Своєю людською природою, а Божеством він ніколи не відділявся від Отця і завжди перебував на небі.

\section{СЬОМИЙ ЧЛЕН СИМВОЛА ВІРИ}

І зпову прийде зі славою судити живих І мертвих, і Царству Його не буде кінця.

В цьому члені Символа Віри говориться про друге пришестя Господа нашого Ісуса Христа, про останній
 
\subsection{Страшний Суд і про вічне нескінченне Царство Христове}

Віра в те, що Господь звершить Свій останній суд над людьми, перебуває в людях з самого початку, починаючи від Адама. Ще Єнох, сьомий від Адама, передсповіщав, кажучи: «Ось іде Господь з безліччю Своїх ангелів учинити суд над усіма і виявити між ними всіх нечестивців» (Юд. 1, 14-15).

Та ж віра в кінець світу й останній суд Божий проходить через усю історію людства. Так пророк Давид про кінець світу каже:

«В початку Ти, Господи, землю заснував, і діло рук Твоїх є небеса. Все те погине, Ти ж перебуватимеш. Все постаріє, і Ти, мов одіж, змотаєш його і (воно) зміниться. А Ти все Той же, і роки Твої не зменшаться» (Пс. 101, 26-28).

А про суд він каже: Господь «іде судити землю, судити вселенну правдою і людей правотою» (Пс. 97, 9).

Все створене має початок і кінець; все матеріальне змінюється й старіє. Так і світ: він створений, як каже премудрий, мірою, числом і вагою (Прем. 11, 21), тобто на час.

Останній же суд Божий є істотною необхідністю Правди Божої.

Бог є Вічна Правда, Чистота й Святість. Бог не створив зла і наслідку його — смерті (Прем. 1, 13). Все те вніс у світ сатана своїм злобним діянням проти Бога. Цей світ, як каже святий Іван Богослов, «у злі лежить» (1 їв. 5, 19). Тому на світі цьому порушена правда Божа: одні живуть неправдою й гнобительством і насолоджуються життям, а інші цілий вік проживають у нужді й стражданнях. Після кінця цього світу має настати вічне Царство Правди, де вже не може бути ніякої кривди, бо там «буде Бог все у всьому» (1 Кор. 15, 28), і смерті там не буде (1 Кор. 15, 26).

Отже не було б правосуддя, коли б пошкоджені ніколи не одержали винагороди за свої страждання, а гнобителі — кари за свої неправди. Це було б противне Правді Божій.

Тільки суд Божий виявить правду, і до Царства Божого ввійдуть тільки праведні. Нечестивці не можуть перебувати у Царстві Правди, у сяйві Бога, серед святих, як темрява не може бути серед світла або скверна серед чистоти.

Про кінець світу засвідчив Сам Господь Ісус Хрис-тос, а також і про останній Свій суд. І свідчення Своє затвердив такими словами: «Небо й земля пройдуть, а слова Мої не пройдуть» (Мт. 24, 35).

«Прийде Син Чоловічеський у славі Отця Свого з ангелами Своїми. Тоді кожному віддасть по ділах його» (Мт. 16, 27).

«Надходить час, в який ті, що в гробах, почують голос Сина Божого і вийдуть: ті, що чинили добро, воскреснуть для життя, а ті, що чинили зло, воскреснуть для суду» (їв. 5, 28-29).

\subsection{Коли ж буде кінець світу і Страшний Суд?}

Час другого пришестя не відкритий нам.

«Про день той і годину ніхто не знає, не знають і ангели, тільки Отець Мій один», — каже Христос (Мт. 24, 36). «А тому завжди будьте готові, бо не знаєте ні дня, ні години, коли Син Божий прийде» (Мт. 25, 13).

Але Господь вказав ознаки, по яких видно буде, що кінець світу і Страшний Суд наближаються. Вони такі:

Євангеліє Царства Божого буде проповідане по всій землі (Мт. 24, 14). Страшенно збільшиться беззаконня між людьми, зникнуть любов і правда. Ослабне віра. Будуть страшні війни. Появляться різні лжехристи і лжепророки, які обманюватимуть народи. Нарешті, прийде антихрист, противник Христа, який явно повстане проти Бога, себе видаватиме за бога й творитиме неправдиві чуда (2 Сол. 2, 8-9). Тоді почнуться ознаки в сонці, і в місяці, і в зорях. На людей нападе великий страх і туга, всі будуть ждати нещастя (Мт. 24, 3-44; Мр. 13, 3-17; Лк. 21, 7-36).
Кінець світу станеться несподівано:

«Як перед потопом їли, пили, женилися... і не думали, аж поки прийшов потоп і не знищив усіх, так буде і в пришестя Сина Чоловічеського» (Мт. 24, 38-39).

Але воно буде явне для всіх:

«Як блискавка блисне на сході і видно її на заході, так буде й пришестя Сина Чоловічеського» (Мт. 24, 27).

У день загального воскресіння мертві воскреснуть, а живі переміняться.

\subsection{Як те станеться?}

Апостол Павло пояснює: «Не всі ми вмремо, але всі перемінимося, дуже швидко, в миг ока, при останній трубі: затрубить, бо і мертві оживуть, а ми перемінимося» (1 Кор. 15, 51-52).

Тіла наші стануть такими, як було тіло Христове після воскресіння, тобто духовними.

«Як носили ми образ земного (Адама), так будемо носити й образ небесного (Христа)» (І Кор. 15, 49). Тому й суд буде не на землі, а на блакиті небесній (1 Сол. 4, 17).

\subsection{Чи можливе воскресіння мертвих?}

Святий апостол Павло каже: «Те, що ти сієш, не оживе, якщо не вмре... Бог дає йому тіло, яке хоче» (1 Кор. 15, 36, 38).

Безвірні кажуть: якщо тіла мертвих згнили і перейшли в інші речі чи навіть істоти, то як же вони воскреснуть?

їм відповідаємо: для людської мудрості то неможливо, а для Бога все можливо. Йому покірні стихії. Він покликав з небуття світ до життя, з нічого створив його. Тому не тяжко Йому повернути всі частки рознилого тіла нашого знову до нього і при тому не матеріально, а духовно.

\subsection{Як же станеться Суд?}

Господь наш Ісус Христос змальовує останні події світу цього і Свій Страшний Суд так:

«Як блискавка блисне на сході і видно її на заході, так буде й пришестя Сина Чоловічеського... І зразу ж після скорбот тих сонце померкне, місяць не дасть світла свого, зорі зникнуть з неба, основи небесні захитаються» (Мт. 24, 27, 29).

Апостол Петро пояснює це місце словами: «В початку Словом Божим небеса й земля складені з води й водою, тому й люди погинули, потоплені водою. А теперішні небеса й земля, що утримуються тим же Словом, готуються до вогню на день суду й загибелі нечестивих людей» (2 Петр. З, 5-7).

«І явиться на небі знамення Сина Чоловічеського (Христа Господа). І заплачуть усі народи землі. І побачать Сина Чоловічеського, що йде на блакиті небесній з силою і славою великою. Він пошле ангелів Своїх з трубою громо-звучною, і збере вибраних Своїх зі всього світу від краю небес до краю їх» (Мт. 24, 30-31).

«Тоді сяде Він на престолі слави Своєї. І зберуться перед Ним усі народи, і розділить їх між собою, як пастух відділяє овець від козлів, і поставить овець праворуч Себе, а козлів ліворуч. Тоді скаже до праведних: «Прийдіть, благословенні Отця Мого, унаслідуйте Царство, приготоване вам від початку світу. Я голодний був, ви нагодували Мене; хотів пити, ви напоїли Мене. Роздягнений був, ви зодягнули Мене, Подорожнім був, ви прийняли Мене; хворий був, ви відвідали Мене; у в`язниці був, ви прийшли до Мене». Тоді праведники скажуть: «Господи, коли ми бачили Тебе й послужили Тобі?» А Господь їм скаже: «Ви це зробили меншим братам Моїм і цим послужили Мені». Тоді скаже й до грішників: «Ідіть від мене, прокляті, у вогонь вічний, приготований для диявола і слуг його. Я голодний Був, ви не нагодували Мене; хотів пити, ви не напоїли Мене. Роздягнений був, ви не зодягнули Мене. Подорожнім був, ви не прийняли Мене. Хворий був і у в`язниці, ви не прийшли до Мене». Тоді ці скажуть: «Господи, коли ми бачили Тебе і не послужили Тобі?» А Господь відповість їм: «Ви нікому того не зробили, значить не зробили й Мені». І підуть ці в муку вічну, а праведники в життя вічне».

Так змальовує долю праведників і грішників євангелист Матвій (Мт. 25, 31-46).

... і Царству Його не буде кінця. Про яке Царство тут говориться?

Царство Христове почалося з того моменту, коли Він смертю Своєю переміг смерть і владу диявола, і сів праворуч Бога (Кол. З, 1).

І, як каже святий апостол Павло, «Йому належить царствувати аж доки покладе всіх ворогів Своїх під ноги Свої. Останнім знищиться ворог — смерть... Коли ж усе покорить Йому Бог, тоді й Сам Син Божий покориться Отцю, і тоді буде Бог все у всьому» (1 Кор. 15, 25-26, 28).

Отже, Син перебуватиме в Отці, Бог буде у всьому, і Царству Його не буде кінця.

Так сказав і святий архангел Гавриїл Пресвятій Діві Марії в час благовістя про народження від неї Сина Божого: «... і буде царювати над домом Якова (Церквою) повіки, і Царству Його не буде кінця» (Лк. 1, 33).
 
\subsection{Чи всіх людей буде судити Господь?}

Всіх без винятку. І не тільки за діла, а й за слова і за помисли. «Кажу бо вам, — каже Господь, — що за кожне погане слово, яке скажуть люди, дадуть за нього відповідь в день судний» (Мт. 12, 36).

Суд той буде воістину Страшний, бо не хтось інший судитиме нас, а діла наші: «Всім нам належить явитися перед судищем Христовим, щоб кожному одержати, відповідно до того, що він робив, коли в тілі жив — чи добре, чи погане» (2 Кор. 5, 10).

Тоді ніхто не зможе допомогти нам, бо буде «суд без милості тим, хто не був милостивим» (Як. 2, 13).

Суд той останній, і не буде вже надії на помилування. Страшні демони схоплять осуджених грішників, а грізні ангели Божі кинуть їх і грішників у вогонь вічний. Тому то й каже Господь: «Пильнуйте й моліться, бо не знаєте, коли настане той час» (Мр. 13, 33).

\section{ВОСЬМИЙ ЧЛЕН СИМВОЛА ВІРИ}
І в Духа Святого, Господа, Животворчого, що від Отця сходить, що з Отцем і Сином рівнопоклоняємий і рівнославимий, що говорив через пророків.

В цьому члені Символа Віри говориться про третю Особу (Іпостась) Пресвятої Тройці — про Духа Святого.

Божественний Дух зветься Святим не тому тільки, що Він Святий по єству, як Бог, а ще й тому, що Він усе Собою освячує.

Животворчим Дух Святий зветься тому, що Він все Собою оживотворює і впорядковує від самого початку світу. Так Дух Святий носився над першоствореним хаосом матерії і вселяв у неї життя (Бут. 1, 2).

«Словом Господнім небеса створені, і Духом уст Його вся сила їх» (Пс. 32, 6).

Духом Своїм Господь дарував життя людині, вдихнувши в неї душу розумну, вільну й безсмертну (Бут. 2, 7).

Духом Своїм Господь завжди оновляє природу. «Однімеш душу (життя) їх (тварин), вони вмирають і в порох повертаються. Пошлеш Духа Твого, вони знову створюються і оновляєш лице землі» (Пс. 103, 29-30). Дух Святий всюди вносить життя, і без благодаті Духа Святого неможливе життя праведне.
 
Дух Святий є третя Особа (Лице, Іпостась) Пре-божественної Тройці, невіддільна і незлитна. Дух Святий превічно сходить від Отця і цим відрізняється від інших Осіб Пресвятої Тройці. Це сходження є істотне вічне. Сам Господь наш це засвідчив словами: «Я ублагаю Отця, і дасть вам іншого Утішителя... Духа Істини... Утішитель же — Дух Святий, Якого пошле Отець в ім`я Моє, навчить вас всього, що Я говорив вам» (їв. 14, 16, 17, 26).

«Коли ж прийде Утішитель, Якого Я пошлю від Отця, Дух Істини, Який від Отця сходить, Він посвідчить про Мене» (їв. 15, 26).

Дух Святий в інших місцях Святого Письма зветься і Духом Христовим (1 Петр. 1, 11).

Також і після воскресіння, коли Христос явився апостолам, то, дихнувши на них, сказав: «Прийміть Духа Святого» (їв. 20, 22).

Однак те сходження Духа Святого від Ісуса Христа, як Сина Божого, є часове. В цьому сходженні давалася не вся повнота Божества, яка перебуває в Дусі Святому, а сила і влада Духа Святого, потрібна для даної місії апостолам.

Отже, істотне сходження Бога Духа Святого у всій повноті звершується лише від Отця превічно, як і родження Бога Сина звершується превічно лише від Отця. Цей догмат затвердив Третій Вселенський Собор (правило 7) і заборонив зміняти його.

Святий Іван Дамаскин так пише:

«Дух Святий від Отця сходить, тому звемо Його Духом Отцевим, од Сина ж Дух Святий не походить, а тільки звемо Його Духом Синовим» («Бог», кн. 1, розд. 11).

... що з Отцем і Сином рівнопоклоняємий і рівнославимий...

У Святій Тройці Єдине Божество, Єдине Господство, Єдина Влада, Єдина Сила і Єдине Єство Боже. Осіб Пребожественної Тройці три: Отець, Син і Святий Дух. Ніхто з Них не старший, не більший, не менший, а всі в Собі рівні. Бо один і незмінний завжди рівний Сам Собі. Коли ж у перечисленні Бог Отець є першим, то тільки в перечисленні. Він — початок або вмістилище Божества.

Тому то Духові Святому, як Богові, належить однакове поклоніння і славлення, як Отцю й Синові, бо Вони — Одно. І тому то Господь Ісус Христос заповідав хрестити в ім`я Отця і Сина і Святого Духа (Мт. 28, 19), як у Одного.

... що говорив через пророків.

Цими словами відзначається, що Господь сповіщав волю Свою людям через пророків. Через них говорив Дух Святий (1 Петр. 1, 11). «Дух Святий передсповістив устами Давида про Юду, що привів тих, які взяли Ісуса» (Діян. 1, 16).

«Ніколи пророцтво не виходило з волі людської, а (його) виголошували святі Божі люди, керовані Духом Святим» (2 Петр. 1, 21).

Той же Дух Святий зійшов на апостолів у виді вогненних язиків у 50-й день після воскресіння Христового:

«І з`явились язики поділені, немов би вогненні і на кожному з них по одному осів. Всі ж вони переповнились Духа Святого й почали говорити іншими мовами, як їм Дух промовляти давав» (Діян. 2, 2-4). Святий Дух керував апостолами у проповіді євангельській.

\subsection{Дія Духа Святого}

Дух Святий діє в кожній достойній людині. «Хіба ви не знаєте, що ви — храм Божий, і Дух Божий живе у вас?» (1 Кор. З, 16). Отже через побожне життя й усердні молитви ми можемо сподоблятися благодаті Духа Святого.

Дари Духа Святого численні, а головніші з них сім:
\begin{enumerate}
    \item Дух страху Божого (страх прогнівити Бога).
    \item Дух пізнавання.
    \item Дух сили (могутність духовна).
    \item Дух поради (мудрість радити, навчати).
    \item Дух розуміння.
    \item Дух мудрості (у всьому).
    \item Дух Господній (побожність, горіння у вірі та любові до Бога) (Іс. 11, 2).
\end{enumerate}

Плоди діяння Святого Духа в нас такі: любов, радість, мир (спокій, лагідність), довготерпіння (в бідах і до кривд людських), доброта (до всіх), милосердя, віра, лагідність, стриманість (Гал. 5, 22-23).

Немає страшнішого гріха, як хула на Духа Святого. Вона не проститься ні в цей вік, ні в будучий (Мт. 12, 31-32).
 
\section{ДЕВ`ЯТИЙ ЧЛЕН СИМВОЛА ВІРИ} 
В Єдину Святу, Соборну і Апостольську Церкву.

Церква — це установлена Богом громада людей, об`єднаних Православною вірою, священноначалієм (ієрархією) і таїнствами.

\subsection{Що значить вірувати в Церкву?}

Це значить побожно шанувати істинну Церкву Христову, коритися її навчанням і настановам з повною певністю, що в ній перебуває, спасенно діє, навчає й керує благодать Божа, що зливається на неї від її Вічного Голови, Господа Ісуса Христа.

В тому ми упевнюємося з того, що:
\begin{enumerate}
    \item Церкву встановив Сам Господь Ісус Христос, як і обітував, коли сказав: «Збудую Церкву Мою, і пекельні сили не подолають її» (Мт. 16, 18).
    \item Вічним Головою Церкви є Богочоловік Христос Ісус, повний благодаті й істини. Він і тіло Своє, тобто Церкву, завжди наповняє благодаттю й істиною.«Бог Отець дав Його (Христа), як Голову Церкви, яка вище всіх, а Церква є тіло Його» (Ефес. 1, 22-23).
    \item Він обітував ученикам Своїм Святого Духа, щоб був з ними (в Церкві) вічно. І дійсно послав Духа Святого на учеників у день П`ятдесятий у виді вогненних язиків (Діян. 2, 2-4).
    
    \item Дух Святий настановляє пастирів Церкви (їв. 14, 16-17, 26). Так каже апостол Павло до пастирів: «Пильнуйте себе і всього стада, над яким вас Дух Святий настановив єпископами, щоб ви пасли Церкву Господа й Бога, що її придбав Він Собі Своєю кров`ю» (Діян. 20, 28). 
\end{enumerate}

Благодать Божа діє в Церкві й донині і буде діяти до кінця віку, як і сказав Господь: «Я з вами по всі дні до кінця віку» (Мт. 28, 20).

Святий апостол Павло ту ж істину відзначає словами: «Йому (Богу Отцеві) слава в Церкві через Христа Ісуса у всі роди на віки вічні. Амінь» (Еф. З, 21).
\subsection{Церква Христова Єдина}

Вона — одне суцільне тіло, що має Єдиного Голову Христа Господа і одушевляється Єдиним Духом Святим.
 
«Єдине тіло і Єдиний Дух, як ви й покликані в єдиній надії вашого покликання. Один Господь, одна віра, одне хрещення, один Бог і Отець усіх, Який над усіма, через усіх і в усіх нас» (Еф. 4, 4-6).

Так каже апостол і додає, що «для Церкви, як Божої будівлі, підвалини іншої ніхто не може покласти ліпшої, ніж та, що лежить, якою є Ісус Христос» (1 Кор. З, 11).

Єдинство Церкви заповіджене Самим Господом Ісусом Христом. Перед Своїми стражданнями в молитві в саду Гефсиманському, Він благав Отця про те єдинство:

«Не за них (апостолів) тільки благаю, а й за тих, що увірують у Мене через слово їх. Щоб усі були одно, як Ти, Отче, в Мені і Я в Тобі, щоб і вони в Нас були одно, щоб увірував світ, що Ти Мене послав, і славу, що Ти дав Мені, Я дав їм, щоб були одно, як і Ми одно. Я в них і Ти в Мені» (їв. 17, 20-23). Тому й апостол навчає: «оберігати єдність духа в союзі миру» (Еф. 4, 3).

\subsection{Якщо Церква єдина, то як розуміть, що існує багато Церков, як наприклад Єрусалимська, Грецька, Російська, Українська, Румунська і інші?}

То лише адміністративний поділ Єдиної Христової Церкви для зручності керування і впорядкування. Поділ вимагається ще й тим, що до єдиної Церкви входять різні народи, що живуть у різних державах. Хоч ті Церкви (частини Єдиної Вселенської Церкви) і незалежні одна від одної в адміністративному керуванні (автокефальні), вони керуються тими ж самими законами Святого Євангелія, постановами св. апостолів і Вселенських Соборів, і тим являють собою Єдину Соборну Апостольську Церкву, яка в цілому є Єдиним тілом Христовим. Вони об`єднуються між собою єдністю віри, молитвами й таїнствами.

\subsection{На які частини поділяється Єдина Церква?}

Новозаповітню Церкву придбав Господь Ісус Христос Своєю неоцінимою кров`ю, але до Єдиної Церкви належать як новозаповітні, так і старозаповітні люди, бо Господь один. Тому Єдина Церква поділяється ще й так:
\begin{enumerate}
    \item На Старозаповітню (Церква підзаконна).
    \item Новозаповітню (Церква благодаті), Церква діюча,
    змагальна (та, що змагається).
    \item Небесну Церкву (Церква слави), Церква блаженних.
\end{enumerate}

\subsubsection{Церква Старозаповітня (Підзаконна)}

Цю Церкву створив Господь ще в раю після гріхопадіння перших людей Адама й Єви, коли обітував їм Спа-сителя і навчив приносити кровні жертви, як прообрази будучої Хресної Жертви Христової.

Ця підзаконна Церква готувала людей до прийняття Спасителя до самої смерті Христової, коли завіса церковна роздерлася (Мт. 27, 51). Розідрання завіси відзначило, що Старий Заповіт закінчився. Богослуження Старозаповітньої Церкви полягало у прославленні Бога, у подяках за благодіяння і в прохальних молитвах, а центром його було принесення жертв.

Підзаконною зветься вона тому, що Закон, даний Богом через Мойсея на горі Синай, був законом примусу, відповідно до непокірливості народу. Він вів людей так, як наказний учитель, і тому був суворий. За порушення його наказів вимагалася тяжка кара. Наприклад за богохульство, порушення суботи, зневагу батьків, за перелюбство каралося смертю, за вбивство також смерть: око за око, зуб за зуб (Вих. 21, 24; Мт. 5, 38).

Закон старозаповітній не визволяв від прародительського гріха і не спасав від шеолу (пекла), а був лише вихователем до майбутнього (тінню грядущого).

Всі надзвичайні події у Старозаповітній Церкві, як принесення в жертву Ісаака, продаж Йосифа в неволю, пасхальний агнець, перехід через море, манна, мідний змій, скінія, ковчег Заповіту, жезло Арона і інше, також і всі жертви — все то були прообрази Спасителя Христа, Церкви й подій новозаповітніх. Тому то Церква Старозаповітня у церковних піснях зветься тінню Новозаповітньої Церкви благодаті: «Минулася тінь закону, коли прийшла благодать...» (догматик 2-го голосу, див. Євр. 10, 1).

\subsubsection{Церква Новозаповітня (Церква благодаті)}

Про Новозаповітню Церкву так говорить Господь через пророків:

«Ось дні надходять, і Я укладу з домом Ізраїля і з домом Юди Заповіт Новий... і напишу Закон Мій на серцях їх» (Євр. 8, 8-10; 10, 16-17) «і буду їм Богом, а вони будуть Мені людьми» (Лев. 26, 12), «оселюся в них і ходитиму (в них) і буду їм Отцем, а вони будуть Мені синами й дочками» (Єр. З, 19; Осії 1, 10; 2 Кор. 6, 16).

Початок Новозаповітньої Церкви поклав Господь Ісус Христос, коли покликав учнів Своїх і Сам став її видимим Головою. Затвердив же її Своїми хресними стражданнями і освятив Своєю дорогоцінною кров`ю.

Перед Своїми стражданнями, на Тайній Вечері, Христос Господь Церкві Своїй через святих апостолів дав на їжу Своє Пречисте Тіло і Найсвятішу Кров Свою під видами хліба й вина (Мт. 26, 26-28; Мр. 14, 22-24; Лк. 22, 19-20), щоб Святе Його Тіло перебувало в нашому тілі і Пречиста Його Кров текла в нашій крові, так що всі, хто достойно причащається, фактично стають членами Тіла Його. Тому й Церква, що складається з таких членів, стала фізично Тілом Його.

Після воскресіння з мертвих, у той же перший день тижня, Христос Спаситель явився ученикам Своїм у сіонсь-кій горниці, як двері були замкнені, і сказав їм: «Мир вам! Як послав Мене Отець, так і Я посилаю Вас». І, сказавши це, дихнув на них і каже: «Прийміть Духа Святого. Кому відпустите гріхи, відпустяться, на кому зоставите — зостануться» (їв. 20, 19-23).

Цими словами Господь передав апостолам, а через них і Церкві Своїй в особах її ієрархії, повноту влади священства — священнодіяти й прощати гріхи тим, хто щиро кається.

А посилаючи апостолів на проповідь, Христос сказав: «Дана Мені всяка влада (вся повнота влади) на небі й на землі. Ідіть, навчайте всі народи, і хрестіть їх в ім`я Отця і Сина і Святого Духа. Навчайте їх додержуватися всього, що Я заповідав вам. І ось я з вами по всі дні до кінця віку. Амінь». (Мт. 28, 18-20).

Цим повелінням Господь Ісус Христос дав апостолам уповноваження будувати Церкву Христову, дав напрям її діяння і визначив час її перебування — до кінця віку.

У 50-й день після Свого воскресіння, Христос послав апостолам від Отця Духа Святого, щоб нагадав їм усе, що Він казав їм (їв. 14, 26), щоб і майбутнє сповістив їм (їв. 16, 13), тобто керував ними і перебував з ними (з Церквою) навіки (їв. 14, 16). «І вони пішли і проповідували всюди» (Мр. 16, 20). Засновували Церкви між народами, навчаючи їх жити по науці Христовій; встановили чини богослужень, навчили звершувати таїнства. А владу в Церкві, дану їм від Господа, передали своїм наступникам — архиєреям і пресвітерам, встановивши ступені ієрархії священства.
Так Господь створив Церкву Своєю Чесною Кров`ю (Діян. 20, 28).

Церква Новозаповітня зветься ще «Церквою воюючою» або «Церквою змагальною», бо вона веде боротьбу з ворогами Божими — сатаною та його послідовниками, взагалі з темними пекельними силами, які даремно намагаються повалити її. Бо Господь сказав: «Збудую Церкву Мою і пекельні сили не подолають її» (Мт. 16, 18).

Про ту боротьбу так каже апостол Павло: «Паша боротьба не проти тіла і крові, а проти начальства, проти власті, проти світоправителів темряви віку цього, проти піднебесних духів злоби» (Еф. 6, 12).

\subsubsection{Церква Небесна, Церква Святих, Церква Слави}

До Небесної Церкви, Церкви Слави належать перш за все незчисленні сонми ангелів, а потім душі (духи) святих людей, які сподобилися стати членами Царства Божого Небесного. Між ними першою є Свята Діва Марія, Пренепорочна Мати Сина Божого, Христа Господа.

Про життя святих на небі так каже Господь: «Ті, що сподобилися досягти того віку і воскресіння з мертвих, не женяться ні заміж не виходять, і вмерти не можуть, бо вони рівні з ангелами і є (вони) сини Божі, ставши синами воскресіння (Лк. 20, 35-36).

Тому то апостол каже до тих, що увірували в Христа: «Ви приступили до гори Сіону, до міста Бога живого — Єрусалиму Небесного, до безлічі ангелів, до урочистих зборів Церкви первородних, на небесах записаних, до Судді всіх Бога, до духів праведників досконалих, до посередника Нового Заповіту — Ісуса» (Євр. 12, 22-24).

Між земною й Небесною Церквами завжди перебуває безперервна єдність у тому, що як Небесна, так і земна, мають своїм Головою Єдиного Господа Ісуса Христа і оживляються Єдиним Духом Святим. Між ними також безперервно перебуває спілкування у вірі, молитві й любові.
 
Вірні, що належать до Церкви земної, приносячи молитви Богу, призивають до помочі святих, які належать до Церкви Небесної. А ті, стоячи близько до Господа, моляться, щоб Господь прийняв молитви земних. І по волі Божій допомагають нам невидимою силою, чи своїми явліннями, чи іншими якими засобами.

\subsection{Чи дозволено призивати святих у молитвах?}

За словами Спасителя, святі стали синами Божими (Лк. 20, 36). І молитви їх, як синів, угодні Богові.

Коли Господь обіцяв задовольнити все, чого з вірою попросимо (Мр. 11, 24), то тим більше задовольняє прохання угодників Своїх (їв. 16, 23-24, 26-27).

Святі — то наші брати по крові й по земному життю. Вони близькі нам і знають усі наші немочі, тому не можуть не клопотатися за нас перед Богом, особливо Пренепорочна Мати Господня.

Так само й святі ангели Божі намагаються допомогти нам спастися уже тому, що Господь бажає нашого спасіння. Для того й дані вони нам, як наші охоронителі (Лк. 15, 10).

Про ангелів апостол каже: «Всі вони служебні духи, які посилаються на служіння тим, що мають у наслідувати спасіння» (Євр. 1, 14).

А про молитви (за нас) святий євангелист Іван Богослов пише так: «... і я бачив... сімох ангелів, що стояли перед Богом... і підійшов інший ангел і став перед жертівником, держачи золоту кадильницю, і дано було йому багато ладану, щоб він з молитвами святих поклав його на золотий жер-тівник, що перед престолом. І піднявся дим ладану з молитвами святих од руки ангела перед Богом» (Об. 8, 2-4).

Померлі святі являються земним людям, як то засвідчено і в Євангелії. В момент смерті Христа на хресті «каміння розпалося... гроби відкрилися і багато тіл померлих святих воскресли і, вийшовши з гробів... увійшли в святе місто і явилися багатьом» (Мт. 27, 51-53).

Таке велике чудо не могло бути без важливої мети. Тому треба думати, що воскреслі святі явилися для того, щоб сповістити про зшестя Ісуса Христа в пекло, звідки Він вивів померлих, та щоб допомогти тим, що народилися у Старому Заповіті (у Старозаповітній Церкві), перейти в Новозаповітню, яка в цей момент відкрилася.

Маємо також докази і з Святого Письма, що святі й після смерті своєї не перестають чудодіяти для нас, як і за життя свого. Коли тіло мертвого чоловіка доторкнулося до тіла спочилого пророка Єлисея, то мертвий ожив (4 Цар. 13, 21).

Святі творять чуда не тільки самі, а силу свою являють і через різні речі. Так, пов`язки з голови святого апостола Павла й рушники, взяті від нього, зціляли хворих (Діян. 19, 12).

Особливо часто Пресвята Богомати і святі являють свою поміч через святі ікони.

На цій підставі земна Церква завжди призиває святих Церкви Небесної у своїх молитвах. Особливо на Божественній Літургії призиває вона святих патріархів, пророків, апостолів, мучеників і всіх святих, щоб за їх молитви й заступництво Бог прийняв молитви наші (св. Кирил Єрусалимський, «Поучення тайновод.» 5, розд. 19).

Святі угодники Божі являють свою благодатну допомогу і через свої святі останки — мощі.

Святий Іван Дамаскин так пише:

«Мощі святих, наче святі джерела, дарував нам Владика Христос. Вони многовидні благодіяння нам виточують... бо через їх розум і в тіло їхня вселився Бог» (Богос. кн. 4, розд. 15).

\subsection{Церква свята}

Вона освячена Ісусом Христом: Його наукою, Його стражданнями, Його молитвою в саду Гефсиманському:

«Отче... освяти їх Істиною Твоєю; Слово Твоє — Істина... і за них я освячую Себе, щоб і вони були освячені Істиною» (їв. 17, 17-19). Церква завжди освячується Духом Святим у святих таїнствах.

«Христос полюбив Церкву й Себе віддав за неї, щоб освятити її, очистивши купіллю водною через слово. Щоб показати її славною Церквою, яка не має плями або пороку, або чогось такого, а щоб була свята й непорочна» (Еф. 5, 25-27).

«... не за них (апостолів) тільки я благаю, а й за тих, що увірують у Мене через слово їх» (їв. 17, 20).
 
Церква не перестає бути святою, хоч у ній перебувають і грішні люди — подібно, як ліс не перестає бути живим, хоч у ньому є й сухі гілки й дерева. Бо ті, що согрішають, через каяття очищаються і через святу Євхаристію освящаються. А нерозкаяні грішники, як сухі гілки, відсікаються від неї видимим дійством церковної влади, як заповідав Господь: «... якщо й Церкви не послухає, нехай буде для вас, як чужий... істинно бо кажу вам: що зв`яжете на землі, буде зв`язано й на небі, і що розв`яжете на землі, буде розв`язано й на небі» (Мт. 18, 17-18).

І апостол завіщав: «Виключіть злого з-поміж себе» (1 Кор. 5, 13).

Або через крайню гріховність вони духовно вмирають і, як сухі безплідні вітки, Самим Богом відсікаються (їв. 15, 2).

Таким чином, святість Церкви не зменшується. Твердо будівля Божа стоїть, маючи ознаку цю:

«Пізнав Бог Своїх; і нехай відступить від неправди кожний, хто іменує ім`я Господнє» (2 Тим. 2, 19).

\subsection{Кафолицька чи Вселенська}

Вона не обмежується одним місцем, часом, або народом, а вміщає в собі правовірних усіх часів, країв і народів.

«В Церкві Христовій немає ні юдея, ні грека, ні обрізання, ні необрізання, чужинця, чи скифа, чи раба, чи вільного, а все і у всьому Христос» (Кол. З, 11).

«І всі, що у вірі, благословляються з вірним Авраамом» (Гал. З, 9).

Вселенська Церква не може впадати в помилки і проповідувати якісь вигадки людські або неправду замість правди, тому що:

1. В ній завжди діє Дух Святий через праведних Отців і Вчителів Церкви (Послання Східних патріархів про Православну Віру, чл. 12).
2. Вона пильно дотримується всього того, що завіщали святі апостоли і що розкрили і утвердили керовані Духом Святим Вселенські Собори. Звідси вона — Православна, бо право (правдиво) навчає вірувати і право ж, правдиво, істинно прославляє Бога.
 
Для того, щоб осягнути спасіння через заслуги Господа нашого Ісуса Христа, необхідно належати до Вселенської Православної Правовірної Церкви. Бо, як каже апостол: «Христос Господь є Головою Церкви і Він (є) Спаситель тіла (Церкви)» (Еф. 5, 23).

Отже, щоб мати участь у спасінні, необхідно бути членом Тіла Христового — Церкви.

Святий апостол Павло навчає нас, що тайна святого Хрещення спасає нас по зразку Ноєвого ковчега. Спаслися лише ті, хто був у ковчезі, а всі, що були поза ковчегом, загинули. Так загибають і ті, що перебувають поза Церквою Христовою (1 Петр. З, 20-21).

Не можна вважати за Церкву ті об`єднання людей, які порвали з Вселенською Церквою і базуються на навчанні новітніх самочинних учителів, що іншого, ніж Вселенська Церква, учать. Святий апостол каже: «Коли б ми, або навіть ангел з небес почали навчати іншого ніж ми (апостоли) навчаємо, то нехай буде анафема, цебто відлучені від Церкви. Таких і Христос називає лжехристами і лжепророками» (Мт. 24, 11).

А святий апостол Павло про таких каже: «Знаю, що після мого одшестя увійдуть до вас і з-між вас повстануть вовки люті, які не пошкодують отару» (Діян. 20, 29).

\subsection{Церква Христова іменується Апостольською}

Це тому, що вона безперервно і незмінно оберігає вчення, прийняте через апостолів і переємство Дарів Благодаті Святого Духа через переємственне рукоположення священства. «Ви не чужі й не захожі, — каже апостол, — а спільники життя святих і свої Богові, бувши збудовані на підвалинах апостолів і пророків, де наріжним каменем є Сам Ісус Христос» (Еф. 2, 19-20).

Через те необхідно твердо триматися вчення й передань апостольських і відхилятися від такого вчення й таких учителів, які не тримаються вчення апостольського. Так і апостол наказує: «Отже, браття, стійте (твердо) й тримайтеся переданая, якого ви навчилися чи через слово, чи через послання наше» (2 Сол. 2, 15).

«Єретика чоловіка після першого й другого умовляння одцурайся» (Тит. З, 10). «Бо багато є непокірних пустословів, обманутих в розумі, які збаламучують родини, навчаючи того, чого не годиться, ради скверного прибутку» (Тит. 1, 10-11).

\subsection{Яким чином у Церкві передається і оберігається преємство апостольського служіння?}

Через ієрархію священства. Бо ієрархія священства християнської Православної Церкви встановлена Самим Господом нашим Ісусом Христом. Вона почала діяти з дня зшестя Святого Духа на апостолів, а через переємственне рукоположення в таїнстві священства діє й до нині.

«Той (Христос) настановив одних апостолами, інших пророками, інших же пастирями й учителями на удосконалення святих, на діло служіння, на збудування тіла Христового (Церкви)» (Еф. 4, 11-12).

\subsection{Як здійснюється керування всією Вселенською Церквою?}

Частинами Вселенської Церкви — помісними Церквами — керують патріархи, митрополити, архиєпископи і єпископи, а на всю Вселенську Церкву свою владу поширює лише Вселенський (Всесвітній) Собор.

\section{ДЕСЯТИЙ ЧЛЕН СИМВОЛА ВІРИ}

Визнаю одне хрещення на відпущення гріхів.

В цьому члені Символа віри говориться про таїнство святого Хрещення. Але Христова Церква має ще й шість інших таїнств. Це — така широка тема, що вона буде розглянута в окремому, наступному, розділі.

\section{ОДИНАДЦЯТИЙ ЧЛЕН СИМВОЛА ВІРИ}

Чекаю воскресіння мертвих.

У цьому члені Символа Віри говориться про передсповіщене Господом загальне воскресіння померлих. Тіла померлих людей знову з`єднаються з душами своїми, оживуть і будуть духовними, як тіло Христове після воскресіння, і безсмертними. Все це станеться всемогутньою силою Божою.

Пояснюючи воскресіння, святий апостол Павло пише: «Сіється тіло звичайне (матеріальне), а встає тіло духовне» (1 Кор. 15, 44).
«Бо належить тлінному цьому стати нетлінним, і смертному цьому стати безсмертним» (1 Кор. 15, 53).
 
\subsection{Як можуть тіла померлих воскреснути, коли вони розсипаються або навіть переходять в інші речі?}

Святий апостол каже: «Тіло й кров не можуть бути у Царстві Божому (в Царстві Духа), і тлінне не може унаслі-дувати нетлінне» (1 Кор. 15, 50). Отже:
Мертві воскреснуть не в таких тілах, в яких померли, а в тілах духовних, подібних до Христового після воскресіння.

«Який земний (Адам), такі й земні (люди); який небесний (Христос), такі й небесні всі після воскресіння» (1 Кор. 15, 48).

«Як в Адамі всі вмерли, так у Христі всі оживуть» (1 Кор. 15, 22).

Бог у початку з землі створив тіло людині, Він же й воскресить тіла померлих людей. Святий апостол Павло пояснює це прикладом посіяного зерна: «Те, що ти сієш, не оживе, якщо не вмре» (1 Кор. 15, 36).

\subsection{Мертві воскреснуть, а що буде з живими?}

«Не всі ми вмремо, але всі перемінимося дуже швидко, в миг ока, при останній трубі, затрубить бо, і мертві оживуть, а ми перемінимося» (1 Кор. 15, 51-52).

Все те станеться при кінці цього видимого світу, бо цей тлінний світ переміниться на нетлінний (1 Кор. 15, 53).

«Саме творіння визволиться з неволі тління для волі слави дітей Божих» (Рим. 8, 21). Бо «нового неба й нової землі ми по Його (Божому) обітуванню чекаємо, на яких правда живе» (2 Петр. З, 13).

Все преобразиться через вогонь:

«Теперішні небеса й земля, що утримуються тим же Словом (Божим), зберігаються для вогню на день суду й загибелі нечестивих людей» (2 Петр. З, 7).

\subsection{Де перебувають душі померлих до часу воскресіння мертвих?}

Згідно з Словом Божим (Лк. 16, 23), душі праведних перебувають у сяйві Божому в покої й радості. Це вже є початком вічного блаженства. А душі грішних в протилежному стані, бо бувши осквернені гріхом, вони не можуть наблизитися до Бога (Лк. 16, 19-31).
 
Але повна нагорода або кара настануть тоді, коли приведеться Правда вічна в день Страшного Суду Христового, як каже апостол: «... тепер готується мені вінець правди, якого дасть мені Господь, Праведний Суддя, в той день, і не тільки мені, а й усім, хто полюбив явління Його» (2 Тим. 4, 8).

«... бо всім належить явитися перед судищем Христовим, щоб одержав кожний за те, що в тілі содіяв, чи добре, чи зле» (2 Кор. 5, 10).

В якому стані перебувають до часу воскресіння душі тих людей, які померли в добрій вірі, але не змогли принести плодів покаяння?

Вони далеко від Бога через гріховність і мучаться, але їм допомагають молитви живих рідних, милостиня за їх душі, а особливо молитви Церкви з принесенням за них безкровної Жертви на Божественній Літургії.

Віра в те стверджується безперервним церковним переданням, починаючи з часів апостольських. Бо молитви за померлих завжди були невід`ємною частиною Святої Літургії, починаючи від Літургії св. апостола Якова.

Святий Кирил, патріарх Єрусалимський, так каже:

«Превелика буде користь душам, за яких приносяться молитви в той час, коли перед нами лежить Свята і Страшна Жертва» (Тайновод. повч. 5).

Святий Василій Великий (IV стол.) в молитвах у день П`ятидесятниці (Святої Тройці) каже, що Господь сподобляє приймати від нас молитви умилостивлення й жертви за тих, що перебувають у пеклі з надією для заспокоєння для них, полегшення й звільнення.

\section{ДВАНАДЦЯТИЙ ЧЛЕН СИМВОЛ А ВІРИ}

...і життя будучого віку. Амінь.

Життя будучого віку почнеться після воскресіння мертвих і всезагального Суду Христового. Але не для всіх воно буде однакове.

\subsection{Яка доля чекає праведників?}

Для тих, що вірували в Бога, любили Його й чинили добро по заповідях Його, воно буде таке блаженне (щасливе), що ми тепер і уявити собі не можемо. «Ще не відкрилося, що ми будемо», — каже святий апостол Іван Богослов (1 їв. З, 2).

Апостол Павло був Духом Божим піднесений «до третього неба» (в рай, оселі небесні) і чув там невимовні слова, яких неможливо людині висловити» (2 Кор. 12, 2-4).

Там Господь показав йому оселі для праведників, і він свідчить:

«Око не бачило, вухо не чуло, і на серце людині не сходило те, що наготовив Господь для тих, що люблять Його» (1 Кор. 2, 9).

Святий Іван Богослов у Об`явленні оповідає, що він на небесах чув великий голос з неба:

«Ось Скінія Бога з людьми, і Він буде перебувати з ними. Вони будуть Його народом, і Сам Бог з ними буде їхнім Богом. І зітре Бог усяку сльозу від очей їх, і не буде вже смерті, ані плачу, ані стогону, ані страждань, бо все минулося» (Об. 21, 3-4).

«Храму я не бачив у ньому (небесному Єрусалимі), бо Господь Бог Вседержитель є храмом для нього і Агнцем. І місто не має потреби ні в сонці, ні в місяці для освітлення, бо слава Божа освітить його, і світильником його є Агнець» (Об. 21, 22-23).

«І не ввійде в нього ніщо нечисте, ніхто з тих, хто віддавався непотребству й неправді, а тільки ті, які записані в Агнця у Книзі Життя» (Об`яв. 21, 27), і ночі там не буде (Об. 21, 25), і не буде потреби в світильниках, бо Господь освітлюватиме їх і буде царювати по віки вічні (Об. 22, 5).

Там уже не буде й часу, а буде вічність.

Духи праведних будуть у вічній радості, яка ніколи не зменшиться і не обридне їм, бо дух ніколи не втомлюється, подібно як діамант ніколи не перестає сяяти від сонця. Така радість буде від того, що праведники лицем до лиця бачитимуть Бога, перебуватимуть у Його сяйві й славі, з`єднаються з Ним. Як каже апостол:

«Ми тепер уявляємо наче в дзеркалі, в загадці, а тоді побачимо лицем до лиця; тепер пізнаю тільки дещо, а тоді пізнаю так, як себе знаю» (1 Кор. 13, 12).

«Тоді праведники засяють, як сонця, у Царстві Отця їх» (Мт. 13, 43). «І буде Бог все у всьому» (1 Кор. 15, 28).

В тому блаженстві матимуть участь і тіла наші, але одухотворені, тобто духовні.
 
«Сіється без честі (кладеться в землю), а встає у славі. Сіється (ховається) тіло звичайне (матеріальне), а встає тіло духовне» (1 Кор. 15, 43-44).

«Як носили ми образ земного (Адама), так будемо носити й образ небесного (Господа Ісуса Христа)» (1 Кор. 15, 49).

Але слава і у святих не буде однакова, і очевидно по заслузі, як каже апостол: «Інша слава сонцю, інша слава місяцеві, інша слава зорям, бо й зоря від зорі відрізняється у славі. Так і воскресіння мертвих» (1 Кор. 15, 41-42).

\subsection{Що чекає неправедних?}

Безбожників і беззаконників чекає зовсім інша доля. Вони уготували собі вічну смерть, вічний вогонь, вічну муку разом з сатаною й демонами його.

«Хто не знайдеться записаним у Книзі Життя, той буде вкинутий у вогняне озеро» (Об. 20, 15). «І це друга смерть» (Об. 20, 14).

їм скаже Господь в день Суду: «Ідіть від мене, прокляті, у вогонь вічний, приготований для диявола і слуг його» (Мт. 25, 41).

»Підуть ці в муку вічну, а праведники в життя вічне» (Мт. 25, 46), «... в геєну вогняну, де чеовяк (докори сумління) не вмирає і вогонь не гасне (Мр. 9, 48).

Так Господь покарає їх не тому, що ніби хотів їх загибелі, а тому, що вони самі не послухали заклику євангельського, зненавиділи Бога і, як сатана, все чинили наперекір Йому. Сквернили себе в огидних ділах і цим зробили себе чужими для Бога, негідними бути серед праведних у сяйві Бога.

«Я вас кликав, а ви не слухали, простягав руку Мою, а ви не звертали уваги. Ви відкинули всі Мої поради, і наказів Моїх не прийняли» (Прит. 1, 24-25).

«За те будуть вони вкушати від плоду путів своїх і наситяться від помислів своїх» (Прит. 1, 31). Тоді побачать вони, чого позбавили себе навіки. І в злобі на самих себе вони будуть мучитися й ненавидіти одне одного. Пекельний неречовинний вогонь пожиратиме їх, але ніколи не пожере. І не буде вже надії вийти з того стану.

Боже, не допусти нас до того тартару!
 
Ніколи не треба забувати про смерть і про свою останню долю. Це відбиратиме охоту грішити й прив`язуватися до земних пристрастів. Живімо по-Божому й для Бога, щоб осягнути вічне Царство з праведними.

Амінь (істинно так)

Цим словом, мов би печаттю, стверджується все сказане у Символі Віри, як незмінна Істина.

\section{ВСЕЛЕНСЬКІ СОБОРИ}

Православна Церква визнає сім Вселенських Соборів і десять помісних. Вселенські Собори відбувалися тоді, коли на всій землі була лише одна Православна, неушкоджена у вченні, Церква. На цих Соборах були єпископи зі всього світу, тому й Собори правдиво названо Вселенськими, тобто всесвітніми. Після розділення між Сходом і Заходом у 1054 році Вселенських Соборів уже не могло бути. Тому всі собори, які відбулися після розділення Церков, вважаються лише помісними.

ПЕРШИЙ Вселенський Собор відбувся у 325 році в місті Нікеї проти єресі Арія, що не визнавав Ісуса Христа за рівного з Богом Отцем. На цьому Соборі укладено перші сім членів Символа Віри.

ДРУГИЙ Вселенський Собор скликано у місті Константинополі (Царгороді) 381 року проти єресі Македонія, що неправдиво вчив про третю Особу (Іпостась) Пресвятої Тройці — про Духа Святого. На цьому Соборі укладено останні п`ять членів Символа Віри (8-12).

ТРЕТІЙ Вселенський Собор, що відбувся у місті Ефесі в 431 році, був скликаний проти єресі Несторія. Несторій не визнавав Пресвяту Діву Марію за Богородицю. Собор ствердив догмат, що Пресвята Діва Марія є істинно Богородиця, бо Вона породила Сина Божого тілесно Ісуса Христа, Який у Ній одягнувся в єство людське і став Бого-чоловіком.

ЧЕТВЕРТИЙ Вселенський Собор відбувся у місті Халкидоні 451 року проти єресі Євтихія, що визнавав у Христі лише одну природу — Божественну (монофізитська єресь). Собор засудив єретичне вчення Євтихія і утвердив догмат, що Ісус Христос має дві природи — Божественну й людську. Він є Істинний Бог і в той же час і істинний чоловік (людина). Як Бог, Він творив чуда і воскрешав мертвих і як Бог же Сам воскрес із мертвих силою Божества. Як людина, він стомлювався, хотів їсти й пити, страждав на хресті і вмер. Отже Він — Богочоловік.

П`ЯТИЙ Вселенський Собор скликано в Константинополі (Царгороді) 553 року проти трьох листів єпископів Феодора Мопсуетського. Феодорита Кіпрського та Іова Едеського, що підтримували єресь монофізитів (євтихіян). Собор засудив ті писання і потвердив ухвали Четвертого Вселенського Собору.

ШОСТИЙ Вселенський Собор відбувся також в місті Царгороді в 680 році проти єресі монофелітів, що визнавали в Ісусі Христі при двох природах одну волю — Божественну. Собор засудив таке неправдиве вчення і постановив визнавати в Ісусі Христі дві волі — Божественну й людську, але так, що людська підкорилася волі Божій, як і Сам Христос у молитві казав: »0тче... не Моя, а Твоя воля нехай буде» (Лк. 22, 42).

СЬОМИЙ Вселенський Собор скликано в місті Нікеї 787 року проти єресі іконоборців. Собор ту єресь засудив і утвердив догмат шанування й поклоніння святим іконам\footnote{Примітка: Ми поклоняємося не матеріалові, з якого зроблена ікона, а (через образ) тій живій істоті, образ якої написаний на іконі.}, на тій підставі, що Сам Бог наказав Мойсееві поставити в Скі-нії херувимів (Євр. 9, 5). Бо святі ікони — то образи Бога, в яких Він являвся людям, і образи друзів Божих — Богоматері, святих ангелів і святих угодників Божих.

Вселенські Собори прийняли й затвердили постанови десяти помісних соборів, які мають значення для всієї Церкви.

Всі постанови Вселенських Соборів і признаних ними помісних Соборів зібрані в одну книгу, яка зветься «Номоканон» або «Канони Святої Православної Церкви». Канони Вселенських Соборів обов`язкові для кожного християнина, бо вони виявляють волю всієї Церкви. Христос же дав Своїй Церкві повноту влади, коли сказав: «Якщо (хто) й Церкви не послухає, нехай буде (для вас) чужий, як поганин» (Мт. 18, 17).

Перший Собор зібрали апостоли в Єрусалимі (Діян. розділ 15), і тому він відомий, як Апостольський Собор. За прикладом апостолів і Церква Христова почала скликати Собори.

\end{document}