\documentclass[main.tex]{subfiles}

\begin{document}
\chapter{Джерела Богопізнання}

\section{Звідки, ми, довідуємося, що Бог є і про Його властивості?}

До пізнання Бога маємо дві дороги:
\begin{enumerate}
    \item Об`явлення Боже.
    \item Пізнавання Бога через вивчення явищ Всесвіту і явищ видимої природи (див. розділ III).
\end{enumerate}
\section{Об'явлення боже}
Що ми, розуміємо під цією назвою?

Під Об`явленням Божим ми розуміємо те, що Бог Сам явив Себе першим людям — Адамові та Єві, їхнім дітям, внукам, а далі являв Себе багатьом святим людям, але не в образі людини, а в образі Божества-Духа \emph{(Іс. 6, 1)}.

Правда, розсіявшись по світу, люди потроху розгубили правдиве розуміння Бога і стали поклонятися та служити силам природи, тваринам та рукотворним ідолам, вважаючи їх за богів. Проте ідея Бога ніколи не зникала серед людей. На світі не було і немає такого народу, який так чи інакше не шанував би Бога.
\subsection{Бог часто свідчив про Себе}

Даючи людям через Мойсея Закон на горі Синаї, Бог сказав: \emph{{\color{red} «Я Господь Бог твій...»} (Вих. 20, 2)}. Те саме Він багато разів повторив через пророків.

Всі ці об`явлення Божі записані у книгах Святого Письма (Біблії).

Об`явлення Боже ми приймаємо, як безсумнівну істину, засвідчену Самим Богом.

Богооб`явлені істини поширюються між людьми і зберігаються в Церкві двома способами: через Святе Передання і Святе Письмо.

\section{Святе передання}

Під Святим Переданням розуміємо те, що святі й достойні люди передавали наступним поколінням про віру, закони Божі, церковні таїнства, побожні звичаї і про все те святе, що не записано у святих книгах. Святе Передання переховується у Церкві Христовій у молитвах, у церковних піснях, у писаннях святих Отців Церкви.

Вірною скарбницею Святого Передання є Церква, як і каже апостол Павло: \emph{«Церква Бога Живого — стовп і твердиня істини» (1 Тим. З, 15)}.

Одначе Передання тільки тоді дійсно святе, коли воно не суперечить Святому Письму й ученню Церкви. Вигадані перекази, навіть якщо вони побожного змісту, не можна вважати за Святе Передання.

Святе Передання давніше, ніж Святе Письмо. До всесвітнього потопу і до Синаю Святого Писання не було. Все учення про Бога й діла Його передавалося усно від одного покоління до другого. Святе Передання не втратило свого значення й тоді, коли появилося Святе Письмо, бо у книгах Святого Письма записано не все. Євангелист Іван пише: \emph{«Багато й інших чудес створив Ісус перед учнями Своїми, про які не написано у книгах цих» (їв. 20, 30)}. Та й неможливо все записати, бо тоді б книг було стільки, що їх ніхто не перечитував би.

Господь наш Ісус Христос Своє Божественне учення і Свої постанови передав учням усно, словом і прикладом, а не на письмі. Тим же способом спочатку й апостоли поширювали Віру Христову й засновували Церкви.

Передання, згідне із Святим Письмом, треба дотримуватися, як про це навчає апостол Павло: \emph{«Отже, браття, стійте й тримайтесь Передання, якого ви навчилися через слово чи через послання наше» (2 Сол. 2, 15)}.

\subsection{Для чого й тепер потрібне Святе Передання?}

Для нашого керівництва у правдивому розумінні Святого Письма, для правильного виконання таїнств та для оберігання священних обрядів у чистоті їхнього первісного встановлення. Святий Василій Великий про це каже:
\begin{FlushRight}
    «Із догматів, що зберігаються у Церкві і проповідань деякі ми маємо від писаного настановлення, а деякі прийняли від апостольського передання по переємству у таємниці. Одні й другі мають одну й ту ж силу для побожності, і цьому ніхто не стане заперечувати, хоча б і мало свідомий в установленнях церковних. Бо якщо одважимося відкидати написані звичаї, як щось дуже неважливе, то безперечно пошкодимо Євангелії в найважливішому, або ще більше: від проповіді апостольської залишимо порожнє ім`я.

    Наприклад: згадаємо насамперед про найперше й загальне: хто вчив писанням, щоб ті, хто надіється на Господа нашого Ісуса Христа, осіняли себе образом хреста? Яке писання нас учило, щоб у молитві повертатися на схід? Хто із святих залишив нам писане слово призивання у перетворенні хліба у Святій Євхаристії й вина у Чаші Благословення? Бо ми не задовольняємося тими словами, що їх апостоли або Євангелії згадують, але й перед ними і після них виголошуємо й інші слова, як такі, що мають велику силу для таїнства, прийнявши їх від неписаного учення.

    По якому писанню благословляємо й воду хрещення й оливу помазання, а також і самого хрещеника? Чи не по замовчаному переданню? Або ще: самого помазання яке писане слово навчило нас? Звідки трикратне занурення чоловіка у воду? Та й інше, що стосується хрещення: одречення від сатани і слуг його з якого взято писання? Чи ж не з цього неоприлюдненого й невисловленого учення, яке Отці наші зберегли в недоступному для цікавих і вивідування мовчанні? Бо чи розумно було б писанням оголошувати те, на що нехрещеним і глянути не дозволяється?» (Василій Великий, «Правило про Святого Духа», розділ 27).
\end{FlushRight}

Святе Передання треба відрізняти від вигаданих переказів (апокрифів), яких Церква не визнає.
 
\section{Святе Письмо}

Книги Святого Письма написані в різні часи: одні до приходу на землю Спасителя Христа, інші після того. Книги, писані до Христа, становлять Старий Заповіт, а книги, писані після Його приходу — Новий Заповіт. Слово Заповіт означає союз або спілка Бога з людьми.

\subsection{В чому полягає Старий Заповіт?}

В тому, що Бог обітував людям Божественного Спасителя й готував їх до прийняття Його. Це готування виявляв Господь через поступові об`явлення про Нього, через пророцтва та прообрази. Книги Старого Заповіту оповідають, як люди чекали приходу Спасителя, які закони давав Бог для того, щоб зберегти людей в істинній вірі, які творив чуда, як давав силу переборювати зло.

\subsection{В чому полягає Новий Заповіт?}

Новий Заповіт полягає в тому, що Господь дарував людям обітованого Спасителя — Єдинородного Сина Свого, Господа нашого Ісуса Христа. У книгах Нового Заповіту оповідається про пришестя Спасителя на землю, про Його вчення, про чуда, які Він звершив для упевнення людей в тому, що саме Він є Посланник Божий, про страждання Христові, смерть, воскресіння і вознесіння на небо. Про це оповідають чотири Євангелія. В інших книгах (Діяннях святих апостолів та апостольських посланнях) оповідається про те, як поширювалось Христове вчення між народами, як засновувалися Церкви.

\subsection{Біблія — священна книга}

Книги Старого і Нового Заповіту зібрані в одну книгу, що зветься Біблією (в перекладі це слово означає книги). Написані вони не з мудрості людської, а по натхненню Духа Святого. Писали їх освячені Богом мужі — пророки і апостоли. Біблія — це священна книга, вища за всі книги на світі.

Святе Письмо дане для того, щоб об`явлення Божі збереглися найточніше й без змін. У Святому Письмі ми читаємо слова пророків і апостолів так, наче б ми з ними жили і їх слухали, не дивлячись на те, що святі книги написані тисячі років тому.
 
\subsection{Як належить читати Святе Письмо?}

Святе Письмо — це Слово Боже. В ньому заховані величні Божі істини. Тому не можна читати книги Святого Письма так, як ми читаємо звичайні людські книжки. Біблію треба читати з великим благоговінням, як слова Самого Бога, і з молитвою, щоб Господь дав мудрість розуміти їх. Святе Письмо неможливо кожному тлумачити по своєму власному розумінню. В ньому багато таємниць Божих, яких з одного рядка або з одного місця неможливо зрозуміти. Тому, щоб не впасти в помилку, треба триматися того розуміння, якого в даному питанні тримається Свята Вселенська Церква Православна в писаннях святих Отців і Вчителів Церкви та Вселенські Собори.

\subsection{Які є свідчення про Божественне походження Святого Письма?}

Святе Письмо само в собі містить свідчення, що вчення його — це Слово Боже. Свідчення ці такі:
\begin{enumerate}
    \item Високість цього вчення, що могло вийти тільки з уст Божих, бо розум людський не зміг би видумати його.
    \item Чистота цього вчення, що могло вийти тільки з найчистішого Розуму Божого.
    \item Пророцтва, які вже справдилися і тому видно, що вони не були видумані людьми.
    \item Чудеса, які творив Господь при очевидцях і які записані в цих книгах.
    \item Вплив цього вчення на серця людей, можливий тільки Божій силі. Наприклад, цим ученням прості рибалки-апостоли підкорили Христові увесь світ.
\end{enumerate}

\begin{FlushRight}
    \emph{«Все писання богонатхненне й корисне для научення, для виявлення гріхів, для виправлення життя, для керівництва до праведності» (2 Тим. З, 16).}
\end{FlushRight}

\subsection{Чи не шкодить святості Біблії те, що у ній в деяких, місцях згадуються жорстокі вчинки окремих людей?}

Біблія описує людей такими, якими вони були, тобто з гріховними нахилами їхньої ушкодженої гріхом природи, показує, як Господь допомагав тим, що каялися, і як тяжко карав нерозкаяних грішників. Святе Письмо, згадуючи погані вчинки людей, засуджує їх, але з особливою силою підносить праведних, які за своє праведне життя одержали Боже благовоління й благословення.
\section{Книги Старого Заповіту}

Головним предметом Старозаповітніх книг є: створення світу і людей, гріхопадіння перших людей, Боже піклування про їх спасіння, обітування їм Спасителя та готування людей до прийняття Його.

Ісус Христос часто посилається на Святе Письмо:

\begin{FlushRight}
    \emph{{\color{red} «Дослідіть Писання... вони свідчать про Мене»} (їв. 5, 39).}
\end{FlushRight}
\begin{FlushRight}
    \emph{«Тоді розкрив їм розум розуміти Писання» (Лк. 24, 45).}
\end{FlushRight}
\begin{FlushRight}
    \emph{«І почав від Мойсея й від усіх пророків і виясняв їм з усіх Писань про Нього» (Лк. 24, 27).}
\end{FlushRight}

І апостоли свою проповідь стверджували Святим Письмом. Бо \emph{«ніколи пророцтво не виходило з волі людської, а керовані Духом Святим (його) виголошували святі Божі люди» (2 Петр. 1, 21)}.

\subsection{Скільки є книг Старого Заповіту?}

Їх властиво 50. Але св. Кирило, патріарх Єрусалимський, св. Афанасій Великий і св. Іван Дамаскин нараховують їх 22, згідно з тим, як їх рахували євреї до пришестя Христового на їх первісній мові. Таке розходження в рахунку полягає в тому, що у той час деякі книги зводили в одну, а деякі з різних причин не вводили до канону. Той перерахунок і ми приймаємо тому, що їм (євреям) були доручені Слова Божі (Рим. З, 2), а новозаповітня християнська Церква прийняла книги Старого Заповіту від старозаповітньої.
Книги Старого Заповіту за списком, прийнятим Православною Церквою:
\begin{enumerate}
    \item Буття
    \item Вихід
    \item Левит
    \item Числа
    \item Второзаконня (або Повторення закону) \footnote{Ці перші п`ять книг Старого заповіту називаються Книгами (Законом) Мойсея.}
    \item Книга Ісуса Навина
    \item Судді (з книгою Руф)
    \item Перша і Друга книга Царств
    \item Третя й Четверта книга Царств
    \item Перша й Друга книга Параліпоменон
    \item Перша й Друга книга Ездри (Неємії)
    \item Книга Естер
    \item Книга Іова
    \item Псалтир
    \item Книга Притч (Приповістки Соломонові)
    \item Книга Екклезіаста
    \item Пісня Пісень
    \item Пророка Ісайї
    \item Пророка Єремії
    \item Пророка Єзекіїла
    \item Пророка Даниїла
    \item 12-ти малих пророків.
\end{enumerate} 

У цей список не увійшли книги: Ісуса сина Сирахового, Товіта, Премудрості Соломона, Юдифи, Третя книга Ездри, книги Маккавеїв. Не ввійшли вони тільки тому, що вони не числяться в єврейському списку. Але вони також священні і, як зазначає св. Афанасій, призначені святими Отцями головно для тих, що вступають у Церкву. Для всіх же вірних вони мають велике виховавче значення.

\subsection{Поділ книг Старого Заповіту}

\noindent Старозаповітні книги поділяються на: законодавчі, історичні, учительні і пророчі.

\noindent Законодавчі: п`ять книг Мойсея: Буття, Вихід, Левит, Числа, Второзаконня.

\noindent Історичні: Ісуса Навина, Суддів, Руф, Царств, Параліпоменон, Ездри, Есфір, Маккавеїв.

\noindent Учительні: Іова, Псалтир (книга псалмів), книги Соломона, Товіта, Ісуса сина Сирахового.

Пророчі: Ісайї, Єремії, Єзекіїла, Даниїла, 12-ти малих пророків.

Примітка: Про Псалтир треба знати, що це особливо висока богонатхненна книга, повна молитовного піднесення, найвищої святої поезії. Вона разом і учительна, бо навчає високої побожності, разом і історична, бо показує історію тієї побожності, разом і пророча, бо вміщає в собі багато пророцтв про Христа Спасителя. Псалтир — це найкраще керівництво до молитви, і в молитві та прославленні Бога підносить дух наш до найвищого ступеня. Тому ця книга безперестанно вживається у церковному богослужінні. Псалтир — це настільна книга усіх побожних християн. Вони щодня прочитують по одній або й більше кафізм після ранішніх молитов.

\section{Книги Нового Заповіту}

Новий Заповіт складається з 27 книг: Законодавчих (4 Євангелія), історичної (Діяння святих апостолів), учительних (послання святих Апостолів) та пророчої (Об`явлення св. Івана Богослова).

\subsection{Євангеліє}

Це слово грецьке, що означає благовістя, радісна вість. Євангеліє благовіствує про пришестя на землю обітованого Спасителя світу, Господа нашого Ісуса Христа, про Його Божество, про втілення від Духа Святого і Марії Діви, про Його дивне народження, про життя на землі, про чудесні діла Його та спасительне вчення і, нарешті, про Його хресні страждання, смерть на хресті, поховання, преславне на третій день воскресіння, вознесіння на небо та сидіння праворуч Бога Отця.

Книга ця названа Євангелієм тому, що для людини не може бути радіснішої вістки, як вістка про Божественного Єпасителя та про вічне спасіння. Тому то перед читанням Євангелії і після читання ми радісно співаємо: «Слава Тобі, Господи, слава Тобі».

Євангеліє — це книга законодавча у найвищому розумінні.

Євангеліє написали чотири учні Христові, які називаються євангелистами: Матвій, Марко, Лука та Іван Богослов.

\subsection{Діяння святих апостолів}

У цій історичній книзі Нового Заповіту повіствується про зшестя Євятого Духа на апостолів та поширення через них Церкви Христової по світі між народами.

Слово апостол також грецьке і означає посланник, від того, що Христос послав своїх учнів-апостолів на проповідь.
 
\subsection{Послання апостольські}

Послання святих апостолів — це учительні книги Нового Заповіту, їх є разом 21:

Сім соборних послань, що призначалися для всіх людей (Церков), а саме: послання апостола Якова, 2 послання апостола Петра, 3 послання апостола Івана Богослова і послання апостола Юди (брата Господнього).

Чотирнадцять послань апостола Павла: до Римлян, 2 послання до Коринфян, до Галатів, до Єфесян, до Филіпійців, до Колосян, 2 послання до Солунян, 2 послання до Тимофія, до Тиха, до Филимона, до Євреїв.

\subsection{Об`явлення св. Івана Богослова (Апокаліпсис)}

Це — пророча книга, яка містить у собі таємниче зображення майбутньої долі Церкви Христової та всього світу. Як старозаповітні пророцтва не були зрозумілими, доки не справдилися, так і ця пророча книга Нового Заповіту є таємницею для нас. А все ж таки по ній можна багато догадуватися.

\subsubsection{Яке значення мають пророцтва після того, як вона справдилася?}

Апостол каже, що вони світять, як світильник у темному місці \emph{(2 Петр. 1, 19)}. Вони показують нам, що святі книги написані по натхненню Святого Духа, бо ніхто з людей не може знати майбутнього. Знає тільки Дух Божий.

Здійснені пророцтва наочно показують нам, що Господь дійсно опікується долею людей і скеровує їх до спасіння.

\subsubsection{Як ми довідуємося, що пророцтва від Бога?}

Коли пророк Ісайя провістив про народження Христа Господа від Діви, чого розум людський і помислити не міг, і коли через кілька століть Христос народився від Пресвятої і ПреІІ е порочної Діви Марії, тоді не можна не бачити, що здійснення пророцтва — це діло Боже і що Сам Бог наперед сповістив його. Євангелист Матвій, повіствуючи про народження Христа, наводить і пророцтво Ісайї. Це все сталося, каже він, щоб збулося сказане від Господа: \emph{«Ось Діва в утробі прийме, породить Сина і дадуть йому ім`я Еммануїл, що означає: з нами Бог» (Мт. 1, 22-23; Іс. 7, 14)}.
Таким же доказом Божого піклування про нас є чуда, записані у святих книгах.

\subsubsection{Що таке чудо?}

Це те, чого неможливо зробити ні силою, ні умінням людським\footnote{Не можна вважати за чуда фокусів або наукових відкриттів, бо всі вони діються згідно з законами природи, а чудо — це дія вища від законів природи.}, а тільки всемогутньою силою Божою, наприклад воскресити мертвого.
Чуда також є ознакою Слова Божого. Хто творить правдиве чудо, той діє Божою силою. Значить, він угодний Богові, і в ньому діє Дух Святий. Тому, коли він говорить іменем Божим, то через нього промовляє Боже Слово.

Христос посилається на створені Ним чуда, як на доказ Свого Божественного Посланництва: \emph{{\color{red} «Діла, які дав Мені Отець, щоб чинив Я, ті діла, які Я чиню, свідчать про Мене, що Отець Мене послав»} (їв. 5, 36)}.

\end{document}