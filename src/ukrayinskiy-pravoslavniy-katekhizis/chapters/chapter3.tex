\documentclass[main.tex]{subfiles}

\begin{document}
\chapter{Катехизис}
\section{Що таке релігія?}

Катехизис — це наука релігії.

Під назвою релігія ми розуміємо ставлення наше до Бога і служіння Йому.

Дізнавшись про Бога, що Він Творець усього видимого і невидимого, в тому числі і нас, а також і про Його милі й безмежно високі якості, ми не можемо лишатися байдужими до Нього. Пізнавши Його безмежну ласку та любов до нас, ми не можемо не благоговіти перед Ним, не любити і не прославляти Його.

Ми знаємо, що Бог наш Творець і Отець. Знаємо, що він не байдужий до нас, бо дає нам життя і все потрібне для нього \emph{(Діли. 17, 25)}. І не тільки те, бо:
\begin{FlushRight}
    \emph{«Так Бог полюбив світ, що й Сина Свого Єдинородного віддав, щоб усякий, хто вірує в Нього, не загинув, а мав би життя вічне» (Iв. 3,~16).}
\end{FlushRight}

А коли так, то ми не можемо не дякувати Йому, не прославляти Його і не просити у Нього всього необхідного нам.

Отже, устремління наше до Бога, наша пошана до Нього, наша віра в Нього і виявлення її назовні відповідними ділами, служіння Богові духом та істиною, цебто душею і тілом — це наша релігія. А комплекс священнодій, скерованих на прославлення Бога, як молитви, богослужби, складають наше служіння Богові.

\section{Чому на світі існувало і ще й досі існує не одна, а багато релігій?}

На початку людського життя на землі була лише одна релігія. Люди не мали сумніву, що Бог є, що він — їхній творець, що вони прогнівили Його і за те позбавлені райського блаженства. Вони старалися умилосердити Його праведним життям, молитвами та приношенням жертв, як навчилися вони від праотця Адама. Таке служіння було угодне Богові, і Він багато разів являвся тим праведним людям, як от Єнохові, Авраамові та іншим патріархам.

Одначе ті, що відходили далеко від патріарших осель, поступово втрачали правдиве розуміння Бога. Ідею про Бога вони скрізь понесли з собою, але поступово замість Бога почали шанувати ті творіння, які так або інакше вражали їх, наприклад сонце, місяць, зорі, море, ліси, грім, блискавку, звірів і т. п. Почали їм служити і приносити жертви.

Св. апостол Павло каже про поган:
\begin{FlushRight}
    \emph{«Вони знали Бога, але не прославляли Його, як Бога, а осуєтилися в мудруваннях своїх... і славу нетлінного Бога змінили на образ тлінної людини, птахів, чотириногих та гадів» (Рим. 1, 21, 23).}
\end{FlushRight}

Крім того, різні народи по-різному розуміли Бога. По іншому стали служити тому, кого вважали за бога. Навіть у християнстві багато різноманітностей у служінні Богові, тому що кожний народ вніс дещо своє в розуміння тих чи інших істин, а дехто й просто вносив свої людські міркування, хоч Євангеліє для всіх одне.

\section{Яке ж християнське віровизнання найправдивіше?}

Очевидно, таке, яке в непорушній чистоті оберігає передані святими апостолами Боговідкриті Істини, без додатку будь-яких людських земних міркувань, і не відхилилося від апостольського вчення.
\end{document}