\documentclass[main.tex]{subfiles}

\begin{document}
\chapter{Про автора}

\section{БЛАЖЕНІШИЙ МИТРОПОЛИТ МИХАІЛ}

Він жив на землі, а одночасно перебував у небі, в постійному спілкуванні з Богом. Це була надзвичайно побожна, глибоко релігійна людина. Служінню своєму Творцеві Владика митрополит Михаїл присвятив усе своє життя.

Народився він 10 липня 1885 року у селі Федорівці біля Чигирина. Уже з дитячих років Федот Хороший (це було світське ім`я і прізвище митрополита Михаїла) виростав в атмосфері побожності й любові до Бога, Православної Віри та Церкви, що панувала в родині його батьків — нащадків козацької старшини Никифора і Анастасії Хороших. Успадкувавши від батьків гарний голос, Федот уже з 8-го року життя почав співати на кліросі. Церковний спів він любив усім своїм єством і невдовзі почав писати свої власні композиції.

Вже в ранньому віці Федот Хороший вирішив присвятити себе Христові та Його Церкві. Чи то як диригент церковного хору, чи учитель церковних шкіл, чи паломник до Києва та його святинь, він знав, що його покликання — бути служителем Спасителя Ісуса Христа.

16 грудня 1912 року 27-семирічний Федот Хороший, що до того часу вже закінчив дяківсько-дияконські курси та вищі диригентські курси у Свято-Михайлівському монастирі в Києві, прийняв дияконську хіротонію з рук єпископа Нико-дима. 1915 року вступив екстерном на 5-й курс Київської Духовної Семінарії. Після закінчення Семінарії, він вступив до Київського університету, поєднуючи навчання з дияконським служінням у київських церквах.

24 квітня 1920 року єпископ Димитрій Вербицький рукоположив молодого диякона в сан священника. Пастирське служіння о. Федот почав у селі Тернівці на Черкащині. Вже тут йому довелось стати на захист Православної Віри, даючи гідну відсіч атеїстам, які намагались підірвати авторитет Церкви.

Коли почалось відродження Української Автокефальної Православної Церкви, молодий священник з запалом включився в працю для її розбудови. Маючи на увазі, що однією з головних засад відродженої Церкви була українізація, він запропонував ВПЦРаді свій переклад Псалтиря, виконаний ще 1919 року. Діяльність отця Федота розгортається. Він став настоятелем кафедрального собору Різдва Богородиці в Черкасах, потім головою Черкаської Окружної Ради УАПЦ.

На популярного священника швидко звернула увагу богоборна влада. Його часто викликали на допити, переконували зректися сану, погрожували, залякували. Але ніщо не зламало вірності о. Федота Христові та Його Церкві. З вересня 1929 року його заарештували, піддавали жорстоким тортурам, намагаючись зламати його стійкість, а ЗО квітня 1930 року засудили на 8 років далеких таборів і вислали на каторжні роботи на Кольський півострів. Потім, немов у страхітливому калейдоскопі, острів смерті Конд, Соловки, Ухта-Печорські табори. Та Господь не дав загинути праведникові. Він вижив і 1937 року був звільнений з заслання, до краю виснажений, з підірваним здоров`ям.

Прихід німців у 1941 році застав о. Федота в Кіровограді. В умовах відносної релігійної свободи на початку окупації, він знову віддався праці для Церкви. Організував і очолив Вище Церковне Управління Кіровоградщини, став настоятелем парафії УАПЦ в Кіровограді. Коли стало питання потреби рукоположення нових ієрархів для відродженої УАПЦеркви, отець Федот був одним з логічних кандидатів.
 
12 травня 1942 року в Андріївському соборі у Києві відбулась єпископська хіротонія архимандрита Михаїла (таке ім`я отець Федот вибрав собі, приймаючи чернецтво). Його святителями були єпископи Ніканор та Ігор. Нововисвячений ієрарх був призначений на Кіровоградську єпархію, де він віддано служив своїй великій пастві, рукополагаючи священників, допомагаючи засновувати нові парафії. Все це, незважаючи на тяжкі умови воєнного часу і окупації, ворожого ставлення німців до УАПЦеркви.

На початку 1943 року владика Михаїл, який був до того часу підвищений до сану архиєпископа, очолив Миколаївську єпархію, де продовжував невтомну апостольську діяльність для розвитку церковно-релігійного життя.

З наближенням фронту, ієрархи УАПЦ виїхали на Захід і кінець кінцем опинилися в Німеччині. Після війни вони роз`їхались по різних місцевостях окупованої Німеччини та почали організовувати церковне життя серед тисяч православних утікачів-українців. Осідком архиєпископа Михаїла стало місто Мюнхен. І тут він розгорнув активну діяльність. Організував Церковне Управління, об`їжджав парафії в таборах переміщених осіб, посвячував споруджені храми, руко-полагав священників, заохочував навчання дітей релігії.

Мрією архиєпископа Михаїла було заснування вищої богословської школи для готування кадрів науковців-богословів та високоосвічених священнослужителів. Ця мрія стала дійсністю 24 серпня 1946 року, коли Священний Синод УАПЦеркви схвалив заснування Богословсько-Педагогічної Академії в Мюнхені. На почесного голову кураторі! покликано Митрополита Полікарпа, а на урядуючого куратора — архиєпископа Михаїла. Ректором Академії став професор Пантелеймон Ковалів.

Діючи в тяжких умовах повоєнного часу, Богословсько-Педагогічна Академія успішно виконувала своє завдання. Владика Михаїл був душею і натхненням для студентів. Його лекції слухали з непослабною увагою. Участь в богослуженнях, що їх він відправляв у церкві на Дахауерштрассе або в академічній каплиці були для студентів-іподияконів, дияконів і священників справжньою духовною насолодою.

У 1948 році архиєпископ Михаїл виїхав до Бельгії, де організував життя УАПЦ у тій країні та в Голландії і Люксембурзі. А 1951 року виїхав на запрошення консисторії Української Греко-Православної Церкви в Канаді на становище правлячого єпископа. Одначе у той час до УГПЦеркви виявив бажання приєднатися митрополит Іларіон. Для добра, єдності і спокою в Церкві владика Михаїл на Надзвичайному Соборі 8-9 серпня 1951 добровільно зняв свою кандидатуру на правлячого єпископа і запропонував кандидатуту митрополита Іларіона, а сам погодився стати його заступником і архиєпископом Торонта та Східної Канади.

Під керівництвом владики Михаїла, Східна єпархія УГПЦеркви швидко росла і розвивалась. Виростали нові парафії, будувались церкви, серед них такі перлини української церковної архітектури, як собор св. Софії у Монтреалі, св. Володимира у Гамільтоні, св. Димитрія у Лонґ Бренч та інші. Архиєпископ Михаїл регулярно об`їжджав громади своєї єпархії, цікавився їхнім життям, давав поради, заохочення.

Після смерті митрополита Іларіона в 1972 році владика Михаїл став першоієрархом УГПЦеркви в Канаді, прийнявши титул Митрополита Вінніпегу і всієї Канади. Але з огляду на стан здоров`я, він 1975 року попросив звільнити його від обов`язків голови Церкви.

Пауково-богословський доробок архиєпископа Михаїла включає, крім «Поширеного катехизиса», такі твори: «Пояснення трудних місць Євангелії від Івана (1956), трилогія «Світова епопея» (1954-1956), «Духовний світ і душа людини» (1961), «Біблійне оповідання про походження світу, землі й людини в порівнянні до наук астрономії, геології, біології та інш.» (1963), «Віра в Месію — Христа Спасителя в Старому Заповіті і Христова Церква», переклади Псалтиря, акафіста св. Пантелеймону і служби святителю Христовому Миколаю, багато проповідей, статей і рецензій.

Серед численних музичних композицій митрополита Михаїла — 4 Херувимські, 2 «Милість миру», постові й великодні пісні, канони на день Св. Тройці та Преображення Господнього, колядки, ораторія на слова Шевченка «Муза» й ін.

Митрополит Михаїл помер у Торонті 18 травня 1977 року. Тисячі православних українців Канади з глибоким сумом проводили у вічну дорогу свого улюбленого архипастиря. Вічна йому пам`ять!
\end{document}