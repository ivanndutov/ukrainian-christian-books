\documentclass[chapters.tex]{subfiles}

\begin{document}
\chapter{Молитви до святих}
\section{Молитва до святого Івана Предтечі}
Хрестителю Христів, проповіднику покаяння, не відкинь мене, що каюся, але, з’єднуючись із небесними силами, молися до Владики за мене, недостойного, сумного, немічного і печального, що потрапив у різні біди житейські, змученого хвилюючими думками мого розуму; бо я — вертеп злих діл, зовсім не маю кінця гріховним звичкам, бо розум мій прикутий до земних речей. Що маю робити? Не знаю. І до кого звернуся, щоб спасти свою душу? Тільки до тебе, святий Іоанне, благодаті тезоіменний, звертаюся, бо перед Господом після Богородиці немає більшого від народжених. Ти сподобився торкнутися глави Царя Христа, Агнця Божого, що взяв на Себе гріхи світу. Його ж благай за грішну мою душу, щоб хоч віднині, в одинадцяту годину, я поніс тягар благий і нагороди прийняв з останніми. Так, Хрестителю Христів, чесний Предтече, перший у благодаті, мученику, посників і пустельників наставнику, чистоти учителю і близький Христа друже! Тебе молю і до тебе вдаюся: не відкинь мене від твого заступництва, але підніми мене, що впав у великі гріхи. Обнови душу мою покаянням, як другим хрещенням, бо ти обох начальник: хрещенням омиваєш праотцівський гріх, покаянням же очищаєш кожного скверне діло. Очисти мене, оскверненого гріхами, і допоможи ввійти туди, куди ніщо скверне не входить — у Царство Небесне. Амінь.

\section{Молитва до святого апостола і євангеліста Івана Богослова}
О, великий апостоле, євангелісте громогласний, Богослове і тайновидче невимовних одкровень, дівственику і улюблений наперснику Христів Іоанне! Прийми нас грішних, що до міцного заступництва твого звертаємося. Випроси у Всещедрого Чоловіколюбця Христа Бога нашого, бо Він перед очима твоїми Кров Свою за нас пролив, недостойних рабів Своїх, щоб не пам’ятав беззаконь наших, а помилував нас і вчинив з нами по милості Своїй; нехай дарує нам здоров’я душевне і тілесне, благоденственне життя і достаток, наставляючи нас звертати все це на славу Його, Творця, Спасителя і Бога нашого, а після закінчення тимчасового життя нашого нехай позбавить нас немилосердних мучителів на повітряних митарствах, щоб досягли ми, тобою ведені, небесного Єрусалима, його ж славу ти в одкровенні бачив і нею в нескінченній радості насолоджуєшся. О великий Іоанне! Збережи всі міста і країни християнські, храм цей і всіх, хто служить і молиться в ньому, від голоду, мору, землетрусу і потопу, вогню і меча, нашестя іноплемінників та міжусобної боротьби; визволи нас від усякої біди та напасті і молитвами твоїми відверни від нас праведний гнів Божий, а Його милосердя нам випроси, щоб разом з тобою сподобилися прославляти в невечірньому дні пресвяте ім’я Отця, і Сина, і Святого Духа навіки-віків. Амінь.

\section{Молитва до святого отця Миколая Чудотворця}
О, всеблагий отче наш Миколаю, пастирю й вчителю всіх, хто з вірою звертається під захист твій і теплою молитвою благає тебе: поспіши й захисти стадо Христове від вовків, що гублять його. Україну нашу й всяку країну християнську та й увесь світ захисти й збережи святими твоїми молитвами від землетрусу, наступу чужинців і народних заколотів, міжусобиці; від голоду, пошесті, потопу, вогню, меча й наглої смерті всяку людину охорони. І як помилував ти трьох чоловіків, що у в’язниці сиділи, та спас від гніву царського й кари смертної, так помилуй і нас, що розумом, словом і ділом у пітьмі гріхів перебуваємо. Спаси нас від гніву Божого і вічної кари, щоб за твоє клопотання й поміч та Своєю благодаттю й милосердям подав нам Христос Бог тихо і без гріха прожити на цім світі, і не засудив нас стояти ліворуч, а сподобив стати праворуч зі всіма святими й увійти в Царство вічної слави Бога. Амінь.

\section{Молитва до святого рівноапостольного князя Володимира}
О, великий угоднику Божий, богообраний і богопрославлений, рівноапостольний княже Володимире! Ти відкинув зловір’я і нечестя язичницьке, увірував у єдиного істинного триіпостасного Бога, прийняв святе хрещення і просвітив світлом божественної віри і благочестя усю країну Руську. Тому, славлячи і дякуючи премилосердному Творцеві й Спасителеві нашому, славимо й дякуємо тобі, просвітителю наш і отче, бо через Тебе ми пізнали спасительну віру Христову й хрестилися в ім’я Пресвятої і Пребожественної Тройці; цією вірою ми визволилися від праведного засудження Божого, вічного рабства дияволу і пекельних мук; цією вірою ми прийняли благодать усиновлення Богові й надію на успадкування небесного блаженства. Ти є перший вождь до Начальника та звершувача нашого вічного спасіння Господа Ісуса Христа; ти є теплий молитовник і заступник землі нашої і всіх людей твоїх. Не може язик наш висловити велич та висоту благодіянь, вилитих через тебе на землю нашу, отців і праотців наших і на нас недостойних.

О всеблагий отче і просвітителю наш! Зглянься на немочі наші і благай премилосердного Царя Небесного, щоб не прогнівався Він дуже на нас, бо через немочі наші ми щодня чинимо гріхи, щоб не погубив Він нас за беззаконня наші, але нехай помилує і спасе нас, з милості Своєї, нехай вселить у серце наше спасенний страх Свій, нехай просвітить Своєю благодаттю розум наш, щоб ми пізнавали шляхи Господні, залишили стежки нечестя й омани і подвизалися на стежках спасіння та істини, неухильно виконуючи заповіді Божі та настанови Святої Церкви. Благай, благосердний, Чоловіколюбця Бога, щоб Він продовжив на нас велику милість Свою: нехай визволить нас від нападу чужинців, від внутрішнього безладдя, заколотів і чвар, від голоду, смертельних хвороб і від усякого зла; нехай подасть нам добре полиття і врожай плодів земних, нехай визволить нашу землю від усякого лиха, нехай дасть пастирям ревність за спасіння пастви, а всім людям поміч у старанному звершенні служби своєї, і щоб вони мали між собою любов та однодумність і подвизалися з вірою на благо Святої Церкви; і нехай засяє світло спасенної віри в Україні нашій по всіх кінцях її, і нехай навернуться до віри невіруючі, і нехай зникнуть усі єресі і розколи, щоб так поживши у мирі на землі, ми сподобилися з тобою вічного блаженства, хвалячи і величаючи Бога навіки-віків. Амінь.

\section{Молитва до святої рівноапостольної княгині Ольги}
О, свята рівноапостольна княгине Ольго, першоугоднице руська, тепла за нас перед Богом заступнице і молитовнице. До тебе вдаємося з вірою і з любов’ю благаємо: будь нам у всьому доброму помічницею і сподвижницею. Як ти за життя намагалася просвітити праотців наших світлом святої віри і навчити їх творити волю Господню, так і нині, перебуваючи в небесних оселях, богоприємними твоїми молитвами допомагай нам у просвітленні розуму і серця нашого світлом Христового Євангелія, щоб ми вдосконалювалися у вірі, благочесті й любові Христовій. Тих, що у скорботах і убогості, утіш; тим, що у бідності, подай руку допомоги; покривджених і тих, хто терпить напади, захисти, а засліплених єресями, відпалих від істинної віри надоум. Виблагай для нас у Всещедрого Бога всілякі блага, потрібні для життя земного і вічного, щоб ми, богоугодно проживши, сподобилися стати спадкоємцями в Царстві Христа Бога нашого, Якому з Отцем і Святим Духом належить всяка слава, честь і поклоніння навіки-віків. Амінь.

\section{Молитва до святого великомученика і цілителя Пантелеймона}
О, великий Христів угоднику і преславний цілителю, великомученику Пантелеймоне! Ти душею на небесах перед престолом Божим предстоїш і триіпостасною Його славою насолоджуєшся, а тілом і іконою святою на землі в Божественних храмах перебуваєш і даною тобі благодаттю різні чудеса твориш. Споглянь милосердним оком твоїм на людей, що стоять і перед чесною іконою твоєю моляться, благаючи від тебе цілющої допомоги і заступництва; піднеси до Господа Бога нашого теплі твої молитви і випроси душам нашим прощення гріхів. Бо ми через беззаконня наші не можемо підняти очей до висоти небесної, ні піднести голосу моління до Нього в Божестві неприступної слави; серцем сокрушеним і духом смиренним тебе, заступника милостивого перед Владикою і молитовника за нас, грішних, закликаємо, бо ти отримав благодать від Господа недуги відганяти і пристрасті зціляти. Тебе просимо: не зневаж нас, недостойних, що молимося тобі й помочі твоєї потребуємо, будь нам у печалях утішителем, у недугах лютих лікарем, у напастях покровителем, для хворих очей прозріння подателем, дітям хворим помічником. Випроси всім все, що для їхнього спасіння корисне, щоб твоїми до Господа Бога молитвами отримали благодать і милість і прославляли всіх благ Джерело і Дароподателя Бога, Єдиного в Тройці Святій, Отця, і Сина, і Святого Духа, нині, і повсякчас, і навіки-віків. Амінь.

\section{Молитва до великомученика Юрія Переможця}
О, всехвальний святий великомученику і чудотворцю Юрію! Зглянься на нас швидкою твоєю допомогою і ублагай чоловіколюбця Бога, щоб не осудив нас, грішних, за беззаконня наші, а сотворив з нами за великою Своєю милістю. Не зневаж моління нашого, але виблагай нам у Христа і Бога нашого тихе і богоугодне життя, здоров’я ж душевне і тілесне, землі плодючість і в усьому достаток; і щоб ми не перетворювали на зло добро, що дарується нам тобою від всещедрого Бога, але на славу святого імені Його і на прославлення кріпкого твого заступництва, щоб подав Він країні нашій, боголюбивому воїнству її і всім православним християнам на супротивників перемогу і щоб укріпив державу нашу постійним миром і благословенням. Особливо ж щоб охороняв нас ополченням святих Своїх ангелів і щоб визволитися нам, після відходження нашого із цього життя, від підступів диявола і від тяжких повітряних митарств і неосудженими стати перед престолом Господа слави. Почуй нас, страстотерпцю Христів Юрію, і благай за нас безперестанно Триіпостасного Владику Бога всіх, щоб Його благодаттю і чоловіколюбством, твоєю ж допомогою і заступництвом ми знайшли милість і стали праворуч Праведного Судді з ангелами і архангелами і всіма святими, і Його завжди славили з Отцем і Святим Духом нині, і повсякчас, і навіки-віків. Амінь.

\section{Молитва до святої великомучениці Варвари}
Свята славна і всехвальна великомученице Христова Варваро! Зібрані сьогодні в храмі твоєму Божественному люди, перед ракою мощей твоїх поклоняючись і з любов’ю їх цілуючи, страждання ж твої мученицькі й у них Самого страстоположника Христа, який дав тобі не тільки в Нього вірувати, але й за Нього страждати, похвалами ублажаючи, молять тебе, відому бажань наших просительку, моли з нами і за нас умоленого від Свого благоутробія Бога, щоб милостиво вислухав тих, хто просить Його благостиню, щоб не відвів від нас всі для спасіння і життя потрібні прохання і дарував християнську кончину життя нашого безболісну, безгрішну і мирну, до Божественних Тайн причетну; і всім, що на всякому місці, в усякій скорботі й напастях потребують Його чоловіколюбства і допомоги, велику Свою подав милість, щоб благодаттю Божою і теплим заступництвом, душею і тілом завжди в здоров’ї перебуваючи, славили дивного у святих Своїх Бога Ізраїлевого, Який не віддаляє допомоги Своєї від нас, завжди, нині, і повсякчас, і навіки-віків. Амінь.

\section{Молитва до святого священномученика Макарія}
О, великий угоднику Христів, святителю отче Макарію, в житії твоєму вірних твоїх любов’ю, більшою за смерть, ти полюбив і за них душу свою віддав, споглянь з оселі небесної на нас, зваж на зітхання наші сердечні й допоможи нам недостойним і смиренним. Молися, святителю милостивий, щоб послав Він мир на людей Своїх, і на преподобних Своїх, і на тих, хто звертає до Нього серця свої. Даруй нам, отче наш, дух любові твоєї до святої Матері нашої Церкви Христової; укріпи нас, розслаблених духом, отцівські передання зберігати і про минулі напасті пам’ять спасительну мати, і завжди подячні пісні благодійнику Богові співати навчи; заступництвом твоїм перед престолом Господа сил виблагай Церкві Христовій від єресей і розколів стіну непорушну; для тих, хто до мощей твоїх приходить і шанує святу пам’ять твою, сповнення благих їхніх прохань подай і всім християнам все для тимчасового і вічного життя потрібне даруй, щоб прославлялося через тебе пречесне і величне ім’я Отця, і Сина, і Святого Духа нині, і повсякчас, і навіки-віків. Амінь.

\section{Молитва до преподобних Антонія і Феодосія Києво-Печерських}
Преподобні і богоносні отці наші, Антонію і Феодосію, ми, грішні і смиренні, щиро звертаємося до вас як до теплих заступників, і скорих помічників, і відомих предстоятелів, смиренно просячи вашої допомоги і заступництва, бо занурені в безодні зла і бід, які кожного дня і години надходять на нас і від лукавих людей, і від злих духів піднебесної, що прагнуть завжди, всюди і різними засобами нашої загибелі. Ми, без сумніву, знаємо, яку велику сміливість ви маєте перед милосердним Богом, бо, ще на землі перебуваючи між подорожуючими до Небесної Вітчизни, ви на собі показали велику силу благодаті Божої, яка чудодіяла вашими устами і руками: бо звели і вогонь з неба, подібно до Іллі, для визначення місця, на якому мала бути заснована на честь і постійне славослов’я Бога і Матері Божої велика Печерська церква, і росу звели по образу Гедеона на очищення і вічне прославлення того ж святого місця. Також знаємо, яка велика кількість мужів і жон із різних країн і народів, скорботних і в напастях сущих або одержимих тяжкими хворобами, або тих, що підпали під насилля і утиски нестерпні і зневірилися у своєму житті, вашими молитвами і заступництвом одержали швидке полегшення і визволення. Якщо, будучи ще у земному смертному подорожуванні, ви милостиво подавали таку допомогу бідуючим, то тим більше тепер, коли ви стоїте перед Всемогутньою Тройцею і більшу маєте сміливість, щоб благати за нас недостойних і нас, сущих у бідах і печалях, утішати, в тяжкій недолі і напастях за нас заступатися, в нещастях і бідах нас захищати. Заради цього і ми, що перебуваємо всюди під загрозами і утисками від безлічі ворожих наклепів, скорбот і зловмисних підступів, доручаємо вашому теплому і сильному після Бога покрову і захисту і щиро молимо вашу благість: збережіть нас неушкодженими від усіх бід і лиха, а найбільше від бісівських підступів і хитрощів, лестощів і несподіваних нападів, щоб не стали ми ганьбою і посміянням, але відженіть їх від нас міцною вашою силою, як у колишні дні ви відганяли від обителі тих, що велику шкоду чинили. Такі ж їхні шкідливі на нас повстання приборкавши, утвердіть нас у вірі, надії і любові, щоб ніколи не оволоділо нами ніяке нерозуміння або сумнів у тому, чого навчає нас вірувати Свята Матір Церква і повеліває сміливо сповідувати; надію нашу на Господа Бога в душах наших влаштуйте вагою і мірою правди і милості Божої, щоб ми безмірно не сподівалися на Бога без трудів і подвигів одержати, а також, дивлячись на великі гріхи і тяжкі злочини, не зневірялися у Божому милосерді. Любов у серцях наших утвердіть і вчиніть достойною, щоб ні про що земне і швидкоплинне більше за Бога, Який все створив і все Собою тримає, не помишляли, не бажали, не віддавали перевагу. Почуття наші душевні й тілесні у такій благоліпній мірі кожного дня і години зберігайте, щоб ми ніколи ними не прогнівляли благого і чоловіколюбного Бога. Розум очистіть, щоб він думав про Бога і Його всюдиприсутність і благе провидіння, ніж турбувався про тимчасове і нікчемне. Розбещену волю нашу виправляйте, щоб вона ніколи не бажала противного волі Божій, але тим задовольнялася і щоб у тому безтурботно і спокійно перебувала, що Богу угодне і приємне, людині ж спасительне і корисне. Пам’ять витверезуйте, щоб вона постійно представляла розуму те, чим прогнівила всемилостивого Бога і Його благосердя, і те, що кожну людину, без усякого сумніву, після земного тимчасового життя чекає. Крім того, і Вітчизни вашої не забувайте, але безперестанно благайте милостивого Бога про мир і спокій її. Всіх людей, сущих в країні нашій, зберігайте в спокої і без печалі і визволяйте від усякої злої недолі. Коли ж прийде час нашого відходу із тимчасового життя і переселення у вічність, прийдіть нам на допомогу і від насильства ворожого визволіть, як прийшли ви колись до ченця Еразма, який перебував у тяжких смертних обставинах, і спонукайте серце наше до істинного покаяння у гріхах, як і його спонукали, щоб він відійшов до Бога в істиннім покаянні, щоб і ми в чистій совісті постали перед Пресвятою і нероздільною Тройцею, щоб прославляти Її разом з вами і всіма святими в безкінечні віки. Амінь.

\section{Молитва до преподобного отця нашого Іова Почаївського}
О, всесвятий і преславний угоднику Божий, преподобний отче наш Іове, постійний за нас до Господа молитвенику і теплий за душі наші заступнику; до тебе нині з розчуленістю звертаємося і, згадуючи подвиги і чудеса, які ти створив і твориш на землі, просимо і молимо твою благість: як твердо і непохитно ти потрудився у вірі Христа Бога нашого і її до кінця в собі й у всіх ближніх твоїх цілою і неушкодженою зберіг від усяких нападів ворожих і єресей розтлінних, так і нас у православ’ї і однодумності укріпи, відганяючи твоїми молитвами всяку темряву невір’я і неправомислення від сердець і думок наших. Ти послужив Господу і Богу твоєму благими ділами і самозреченням у трудах, нічних молитвах і пості, настав і нас на путь усяких чеснот і благості, визволяючи від спокус і гріхів, що віддаляють нас від Бога і опускають у безодню зла все життя наше; ти колись явився з Пречистою Дівою Богородицею на верху гори Почаївської, щоб спасти твою обитель від нашестя і оточення агарянського, і нині прийди на поміч Богом береженій Україні нашій проти усіх ворогів наших зовнішніх і внутрішніх, утверджуючи мир і спокій на землі нашій, щоб твоїми молитвами і заступництвом ми тихе і безмовне житіє прожили у всякому благочесті й чистоті. І всім, хто до тебе вдається і припадає до раки чесних і многоцілющих мощей твоїх і твоєї допомоги і заступництва потребує незаздрісно, подавай невичерпні милості; не залиш і нас сиротами і безпомічними, що молимося тобі, визволяючи від усякої скорботи, гніву і недолі, від голоду, згуби, землетрусу, повені, вогню, меча, нашестя іноплемінників і міжусобної боротьби. Так, угоднику Божий, зглянься милостиво від престолу Царя слави, Якому ти нині предстоїш з архангелами і ангелами і зі всіма святими, на обитель твою Почаївську, якою ти мудро правив, укріпивши її всехвальним і дивним твоїм житієм, і збережи її молитвами твоїми, і всяке місто, і країну, і всіх усюди: на морі, на суші, в повітрі, в пустелях і в ув’язненнях різноманітних, що тебе закликають, від усякого видимого і невидимого зла; щоб так твоєю допомогою і заступництвом спасенні, у цьому віці і після кінця життя нашого сподобилися разом з тобою славити і оспівувати всечесне ім’я Отця, і Сина, і Святого Духа навіки-віків. Амінь.

\section{Молитва до преподобного Серафима Саровського}
О, предивний отче Серафиме, великий Саровський чудотворцю, всім, хто до тебе звертається, скоропослушний помічнику! У дні земного твого життя ніхто від тебе без нічого і непочутим не відходив, але для всіх був солодким вигляд лиця твого і благопривітним голос слів твоїх. До цього ж ти мав великий дар лікування немічних душ, дар зцілення, дар прозріння. Коли ж покликав тебе Бог від земних трудів до небесного упокоєння, любов твоя до нас ніколи не переставала, і неможливо перелічити чудеса твої, що примножилися, як зірки небесні: ось бо ти являєшся людям Божим усіх країв землі нашої і даруєш їм зцілення. Тому ми взиваємо до тебе: о претихий і лагідний угоднику Божий, сміливий перед Ним молитвенику, що ніколи не відганяєш тих, хто закликає, піднеси за нас благомогутню твою молитву до Господа сил, щоб Він дарував нам усе благоподібне для цього життя і все, що корисне для душевного спасіння, щоб Він оберігав нас від гріховних падінь і істинного покаяння навчив нас, щоб нам безперешкодно ввійти в Небесне Царство, де ти нині у незаходимій славі сяєш, і там оспівувати з усіма святими Живоначальну Тройцю до кінця віків. Амінь.

\section{Молитва до пророка Божого Іллі}
О, прехвальний і пречудний пророче Божий Іллє, що засяяв на землі рівноангельським житієм твоїм, полум’яною ревністю по Господу Богові Вседержителеві, ще ж знаменнями і чудесами преславними, також по великому благоволінню до тебе Божому ти був узятий на вогненій колісниці з тілом твоїм на небо, сподобився розмовляти зі Спасителем світу, Який преобразився на Фаворі, і нині в райських оселях безперестанно перебуваєш і стоїш перед престолом Небесного Царя! Почуй нас, грішних і недостойних, що стоїмо перед святою твоєю іконою і щиро звертаємося до твого заступництва. Моли за нас Чоловіколюбного Бога, щоб подав нам дух покаяння і сокрушення за гріхи наші, і всесильною Своєю благодаттю нехай допоможе нам залишити путі нечестя, мати ж успіх у всякій добрій справі, нехай укріпить нас у боротьбі з пристрастями і похотями нашими, нехай вселить у серця наші дух смирення і лагідності, дух братолюбства і незлобливості, дух терпіння і цнотливості, дух ревнощів до слави Божої і щоб про спасіння своє і ближніх добре піклувалися. Знищ молитвами твоїми, пророче, злі звичаї світу, найперше ж тлінний дух віку цього, що розтліває християнський рід неповагою до Божественної православної віри, до уставу Святої Церкви і до заповідей Господніх, неповагою до батьків і влади і кидає людей у безодню нечестя, розбещення і погибелі. Відверни від нас, пречудний пророче, заступництвом твоїм праведний гнів Божий і позбав усі міста і села України нашої бездощів’я і голоду, страшних бур і землетрусів, смертоносних моровиць і хвороб, нашестя ворогів і міжусобної боротьби. Укріпи твоїми молитвами, преславний, владу нашу, яка у великому і важкому подвигу трудиться, сприяй їй у всіх благих діяннях і в починаннях по впровадженню миру і правди в країні нашій. Допомагай христолюбивому воїнству в боротьбі з ворогами нашими. Виблагай, пророче Божий, від Господа пастирям нашим ревнощі по Богу, сердечне піклування про спасіння пастви, мудрість у навчанні й управлінні, благочестя і міцність у спокусах, суддям виблагай непідкупність і безкорисливість, правоту і співчуття до скривджених, всім, хто начальствує, піклування про підлеглих, милість і правосуддя, підлеглим же покірність і слухняність. Щоб так у світлі і благочесті поживши в цьому віці, сподобилися причастя вічних благ у Царстві Господа і Спаса нашого Ісуса Христа, Йому ж належить честь і поклоніння з Безначальним Його Отцем і Пресвятим Духом навіки-віків. Амінь.

\section{Молитва до святих першоверховних апостолів Петра і Павла}
О, преславні апостоли Петре і Павле, ви душі за Христа віддали і кров’ю вашою пасовище Його удобрили! Почуйте дітей ваших молитви і зітхання, що з серцем сокрушеним нині вам приносяться. Ось бо ми беззаконнями затьмарились і через це бідами, наче хмарами, покрилися, на єлей доброго життя зубожіли сильно і не можемо противитися вовкам хижим, які зухвало намагаються розкрадати спадок Божий. О сильні! Понесіть неміч нашу, не відлучайтеся духом від нас, щоб не відлучились ми від любові Божої, але твердим заступництвом вашим захистіть, щоб помилував Господь усіх нас, молитов ваших ради, щоб знищив рукописання безмірних гріхів наших і сподобив зі всіма святими блаженного Царства і весілля Агнця Свого, Йому ж честь і слава, і подяка і поклоніння навіки-віків. Амінь.

\section{Молитва до святого апостола Андрія Первозванного}
Первозванний апостол Бога і Спаса нашого Ісуса Христа, Церкви послідовнику верховний, всехвальний Андрію, славимо і величаємо апостольські труди твої, з радістю згадуємо твоє благословенне до нас пришестя, ублажаємо чесні страждання твої, шануємо святу пам’ять твою і віруємо, що живий Господь і жива душа твоя і з Ним повіки перебуваєш на небесах, звідки і любиш нас тією ж любов’ю, якою ти полюбив нас, коли Духом Святим передбачив наше до Христа навернення, і не тільки любиш, але і молишся за нас Богу, бачачи у світлі Його всі наші біди. Так віруємо і так цю нашу віру сповідуємо у храмі, що в ім’я твоє, святий Андрію, славно збудований; віруючи ж, просимо і благаємо Господа і Бога і Спаса нашого Ісуса Христа, щоб твоїми молитвами, які Він завжди слухає і приймає, подасть нам усе потрібне для спасіння нас грішних; щоб як ти зразу за голосом Господа, залишивши сіті свої, впевнено пішов за Ним, так і кожний із нас нехай не шукає своєї користі, але про те, що на благо ближнього, і про небесне покликання нехай помишляє. Маючи ж тебе заступником і молитвеником за нас, уповаємо, що молитва твоя багато може перед Господом і Спасителем нашим Ісусом Христом, Йому ж подобає всяка слава, честь і поклоніння з Отцем і Святим Духом навіки-віків. Амінь.

\section{Молитва до трьох святителів}
О, пресвітлі світильники Церкви Христової, Василію, Григорію і Іоане, світлом православних догматів ви осяяли всі краї землі і мечем Слова Божого засмучення і хитання єресей погасили; припадаючи до нашого милосердя, з вірою і любов’ю із глибини душі взиваємо: стоячи перед престолом Пресвятої, Єдиносущної, Животворчої і Нероздільної Тройці, за Яку ви словом, писанням і житієм добре подвизалися і душі свої поклали, завжди моліться до Неї, щоб Вона укріпила і нас у православ’ї і однодумності, і непохитному навіть до смерті сповіданні віри Христової, і у вседушевному послуху Його Церкві святій; щоб Він опоясав нас із висоти силою на всіх видимих і невидимих ворогів наших; щоб Він зберіг Церкву Свою нерозхитаною від невір’я, марновір’я, єресей і розколів; щоб Він дарував нашим архіпастирям здоров’я, довгоденство і у всьому благий успіх; пастирям нашим щоб подав духовну тверезість і ревність у спасінні пастви; правителям — суд і правду, воїнам — терпіння, мужність і подолання ворогів, сиротам і вдовам — заступництво, хворим — зцілення, юним — добре у вірі зростання, старцям — утіху, покривдженим — захист, і всім усе, що для тимчасового життя потрібне, щоб у мирі та покаянні, палаючи бажанням спасіння, Господеві працюючи, добрим подвигом подвизаючись, життєвий шлях закінчивши, ми сподобилися у Царстві Небесному разом з вами завжди оспівувати і славити Пресвяте і величне ім’я Отця, і Сина, і Святого Духа навіки-віків. Амінь.

\section{Молитва до всіх святих, що відвіку Богу вгодили}
О, преблаженні угодники Божі, всі святі, що стоїте перед престолом Пресвятої Тройці і насолоджуєтеся невимовним блаженством! Ось нині, в день загального вашого торжества, милостиво спогляньте на нас, менших ваших братів і сестер, що приносимо вам цей похвальний спів і заступництвом вашим просимо милості і відпущення гріхів у Преблагого Господа; знаємо, воістину знаємо, що все, що ви хочете, можете виблагати у Нього. Тому смиренно молимося до вас: моліть милостивого Владику, щоб Він подав нам дух вашої ревності у виконанні святих Його заповідей, щоб, ідучи вашими стопами, ми змогли пройти земне поприще у побожному без пороку житті і в покаянні досягти преславних осель райських, і там разом з вами прославляти Отця, і Сина, і Святого Духа навіки-віків. Амінь.

\section{Молитва до страстотерпців Бориса і Гліба}
О, двоїце священна, братіє прекрасна, мужні страстотерпці Борисе і Глібе, ви від юності Христу вірою, чистотою і правдою послужили і кров’ю своєю, наче багряницею, прикрасились; і нині з Христом царюючи, не забудьте і нас, на землі сущих, але, як теплі заступники, вашим сильним заступництвом перед Христом Богом збережіть юних у святій вірі і чистоті неушкодженими від усякого навіювання невір’я і нечистоти, огородіть усіх нас від усякої скорботи, озлоблень і несподіваної смерті, втихомирте всяку ворожнечу і лють, що викликаються діями близьких і чужих. Молимо вас, христолюбиві страстотерпці, виблагайте у великодаровитого Владики всім нам відпущення гріхів наших, однодумність і здоров’я, визволення від нашестя іноплемінників, міжусобної боротьби, язви і голоду. Обдаровуйте своїм заступництвом усіх, хто шанує святу пам’ять вашу навіки-віків. Амінь.

\section{Молитва до всіх преподобних Печерських}
Преподобні і богоносні отці наші Антонію і Феодосію та всі преподобні Печерські, світила землі пресвітлої, що від темних печер землі нашої славно засяяли і її всю багатьма світло-сяйними ангелоподібними зорями освітили, всю ж вселенну високих чеснот і чудес Божественних сяянням здивували, і сьогодні, у смертному гробі тілом перебуваючи, душі ваші з Христом, Сонцем Правди, разом з усіма праведниками сяють, як сонце в Царстві Небеснім; звідтіля молитовне ваше проміння до Бога Світла за вітчизну свою посилаєте, не забудьте і нас, що молимося до вас, перебуваючи в ночі скорбот і пристрастей наших, промінням благодаті осяйте нас, щоб ми ходили правдиво у світлі чеснот вашого життя і сподобилися бачити світло недоторканної слави Божої, прославляючи Його разом з вами навіки-віків. Амінь.

\section{Молитва до преподобного Антонія Печерського}
О, добрий пастирю і наставнику, приснозгадуваний ченців руських першоначальнику, преподобний і богоносний отче наш Антонію! Ти вгорі на небесах, а ми внизу на землі, віддалені від тебе не стільки місцем, скільки гріховною нашою нечистотою, однак, пам’ятаючи твою батьківську любов до людей, рідних тобі, припадаємо і молимося зі зворушенням і вірою: допоможи нам, грішним, очиститися покаянням і бути достойними помилування і прощення від Господа і Сотворителя нашого. Умоли Його благість дарувати нам великі і багаті милості, добре поліття, міцний мир, нелицемірну братолюбність і благочестя, достаток плодів земних, і щоб ми не перетворювали на зло добро, що дарується нам від щедроподательної правиці Його, але на славу імені Його святого і для нашого спасіння. Збережи, угоднику Божий, твоїм святим заступництвом усіх нас. Охорони, чудотворцю святий, благосильними твоїми молитвами обитель твою, і всю Україну нашу неушкодженими від усякого зла, і всіх людей, що живуть в обителі твоїй і тих, що на поклоніння до неї приходять, осіни небесним твоїм благословенням і в скорботах, бідах і хворобах подай їм утіху, визволення і зцілення; щоб ми подячно славили, хвалили і величали Господа, Який прославив тебе, і через тебе дивно благодіє нам, безначального Отця, Єдинородного Його Сина і Єдиносущного Його Духа, Тройцю живоначальну і нероздільну, і твоє святе заступництво навіки-віків. Амінь.

\section{Молитва до преподобного Феодосія Печерського}
О, священна главо, ангеле земний і людино небесна, преподобний і богоносний отче наш Феодосію, прекрасний слуга Пресвятої Богородиці, що в ім’я Її святе предивну обитель на горах Печерських влаштував і в ній безліччю чудес засяяв! Благаємо тебе з великою щирістю, молися за нас Господу Богу і виблагай у Нього великі і багаті милості: віру праву, надію на спасіння несумнівну, любов до всіх нелицемірну, благочестя непохитне, душ і тіл здоров’я, достаток земних потреб, і щоб не обертали на зло добро, що дарується нам від щедроподательної правиці Його, але у славу імені Його святого і для нашого спасіння. Збережи, преподобний чудотворцю, вітчизну твою земну, Україну, всі міста і села її, і лавру твою неушкодженими від усякого зла; і всіх людей, які приходять на поклоніння до чесного твого гробу і перебувають у святій твоїй обителі, осіни небесним твоїм благоволінням і милостиво позбав від усякого зла. Найбільше ж у час кончини нашої покажи нам багатосильне твоє покровительство, щоб ми позбавилися, твоїми молитвами до Господа, влади лютого світоправителя і сподобилися успадкувати Царство Небесне. Яви нам, отче, твоє благосердя і не залиш нас сиротами і безпомічними; щоб ми завжди славословили дивного у святих Своїх Бога, Отця, і Сина, і Святого Духа, і твоє святе заступництво навіки-віків. Амінь.

\section{Молитва до святих мучениць Віри, Надії і Любові та матері їхньої СОФІЇ}
О, святі і достохвальні мучениці Віро, Надіє і Любове і мужніх дочок мудра мати Софіє, до вас нині звертаємося зі щирою молитвою; бо хто може заступатися за нас перед Господом, якщо не віра, надія і любов, ці три наріжні чесноти, образ яких ви на собі показали! Ублагайте Господа, щоб Він покрив нас Своєю невимовною благодаттю в скорботах і напастях, спас і зберіг як благий і чоловіколюбець. Його славу, як незаходиме сонце, нині бачачи, допомагайте і нам у наших смиренних молитвах, щоб простив Господь Бог наші гріхи і беззаконня і щоб помилував нас грішних і недостойних заради Його щедрот. Моліть же за нас, святі мучениці, Господа нашого Ісуса Христа, Йому ж славу возсилаємо, з безначальним Його Отцем і Пресвятим, і благим, і животворчим Духом нині, і повсякчас, і навіки-віків. Амінь.

\section{Молитва до святих мучеників Гурія, Самона і Авіва}
О, святі мученики і сповідники Христові Гурію, Самоне і Авіве! Теплі за нас заступники і молитвеники перед Богом, у зворушенні наших сердець, дивлячись на пречистий ваш образ, смиренно благаємо вас: почуйте нас, грішних і недостойних рабів своїх, сущих у бідах, скорботах і недолі, і, зневаживши наші тяжкі і незчисленні гріхи, покажіть нам велику свою милість, підніміть нас із глибини гріховної, просвітіть наш розум, пом’якшіть зле і окаянне серце, припиніть заздрість, ворожнечу і сварки, що живуть у нас. Осініть нас миром, любов’ю і страхом Божим, ублагайте милосердного Господа, щоб Він покрив безліч гріхів наших Своїм невимовним милосердям. Нехай збереже Церкву Свою святу від невір’я, єресей і розколів. Нехай подасть країні нашій мир, благоденство, плодючість землі, подружжям — любов і злагоду, дітям — послух, покривдженим — терпіння, кривдникам — страх Божий, скорботним — благодушність, тим, що радіють, — стриманість. Усіх же нас нехай покриє Своєю всесильною правицею і позбавить від голоду, згуби, землетрусу, повені, вогню, меча, нашестя іноплемінників, міжусобиці і несподіваної смерті. Нехай огородить нас ополченням святих Своїх ангелів, щоб позбутися нам після нашого виходу із цього життя підступів лукавого і таємничих повітряних митарств і незасудженими стати перед престолом Господа Слави, де собори святих ангелів зі всіма святими завжди прославляють пресвяте і величне ім’я Отця, і Сина, і Святого Духа нині, і повсякчас, і навіки-віків. Амінь.

\section{Молитва до священномученика Кіпріана}
О, святий угоднику Божий, священномученику Кіпріане, скорий помічнику і молитвенику за всіх, хто до тебе звертається! Прийми від нас, недостойних, похвали ці; виблагай у Господа Бога для нас у немочах — зміцнення, в скорботах — утіху і всім усе корисне для життя цього. Піднеси до Господа твою благу і сильну молитву, щоб охороняв і зберігав нас від гріхопадінь, а тим, які впали у гріх, дарує істинне покаяння і збереже всіх нас від полону диявольського і від усякого дійства злих духів та приборкає тих, які кривдять нас. Будь нам поборником могутнім від усіх ворогів видимих і невидимих. Подай нам терпіння, а в час кончини нашої вияви нам заступництво від мучителів на митарствах, щоб під твоїм проводом досягли ми Єрусалима Вишнього і удостоїлися в Царстві Небесному зі всіма святими славити і оспівувати пресвяте ім’я Отця, і Сина, і Святого Духа навіки-віків. Амінь.

\section{Молитва до святої великомучениці Катерини}
О, святая Катерино, діво і мученице, істинна Христова невісто! Молимо тебе через особливу благодать, якою нагородив тебе Жених твій, найсолодший Ісус. Ти посоромила спокуси мучителя мудрістю твоєю, п’ятдесят мудреців перемогла ти і, напоївши їх небесним вченням, до світла істинної віри наставила; так і нам виблагай цю Божу мудрість, щоб ми, всі спокуси пекельних мук подолавши і відкинувши світу і плоті гріхи, достойними явилися божественної слави, і для розширення святої нашої православної віри сосудами достойними стали, і з тобою в небесній оселі Господа і Владику нашого Ісуса Христа, з Отцем і Святим Духом, величали і прославляли навіки-віків. Амінь.

\section{Молитва до преподобного Агапіта, лікаря Печерського}
О, всеблаженний Агапіте, земний ангеле і небесна людино! Припадаємо до тебе з вірою і любов’ю і молимо тебе старанно: яви нам, смиренним і грішним, святе твоє заступництво, бо ради гріхів наших ми не маємо свободи дітей Божих просити за наші потреби Господа і Владику нашого. А тебе, молитвеника благоприємного до Нього, просимо з великим старанням: виблагай нам у Його благості благопотрібні дари душам і тілам нашим, віру праву, любов до всіх нелицемірну, у злостражданнях терпіння, тяжкими хворобами одержимим — від недугів зцілення, тим, хто під тягарем скорбот і напастей нестерпних падає і у житті своєму впадає у відчай, твоїми молитвами одержати швидке полегшення і визволення.

Не забудь, блаженний отче, і обитель цю святу, яка завжди тебе шанує, але збережи її і всіх, хто живе і подвизається в ній, і тих, хто на поклоніння до неї приходить, неушкодженими від спокус диявольських і всякого зла. Коли ж прийде наш відхід від цього тимчасового життя і переселення до вічного, не позбав нас небесної допомоги, але твоїми молитвами всіх нас приведи до пристановища спасіння і покажи нас спадкоємцями всесвітлого Царства Христового, щоб ми оспівували і славили невимовні щедроти Чоловіколюбця Бога, Отця, і Сина, і Святого Духа, і твоє разом з Антонієм і Феодосієм батьківське заступництво навіки-віків. Амінь.

\section{Молитва до преподобного Сергія, ігумена Радонезького}
О, священна главо, преподобний і богоносний отче наш Сергію! Молитвою твоєю, і вірою, і любов’ю до Бога, і чистотою серця в обитель Пресвятої Тройці душу твою ще на землі влаштував ти, і спілкування з ангелами, і відвідання Пресвятої Богородиці сподобився, і дар чудодійної благодаті прийняв. Після ж відходу твого від земного життя ти ще більше до Бога наблизився і небесної сили сподобився, але й від нас духом любові своєї не відступив, і чесні твої мощі, як сосуд, сповнений і переповнений благодаті, нам залишив. Маючи велику відвагу до всемилостивого Владики, благай, щоб спаслися раби Його, які вірять, що в тобі діє Його благодать, і до тебе з любов’ю звертаються. Вимоли у великодаровитого Бога нашого всякий дар, усім і кожному потрібний на благо: віри непорочної збереження, міст наших утвердження, світу примирення, від голоду й пошесті позбавлення, заблудлим на путь істинну та спасіння повернення, від нашестя чужинців оборону, скорботним — утіху, хворим — зцілення, упалим — піднесення, тим, хто подвизався, — укріплення, благодійникам — добрий успіх і благословення, дітям — виховання, юним — настановлення, нерозумним — напоумлення, сиротам і вдовам — заступництво, тим, хто відходить від тимчасового життя до вічного, — добре приготування й напуття, хто відійшов, — блаженне упокоєння. І всіх нас, з допомогою твоїх молитов, сподоби в день Страшного суду позбутися стояння ліворуч, а стати спільниками тих, хто стоїть праворуч і почути блаженний голос Владики Христа: прийдіть, благословенні Отця Мого, успадкуйте приготоване вам від створення світу Царство. Амінь.

\section{Молитва до святителя Димитрія Ростовського}
О, всеблаженний святителю Димитрію, великий угоднику Христів, золотоусте слов’янський, почуй нас, грішних, що молимося до тебе, і принеси молитву нашу до милостивого і чоловіколюбного Бога; Йому ж ти нині в радості святих із ангелами предстоїш, ублагай Його благосердя, щоб не осудив нас за беззаконня наші, а помилував нас із милості своєї, виблагай нам у Христа і Бога нашого мирне і безтурботне життя, здоров’я душевне і тілесне, у всьому достаток і благоденство, і щоб не на зло перетворили ми блага, даровані нам Господом, а на славу Його і прославлення твого заступництва. Даруй нам богоугодно перейти це тимчасове життя, позбав нас від повітряних митарств і настав нас на путь, що веде до небесних осель, де лунає праведників голос невмовкний, бо бачать вони красу Божого лиця невимовну; Церкву ж святу від єресей і розколів неушкодженою збережи, заблудлих наверни і даруй нам усе, для спасіння і слави Божої потрібне. Вітчизну твою непоборною для ворогів збережи і подай нам усім твоє архіпастирське благословення, щоб, ним осінені, ми позбавилися тенет диявольських і визволилися від усякої біди і недолі; почуй моління наше, отче Димитрію, і моли безперестанно за всіх нас Всесильного Бога, славленого і глибоко шанованого в трьох іпостасях, Йому ж подобає всяка слава, честь і поклоніння навіки-віків. Амінь.

\section{Молитва до святого Феодосія Чернігівського}
О, священна главо, кріпкий наш молитвенику і заступнику, святителю Феодосію! Почуй нас, що з вірою закликаємо тебе і старанно припадаємо до раки чесних і багатоцілющих мощей твоїх (або: ікони твоєї). Поминай нас біля престолу Вседержителя і не переставай молитися за нас. Знаємо, що гріхи наші розлучають нас із тобою, і ми недостойні такого отця і заступника. Але ти, уподібнюючись чоловіколюбству Божому, не переставай взивати за нас до Господа; виблагай у Всемилостивого Бога нашого мир Церкві Його, що на землі воює; пастирям її силу ревно подвизатися за спасіння людей. Ублагай Небесного Отця всім нам подати дар, що кожному потрібний: віру істинну, надію тверду і любов неослабну, міст наших утвердження, світу примирення, від голоду і згуби позбавлення, від нашестя іноплемінників збереження, юним і дітям благе у вірі зростання, старим і немічним утішення і підкріплення, хворим зцілення, сиротам і вдовам милість і заступництво, заблудлим виправлення, бідуючим благовчасну допомогу. Не посором нас у нашому упованні, сприяй, як отець чадолюбивий, і нам понести ярмо Христове благодушно і терпеливо, і всіх управ у мирі і покаянні неосоромлено закінчити життя своє і стати спадкоємцями Царства Божого, де ти нині перебуваєш з ангелами і всіма святими, прославляючи Бога, в Тройці славленого, Отця, і Сина, і Святого Духа. Амінь.

\section{Молитва до святого Іоана Воїна}
О, преславний угоднику Христів Іоане воїне, хоробрий у битвах, ворогів прогонителю і покривджених заступнику, всіх православних християн великий захиснику і угоднику Христів, пом’яни нас, грішних і недостойних рабів, у бідах, і скорботах, і печалях, і у всяких лихих напастях, і від усякого злого кривдника захисти нас, бо дано тобі благодать від Бога молитися за нас; будь нашим поборником кріпким на всіх видимих і невидимих ворогів наших, щоб твоєю допомогою і міцним заступництвом були посоромлені всі, що чинять нам зло. О великий поборнику, Іоанне воїне, не забудь нас, грішних, що молимося тобі і просимо твоєї допомоги і невичерпної милості, і сподоби нас, грішних і недостойних рабів, одержати від Бога невимовні блага, які приготовлені люблячим Його, бо Йому належить усяка слава, честь і поклоніння, Отцю, і Сину, і Святому Духу, нині, і повсякчас, і навіки-віків.

\section{Молитва до великомучениці Анастасії Узорішительки}
О, багатостраждальна і премудра великомученице Христова Анастасіє! Ти душею на небесах перед престолом Господнім стоїш, на землі різні зцілення звершуєш благодаттю, даною тобі; зглянься милостиво на людей, що моляться перед твоїми мощами і просять твоєї допомоги, піднеси до Господа святі твої молитви за нас і виблагай нам прощення гріхів наших, недужим зцілення, скорботним і бідуючим швидку поміч; ублагай Господа, щоб Він подав усім нам християнську кончину і добру відповідь на Страшному суді Своєму, щоб і ми сподобилися разом з тобою славити Отця, і Сина, і Святого Духа навіки-віків. Амінь.

\section{Молитва до святої великомучениці Ірини}
О, багатостраждальна і преславна Ірино, невісто Христова всехвальна, угоднице Божа! Стоячи перед престолом Пресвятої Тройці і насолоджуючись невимовним блаженством, споглянь милостиво на нас, що приносимо тобі цей похвальний спів. Виблагай нам милість, відпущення гріхів у Преблагого Господа; знаємо бо, воістину знаємо, що все, що ти захочеш, виблагати у Нього можеш. Тому ми смиренно припадаємо до тебе і просимо: умилостив Владику неба і землі, щоб Він подав нам дух твоєї ревності для виконання Його заповідей, щоб ми змогли провадити земне життя доброчинно, успадкувати райські оселі і з усіма святими прославляти Отця, і Сина, і Святого Духа навіки-віків. Амінь.

\section{Молитва до святої мучениці Параскеви}
Свята невісто Христова, багатостраждальна мученице Параскево! Знаємо, що ти з юності всією душею твоєю і всім серцем твоїм полюбила Царя Слави, Христа Спасителя, і Йому єдиному уневістила себе, роздавши майно своє бідним і убогим. Ти силою твого благочестя, твоєю цнотливістю і праведністю, як сонячним промінням, засяяла, живучи побожно серед невірних і безбоязно проповідуючи їм Христа Бога. Ти з днів юності твоєї, навчена твоїми батьками, завжди благоговійно шанувала дні спасительних страждань Господа нашого Ісуса Христа, Його ж ради і сама добровільно постраждала. Ти рукою ангела Божого від незцілимих ран дивно зцілилася і, прийнявши невимовну світлість, здивувала невірних мучителів. Ти ім’ям Господа нашого Ісуса Христа і силою твоєї молитви у язичницькому капищі всіх ідолів на землю скинула і на порох перетворила; ти, обпалена свічками, єдиною твоєю молитвою до всесильного Господа погасила природний вогонь і тим же полум’ям, чудесно запаленим через ангела Божого, попаливши нечестивих беззаконників, багато людей привела до пізнання істинного Бога. Ти, прийнявши у славу Господа відсічення мечем твоєї глави мучителями, мужньо закінчила свій страдницький подвиг, зійшовши душею на небеса, в чертог улюбленого твого Жениха Христа, Царя Слави, що радісно зустрів тебе цим небесним гласом: «Радуйтеся, праведні, бо мучениця Параскева увінчалася!» Тому і ми нині вітаємо тебе, багатостраждальна, і, дивлячись на святу ікону твою, з розчуленістю взиваємо до тебе: всечесна Параскево, знаємо, що ти маєш велику сміливість перед Господом, ублагай Його чоловіколюбство за нас, що стоїмо і молимося тобі. Нехай подасть Він і нам, як тобі, терпіння і благодушність у бідах і скорботних обставинах; нехай Він дарує, твоїм заступництвом і клопотанням, радісне, благоденственне і мирне життя, здоров’я і спасіння, і у всьому благий успіх улюбленій Вітчизні нашій Україні, нехай пошле Своє святе благословення і мир; і всім православним християнам нехай подасть твоїми молитвами утвердження у вірі, побожності і святості, зростання у християнській любові і всяких чеснотах; нехай очистить нас від усякої скверни і пороку, нехай огородить нас святими Своїми ангелами, нехай захистить, збереже і помилує всіх святою Своєю благодаттю і вчинить спадкоємцями і причасниками Небесного Свого Царства. І так досягнувши спасіння твоїми святими молитвами та заступництвом, всеславна невісто Христова Параскево, прославимо всі пречесне і величне ім’я дивного у святих Своїх істинного Бога, Отця, і Сина, і Святого Духа, завжди, нині, і повсякчас, і навіки-віків. Амінь.

\section{Молитва до святої мучениці Тетяни}
О, свята мученице Тетяно, невісто найсолодшого Жениха твого Христа! Агнице Агнця Божественного! Голубко цнотливості, запашне тіло ти одягнула стражданнями як царськими ризами, до соборів небесних приєднана, торжествуєш нині у славі вічній, від днів юності ти обіцяла Богові бути служителькою Церкви, цнотливість ти зберегла і полюбила Господа більше за всі блага! До тебе ми молимося і тебе просимо: вислухай наші сердечні благання і, не відкидаючи наших молінь, подаруй чистоту тіла і душі, надихни любов до божественних істин, введи нас на путь чеснот, виблагай нам у Бога ангельську охорону, вилікуй наші рани, огороди юність, подаруй старість безболісну і безбідну, відвідай нас, сущих у в’язниці гріха, скоро настанови нас на покаяння, розпали полум’я молитви, не залиш нас сиротами, щоб, славлячи твої страждання, ми возсилали хвалу Господу нині, і повсякчас, і навіки-віків. Амінь.

\section{Молитва перша до святого Боніфатія}
Багатостраждальний і всехвальний мученику Боніфатію, до твого заступництва нині звертаємося: моління наші, якими оспівуємо тебе, не зневаж, але милостиво почуй нас, подивись на братів і сестер наших, одержимих тяжким недугом пияцтва і через те від своєї матері — Церкви Христової, від вічного спасіння відпадаючих. О святий мученику Христів Боніфатію, доторкнися до сердець їхніх даною тобі від Бога благодаттю, швидко підніми від падіння гріховного і до спасительної стриманості приведи їх. Ублагай Господа Бога, заради Якого ти страждав, щоб, простивши нам гріхи наші, не позбавив милості Своєї синів Своїх, але укріпив нас у тверезості і цнотливості, щоб допоміг Своєю десницею зберігати до кінця кріпкою і спасительною стриманість, обіцяну Богові, пам’ятаючи про неї вдень і вночі, щоб отримати добру відповідь на Страшному суді. Прийми, угоднику Божий, молитви матерів, що проливають сльози за дітей своїх; чесних дружин, що за чоловіків своїх ридають; дітей, самотніх і убогих, залишених п’яницями, і всіх нас; і нехай прийде волання наше молитвами твоїми до престолу Всевишнього, дарувати всім за молитвами їхніми здоров’я і спасіння душ і тіл, а найбільше ж Царство Небесне. Покрий і збережи нас від ловлення лукавого і всіх ворожих підступів, у страшну годину кончини нашої допоможи пройти безперешкодно повітряні митарства і молитвами твоїми позбутися вічного засудження. Ублагай Господа нашого дарувати нам до України нашої нелицемірну любов і непохитну волю перед ворогами святої Церкви, видимими і невидимими, нехай покриє нас милість Божа в безкінечні віки віків. Амінь.

\section{Молитва друга до святого Боніфатія}
О, святий угоднику Христів, страстотерпче і мученику Боніфатію, ти душею на небесах стоїш перед престолом Божим і триіпостасною славою Божою насолоджуєшся, іконою ж святою на землі перебуваєш у божественних храмах і даною тобі згори благодаттю виточуєш різні чудеса; поглянь милостивим оком на людей, що стоять і зворушено моляться перед чесною твоєю іконою і просять у тебе цілительної помочі і заступництва, виблагай душам нашим прощення гріхів; тебе, заступника милостивого перед Владикою і молитвеника за нас, грішних, серцем сокрушеним і духом смиренним закликаємо, бо ти прийняв дар від Нього недуги відганяти і пристрасті зціляти; тебе ж просимо: не зневаж нас, недостойних, що молимося і твоєї допомоги потребуємо, будь нам у печалях утішителем, тим, що страждають від пияцтва, лікарем і цілителем, тим, що у славу Божу проводять тверезий спосіб життя, скорим покровителем і готовим заступником; виклопочи всім усе, що для спасіння корисне, щоб твоїми до Господа Бога молитвами, одержавши благодаті, і милість, ми прославляли джерело всіх благ Єдиного Бога, в Тройці Святій славленого, Отця, і Сина, і Святого Духа, нині, і повсякчас, і навіки віків. Амінь.

\section{Молитва до преподобних і богоносних отців наших Іова і Феодосія Манявських}
О, преподобні і всеблаженні отці і вчителі наші Іове і Феодосію, земні ангели, ближні друзі Христові і угодники Божі, обителі вашої славо і окрасо, всієї Галицької землі і всієї православної вітчизни нашої незборима стіно, наставники чистоти і цнотливості, невтомні подвижники, віри православної поборники, що збагатили і примножили преподобних старців землі Української, святої обителі Манявської непереможні охоронителі.

О, всеблаженні і преподобні Іове і Феодосію, православної віри великі ревнителі, малого цього моління від убогих сердець наших не відкиньте, але будьте нам у вірі і покаянні наставниками. Отці святі, Іове і Феодосію, не забувайте нас, дітей своїх, і з соборами преподобних отців і мучеників до Владики Христа моліться за нас. Збережіть Церкву нашу святу Православну і нас, вірних дітей її, від ворогів видимих і невидимих, додайте нам, немічним, сили уподібнитися вам у цьому житті.

Вознесіть молитву, преподобні отці, і хворих зціліть, малодушних підтримайте, скорботних утіште і випросіть заступництвом вашим Україні нашій мир, тишу і спокій, а обителі вашій Манявській благочестя і процвітання.

Молимо вас, пастирі наші, охороніть нас молитвами вашими і допоможіть нам дотримуватися заповідей Божих зі смиренням і терпінням, у лагідності та любові, у послуху та праці, приборкувати волю та похіть і утримуватися від усякого зла. Прославляємо і ублажаємо життя ваше не тільки словами, але й життям своїм намагаємося за допомогою Божою уподібнюватися подвигам вашим, щоб і нам після відходу зі світу цього з вами нерозлучно сподобитися праворуч Бога стати, бо Йому Єдиному належить усяка слава, честь і поклоніння, Отцю, і Сину, і Святому Духу, нині, і повсякчас, і навіки-віків. Амінь.

\section{Молитва до св. прор. Малахії}
\emph{Кондак}

Як ангел послужив ти, всеблагий, і сподобився майбутнє провіщати, бо Господнє вочоловічення всім вияснив ти. Тому співаємо тобі: радуйся, пророче божий Малахіє, приснопам’ятний.

\section{Молитва до св. прор. Захари}
\emph{Кондак}

Нині пророк і священик Вишнього, Захарія, Предтечі батько, пропонує трапезу своєї пам’яті і пиття правди всім готує, вірних живлячи, як Божественний тайновидець і тайнослужитель Божої благодаті.

\section{Молитва до св. ап. Марка}
\emph{Тропар}

У верховного Петра навчившись, апостолом Христовим був ти. І як сонце, народам возсіяв ти, Олександрії бувши прикрасою, блаженний. Завдяки тобі, Єгипет від облуди визволився. Як світлом, євангельським твоїм вченням усе просвітилося, стовпе церковний. Через це пам’ять твою світлим святом вшановуючи, Марку Богогласний, моли Бога, якого ти проповідував, щоб гріхів відпущення подав душам нашим.

\section{Молитва до св. ап. Симона Зилота}
Святий славний і всехвальний апостоле Христовий Симоне, що сподобився прийняти в оселі у Кані Галилейській Господа нашого Ісуса Христа та Його Пречисту Матір, Владичицю нашу Богородицю і, преславного чуда Христового очевидцем був! З вірою та любов’ю молимо тебе: ублагай Христа Господа перетворити гріхолюбні душі на достойні сосуди благодаті. Збережи і охорони нас молитвами твоїми від принад диявольських та падінь гріховних. Виблагай для нас у Вишнього допомогу під час смутку, щоб ми не стикнулися з каменем спокуси, але неухильно прямували шляхом спасительних Христових заповідей поки не досягнемо блаженних райських обителей, у яких ти нині перебуваєш і веселишся.

Отже, апостоле Спасів! Не осором нас, що дуже на тебе надіємось, але протягом всього життя нашого будь нам помічником та покровителем. Допоможи нам доброчесно та богоугодно життя це тимчасове закінчити, благий та мирний християнський кінець отримати і доброї відповіді на Страшному Суді Христовому сподобитися. Щоб, уникнувши небесних митарств та влади злого світоправителя, ми успадкували Царство Небесне і прославляли величне ім’я Отця, і Сина, і Святого Духа на віки віків. Амінь.

\section{Молитва до св. ап. Филипа}
\emph{Тропар}

Благо прикрашається вселенна. Ефіопія, що тобою просвічена, як вінцем прикрашена торжествує і святкує пам’ять твою, проповіднику Божий Филипе, бо ти всіх вірувати в Христа навчив і життя достойно Євангелія провадив. Тому сміливо спрямована Ефіопська рука до Бога, Його ж моли дарувати нам велику милість.

\section{Молитва до св. ап. Фоми}
\emph{Ікос}

Господньому учню та великому віснику тайни, Фомі Божому проповіднику, Петро промовив: бачив я Господа. Але той відповів: поки не побачу я ран на руках Його, не увірую. Тоді Владика, як раб приходить, бажаючи спасти всіх, і промовляє до Фоми: поглянь на руки та ребра Мої і увіруй. Бо Я — Господь і Бог твій. Він же з покаянням вигукнув: Ти — мій Бог і Господь!

\section{Молитва до свв. рівноап. Кирила і Мефодія, першовчителів слов’янських}
О, преславні народів слов’янських вчителі і просвітителі, святі рівноапостольні Мефодію та Кириле! До вас, як діти до отців, святим вченням та писанням вашим просвічені й у вірі христовій наставлені, нині щиро прибігаємо і з сокрушенням сердець наших молимось. Ми і завітів ваших, як діти непокірні, не дотримуємось; і працю для Бога, про яку ви навчаєте, занедбуємо; і про однодумність та любов, яка личить нам, слов’янам, як братам по вірі та плоті, про яку заповідали ви, не дбаємо. Проте, колись ви від недостойних та невдячних не відверталися, але добром за зло віддячували, так і нині грішних та недостойних чад ваших молитви не відкиньте. Маючи велику сміливість перед Господом, щиро Його моліть, щоб наставив нас і навернув на шлях спасіння; розбрат і незгоди, які серед одновірців виникають, щоб припинив; тих, що відпали, щоб знову до однодумності привів і усіх нас в єдності Духа та любові до Єдиної Святої Соборної і Апостольської Церкви приєднав. Знаємо бо, як багато може молитва праведних до милосердя Владики, хоч і за грішних людей приноситься. Не залишайте, отже, нас безпечних та негідних чад ваших. Бо через гріхи наші, паства вами зібрана, у ворожнечі єдність втрачаючи та іновірцями спокушена, зменшується, а вівці її словесні вовками духовними розсіваються. Надихніть молитвами вашими нас ревністю до Православ’я. Щоб нею ми живлені, отцівські передання добре зберігали, уставів та звичаїв церковних вірно дотримувалися, лжевчень різних цуралися і так в житті богоугодному на землі перебуваючи, життя райського на Небі сподобилися, і там разом з вами Владику всіх у Тройці Єдиного Бога прославляли на віки віків. Амінь.

\section{Інша молитва свв. рівноап. Кирилу та Мефодію}
О, всехвальні рівноапостольні Мефодію та Кириле, припадаючи перед чесною вашою іконою, щиро молимо вас: погляньте милостиво на нас, що завдяки вашій праці просвітилися, і постійним вашим заступництвом від ворожих підступів нас охороніть! Згляньтеся над виноградом цим, який насадили ви, і не дайте дикому вепру його спустошити. Збережіть, святі угодники Божі, Церкву нашу Православну, яку збудували ви на наріжному камені, Христі, щоб була вона непорушною, і щоб розбилися об камінь цей хвилі всякого маловірства. Укріпіть пастирів наших у всяких доброчинностях та подвигові проповідування. Навчіть паству слухатися голосу їхнього. Збережіть всі землі слов’янські від усякого виснаження, від вогню і меча, від смертоносної рани і всякого зла. Почуйте і кожну людину, яка з вірою до вас приходить і вашої благодатної допомоги потребує. У страшний же смертний час станьте всім нам благими заступниками і темних демонських образів в’язничними. Щоб у мирі та покаянні земний шлях закінчивши, ми досягли вічних благ насолоди і разом з вами прославляли Пресвяту Тройцю — Отця, і Сина, і Святого Духа, нині, і повсякчас, і на віки віків. Амінь.

\section{Святим рівноапп. Костянтину і Єлені}
\emph{Кондак}

Костянтин з матір’ю Єленою нині являють всечесне древо Хреста — юдеїв посоромлення і зброя проти противників благочестивих царів. Бо заради нас явилося знамення велике і в боях грізне.

\section{Молитва до святих рівноапостольних Костянтина та Єлени}
О, предивні та всехвальні царі, святі рівноапостольні Костянтине та Єлено! До вас, щирих заступників, наші недостойні молитви возносимо, бо ви маєте велику сміливість перед Господом. Виблагайте в Нього для Церкви та всього світу благі дні, начальникам — мудрість, пастирям — піклування про паству, пастві — смирення, дівственникам — чистоту, дітям — послух і християнське виховання, хворим — зцілення, старцям — спокій, чоловікам — міцність, жінкам — прикрасу, ворогуючим — примирення, ображеним — терпіння, образникам — страх Божий; всім, що приходять до цього храму і моляться в ньому — святе благословення, і всім за їхнім проханням корисне. Щоб ми хвалили і оспівували Благодійника всіх Бога в Тройці славимого Отця і Сина, і Святого Духа, нині, і повсякчас, і на віки віків. Амінь.

\section{Молитва святого Симеона Богоприїмця}
Нині відпускаєш раба Твого, Владико, за словом Твоїм з миром, бо побачили очі мої спасіння Твоє, що Ти приготував перед лицем всіх людей: світло на просвіту народів, і славу людей твоїх, Ізраїля.

\section{Молитва до святого Симеона Богоприїмця}
О, великий угоднику, Божий Богоприїмцю Симеоне! Стоячи перед Престолом Великого Царя і Бога нашого Ісуса Христа, велику сміливість маєш до Нього, бо Він заради нашого спасіння на твоїх руках лежати зволив. До тебе бо, як до сильного заступника та молитвеника за нас, прибігаємо ми, грішні та недостойні. Моли благість Його, щоб Він відвернув від нас гнів Свій, на який ми заслужили справами своїми, і, зглянувшись на незчисленні гріхи наші, направив нас на шлях покаяння і утвердив нас у виконанні заповідей Своїх. Збережи молитвами твоїми в мирі життя наше, і в усьому благому доброго успіху виблагай, даруючи нам все, що для життя та благочестя нам необхідне. Як колись Великий Новгород явленням чудотворної твоєї ікони від смертної пагуби визволив ти, так нині нас та всі міста і села України нашої від усякої напасті, бід і смерті несподіваної твоїми молитвами визволи, і від всіх ворогів видимих та невидимих покровом твоїм захисти. Щоб ми тихе і мирне життя прожили у всякому благочесті та чистоті. І щоб так від тимчасового життя відійшовши, ми досягли вічного спокою, в якому сподобилися Небесного Царства Христа Бога нашого, Якому належить всяка слава з Отцем і Пресвятим Його Духом нині, і повсякчас, і на віки віків. Амінь.

\section{Молитва до святого праведного Йосифа, Обручника Пресвятої Діви Марії}
О, святий і праведний Йосифе! Ти, ще на землі бувши, велику сміливість перед Сином Божим, Який зволив називати тебе як обручника Своєї Матері, батьком Своїм і слухатися тебе мав. Віруємо, що ти, який нині з ликами праведних у небесних оселях перебуваєш, у всіх твоїх молитвах будеш завжди почутий Богом і Спасителем Нашим. Тому, до твого захисту та заступництва прибігаючи, смиренно молимо тебе. Як сам від бурі сумнівів визволений був ти, так нині визволи і нас, що охоплені хвилями розколів та пристрастей. Як охороняв ти Всенепорочну Діву від наклепу людського, так охорони і нас від усякого несподіваного наклепу. Як оберігав ти від усякої рани та біди втіленого Господа, так твоїм заступництвом оберігай і Церкву Його Православну та всіх нас від усякої рани та біди. Знаєш, святий Божий, що і Син Божий за часів втілення Свого тілесні потреби маючи, сподобив тебе послужити Собі. Тому молимо тебе і наші тимчасові потреби задовольни, подаючи нам усе добре та в житті цьому корисне. Особливо просимо тебе виблагати нам у нареченого твого Сина, Єдинородного ж Сина Божого, Господа нашого Ісуса Христа, прощення наших гріхів. Щоб молитвами твоїми ми стали достойними успадкувати Царство Небесне і в небесних оселях разом з тобою перебуваючи, прославляли Єдиного Триіпостасного Бога, Отця, і Сина, і Святого Духа, нині і на віки віків. Амінь.

\section{Молитва до св. Іоана Золотоустого}
О, святителю великий Іоане Золотоустий! Ти багато різних дарів від Господа отримав: як благий та вірний раб, ти отримані таланти добре примножив. Тому воістину вселенським учителем був ти. Адже люди різного віку та звання від тебе навчаються. Бо дітям образ послуху виявив ти; юним — світло цнотливості; чоловікам — працелюбності наставник; старшим — учитель незлобливості; ченцям — правило стримання; молитвеникам — Богом натхненний вождь; мудрим — розуму просвітлення; доброгласним ораторам — невичерпне джерело живого слова; добротворцям — зірка милосердя; начальникам — образ мудрого правління; поборникам правди — бадьорий натхненник; гнаним за правду — наставник терпіння; для всіх усім був ти, щоб хоч деяких спасти. Але перед усім ти збагачувався у любові, яка є запорукою досконалості. Нею ти всі таланти в душі твоїй у єдине поєднав. Ту ж любов, яка роз’єднаних примирює, ти, пояснюючи апостольські слова, всім вірним проповідував. Ми ж, грішні, один талант кожен маючи, єднання духу у союзі миру не маємо, але буваємо марнославними, один одному заздримо, один одного роздратовуємо. І так наші таланти не для миру та спасіння, а на ворожнечу та осуд використовуються. Тому до тебе, святителю Божий, ми, що розбратом охоплені, припадаємо і з сокрушенням серця просимо: молитвами твоїми віджени від сердець наших усяку гордість та заздрість, через які поділ між нами виникає. Щоб ми, подолавши багато труднощів, у єдності тіло церковне зберегли та, молитвою твоєю, один одного полюбили і разом сповідували Отця, і Сина, і Святого Духа, Тройцю Єдиносущну та Нероздільну нині, і повсякчас, і на віки віків. Амінь.

\section{Молитва до св. Василія Великого, архиєпископа Кесарії Каппадокійської}
О, великий святителю отче Василію, преславний Вселенської Церкви учителю, слави Пресвятої Тройці найщиріший поборнику, Матері Божої та її Пренепорочного Дівства наперед обраний ісповідниче, пресвятої чистоти, смирення і терпіння образ! Ось я, що маю багато гріхів і негідний на висоту небесну споглядати, смиренно молю тебе. О премудрий Церкви Христової учителю, навчи мене так богобоязно життя своє провадити, щоб ніколи на шлях, Богові противний, не звернути, або у спокусу не впасти. Збережи та визволи мене твоїм могутнім заступництвом від спокус та підступів диявольських, як визволив ти від них юнака, який від Благого Спасителя нашого відступив і під владу сатани підпав. Дай мені силу душевну, щоб я міг щиро наслідувати твої великі доброчинності. Вчини, щоб я у вірі православній був твердим та непохитним. Зміцни мене, малодушного, у терпінні та надії на Господа. Виплекай у серці моїм істинну любов Христову, щоб я небесних благ найбільше за все бажав та ними задовольнявся. Виблагай для мене в Господа щире сокрушення за гріхи, щоб я останок життя мого перебував у мирі, покаянні та виконанні Христових заповідей. Коли ж час мого кінця наблизиться, ти, о благий отче, з Преблагословенною Дівою Марією швидко прийди мені на поміч, захисти мене від злих ворожих підступів, і сподоби мене стати спадкоємцем райських осель. Щоб мені з тобою та всіма святими перед неприступним Престолом Божої Величі стати, і Животворчу, Єдиносущну та Нероздільну Тройцю прославляти і оспівувати, повсякчас і в безкінечні віки. Амінь.

\section{Молитва до святих семи Ефеських юнаків: Максиміліана, Ямвлиха, Мартиніана, Іоана, Діонисія, Ексакустодіана (Костянтина) та Антоніна}
О, пречудні святі сім отроків, міста Ефесу похвала, та всього світу надіє! Спогляньте з висоти небесної слави на нас, що з любов’ю пам’ять вашу шануємо, а найперше, на немовлят християнських, що батьками своїми на ваше заступництво віддані. Пошліть на них благословення Христа Бога, Який промовив: дозвольте дітям приходити до Мене. Хворих зціліть, скорботних утіште. Серця їхні в чистоті збережіть та смиренням наповніть. У землю сердець їхніх зерно віри Божої посадіть і закріпіть, щоб набираючись сили їм зростати. І всіх нас, що перед святою вашою іконою припадаємо, до мощей ваших з вірою прикладаємось і щиро до вас молимося: сподобте Царство Небесне успадкувати та невпинно з радістю прославляти нам величне ім’я Пресвятої Тройці, Отця, і Сина, і Святого Духа, на віки віків. Амінь.

\section{Молитва до святителя Тихона Задонського}
О, всехвальний святителю, угоднику Божий, отче наш Тихоне! Ангельське на землі життя провадивши, ти як благий ангел дивно прославився. Віруємо від усієї душі та розуміння нашого, що ти, милостивий наш помічник та молитвеник, щирими молитвами та благодаттю, яка тобі від Господа щедро подається, завжди сприяєш спасінню нашому. Прийми, отже, прославлений угоднику Христовий, і нині наші недостойні молитви. Визволи нас твоїм заступництвом від пустомовства та суєтності, які засмоктують нас, іновірства, та марновірства людського. Поспіши, скорий за нас молитвенику, добрим твоїм заступництвом ублагати Господа, щоб Він завжди подавав Свої великі та багаті милості нам, грішним та недостойним рабам Його. Щоб Він Своєю благодаттю зцілив невиліковні рани та струпи розбитих душ і тіл наших. Щоб пом’якшив скам’янілі серця наші слізьми жалю та сокрушення про безліч гріхів наших і визволив нас від вічних мук та вогню геєнського. Усім же вірним людям Своїм нехай дарує в цьому віці мир та злагоду, здоров’я, спасіння і в усьому добрий успіх. Щоб так, тихе і мирне життя проживши у всякому благочесті та чистоті сподобилися разом з ангелами і з усіма святими славити і оспівувати всесвяте ім’я Отця і Сина, і Святого Духа на віки віків. Амінь.

\section{Молитва до святителя Митрофана Воронезького}
О, всехвальний святителю Христовий і чудотворче Митрофане! Прийми це мале моління від нас, грішних, що до тебе прибігаємо, і щирими молитвами ублагай Господа і Бога нашого Ісуса Христа, щоб Він, зглянувшись милостиво на нас, подав нам прощення гріхів наших вільних та невільних і з великої Своєї милості визволив нас від біди, смутку, скорботи, недугів душевних та тілесних, які нас обтяжують. Нехай подасть землі і плодоносність та все, що для нашого теперішнього життя потрібне. Нехай дарує нам скінчити це тимчасове життя в покаянні і сподобить нас, грішних та недостойних Небесного Царства Свого, щоб ми з усіма святими славили Його невичерпне милосердя з Безпочатковим Його Отцем і Святим і Животворчим Його Духом на віки віків. Амінь.

\section{Молитва до святого Олександра Невського}
Скорий помічнику всіх, хто ревно до тебе вдається, і щирий наш перед Господом заступнику, святий благовірний великий княже Олександре!

Зглянься милостиво на нас недостойних, що через безліч беззаконь наших зробили себе негідними і до святої ікони твоєї нині вдаємося, і з глибини серця до тебе взиваємо. Ти в житті своєму ревнителем та захисником Православної віри був. Тому і нас у ній щедрими твоїми молитвами до Бога непорушно утверди. Ти велике покладене на тебе служіння старанно проходив. Тому і нас твоєю допомогою, кожного, виконувати те, до чого покликаний, навчи. Ти переміг полки ворогів. Тому і всіх видимих та невидимих ворогів, що оточують нас, подолай. Ти, залишивши тлінний вінець царства земного, обрав тихе життя і нині праведно нетлінним вінцем вінчаний, на небесах царюєш. Тому виблагай і нам, — смиренно молимо тебе — життя тихе та спокійне і до вічного Царства шлях наш твоїм заступництвом спрямуй. Молячись же зі всіма святими перед Престолом Божим, молися за всіх православних християн, щоб зберіг їх Господь Бог Своєю благодаттю в мирі, здоров’ї, довголітті та всяких благах на довгі літа. Щоб ми повсякчас славили і благословили Бога в Тройці Святій славимого, Отця, і Сина, і Святого Духа, нині, і повсякчас, і на віки віків. Амінь.

\section{Молитва до святих благовірних князя Петра і княгині Февронії, Муромських чудотворців}
О, великі угодники Божі та предивні чудотворці благовірні, княже Петре і княгине Февроніє, міста Мурома заступники та охоронці і за всіх щирі до Господа молитвеники! До вас прибігаємо і до вас із сильною надією молимось. Вознесіть за нас, грішних, святі молитви ваші до Господа Бога і виблагайте у Нього все, що добре та корисне для душ і тіл наших: віру правдиву, надію благу, любов нелицемірну, благочестя непохитне, в добрих справах успіх, мир для світу, землі плодоносність, добре поліття, тілам здоров’я і душам спасіння. Виблагайте в Царя Небесного Церкві святій і всій державі нашій мир, тишу та зростання, а всім нам — життя благоприємне і добру християнську кончину. Збережіть вітчизну вашу, місто Муром, і всі міста від усякого зла. А всіх людей православних, які до вас приходять і святим мощам вашим поклоняються, покрийте благодатною дією Богові приємних молитов ваших, і всі прохання їхні на благо виконайте. О чудотворці святі! Не зневажте молитов наших, які ми сьогодні вам смиренно возносимо, але будьте перед Господом і сподобіть нас, за допомогою вашою, спасіння вічне отримати і Царство Небесне успадкувати. Щоб ми славили невимовне чоловіколюбство Отця, і Сина, і Святого Духа, в Тройці поклоняємого Бога на віки віків. Амінь.

\section{Молитва до святителя Іоана Милостивого, патріарха Олександрійського}
Святителю Божий Іоане, милостивий захиснику бідних і тих, що у напастях перебувають! До тебе прибігаємо і тобі молимося, як до скорого захисника всіх, що в бідах та скорботах у Бога втіхи шукають. Не переставай молитися до Господа за всіх, що з вірою до тебе приходять! Ти, бувши переповненим Христової любові та благості, явився як чудесний чертог доброчинності та милосердя і отримав собі ім’я «милостивий». Ти був, як ріка, що завжди наповнена щедрим милосердям і всіх спраглих щиро напуває. Віруємо, що після переселення твого від землі на небо збільшився в тобі дар благодаті цієї і ти став невичерпним сосудом усякої благості. Зроби молитвами твоїми та заступництвом перед Богом всякі полегшення, щоб ті, що прибігають до тебе, отримували мир і спокій. Даруй їм полегшення у тимчасових скорботах і допомогу в потребах життєвих, всели в них надію на вічне у Царстві Небесному упокоєння. В житті своєму земному був ти пристановищем для всіх, хто мав якусь біду, потребу, недугу або терпів образу, і ніхто з тих, які приходили до тебе і просили милості без твоєї благості не залишилися. Те саме і нині, царюючи з Христом Богом на небесах, яви всім, що перед чесною твоєю поклоняються іконою і про допомогу та заступництво моляться.

Не лише сам ти чинив милість безпомічним, але й серця інших спонукав ти до допомоги немічним та милості до убогих. Піднеси, отже, і нині серця вірних на заступництво бідних, на утішення скорботних та заспокоєння немічних. І нехай не вичерпуються в них дари милості, але нехай вселиться в них та в дім цей, що з милістю ставиться до страждаючих, мир та радість в Дусі Святому, на славу Господа і Спаса нашого Ісуса Христа на віки віків. Амінь.

\section{Молитва до святителя Спиридона, Триміфунтського чудотворця}
О, преблаженний святителю Спиридоне, великий угоднику Христовий і преславний чудотворче! Молячись на небесах із чинами ангельськими біля Престолу Божого, поглянь милостивим оком на людей, що стоять тут і сильно твоєї помочі просять. Умоли, милостивий, Чоловіколюбця Бога, щоб Він не засудив нас за беззаконня наші, але вчинив нам по милості Своїй. Виблагай нам у Христа Бога нашого мирне та спокійне життя, здоров’я душевне і тілесне, багатство плодів земних і в усьому допомогу та благі дні. Нехай не на зло ми будемо використовувати блага, які отримуємо від Щедрого Бога, але на славу Його та на прославлення твого заступництва. Визволи всіх, що з вірою непохитною до Бога приходять, від усяких бід душевних і тілесних, від всіх мук та диявольських підступів. Будь для скорботних утішителем, для недужих — лікарем, в напастях — помічником, нагим — покровителем, вдовам — заступником, убогим — захисником, немовлятам — годувальником, похилим — опорою, подорожуючим — провідником, плаваючим — кормчим. Виблагай усім, які сильної твоєї допомоги потребують, усе, що для спасіння потрібне. Щоб ми, які твоїми молитвами проваджені та бережені, досягли вічного спокою і разом з тобою прославляли Бога, в Тройці Святій славимого Отця, і Сина, і Святого Духа, нині, і повсякчас, і на віки віків. Амінь.

\section{Тропар святителям Гурію та Варсонофію, Казанським чудотворцям}
Перші вчителі раніше темного, а нині світлого та новопросвіченого міста Казань, перші вісники спасенного шляху, істинні охоронителі апостольських передань, непохитні стовпи, вчителі благочестя і Православ’я і наставники, Гурію та Варсонофію, Владику всіх моліть мир світові дарувати і душам нашим велику милість.

\section{Молитва до святих праведних Йоакима та Анни}
О, завжди славні Христові праведники, святі богоотці Йоакиме та Анно, що молитеся перед Небесним Престолом великого Царя і велику сміливість до Нього маєте, що від Преблагословенної Дочки вашої, Пречистої Богородиці і Приснодіви Марії, втілитися зволив. До вас, багатосильних заступників та молитвеників за нас, ми, грішні та недостойні, прибігаємо. Моліть Його, щоб Він відвернув від нас гнів Свій, який праведно за діла наші на нас посилається, і нехай, на незліченні гріхи наші споглянувши, направить нас на шлях покаяння і на стежки заповідей Своїх нехай нас утвердить. Також молитвами вашими в мирі життя наше збережіть і в усьому благому доброго успіху виблагайте, все, що для життя і благочестя потрібне, нам від Бога даруючи; від усяких мук, бід та несподіваної смерті заступництвом вашим нас визволяючи, і від всіх ворогів видимих та невидимих захищаючи. Щоб ми тихе і спокійне життя прожили в усякому благочесті та чистоті. І так, у мирі це тимчасове життя проживши, досягли вічного покою, де вашими святими молитвами, сподобимось Небесного Царства Христа Бога нашого, Якому з Отцем і Пресвятим Духом належить всяка слава, честь і поклоніння на віки віків. Амінь.

\section{Святителю Стефанові Велико-Пермському}
\emph{Тропар}

Божественною любов’ю з юних років, Стефане премудрий, запалав ти і ярмо Христове взяв, а людям, що раніше занедужали невірством серця, Божественне насіння в них посіяв і євангельськи духовне породив ти. Тому преславну твою пам’ять вшановуючи, молимо тебе: благай Того, Якого проповідував ти, щоб Він спас душі наші.

\section{Молитва до святителя Тихона Амафунського}
О, трисонячної Тройці угоднику! О невичерпне джерело чудес! О Небесної краси ревнителю, святителю Божий, преблагий, обранець Спасів, великий архієрею, преблаженний святителю Христовий Тихоне. Ти, який просвічуєш та дивуєш чудесами твоїми всіх, що знаходяться під сонцем! Завжди перед Святою Тройцею перебуваючи і за весь світ молячись, всіх, хто прикликає ім’я твоє святе, від всякої немочі душевної і тілесної зціли. Відпущення гріхів їхніх виблагай, гріховні думки знищ і всіляку напасть та нашестя диявола від тих, що моляться до тебе далеко відведи і підведи нас від сну гріховного; зроби нас, що входимо до святого храму твого, здоровими душею і тілом, і щоб ми щиро славили ім’я Пресвятої Тройці, Отця, і Сина, і Святого Духа, і твоє молитовне заступництво, завжди, нині, і повсякчас, і на віки віків. Амінь.

\section{Молитва до святителя Миколая, архієпископа Мирлікійського, Чудотворця}
О, всехвальний і всечесний архієрею, великий Чудотворче, Святителю Христовий, отче Миколаю, чоловіче Божий і вірний служителю, муже любові, сосуде обраний, міцний стовпе церковний, світильнику пресвітлий, зірко, яка осяює та освітлює увесь світ!

Ти праведник, — як фінік, що у дворах Господа свого розцвів, живучи в Мирах, миром облагоухав себе, і миро повсякчасної благодаті Божої виточуєш. Твоєю ходою, пресвятий отче, море освятилося, коли многочудесні твої мощі прямували до міста Барського, щоб зі сходу до заходу хвалити ім’я Господнє. О преславний та предивний чудотворче, скорий помічник, щирий заступнику, пастирю предобрий, який спасає словесне стадо від усяких бід! Тебе прославляємо і тебе величаємо як надію всіх християн, джерело чудес, захисника вірних, всемудрого учителя; тих, що голодні — годувальника; тих, що плачуть — веселощі; нагих — одяг, хворих — лікаря; тих, що в морі подорожують — керманича; полонених — визволителя; вдів та сиріт — годувальника та заступника; цнотливості — хоронителя; дітей — смиренного вихователя; посників — наставника; старих — міцність; занепокоєних — спокій; бідних та убогих — щире багатство. Почуй нас, що молимось тобі і до твого захисту вдаємося. Вияви заступництво своє за нас перед Всевишнім і вимоли твоїми богонадхненними молитвами все, що корисне для спасіння душ і тіл наших. Збережи святу оселю цю, всяке місто й село, і всяку країну християнську від усяких утисків, бо знаємо, як багато може молитва праведника посприяти на благо. Тебе праведного після Преблагословенної Діви Марії заступника перед Всемилостивим Богом маємо, і до твого, преблагий отче, щирого заступництва та захисту смиренно припадаємо. Ти, як добрий та пильний пастир, збережи нас від усяких ворогів, згуби, землетрусу, граду, голоду, потопу, вогню, меча, нашестя чужоземного, і у всіх бідах та скорботах подавай нам руку допомоги.

Відкрий нам двері милосердя Божого, бо через безліч неправд наших ми недостойні бачити висоту небесну і, перебуваючи в гріховних кайданах, благої волі Творця нашого ми не виконали. Богомудро ми приклоняємо наші сокрушенні коліна та смиренні серця до нашого Сотворителя, і твого отцівського перед Ним заступництва просимо: допоможи нам, угоднику Божий, щоб ми не загинули з беззаконнями нашими, визволи нас від усякого зла та усього супротивного, направ розум наш і зміцни серце наше в правдивій вірі, щоб у ній перебуваючи, ми за твоєю молитвою та заступництвом, ані ранами, ні злобою людською, ні моровицею, ні гнівом Творця свого охоплені не були. Щоб ми тут мирне життя провадили і сподобилися бачити добро на землі живих, прославляючи Отця і Сина, і Святого Духа, Єдиного в Тройці славимого і поклоняємого Бога нині, і повсякчас, і на віки віків. Амінь.

\section{Молитва 2-а}
О, всесвятий Миколаю, величний угоднику Господній, щирий наш заступнику і всюди в скорботах скорий помічнику! Допоможи мені, грішному та безпомічному, в цьому житті. Ублагай Господа Бога дарувати мені прощення всіх моїх гріхів, якими я згрішив від юності моєї, протягом усього життя мого ділом, словом, думкою і всіма моїми почуттями. Також під час відходу душі моєї допоможи мені, окаянному. Ублагай Господа Бога, всього світу Сотворителя, позбавити мене небесних митарств і вічних мук, щоб я завжди прославляв Отця, і Сина, і Святого Духа, і твоє милостиве заступництво нині, і повсякчас, і на віки віків. Амінь.

\section{До свт. Модеста, Патріарха Єрусалимського}
\emph{Тропар}

Преподобно проживши, Богомудрий Модесте і святительським одягом прикрашений, блаженніший пастирю Єрусалима, ти в радості нині перед Христом перебуваєш і всі кінці світу чудесами просвіщаєш. Тому будь помічником всім, хто до тебе взиває, і молися, отче, за всіх нас.

\section{До св. Філарета Милостивого}
\emph{Тропар}

Через терпіння твоє отримав ти винагороду свою, праведний, і заповіді Господні досконало виконував, убогих полюбив ти і їм допомагав. Молися, блаженний, до Христа Бога, щоб спастися душам нашим.

\section{До святого праведного Іова Багатостраждального}
\emph{Тропар}

Пам’ять праведного Твого Іова, Господи, святкуючи, тим же молимо Тебе: визволи нас від підступів і тенет лукавого диявола і спаси, як Чоловіколюбець, душі наші.

\section{Молитва до св. сщмч. Власія, єпископа Севастійського}
Преблаженний та достойнопам’ятний священномученику Власію, дивний страждальнику та щирий наш заступнику. Ти, який обіцяв після відходу свого до життя вічного вислуховувати молитви та допомагати всім, що ім’я твоє святе будуть прикликати! Ось нині ми до тебе, угодника Божого, як до істинного помічника спасіння, приходимо і смиренно молимось: допоможи нам, що кайданами гріхів скуті; стань на всемогутні твої до Бога молитви і помолися за нас, грішних. Бо тебе ми, недостойні, прикликати на заступництво насмілюємося і жадаємо, за твоєю допомогою, від усіх гріхів наших звільнитися. О святий Божий Власію! Зі смиренним та сокрушенним нашим серцем перед тобою припадаємо і молимось. Осяй нас, ворожими підступами вражених, світлом вишньої благодаті, щоб у ній перебуваючи, ми не спіткнули об камінь ніг наших. Тебе, як сосуд, що на честь обраний та наповнений Божою благодаттю, молимо: сподоби нас, грішних від твоєї повноти бажане прийняти. Зціли наші душевні та тілесні рани. Гріхів наших прощення для душевного й тілесного нашого здоров’я та спасіння корисного у Господа виблагай. Щоб ми завжди прославляли Отця, і Сина, і Святого Духа, і твоє милостиве заступництво за душі та тіла наші нині, і повсякчас, і на віки віків. Амінь.

\section{Молитва до св. вмч. Микити}
До тебе як до скорого та вибраного помічника у нашому насінні, богообраного воєводи, який зброєю хреста ворогів переміг, великомученику Микито, всією душею припадаємо. Не відвернися від наших страждань, вислухай молитви наші, нас та обитель нашу від бід збережи. Простягни руку твою, яка скоро допомогу подає, відверни розум наш від думок нечестивих, серця наші осквернені очисти, освяти і до неба піднеси. Від видимих та невидимих ворогів нас збережи, щоб, зі страхом Господнім святе діючи, ми пристрасті перемогли та, від усякої бездіяльності визволившись, радуватися в Господі почали. І так, твоїми молитвами, зі смиренням та чистотою серця сподобились до останнього подиху богоугодно Отця, і Сина, і Святого Духа оспівувати і твої богоувінчані подвиги та чудеса прославляти на віки віків. Амінь.

\section{Молитва до свв. мчч. Адріана та Наталії}
О, священна паро, святі мученики Христові Адріане і Наталіє, блаженне подружжя та хоробрі страждальці! Почуйте нас, що зі сльозами молимось до вас, і пошліть нам усе, що корисне для душ і тіл наших. Моліть Христа Бога, щоб Він помилував нас і вчинив нам за милістю Своєю, щоб ми не загинули у гріхах наших.

О, святі мученики! Прийміть голос моління нашого і визволіть нас молитвами вашими від голоду, згуби, землетрусу, потопу, вогню, меча, нападу чужинців та міжусобиці, від несподіваної смерті та всякої біди, смутку та хвороби. Щоб нині ми, зміцнені вашими молитвами та заступництвом, прославляли Господа Ісуса Христа. Йому ж належить всяка слава, честь та поклоніння з Безпочатковим Його Отцем і Пресвятим Духом на віки віків. Амінь.

\section{Молитва до св. сщмч. Харлампія}
О, священна та багатостраждальна голово, добрий пастирю Христових словесних овець, священномученику Христовий Харлампію, Магнезійська похвало та світу слава, всесвітній світильнику, великий наш заступнику, в скорботах, бідах та всяких потребах благий помічнику! Почуй нас, грішних, що до тебе прибігаємо та молимось, і визволи нас від усякого злого випадку. Бо ти за великі та тяжкі твої страждання і терпіння від Бога прийняв велику силу та благодать, щоб всюди й у всьому нам допомагати; особливо ж там, де пам’ять твоя шануватися буде.

Особливу благодать ти, священномученику Христовий, від Господа Царя Слави, що до тебе явився, отримав, коли на усічення мечем тебе засудили. Бо ти чув цей найбажаніший та найприємніший голос, який до тебе промовляв: «Прийди, Харлампію, друже Мій, який багато мук за ім’я Моє витерпів, і проси у Мене все, що хочеш, і Я дам тобі». І тоді сказав ти Христу Господу: «Господи мій, великим для мене від Тебе, Світла Невечірнього, є дар цей. Якщо угодно буде Величності Твоїй, молю Тебе зволити дарувати мені особливу Твою милість. Нехай там, де покладені будуть мощі мої та шануватиметься пам’ять страждань моїх, не буде ні голоду, ні моровиці, ні повітря згубного, що нищить плоди. Але нехай найбільше на цьому місці панує мир, здоров’я тілесне і душам спасіння, достаток пшениці, вина і оливи, та примноження худоби, на потребу людям». І відповів тобі голос Господній: «Нехай буде так, як бажаєш ти, преславний Мій воїне». І відразу після цих слів Господніх до усічення мечем віддав єси душу свою і вона, зустрінута ангелами, до Господа вознеслась. І так ти прийняв вінець слави від Божественної руки Його ликами святих, які вічно славлять Пресвяте ім’я Господа. Так істинно у славі небесній перебуваючи, зглянься угоднику Божий, і над нами, грішними, що моляться тобі, пом’яни нас перед Господом, щоб дарувати нам на нашу потребу великі Його милості на віки віків. Амінь.

\section{Молитва до св. мц. Фомаїди}
О, всехвальна мученице Фомаїдо! За чистоту шлюбу навіть до крові подвизавшись і заради цнотливості своє життя поклавши, явилася ти перед Господом достойною до лику преподобних дів прилучитися. Почуй нас, що молимось тобі і як колись за дарованою Тобі Богом благодаттю, ти була цілителькою наших плотських напастей, так і нині тих, хто до твого заступництва вдається, зміцни і від плотської напасті звільни. Чисте і бездоганне життя у шлюбі та в дівстві перебування твоїми благоугодними молитвами всім виблагати поспіши. Щоб тіла наші були храмом перебуваючого в нас Святого Духа.

О, вибрана серед жінок та вірна рабо Христова! Допоможи нам, щоб ми не загинули в пристрастях наших, але виправили розум свій і зміцнилися серцем нашим у всякому благочесті та чистоті, славлячи твою допомогу й заступництво, благодать і милість Триєдиного Бога, Отця, і Сина, і Святого Духа на віки віків. Амінь.

\section{Молитва до свв. мчч. Флора та Лавра}
Достойні похвали мученики, всечесні брати Флоре і Лавре, почуйте всіх, хто до вашого заступництва приходить, і як за життя вашого ви зціляли коней, так і тепер позбавте їх від усіх недугів. Прислухайтеся до молитви всіх, хто до вас прибігає, щоб всіма було прославлене пресвяте ім’я Отця і Сина, і Святого Духа, завжди, нині і на віки віків. Амінь.

\section{Молитва до св. вмч. Димитрія Солунського}
Святий і славний великомученику Христовий Димитрію, скорий помічнику та щирий заступнику всіх, хто з вірою вдається до тебе! Сміливо перебуваючи перед Небесним Царем, виблагай у Нього прощення гріхів наших і щоб визволитися нам від всезгубної пошесті, землетрусу, потопу, вогню, меча та вічного покарання. Моли благість Його, щоб Він змилосердився над містом цим, обителлю цією та всякою країною християнською. Виблагай у Царя Царюючих перемогу над ворогами, а всій країні нашій мир, спокій, непохитність у вірі та успіх у благочесті. А нам, що шануємо чесну пам’ять твою, виблагай благодатне зміцнення на благі справи. Щоб угодне Владиці нашому Христу Богу тут чинячи, ми, молитвами твоїми, сподобилися унаслідувати Царство Небесне для вічного прославлення Його з Отцем і Святим Духом. Амінь.

\section{Молитва до сщмч. Антипи, єпископа Пергама Асійського}
О, преславний священномученику Антипо, християнам у хворобах скорий помічнику! Вірую від усієї душі та розуміння, що дав тобі Господь дар хворих зцілювати, недужих лікувати і розслаблених зміцнювати. Через це до тебе, як до благодатного лікаря хвороб, я, немічний, вдаюся і твій достойно шанований образ з благоговінням цілуючи, молюся: своїми молитвами у Царя Небесного виблагай мені, хворому, зцілення від зубного болю, що гнітить мене.

І хоч я і недостойний тебе, милостивого отця та повсякчасного заступника мого, але ти, як наслідувач Божого чоловіколюбства, зроби мене достойним твого заступництва через моє навернення від злих діл до благого життя. Зціли дарованою тобі великою благодаттю рани і струпи душі та тіла мого. Даруй мені здоров’я, спасіння та у всьому добрий успіх. Щоб так тихе і спокійне життя провадивши в усякому благочесті та чистоті, сподобитися мені зо всіма святими славити Всесвяте ім’я Отця, і Сина, і Святого Духа. Амінь.

\section{Молитва до свв. мчч. Віленських Антонія, Іоана і Євстафія}
Святі мученики Антонію, Іоане та Євстафію! Спогляньте з небесної оселі на тих, хто потребує вашої допомоги і не відкиньте прохань наших, але, як постійні благодійники та заступники наші, моліть Христа Бога, щоб Він, чоловіколюбним та багатомилостивим бувши, зберігав нас від усякого зла: від землетрусу, потопу, вогню, меча, нашестя чужинців та міжусобиць. Щоб не засудив нас, грішних, за беззаконня наші, і щоб не на зло ми використали добро, яке дароване нам Всещедрим Богом, але на славу і святого імені Його і на прославлення вашого міцного заступництва. Нехай молитвами вашими Господь подасть нам мир помислів, від пагубних пристрастей стримання і від усякої скверни ухиляння. Нехай зміцнить у всьому світі Свою Єдину Святу Соборну і Апостольську Церкву, що її Він здобув чесною Своєю кров’ю. Моліться старанно, святі мученики, щоб благословив Христос Бог всю нашу державу, щоб утвердив у святій Своїй Православній Церкві живий дух правдивої віри і благочестя. Нехай всі чада її, від марновірства та суєтності звільнившись, в дусі та істині Йому поклоняються і про виконання Його заповідей щиро дбають. Нехай ми всі в мирі та благочесті в нинішньому віці поживши, благодаттю Господа нашого Ісуса Христа блаженного вічного життя досягнемо на небесах. Йому ж належить всяка слава, честь і поклоніння з Отцем і Святим Духом нині, і повсякчас, і на віки віків. Амінь.

\section{Молитва до св. мч. Мини}
О, страстотерпче святий мученику Мино! Споглядаючи на ікону твою і згадуючи про зцілення, які ти подав усім, хто з вірою та благоговінням до тебе приходив, припадаємо до тебе і, схиливши коліна сердець наших, від усієї душі нашої молимо тебе: будь нам перед Господом і Спасом нашим Ісусом Христом заступником у немочах наших, перебувай з нами та втішай у скорботах, дай нам пам’ятати про гріхи наші, допомагай під час спокус та у хвилях бурхливого моря світу цього, в усіх бідах, які ми на цьому місці плачу терпимо.

\section{Молитва до св. мч. архідиякона Лаврентія Римського}
О, пресвятий і предивний мученику Христовий, архідияконе Лаврентію! Віру та страждання твої прославляючи, вшановуємо перемогу та блаженний твій перехід через розпечене вугілля від темряви віку цього до незгасного світла Престолу величі Божої. Тим-то молимо тебе: як колись тих, що з вірою до твого заступництва вдавалися, своїми чудесами ти зціляв, так прийми й нас під покров твій, у хворобах і скорботі заступником будь. І як від сліпоти ти позбавляв, так молитвами своїми біля Престолу Божого і нашу душевну сліпоту зціли. Споглянь на наше тілесне й душевне розслаблення та на видимих і невидимих ворогів наших, від нападів яких ми смуток маємо, хоробрістю та пильністю нас зміцни. Щоб, за твоєю допомогою пройти шлях цього тимчасового життя, подолавши всі біди, скорботи й усі напасті, та досягти Царства Божого, де ти перебуваєш і щиро молишся за людей, що з вірою до тебе припадають і оспівують дивного у святих Бога Ізраїлевого на віки віків. Амінь.

\section{Молитва до свв. дев’яти мчч. Кизицьких: Феогнида, Руфа, Антипатра, Феостиха, Артема, Магна, Феодота, Фавмасія та Филимона}
Всехвальні святі мученики, мужні раби Христові: Феогниде, Руфе, Антипатре, Феостиху, Артеме, Магне, Феодоте, Фавмасію та Филимоне! Ви Христа ради з різних країн зібралися: істинного Бога і Чоловіка сміливо в Кизиці проповідували, ідолів знищивши, князя посоромили, і після різних страждань мечем усічені були, і кров свою як воду за Христа пролили, і нею як багряницею Церкву Христову прикрасили. І нині тим самим нев’янучим вінцем безсмертної слави увінчані, у невимовній радості з ликами ангельськими в небесних оселях перебуваєте, Божественним світлом осяяні і спогляданням Його незбагненної доброти насолоджуєтеся. Тому і ми, смиренні, до вас прибігаємо зі сокрушенням сердець на ваш образ споглядаючи, молимо вас: моліться за нас до Христа Бога, бо ви сміливість перед Ним маєте, щоб вашими святими молитвами ми від усякої болісті і шкоди душевної та тілесної визволилися. Бо вам від Бога дано дар зцілення, щоб тих, хто з вірою до вас вдається, від всіляких мук зціляти, як колись славного мужа від лихоманки зцілили. І ми, вірні, поміч отримавши, шануємо вас, святі мученики, що стільки дарів від Христа Бога маєте. Йому ж слава разом з Отцем і Святим Духом завжди, нині, і повсякчас, і навіки віків. Амінь.

\section{Свв. мчч. Галактіону та Єпістимії}
Тропар

Мученики Твої, Господи, у стражданнях своїх від Тебе, Бога нашого, нетлінні вінці прийняли. Зміцнені Тобою, вони мучителів перемогли та демонів нерозумні спокуси подолали. Їхніми молитвами спаси душі наші.

\section{Св. прпмц. Євдокії}
\emph{Тропар}

Духовною істиною душу свою до любові Христової прив’язала ти як учениця. Слово тлінне, привабливе і тимчасове зневажила: спочатку постом пристрасті знищивши, а потім стражданнями ворога посоромивши. Тому Христос подвійного вінця сподобив тебе, славна Євдокіє, преподобна страстотерпице, моли Христа Бога, щоб спастися душам нашим.

\section{Молитва до 40-а св. мч. Севастійських}
О, святі, славні сорок страстотерпців Христових, що в місті Севастії заради Христа мужньо постраждали. Ви через вогонь та воду пройшовши і як друзі Христові у спокій Небесного Царства увійшовши, щиро молитеся до Пресвятої Тройці за рід християнський, особливо за тих, хто шанує святу пам’ять вашу і з вірою та любов’ю у молитвах звертається до вас: виблагайте у Всещедрого Бога прощення гріхів наших і життя нашого виправлення, щоб, у покаянні та нелицемірній любові один до одного перебуваючи, зі сміливістю стали на Страшному Суді Христовому і вашими молитвами праворуч Праведного Судді стали.

О, угодники Божі, будьте нашими захисниками від усіх ворогів видимих та невидимих, щоб під покровом ваших молитов ми визволялися до кінця життя нашого від усякої біди, зла та небезпеки, і так прославляли велике та гідне поклоніння повновладної Тройці: Отця, і Сина, і Святого Духа нині, і повсякчас, і на віки віків. Амінь.

\section{Молитва до свв. чудотворців та безсрібників, мучеників Кира та Іоана}
Преблаженні Кире та Іоане, Троїчна благодать в серця наші вселившись, виявила вас дивними переможцями нечистих духів. Не лише явні, але й таємні недуги зціляли ви і тому, сміливість до Чоловіколюбця Бога маючи, постійною молитвою душевні наші недуги зціляйте.

\section{Молитва до прпмц. Анастасії Римлянки}
Преподобна Анастасіє, дівствениць похвала і мучениць слава! До тебе з розчуленням сердець наших прибігаємо і твого заступництва перед Господом за нас просимо. Твоєї духовної чистоти та безбоязного сповідання віри ми не наслідували, але багатьма прогріхами, згубними падіннями та відступництвами гнів Божий на себе накликали. Але не хочемо ми померти у гріхах наших, і твої мужні подвиги споглядаючи, знову намагаємося подолати наші пристрасті. Знаючи ж, що без благодатної допомоги ми нічого доброго чинити не можемо, просимо тебе виблагати нам її у Господа. Бо ти, преподобномученице преславна, велику сміливість до Владики здобула, чистотою своєї душі та тілесними стражданнями Його прославивши. Усі погрози та умовляння мучителя зневаживши, виривання зубів і нігтів, відтинання сосків, рук та ніг смиренно перетерпівши, ти щиро взивала: Христос — багатство моє і похвала. Тому від цього багатства й нам, убогим, уділи духовні дари і від безлічі гріхів життя наше збережи, мир та спокій нам виблагай, від смутку та небезпеки своїм заступництвом нас захисти, братолюбності й служінню один одному навчи, і до Господа очі сердець наших завжди спрямовуй, щоб ми повсякчасно славили владу Отця, і Сина, і Святого Духа, і твоє щире заступництво на віки віків. Амінь.

\section{Св. мч. Агрипині}
\emph{Тропар}

Агниця Твоя, Ісусе, Агрипина, взиває сильним голосом: Тебе, Наречений мій, люблю і Тебе, шукаючи, муки терплю і розпинаюся з Тобою, і погребаюся у Твоєму хрещенні, і страждаю Тебе ради, щоб царювати з Тобою, і вмираю за Тебе, щоб жити з Тобою. Тому як жертву непорочну прийми мене, що з любов’ю принесла себе в жертву Тобі, її молитвами, як Милостивий, спаси душі наші.

\section{Св. мч. Діомиду, лікарю}
\emph{Тропар}

Мучительні знаряддя прискорили шлях до Небес тобі, Христовий воїне Діомиде. Перемігши диявольські підступи і на Небесах у Христа прославившись, молися за тих, що з вірою шанують пам’ять твою.

\section{Свв. мчч. Євстратію, Авксентію,Євгенію, Мардарію та Оресту}
\emph{Тропар}

Мучеників всечесних п’ятьох страстотерпців велич оспівуємо: сонцем сяючого Євстратія, премудрого проповідника, зі страждальцями, які за Христа Царя всіх вогонь та муки відважились перетерпіти і від Престолу Його слави переможними вінцями вшановані. Їхніми молитвами, Христе Боже, спаси душі наші.

\section{Св. вмч. Артемію}
\emph{Тропар}

У істинній Христовій вірі утвердившись, страстотерпче, нечестивого царя з його ідолошануванням переміг ти. Тому той, хто вічно царствує, Цар Великий, обдарував тебе сяючим вінцем перемоги. Ти, що зціляєш всіх, хто в хворобах і призиває тебе, Артеміє великий, моли Христа Бога, щоб спастися душам нашим.

\section{Св. вмч. Федору Тирону}
\emph{Тропар}

У джерелі полум’я, як на тихій воді перебуваючи, святий мученик Федір радувався: бо у вогні всепалення, як хліб солодкий, в жертву Тройці самого себе приніс. Його молитвами, Христе Боже, спаси душі наші.

\section{Св. вмч. Іоану Новому, Сочавському}
\emph{Тропар}

Життя на землі вірно скеровуючи, страждальнику, милостинями, частими молитвами і сльозами, а також, на страждання мужньо пішовши, ти перське викрив нечестя. Тому Церкви ти був утвердженням і християнам похвалою, Іоане приснопам’ятний.

\section{Молитва до прп. Олексія, чоловіка Божого}
О, великий Христовий угоднику, святий чоловіче Божий Олексію, що душею на небесах біля Престолу Господнього перебуваєш, а на землі, даною тобі Богом благодаттю, різні чудеса здійснюєш! Зглянься милостиво на людей, які біля ікони твоєї з сокрушенням сердець моляться і просять у тебе допомоги та заступництва. Простягни молитовно до Господа Бога чесні руки твої і виблагай нам у Нього прощення гріхів наших вільних і невільних: тим, що від хвороб страждають, зцілення; тим, що спокуси терплять, заступництво; тим, що у скорботі перебувають, — заспокоєння; тим, що бідують, — швидку допомогу, а всім, що шанують тебе — мирний та християнський кінець життя і добру відповідь на Страшному Суді Христовому.

О, угоднику Божий, не посором надії нашої, яку на тебе після Бога та Владики покладаємо, але будь нашим помічником і покровителем для спасіння. Щоб твоїми молитвами, отримавши благодать та милість від Господа, ми прославляли чоловіколюбність Отця і Сина, і Святого Духа, в Тройці славимого і поклоняємого Бога, і твоє святе заступництво, нині, і повсякчас, і на віки віків. Амінь.

\section{Молитва до прп. Сергія Радонезького, чудотворця}
Як чеснот ревнитель та істинний воїн Христа Бога, у стражданнях вельми подвизався ти, в житті тимчасовому. У співах, нічних молитвах та постах ти був зразком для учнів своїх. Тому й вселився в тебе Пресвятий Дух, дією Якого світло прикрашений ти. Маючи ж сміливість до Святої Тройці, поминай стадо, яке мудро зібрав ти, і не забувай, як і обіцяв, коли відвідував духовних чад своїх, Сергію преподобний отче наш.

\section{Молитва до свв. прпп. Зосима та Саватія Соловецьких чудотворців}
О, преподобні отці, великі заступники, які скоро до молитов прислухаються, угодники Божі, чудотворці Зосимо та Саватію! Не забувайте, як обіцяли, відвідувати чад ваших. Бо хоча ви тілом і відійшли від нас, але духом повсякчасно з нами перебуваєте. Тому молимо вас, о преподобні: визволіть нас від вогню та меча, від нашестя чужинців та міжусобиць, від нищівних вітрів, від несподіваної смерті та від усіх бісівських хитрощів, які спрямовані проти нас. Почуйте нас, грішних, і прийміть молитву цю та моління наше, як кадило запашне, як жертву приємну, і душі наші, що нечестивими ділами, намірами та помислами умертвилися, оживіть. Ви спочилу отроковицю підняли, невиліковні рани багатьох зцілили та тих, що від злих духів потерпають, визволили, так і нас, що в кайданах ворожих перебуваємо, визволіть, від тенет диявольських звільніть, із глибини гріхів виведіть і милостивим вашим піклуванням та заступництвом від ворогів видимих та невидимих захистіть нас, благодаттю і силою Пресвятої Тройці завжди, нині, і повсякчас, і на віки віків. Амінь.

\section{Молитва до прп. Олександра Свірського}
О, священна главо, ангеле земний і чоловіче небесний, преподобний і богоносний отче наш Олександре, преславний угоднику Пресвятої і Єдиносущної Тройці, ти велику милість подаєш тим, що у святій обителі твоїй живуть та всім, що з вірою і любов’ю до тебе прибігають. Виблагай нам все, що для цього тимчасового життя необхідне, а особливо те, що для нашого вічного спасіння погрібне. Допомагай заступництвом твоїм, угоднику Божий, керманичам країни нашої. Нехай у непохитному мирі перебуває свята Православна Церква Христова. Будь для всіх нас, чудотворче святий, у всякій скорботі та напастях скорим помічником. Особливо ж в час кончин нашої будь нам заступником милосердним, щоб не були ми віддані на митарства злобного правителя нечистих духів, але сподобилися безперешкодно увійти до Царства Небесного.

Ти, отче, молитвенику наш повсякчасний! Не посором нас у надії нашій, не зневаж смиренних молінь наших, але повсякчасно молися за нас перед Престолом Живоначальної Тройці, щоб ми, разом із тобою й усім святими сподобилися в райських оселях славити велич благодать і милість Єдиного в Тройці Бога: Отця, і Сина, Святого Духа на віки віків. Амінь.

\section{Молитва до св. прп. Микити стовпника, Переяславського чудотворця}
О, всечесна главо, преподобний отче, преблаженний Микито, преподобномученику! Не забудь убогих твоїх до кінця, але згадуй нас завжди у святих своїх і благоприємних молитвах до Бога і не забувай піклуватися про чад своїх, Моли Бога за нас, отче благий і обранцю Христовий, бо ти маєш сміливість перед Небесним Царем. Постійно молися за нас до Господа і не відкинь нас, що вірою та любов’ю шануємо тебе. Поминай нас, недостойних, коло Престолу Вседержителя і не переставай молитися до Христа Бога; бо тобі дана була благодать молитися за нас. Не вважаємо тебе мертвим: бо, хоча тілом ти і відійшов від нас, але по смерті живим залишаєшся. Не відступай від нас духом, але захищай та зберігай нас від ворожих стріл та всяких бісівських принад, заступнику і молитвенику наш добрий. Бо хоча й рака з твоїми мощами завжди перед очима нашими перебуває, але свята твоя душа з ангельськими воїнствами, з Безплотними ликами, з небесними силами біля Престолу Вседержителя Бога достойно веселиться. Тож, знаючи, що ти після смерті воістину живим залишаєшся, до тебе припадаємо, тобі і смиренно молимось і доручаємо себе твоєму заступництву, щоб ти молився за нас до Всесильного Бога про корисне для душ наших. Виблагай нам час для покаяння і щоб безперешкодно перейти нам від землі до Неба, і тяжких митарств, повітряних князів та вічної муки визволитися, і Небесного Царства спадкоємцями бути зо всіма праведними, які від віку угодили Йому, Господеві нашому Ісусу Христу. Йому ж належить всяка слава, честь і поклоніння з Безпочатковим Його Отцем і Святим Духом нині, і повсякчас, і навіки віків. Амінь.

\section{Молитва до прп. Афанасія Афонського}
Преподобний отче Афанасію, славний угоднику Христовий і великий Афонський чудотворець. Ти, що в дні земного життя свого багатьох на шлях істини направив і до Царства Небесного мудро скеровував, скорботних утішав, тим, що впали — руку допомоги подавав і для всіх люб’язним, милостивим та співчутливим отцем був! Ти і нині, в небесному сяйві перебуваючи, особливо примножуєш любов свою до нас немічних, що в морі життєвому потерпаємо від різних небезпек, спокушені духом злоби та власними пристрастями, що повстали супроти духу нашого. Тому смиренно молимо тебе, святий отче, благодаттю, що дана тобі від Бога, допоможи нам зі щирим серцем та смиренням волю Господню виконувати, спокуси ворожі перемогти і лютих пристрастей море висушити. Щоб ми неушкодженими подолали житейське море і заступництвом твоїм перед Господом, сподобились досягти обіцяного нам Царства Небесного, славлячи Безначальну Тройцю: Отця, і Сина, і Святого Духа нині, повсякчас, і на віки віків. Амінь.

\section{Молитва до прп. Афанасія, ігумена Берестейського}
О, предивний страстотерпцю та великий угоднику Христовий, отче наш Афанасію! Зі смиренням нині схиляємо коліна сердець наших і взиваємо до тебе: принеси молитву нашу до Милостивого і Всесильного Бога, і виблагай нам у Його благості все, що корисне для душ і тіл наших: віру правдиву, надію щиру, любов нелицемірну, під час спокус — мужність, у муках — терпіння, в благочесті — успіх, щоб не на зло ми скористалися дарами Всеблагого Бога, але на славу Його святого імені і для прославлення твого за нас заступництва. Церкву ж святу від розколів і єресей збережи, вірних зміцни, заблудлих на шлях правдивої віри наверни; батьківщину нашу від ворожих утисків збережи і воїнству нашому допомагай у боротьбі. Виблагай, угоднику Христовий, у Царя царюючих і Господа володарюючих благодать, силу і благословення небесне керманичам нашої держави і подай Україні нашій спокій та благоустрій. Не забудь, чудотворцю святий, і про святу обитель цю: збережи і охорони її твоїми молитвами від усякого зла. О, Божий святий, пом’яни всіх, хто до твоєї священної раки приступає і сподоби їх благу кончину отримати і Небесне Царство успадкувати, щоб ми прославили дивного у святих своїх Бога, якого прославляємо в трьох іпостасях, Йому ж належить всяка слава, честь і поклоніння на віки віків. Амінь.

\section{Молитва до прп. Мойсея Мурина про зцілення від пияцтва}
О, велика сила покаяння! О неосяжна глибина Божого Милосердя! Ти, преподобний Мойсею, був спершу розбійником, але потім пройнявся жахом від гріхів своїх, і сповнився скорботою за них і в розкаянні прийшов у монастир, де зі сльозами за свої беззаконня у великих подвигах посту та молитви провів решту днів своїх і був удостоєний Христової благодаті прощення й дару творити чудеса. О, преподобний, ти від тяжких гріхів піднявся до дивовижних чеснот, допоможи і рабам Божим (імена), що прямують до погибелі своєї через надмірне споживання спиртного, яке губить безсмертну душу та тіло — храм Духа Святого. Споглянь на них своїм милостивим оком і не відкинь, але вислухай тих, що до тебе вдаються. Моли, святий Мойсею, Владику Христа, щоб Він, Милостивий, не відкинув їх, і щоб не возрадувався диявол їхній загибелі, але щоб помилував Господь цих безсилих і нещасних, яких охопила згубна пристрасть до пияцтва, бо всі ми створіння Божі, очищені Пречистою Кров’ю Сина Його. Почуй же, преподобний Мойсею, молитву їхню і нашу. Віджени від них диявола, даруй їм силу побороти згубну пристрасть, поможи їм, простягни руку свою, виведи їх на дорогу добра, звільни їх від рабства пристрастей і порятуй від недуги пияцтва, щоб вони, оновлені у тверезості й при світлому розумі возлюбили стриманість і благочестя і вічно прославляли Всеблагого Бога, Який завжди рятує створіння Свої. Йому ж належить всяка слава, честь і поклоніння на віки віків. Амінь.

\section{Молитва до прп. Марона, пустельника Сірійського}
О, дорогоцінна і священна главо, преподобний і богоносний наш отче Мароне! Споглядай милостиво з небес на нас, грішних, і молися до Всещедрого Владики і всіх благ Подателя Бога, за збереження нашої держави, щоб Він цілою та неушкодженою зберіг її від усіх ворогів. Молися і за нас, недостойних. До Всеблагого Господа свої чесні руки простягни і у Нього милості та щедроти виблагай, ними ж від усіх бід і лютих хвороб нас збережи, і від усяких пристрастей визволи, а також від нестерпної та неутішної лихоманки, жару і нападів бісівських твоїми молитвами нас визволи; від видимих і невидимих ворогів збережи, і гріхів наших прощення виблагай, і спасенними нас перед Христом яви, щоб в день Страшного Суду нам з радістю стати перед лицем Його невимовної слави, і Його милостивий голос, що кличе нас до Небесного Царства почути, і у невимовній радості преславну доброту Його лиця побачити, благодаттю і чоловіколюбством Господа Бога і Спаса нашого Ісуса Христа. Бо Його влада благословенна і препрославлена з Безпочатковим Його Отцем і з Пресвятим і Благим, і Животворчим Його Духом нині, і повсякчас, і на віки віків. Амінь.

\section{Молитва до прп. Іринарха, затворника Ростовського}
О, великий угоднику Христовий, добровільний страждальнику, що чудесами просяяв, отче наш Іринарху. Хто не буде дивуватися твоєму вільному і багаторічному страдницькому терпінню: бо на тридцять років у тісній оселі, ти себе замкнув, холод, недоїдання та виснаження тіла заради Небесного Царства ти переніс, до того ж через підступи диявольські без нарікання вигнання з обителі перетерпів. Знаємо ж, що згодом, як незлобивий агнець, ти з братією примирився і знову в обитель свою повернувся, в ту ж тісну оселю вселився і, як міцний діамант, на невидимі бісівські полчища та видимих ворогів терпінням себе озброїв. Коли ж Божим попущенням цю обитель польські вояки захопили, ти їхніх смертельних погроз не злякався, але повчаючи їх своїми настановами, спонукав повернутися назад. Заради цього, Всеблагий Бог, бачачи твою віру та страждальницьке довготерпіння, дар прозорливості та зцілення тобі дарував: біснуватим зцілення, кульгавим одужання, сліпим прозріння подав ти, і багато чого іншого тим, що з вірою до тебе приходять, навіть і донині на благо чудодієш. Ми ж, недостойні, такі чудеса бачачи і радістю наповняючись, взиваємо до тебе: радуйся, мужній страждальнику і бісів переможцю, радуйся, наш скорий помічнику і щирий до Бога молитвенику. Почуй, отже, і нас, грішних, що молимось тобі і до твого захисту прибігаємо: вияви твоє милостиве заступництво за нас перед Всевишнім і виблагай твоїми богоприємними молитвами все, що корисне для спасіння душ і тіл наших, збережи святу обитель цю, кожне місто і  кожну країну християнську від підступів ворожих, в скорботах і хворобах наших подавай нам руку допомоги, щоб твоїм заступництвом та молитвами, благодаттю ж і милосердям Христа Бога нашого і ми, недостойні, після своєї кончини, визволилися від стояння ліворуч, але зо всіма святими, стояти праворуч сподобилися, на віки віків. Амінь.

\section{Св. прп. Андрію Рубльову}
\emph{Тропар}

Божественного світла промінням осяяний, преподобний Андрію, Христа пізнав ти — Божу Премудрість і Силу; іконою Святої Тройці всьому світу проповідував ти Єдність у Святій Тройці. Ми ж із подивом та радістю співаємо тобі: ти, що маєш сміливість до Пресвятої Тройці, молися, щоб просвітитися душам нашим.

\section{Прп. Феодорові Студиту}
\emph{Кондак}

Непереможна предвічна Любове, Господи Вседержителю! Прийми від нас, найостанніших із останніх слуг Твоїх, цю подячну і благальну пісню! А ти, преподобний отче наш Феодоре Студите, що обіцяв неперестанно молитися за нас, випроси у Господа благодать, про яку благаємо!

\section{Прп. Феофанові Сигріанському, сповіднику}
\emph{Тропар}

Стриманням життя прикрасив, і, пристрасті умертвивши, ворожі підступи переміг ти, отче Феофане, перейшов ти до Бога у вічне життя, як достойний причасник, безперестанно молячись, щоб помилувані були душі наші.

\section{Молитва до прп. Ніла, Сорського чудотворця}
О, преподобний і богоблаженний отче Ніле, богомудрий наставнику і учителю наш! Ти, заради Божої любові від мирського життя ухиляючись, в непрохідній пустині та лісових нетрях оселитися зволив і, як плідна лоза дітей пустині намножив; словом, писанням і життям образ усіх чернечих чеснот показав ти і, як ангел у плоті, на землі поживши, нині в небесних оселях, де голос святкуючих невпинно лунає, перебуваєш, і з ликами святих перед Богом стоячи, Йому славу та хвалу безперестанно приносиш. Молимо тебе, богоблаженний, навчи і нас, що під покровом твоїм живемо, непохитно стопами твоїми ходити і Господа Бога всім своїм серцем любити, Його Єдиного жадати і про Нього Єдиного помишляти; мужньо ж і вправно ворожим підступам та думкам, що до пекла впроваджують, протидіяти, і завжди їх перемагати; допоможи нам всілякі труднощі монашеського життя полюбити, а принади цього світу, заради любові Христової, зненавидіти; всілякі чесноти, що в них потрудився ти, в серцях наших зростити. Моли Христа Бога, щоб і всім православним християнам, які у світі живуть, Він просвітив розум і очі сердечні для спасіння, зміцнив їх у вірі та благочесті і виконанні заповідей. Нехай рабів Своїх від хитрощів світу цього збереже, і прощення гріхів всім християнам дарує, також і все, що необхідне для тимчасового життя всім подасть. Нехай усю країну нашу в любові, однодумності та довголітті збереже, усіх керманичів, владу та військо її на шлях правди та вірності наставить. Щоб всі християни, які в пустелі та в мирі перебувають, тихе і спокійне життя прожили у всякому благочесті та чистоті, і устами та серцем прославили Христа з Безпочатковим Його Отцем і Пресвятим, і Благим, і Животворчим Духом завжди, нині, і повсякчас, і на віків. Амінь.

\section{Прп. Зотику}
\emph{Кондак}

За любов Христову, преподобний, убогих возлюбив і їм допомагав ти, Зотику пребагатий. Тому всі ми тебе шануємо, святкуючи пам’ять твою.

\section{Прп. Меланії Римлянці}
\emph{Кондак}

Чистоту дівства полюбивши і обручника на добро наставивши, безліч багатства витратила ти, богоблаженна, і воздвигла обитель для життя іноків. Тим то в Небесних обителях вселившись, поминай нас, всечестива Меланіє.

\section{Молитва до прп. Макарія Олександрійського}
О, священна голово, земний ангеле і небесний чоловіче, преподобний і богоносний отче наш Макарію! Припадаємо до тебе з вірою та любов’ю, і ревно молимось: яви нам смиренним та грішним твоє святе заступництво. Бо через гріхи наші не маємо волі як чада Божі у потребах Господа і Владику нашого просити, але тебе як щирого молитвеника Йому пропонуємо і просимо тебе щиросердно: виблагай у Його благості всі необхідні для душ і тіл наших дари: віру правдиву, надію на спасіння непохитну, любов до всіх нелицемірну, під час напастей — терпіння, в молитвах — сталість, для душ і тіл — здоров’я, землі — плодючість, сприятливої погоди, життєвих потреб задоволення, мирного і безтурботного життя, благої християнської кончини і доброї відповіді на Страшному Суді Христовому. Не забувай, преподобний отче, і пустельне місце подвигів твоїх, але виявляй милість свою до нього і прослав його твоїми чудесами, і всіх, що приходять для поклоніння до твоїх святих мощів, від спокус диявольських та всякого зла милостиво визволи. О, чудотворцю святий! Не позбав нас твоєї небесної допомоги, але молитвами твоїми, всіх нас приведи до пристановища спасіння і яви нас спадкоємцями Всесвятого Царства Христового, щоб ми оспівували і славили невимовні щедроти Чоловіколюбця Бога, Отця, і Сина, і Святого Духа, і твоє святе отцівське заступництво на віки віків. Амінь.

\section{Молитва до прп. Самсона Странноприїмця}
О, щирий молитвенику, благосний отче, преподобний Самсоне! Моли Бога за мене, грішного, і пошли мені від Всеблагого Владики допомогу та визволення, бо життя моє тимчасове і переповнене труднощами, скорботами та хворобами. Зміцни серце моє, щоб я спромігся понести мій тягар, і не попусти, щоб численні спокуси перемогли мої малі сили, але допомагай мені твоїм заступництвом, і, посеред напастей та бід, спрямуй мій шлях до Царства Небесного, щоб мені славити Господа, Який в тобі прославився, навіки. Амінь.

\section{Молитва до прп. Онуфрія Великого}
О, пречудесний і преблаженний угоднику Христовий, Онуфрію Великий! Ти дивовижну любов до Господа твого показав і на дивовижні подвиги благодаттю Його зміцнився і тому великої сміливості до Нього сподобився: бо багато чудес та знамень сили Божої через тебе людям явилося. Споглянь отже і нині, отче предивний, милосердним оком твоєї любові на нас, недостойних рабів, і даруй кожному по його потребі, вірі та надії: скорботних обрадуй; тих, що плачуть, утіш; недужих зціли, тим хто трудиться, допоможи; тих, що з пристрастями та бісівськими хитрощами борються, зміцни і благослови; знемагаючих підтримай, обитель твою і всіх нас від ворогів видимих і невидимих та від усякого зла збережи. Святу православну віру в батьківщині нашій звелич, наверни заблудлих, тих, що відпали, врозуми, утверди хитких, пом’якши упертих, просвіти невірних і всіх приведи до тихого пристановища Небесної Вітчизни.

О, предивне слово монахів і всіх вірних утіхо! Славними твоїми подвигами та солодким спогляданням твого чудотворного образу і нас осяй, і на всякий труд, подвиг і терпіння зміцни на славу імені Божого, щоб ми спаслися і сподобилися з тобою та зо всіма святими Вічного і Преблагословенного Царства слави Отця, і Сина, і Святого Духа на віки віків. Амінь.

\section{Прп. Іоану Кущнику}
\emph{Кондак}

Від життєвих турбот ухилившись і до Божого пристановища свій шлях направивши, ти, блаженний, зцілень потоки виточуєш усім, хто з вірою та любов’ю до тебе приходить і взиває: молися в день Суду за тих, що шанують світлосяйну пам’ять твою, отче, і від мук та страждань визволи.

\section{Прп. Макарію Великому Єгипетському}
\emph{Кондак}

Блаженне життя з ликами мучеників скінчивши, в землі смиренних достойно перебуваєш, богоносний Макарію, і пустелю, як місто, населивши, благодать чудес прийняв ти від Бога. Тому тебе шануємо.

\section{Прп. Антонію Великому}
\emph{Кондак}

Життєві тривоги відкинувши, усамітнено життя скінчив ти, преподобний, Хрестителя всіляко наслідуючи. З ним же тебе, отців начальнику, Антонію, шануємо.

\section{Прпп. Ксенофонту і Марії}
\emph{Кондак}

Від житейського моря ухилившись, праведний Ксенофонт разом з дружиною чесною на Небесах веселиться разом з чадами, Христа величаючи.

\section{Прп. Лаврентію Турівському}
\emph{Кондак}

Блаженний Лаврентію, ти, щиро Печерській Лаврі прославлення від чудес бажаючи, а себе самого смиряючи, відмовився одного чоловіка, що терпів муки від диявола, зцілити і послав його до печери, де він, зцілившись, прославив з тобою Господа, Який прославляє святих Своїх.

\section{Прп. Іоану Рильському}
\emph{Тропар}

Покаяння основою, проповіданням розчулення, образом утішення, духовного вдосконалення, подібним до ангельського життя твоє було, преподобний. Тому в молитвах, постах і в сльозах перебуваючи, моли, отче Іоане, Христа Бога за душі наші.

\section{Прп. Алімпію Печерському}
\emph{Тропар}

Образи святих на дошках зображаючи, їхні чесноти, як умілий художник, всехвальний Алімпію, старанно на скрижалях свого серця написав ти, і тому, як чудово прикрашений образ, благодаттю священства, як золотом, покрився ти від Христа Бога і Спаса душ наших.

\section{Прп. Іпатію Цілителю}
\emph{Тропар}

Безначального і тихого місця, де немає ні журби, ні зітхання, досягнути, преподобний, бажаючи, не давав ти собі тут спокою, але постійно день і ніч у всяких ділах та суворому житті працелюбно подвизаючись, ти бажане вже отримав, Іпатію. Тож молися за душі наші.

\section{Прп. Спиридону, проскурнику Печерському}
\emph{Тропар}

Хліби принесення своїми руками творячи, безперестанно вустами своїми псалмоспів, як жертву хвали Господу, разом із чесним Никодимом приносив ти, блаженний Спиридоне, з ним же молися до Христа Бога за душі наші.

\section{Прп. Аврамію Затворнику}
\emph{Кондак}

Як ангел у тілі на землі явився ти, і в пості, був подібним до дерева, яке водою стриманості зростало, потоками сліз своїх скверну омив ти. Тому явився ти, Авраамію, храмом Божественного Духа.

\section{Молитва до прпп. Даміана, пресвітера і цілителя, Єремії та Матфея прозорливих, Печерських}
Світлом Христових повелінь свої серця просвітивши і морок пристрасті відігнавши, храмом Тройці стали ви, троє отців, Даміане, Єреміє та Матфею. Отримавши від Неї Благодать, недужих зціляєте, майбутнє звіщаєте і ангелів спільниками перебуваєте. Моліться до Христа Бога, щоб дарував нам єдність зо святими. Амінь.

\section{Прп. Роману Солодкоспівцю}
\emph{Кондак}

Божественними чеснотами духу замолоду прикрасившись, Романе премудрий, Церкви Христової пречесною прикрасою був ти, бо чудовим співом прикрасив її, блаженний. Тому молимо тебе: подай всім, що бажають твого божественного дару, щоб ми співали тобі: радуйся, отче преблаженний, церковна окраса.

\section{Прп. Мартиніану}
\emph{Кондак}

Як вправного подвижника благочестя, що добровільно на чесні страждання пішов, і мешканця пустелі, в піснях достойно звеличаємо завжди хвального Мартиніана, який змія переміг.

\section{Прп. Василію Сповіднику}
\emph{Тропар}

Мешканцем пустелі, і ангелом в тілі, і чудотворцем був ти, богоносний отче наш, Василію. Постом, бдінням та молитвою Небесні дарування прийнявши, зцілюєш ти недужих та душі тих, що з вірою до тебе вдаються. Слава Тому, Хто дав тобі силу, слава Тому, Хто вінчав тебе, слава Тому, Хто посилає всім через тебе зцілення.

\section{Прп. Петру Афонському}
\emph{Кондак}

Від співжиття з людьми відокремившись, в кам’яних печерах та проваллях прагненням Бога та любов’ю до Нього жив ти, Петре. Моли повсякчасно Господа твого, від Якого вінець прийняв ти, щоб спастися нам.

\section{Прп. Агапиту Печерському, безкорисному лікарю}
\emph{Тропар}

Смиренність богоносного Антонія наслідуючи, поживними рослинами, наче ліками хворих зціляв ти, преподобний Агапите, чим і лікаря іновірного переконавши, на шлях спасіння наставив. Зціли і наші хвороби і молися до Господа за тих, що оспівують тебе.

\section{Молитва до прп. Марії Єгипетської}
О, велика Христова угоднице, преподобна Маріє! На небесах перед Божим Престолом стоячи, на землі ж духом любові з нами перебуваєш, маючи сміливість перед Господом, молися, щоб спаслися раби Його, що до тебе з любов’ю припадають. Виблагай нам у Великомилостивого Владики і Господа бездоганне дотримання віри, міст і сіл наших утвердження, від голоду й мору визволення, сумуючим — утіху, хворим — зцілення, занепалим — відродження, заблудлим — зміцнення, в благих справах утіху та благословення, сиротам і вдовам — захист і тим, що від цього життя відійшли, вічне упокоєння, всім же нам в день Страшного Суду спільниками тих, що праворуч перебувають, стати і блаженний голос Судді нашого почути: прийдіть, благословенні Отця Мого, успадкуйте уготоване вам Царство від сотворення світу, і там повіки перебувайте. Амінь.

\section{Прп. Паїсію Великому}
Божественною любов’ю від юності запалившись, преподобний, всю красу світу ти зненавидів, Христа ж Єдиного полюбив ти і тому в пустелі оселився, де й Божественного явлення сподобився. Впавши ниць, ти поклонився Тому, на Кого й ангели не можуть дивитися. Великий же Дародатель, як Чоловіколюбець, промовив до тебе: не жахайся, улюблений Мій, твої діла приємні для Мене, ось даю тобі дар: за якого грішника ти помолишся, той отримає прощення гріхів. Ти ж у чистоті свого серця запалав, прийняв воду і доторкнувся до Недосяжного, умивши Його ноги і пивши воду, даром чудес збагатився, щоб хворих зціляти, бісів від людей відганяти і грішників від мук своєю молитвою звільняти. О преподобний отче Паїсію, молю тебе, щоб ти за даною тобі Богом обітницею, ублагав Його за мене, бо серед грішників — я перший, щоб дав мені Господь час для покаяння; простив мій гріх, бо Він Благий і Чоловіколюбець, щоб разом зо всіма я заспівав йому: Алилуя.

\section{Прп. Паїсію Величковському}
Тропар

У тобі, отче, дійсно спаслося те, що було образом Божим; взявши бо хрест, пішов ти за Христом, і ділом навчав нехтувати тілом, бо воно минається, а дбати про душу — єство безсмертне; тому й радіє з ангелами, преподобний Паїсіє дух твій.

\section{Прп. Максимові Греку}
\emph{Кондак}

Проповіддю богонадхненного Писання і Богослов’я марновірство невірних викрив ти, безмежно благий. Православ’ю навчаючи, на шлях істинного пізнання наставив. Як богогласна сопілка, звуком якої слухачі насолоджуються, невпинно звеселяєш ти, Максиме, достойний подиву тим то молимо тебе: молися до Христа Бога, щоб Він послав прощення гріхів тим, що з вірою оспівують всесвяте успіння твоє.

\section{Молитва до святої блаженної Ксенії Петербурзької}
О, свята всеблаженна мати Ксеніє! Ти, що під покровою Всевишнього жила, і Богоматір’ю навчена та зміцнена була, голод і спрагу, холод і спеку, знущання і гоніння перетерпівши, дар прозорливості й чудотворіння від Бога прийняла, і під захистом Всемогутнього перебуваєш. Нині свята Церква, як благоуханний цвіт, прославляє тебе. Стоячи на місці погребіння твого перед образом святим твоїм, як до живої, що з нами перебуваєш, молимось тобі: прийми прохання наші і принеси їх до Престолу Милосердного Отця Небесного, бо ти маєш сміливість перед Ним. Випроси всім, хто приходить до тебе, вічне спасіння, на добрі діла й починання наші щедре благословення, від всяких бід і скорбот визволення. Заступися святими твоїми молитвами перед Всемилостивим Спасителем нашим за нас, недостойних і грішних. Допоможи, свята блаженна мати Ксеніє, младенців світлом святого хрещення осяяти і печаттю дару Духа Святого ознаменувати, юних у вірі, чесності, богобоязливості і цнотливості виховати і успіхи у навчанні дарувати. Хворих і недужих — зціли, сімейним — любов і згоду пошли, чернецтво подвигами добрими подвизатися удостой і від наруги захисти, пастирів у кріпості Духа Святого утверди, народ і країну нашу в мирі і спокої збережи, всіх боголюбивих чад у передсмертний час причастя Святих Христових Таїн сподоби. Бо ти наша надія, що завжди нас вислуховуєш і визволення посилаєш. Тобі подяку возносимо і з тобою славимо Отця, і Сина, і Святого Духа і нині, і повсякчас, і на віки віків. Амінь.

\section{Молитва до святих чудотворців і безсрібників Косми та Даміана}
До вас, святі безсрібники та чудотворці Космо і Даміане, як до скорих помічників та гарячих молитвеників за наше спасіння ми, недостойні, схиливши коліна, вдаємося і, припадаючи, щиро благаємо: не зневажте моління нас, грішних, немічних, що множество беззаконь вчинили, і кожного дня та години грішимо. Ублагайте Господа, щоб подав нам, недостойним рабам Своїм великі й багаті Свої милості. Визволіть нас від всякої скорботи і хвороби. Бо ви прийняли від Бога і Спаса нашого Ісуса Христа невичерпну благодать зцілень заради твердої віри вашої, безоплатного лікування та мученицької кончини. Так, угодники Божі, не переставайте молитися за нас, що з вірою до вас вдаємося: через безліч гріхів наших, ми, недостойні вашого милосердя, але ви, як вірні наслідники Божого чоловіколюбства моліть, щоб ми принесли достойні плоди покаяння і досягли вічного спокою, прославляючи і благословляючи дивного у святих Своїх Господа Бога і Спаса нашого Ісуса Христа, і Пречисту Його Матір, і ваше щире заступництво завжди, нині, і повсякчас, і навіки віків. Амінь.

\section{Молитва до блаж. Василія, Христа ради юродивого}
О, великий угоднику Христовий, істинний друже і вірний рабе Всетворця Господа Бога, преблаженний Василію! Почуй нас, многогрішних, що нині благаємо тебе і святе ім’я твоє прикликаємо: помилуй нас, що сьогодні до раки з твоїми мощами припадаємо, прийми це наше мале і недостойне моління, змилуйся над убогістю нашою і молитвами твоїми зціли всяку недугу та хворобу душі та грішного нашого тіла, і сподоби нас цей скороминущий шлях нашого життя безгрішно, неушкоджено від видимих та невидимих ворогів пройти, і бездоганну, мирну, спокійну християнську кончину та Небесного Царства насліддя отримати зо всіма святими на віки віків. Амінь.

\section{Тропар св. Ісидору Христа ради юродивому}
Просвітившись згори Божественною благодаттю, Богомудрий, з великим упованням тимчасове життя ти пройшов, тому й після упокоєння виявляєш чудеса всім, хто з вірою припадає до раки з твоїми мощами, всеблаженний Ісидоре. Слава Тому, Хто дав тобі силу, слава Тому, Хто прославив тебе в чудесах, слава Тому, Хто посилає тобою всім зцілення.

\section{Кондак св. Андрію Христа ради юродивому}
Добровільно юродствуючи, ти цілковито зневажив принади цього світу. Постом і спрагою, спекою і морозом тілесні бажання ти вгамував, від дощу і снігу та від усякої негоди ніколи не ухилявся, тим і очистив себе, як золото в горнилі, Андрію блаженний.

\section{Пророку}
\emph{Тропар}

Пророка Твого \emph{(ім’я)} пам’ять, Господи, святкуючи, ним Тебе молимо, спаси душі наші.

\section{Кондак}

Просвітлене духом чисте твоє серце було вмістилищем пророцтва пресвітлого пророкування, бо те, що далеко, ти бачиш як нинішнє; тому тебе, пророче (ім’я) блаженний, славний шануємо.

\section{Апостолу}
\emph{Тропар}

Апостоле святий \emph{(ім’я)}, моли милостивого Бога, щоб відпущення гріхів подав душам нашим.

\emph{Кондак}

Як зірку пресвітлу Церква назавжди здобула тебе, апостоле (ім’я), просвітлюючись даруванням многих чудес твоїх. Тому взиваємо до Христа: спаси тих, що з вірою шанують пам’ять Твого апостола, Многомилостивий.

\section{Святителю}
\emph{Тропар}

Правилом віри і образом лагідності, стриманості вчителем явив тебе стаду твоєму Той, Хто є Істиною всіх речей. Ради цього придбав ти смиренням — високе, убогістю — багатство. Отче святителю (ім’я), моли Христа Бога, щоб спастися душам нашим.

\emph{Кондак}

Громе Божественний, сурмо духовна, віри насадителю і єресей нищителю, Тройці угоднику, великий святителю (ім’я), завжди з ангелами стоячи, моли безперестанно за всіх нас.

\section{Молитва до святителя}
О, пречесна, священна, сповнена благодаті Святого Духа главо, Спасова з Отцем обитель, великий архієрею, теплий наш заступнику, святителю (ім’я). Ти, що біля Престолу Царя всіх стоїш і світлом Єдиносущної Тройці насолоджуючись, по-херувимському разом з ангелами трисвяту пісню виголошуєш, велику і незбагненну сміливість до всемилостивого Владики маючи, молися, щоб спаслися люди Христової пастви, благоустрій у святих церквах утверди; архієреїв достойністю святительства прикраси, чернецтво на шляху доброго подвигу зміцни, наше місто і всі міста та країни у добрі збережи; молися, щоб святу непорочну віру ми зберегли, увесь світ своїм заступництвом утихомир, від голоду і згуби нас визволи, від чужоземних нападів захисти, старих утіш, юних навчи, безумних врозуми, вдовиць помилуй, сиріт захисти, дітей зрости, рідних поверни, немічних зціли. І всюди, де до тебе зі щирою вірою припадають і моляться, від всяких напастей і бід своїм заступництвом збережи: моли за нас Всещедрого і Чоловіколюбного Христа Бога, щоб і в день Страшного Його Пришестя від стояння ліворуч нас визволив, але радості успадкування святості сподобив нас з усіма святими на віки віків. Амінь.

\section{Мученикові}
\emph{Тропар}

Мученик Твій, Господи, \emph{(ім’я)}, в стражданні своїм вінець нетлінний прийняв від Тебе, Бога нашого; маючи бо силу Твою, мучителів подолав, сокрушив і демонів немічні спокуси. Його молитвами спаси душі наші.

\emph{Кондак}

Зіркою світлою, необлесливою явився ти світові, страстотерпче (ім’я), Сонце Христа зорею твоєю сповіщаючи; і всю оману ти знищив, а нам подаєш світло, молися безперестанно за всіх нас.

\section{Мучениці}
\emph{Тропар}

Агниця Твоя, Ісусе, \emph{(ім’я)} взиває великим гласом: «Тебе, Наречений мій, люблю і, Тебе шукаючи, страждаю, і співрозпинаюся, і співпогребаюся в хрещенні Твоїм, і страждаю Тебе ради, щоб царювати з Тобою, і вмираю за Тебе, щоб жити з Тобою; прийми ж мене як жертву непорочну, що з любов’ю принесла себе в жертву Тобі». Її молитвами, як Милостивий, спаси душі наші.

\emph{Кондак}

Храм твій всечесний, як лічницю духовну отримавши, ми, всі вірні, вельмигласно взиваємо до тебе: о, дивна мученице (ім’я) славна, моли Христа Бога безперестанно за всіх нас.

\section{Преподобному}
\emph{Кондак}

Чистотою душевною божественно озброївшись, безперестанні молитви, наче спис, міцно в руці тримаючи, переміг ти сили бісівські, отче наш (ім’я), моли безперестанно за всіх нас.

\section{Молитва до преподобного}
О, священна главо, преподобний отче, преблаженний авво (ім’я), не забудь убогих твоїх до кінця, але завжди згадуй нас у твоїх святих і благоприємних молитвах до Бога: пом’яни стадо твоє, яке скерував ти, і не забувай відвідувати чад своїх, молися за них, отче святий, за твоїх духовних дітей, бо ти маєш сміливість перед Небесним Царем. Не умовкай за нас перед Господом і не відкинь нас, що з вірою та любов’ю тебе шануємо. Поминай нас, недостойних, біля Престолу Вседержителя і не переставай молитися за нас до Христа Бога, бо тобі дана була благодать за нас молитися. Ми не вважаємо тебе за мертвого, бо хоча ти тілом і відійшов від нас, але й після смерті живим залишаєшся. Не відступай своїм духом від нас, але зберігай нас, добрий наш пастирю, від ворожих стріл та всіляких бісівських спокус і диявольських підступів, бо хоча й рака з твоїми мощами завжди перед нашими очима, але твоя свята душа з ангельськими воїнствами, з безплотними ликами, з небесними силами, біля Престолу Вседержителя перебуваючи достойно веселиться. Знаючи бо, що ти воістину і по смерті живий, до тебе припадаємо і молимось: молися Всемилостивому Богові за корисне для душ наших і виблагай нам час для покаяння, щоб від тяжких митарств, бісів повітряних князів і вічних мук визволившись, ми безперешкодно перейшли від землі на небо і Небесне Царство успадкували разом зі всіма праведниками, що від віку вгодили Господу нашому Ісусові Христові, Йому ж належить всяка слава, честь і поклоніння з Безначальним Його Отцем, і з Пресвятим, і Благим, і Животворчим Його Духом, нині, і повсякчас, і на віки віків. Амінь.

\section{Преподобній}
\emph{Тропар}

У тобі, мати, вповні спаслася душа, створена за образом Божим; взявши бо хрест, пішла ти за Христом і ділом навчала не про тіло дбати, бо воно тимчасове, а про душу — єство безсмертне; тому, преподобна (ім’я), разом з ангелами і радіє дух твій.

\emph{Кондак}

З любові до Господа, преподобна, бажання спокою зненавиділа єси: постом дух твій просвітивши, сильні пристрасті перемогла єси; подолай же молитвами твоїми гордощі супротивника.

\section{Безсрібникам}
\emph{Тропар}

Святі безсрібники і чудотворці, згляньтеся на немочі наші: задарма ви одержали, задарма і подавайте нам.

\emph{Кондак}

Благодать зцілення прийнявши, ви, лікарі, преславні чудотворці, подаєте здоров’я тим, хто потребує; вашим відвіданням знищіть гординю противників, зціляючи світ чудесами.

\section{Христа ради юродивому}
\emph{Тропар}

Глас апостола Твого Павла почувши, який промовляв: «Ми юродиві Христа ради», раб Твій, Христе Боже, (ім’я) постав на землі Тебе ради. Тим то, пам’ять його шануючи, молимо Тебе, Господи, спаси душі наші.

\emph{Кондак}

Небесної краси прагнучи, земних насолод тілесних щиро ти зрікся. Про суєтний світ не дбаючи, ангельське життя провівши, спочив ти (ім’я), блаженний. З ними ж моли Христа Бога безперестанно за всіх нас.

\section{Священномученикові}
\emph{Тропар}

І співучасником звичаїв, і спадкоємцем влади апостольської бувши, ти, богонатхненний, обрав діяльність на путі дослідження. Тому, правдиво навчаючи слова істини, ти й до крові трудився у вірі, священномученику (ім’я); моли Христа Бога, щоб спастися душам нашим.
\end{document}