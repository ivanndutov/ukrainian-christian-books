\documentclass[chapters.tex]{subfiles}

\begin{document}
\chapter{Молитовне правило за Україну в небезпеці}
\section{У час біди та при нападі ворогів}
\emph{Тропар, глас 4}

Скоро поспіши до нас, доки не поневолив ворог, який ганьбить Тебе та погрожує нам, Христе Боже наш: погуби Хрестом Твоїм тих, що воюють проти нас, нехай зрозуміють, що може віра православних, молитвами Богородиці, єдиний Чоловіколюбче.

\emph{Кондак, глас 8}

Непереможний Воєводо і Господи, пекла Переможцю! Визволившись од вічної смерті, похвальні пісні приносимо Тобі, створіння й раби Твої. А Ти, що маєш милосердя невимовне, від усяких бід визволи нас, що взиваємо Тобі: Ісусе, Сину Божий, помилуй нас.

\emph{Стихи із псалмів:}

Господи, воздвигни силу Твою, і прийди, щоб спасти нас.

Хай воскресне Бог і розвіються вороги Його, і біжать від лиця Його всі, хто ненавидить Його. Як щезає дим, нехай щезнуть.

\section{Коли Вітчизна в небезпеці}
Господи Боже наш, Ти вислухав Мойсея, коли він простягав до Тебе руки, і народ ізраїльський зміцнив на амаликитян, озброїв Ісуса Навина на битву та повелів сонцю спинитися. Ти й нині, Владико, почуй нас, що молимося до Тебе. Зміцни силою Твоєю побожний народ наш, благослови його справи, примнож славу його перемогою над ворогом, зміцни всемогутньою Твоєю правицею нашу державу, збережи військо, пошли ангела Твого на зміцнення захисників народу нашого, подай нам усе, що просимо для спасіння; примири ворожнечу і мир утверди. Простягни, Господи, невидиму правицю Твою, яка рабів Твоїх заступає в усьому. Тим же, кому судив Ти покласти душу свою на війні за віру православну, побожний народ наш і державу, прости їхні провини і в день праведної Твоєї відплати подай вінці нетління. Бо Твоя є влада, Царство і сила, від Тебе допомогу всі приймаємо, на Тебе надію покладаємо і Тобі славу возсилаємо, Отцю, і Сину, і Святому Духові, нині, і повсякчас, і на віки віків. Амінь.

\section{За ворогів України}
Господи, наверни до Себе серця ворогів наших. Якщо ж неможливо запеклим навернутися, то поклади межу їхньому злу і захисти від них обраних Твоїх. Амінь.

\section{До святого Архистратига Михаїла}
Святий Архистратиже Божий Михаїле і всі небесні сили безтілесні, моліть Бога за нас!

\section{До святих}
Всі святі землі Української, моліть Бога за нас!

Всі святі, моліть Бога за нас!

\section{Молитва за спасіння держави Української і втихомирення в ній розбрату і чвар між людьми}
Господи Боже, Ісусе Христе, Спасителю наш! До Тебе припадаємо зі скорботним серцем і сповідаємося у гріхах і беззаконнях наших, що ними зранили Твоє милосердя і зачинили від себе щедроти Твої. Бо відступили від Тебе, Владико, і законів Твоїх не дотримуємося, і не робимо того, що заповідав Ти нам. Тому й уразив Ти нас безладдям і віддав на потоптання й зневаження ворогам нашим, і принижені ми більше інших народів, і зневагу й наругу терпимо від сусідів наших.

Боже Великий і Дивний, що вболіваєш за злобу людську, підводиш повалених і підіймаєш тих, що впали! Небесну Твою силу з неба народові нашому пошли, зціли рани душ наших і підійми нас із постелі немочі, бо вражені розслабленням стегна наші, бо хворі на неправду і породжуємо беззаконня. Вгамуй ворожнечу і розбрат у землі українській, віддали від нас заздрощі, чвари, зарозумілість, злочинність, пияцтво й розпалення пристрастей, попали в серцях наших терня нечистоти, непримиренности й озлоблення, щоб полюбили ми один одного і, як одне ціле, перебували в Тобі, Господі і Владиці нашому, як і вчив і заповідав Ти нам. Помилуй нас, Господи, помилуй нас, бо сповнилися ми приниження і негідні очей своїх на небо звести. Згадай милості, що їх явив Ти отцям нашим, переміни гнів Твій на милість і дай нам поміч у скорботі. Молить бо Тебе Церква Твоя молитвами святого рівноапостольного князя Володимира і блаженної великої княгині Ольги, святих благовірних князів-страстотерпців Оскольда, Бориса, Гліба та Ігоря, митрополитів Київських: Михаїла, Іларіона, Макарія, Петра (Могили); святих митрополитів Феодосія Чернігівського, Димитрія та Арсенія Ростовських, Інокентія та Софронія Іркутських, Іоасафа Білгородського, Іоана та Павла Тобольських, які за Україну страждали; святого великомученика Юрія Побідоносця, великомучениці Варвари, святі мощі якої у благословенному Богом Києві перебувають, преподобномученика Афанасія Берестейського, Миколи Луцького, Макарія Канівського, Федора Острозького, преподобних отців наших Антонія і Феодосія та всіх чудотворців Києво-Печерських, преподобного Іова Почаївського, преподобних Іова та Феодосія Манявських, святої Юліанії Ольшанської, всіх святих, на землі Українській прославлених, а особливо Пресвятої Богородиці і Вседіви Марії, Яка з древніх часів покривала і захищала Своїм святим омофором країну нашу.

Врозуми народ український, а особливо тих, хто при владі стоїть, про нашу помісну Православну Церкву, Україну і весь народ добре дбати та державність відстоювати. Силою Хреста Твого зміцни державу і воїнство наше та захисти від підступів ворожих. Піднеси людей розуму й сили і постав на служіння Вітчизні нашій, а нам усім дай духа мудрости і страху Божого, духа міцности та благочестя і даруй кожному вогонь любові до матері нашої України — єдиної і неподільної.

Господи, до Тебе вдаємося, навчи нас чинити волю Твою, бо Ти Бог наш, бо в Тобі джерело життя і в світлі Твоєму побачимо світло. Подай милості й щедроти Твої народові українському, щоб ми славили святе ім’я Твоє з Отцем і Святим Духом на віки віків. Амінь.

\section{Псалми під час лиха й нападу ворогів}
\subsection{Псалом 9}

Славитиму Тебе, Господи, всім серцем моїм, розповідатиму про всі чудеса Твої. Буду веселитися і хвалитися Тобою, буду співати імені Твоєму, Всевишній. Ось повернули назад вороги мої, знесиліли й погинули перед лицем Твоїм. Бо це Ти вирішив суд мій і справу мою; це Ти сів на престолі, Суддя найсправедливіший. Ти погромив народи, знищив нечестивого, затер ім’я їх на віки вічні. У ворога не стало зброї, міста поруйнував Ти і загинула пам’ять їх із ними. Господь же перебуває повік. Він приготовив для суду престол Свій. Він буде судити вселенну по правді, розсудить народи по справедливості. І буде Господь пристановищем пригнобленому й захистом у час смутку. І будуть надіятися на Тебе всі, хто знає ім’я Твоє, бо Ти не покинеш тих, хто шукає Тебе, Господи. Співайте Господеві, що живе на Сіоні, сповіщайте між народами про діла Його. Бо Він карає за пролиту кров, пам’ятає її й не забуде плачу пригноблених. Помилуй мене, Господи, споглянь на приниження моє від ворогів моїх, Ти, що підняв мене від воріт смерти, щоб я звіщав усі хвали Твої у воротах дочки Сионової. Буду радуватися спасінням Твоїм. Попадали народи в яму, яку викопали, в сітку, яку тайно поставили, зав’язла нога їх. Пізнали Господа по суду Його, що Він звершив: нечестивець заплутався в ділах рук своїх. Нехай зійдуть безбожники в пекло і всі народи, що забувають Бога. Бо не назавжди буде забутий убогий, надія покривджених не загине. Встань, Господи, щоб не підносився чоловік, нехай перед лицем Твоїм приймуть суд усі народи. Постав над ними страх Твій, щоб зрозуміли народи, що вони тільки люди. Чому Ти, Господи, стоїш далеко і не являєш Себе в час скорботи? В гордині своїй безбожний пригноблює вбогого; нехай самі вони будуть уловлені хитрощами, які самі вимишляють. Бо безбожний хвалиться похотями душі своєї, і гнобитель ублажає себе. В гордині своїй нечестивий не визнає Бога, каже він: «не покарає», бо думає він, що Бога нема. Кожного часу шляхи його осквернені. Суди Твої далекі для нього. На ворогів своїх дивиться з погордою. Каже він в серці своїм: «Не похитнуся; з роду в рід не зазнаю лиха». Уста його повні прокльонів, підступів та лукавства; під язиком його мука та загибіль. Він іде на засідку край осель, щоб таємно вбити неповинного, очі його підглядають за безборонними. Підстерігає по закутках, як лев по гущавині, щоб схопити бідного; хапає і тягне в сітки свої. Припадає до землі, пригинається, і бідні попадають у міцні лапи його. Каже він у серці своїм: «Забув Бог, відвернув лице Своє і не побачить ніколи». Воскресни, Господи, Боже мій, нехай піднесеться рука Твоя, і не забудь убогих Твоїх до кінця. Чому безбожник зневажає Бога і каже в серці своїм: «Бог не бачить»? Ти ж бачиш, бо Ти споглядаєш на кривди та пригноблення, щоб віддати рукою Твоєю. До Тебе вдається бідний і сироті Ти будь захистом. Зломи силу нечестивого й лукавого так, щоб шукати і не знайти злодіянь його. Господь — Цар на віки вічні, а поганці згинуть з землі Його. Господи, Ти чуєш бажання вбогих, підкріпи серце їх, відкрий ухо Твоє, щоб дати праведний суд сироті й пригнобленому, щоб не лякав їх чоловік на землі.

\subsection{Псалом 78}

Боже, прийшли чужинці в насліддя Твоє, осквернили святий храм Твій, Єрусалим обернули на руїну. Кинули трупи рабів Твоїх на поживу птахам небесним, тіла святих Твоїх — звірам земним. Розлили кров їх, як воду, навкруги Єрусалиму, і не було кому ховати. Ми стали посміховиськом для сусідів наших, наругою й соромом для тих, що оточують нас. Доки, Господи, будеш безупинно гніватися на нас, доки буде палати, як вогонь, обурення Твоє? Вилий гнів Твій на народи, які не знають Тебе, і на царства, які ймення Твого не призивають. Бо вони пожерли народ Твій і місце його спустошили. Не згадуй давніх гріхів батьків наших, але скоро пошли нам щедроти Твої, Господи, бо зубожіли дуже ми. Поможи нам, Боже, Спасителю наш, задля слави імени Твого. Господи, визволи нас і очисти гріхи наші задля імени Твого. Щоб не сказали безбожні: «Де Бог їх?» Нехай на очах наших зазнають невірні помсти за пролиту кров вірних Твоїх. Нехай прийде перед лице Твоє стогін ув’язнених; могутністю Твоєю збережи засуджених на смерть. Всемеро поверни в серце ворогам нашим наругу їх, що нею Тебе, Господи, вони зневажали. А ми, народ Твій і вівці пасовиська Твого, будемо вічно прославляти Тебе та з роду в рід сповіщати хвалу Твою.

\subsection{Псалом 84}

Господи, Ти змилувався над землею Твоєю, повернув з неволі синів Якова. Ти простив провину народу Твого, покрив усі гріхи їх. Ти спинив увесь гнів Твій і відвернув від гніву ярости Твоєї. Поверни нас, Боже, Спасителю наш, і спини гнів Твій проти нас. Чи ж довіку будеш гніватися на нас? Невже ж продовжиш гнів Твій з роду в рід? Боже, повернися і оживи нас, щоб зрадувався Тобою народ Твій. Яви нам, Господи, милість Твою, і Твоє спасіння дай нам! Почую, що скаже про мене Господь Бог. Він скаже мир народові Своєму і вибраним Своїм, тим, що звертають серця свої до Нього. Так близько спасіння Його до тих, що бояться Його, щоб слава Його перебувала на землі нашій. Милість і справедливість зустрілися, правда і мир привіталися. Істина від землі засяяла, а правда з неба прихилилася. Бо Господь пошле нам добро, і земля наша дасть плід свій. Правда піде перед Ним і направить на путь стопи наші.

\subsection{Псалом 90}

Хто живе надією на Всевишнього, той під покровою Бога Небесного спочиває. Каже він до Господа: «Ти моє пристановище і захист мій, Бог мій, і я надіюся на Нього». Він спасе тебе від сітки ловця і від пошести згубної. Плечима Своїми Він захистить тебе, і під тінню крил Його ти надійно спочиватимеш. Обороною тобі буде правда Його. Не побоїшся страху вночі, ані стріли, що летить удень. Ані пошести, що діє в темряві, ані напасти духа зла опівдні. Впаде коло тебе тисяча, і десять тисяч правобіч тебе, але до тебе не наблизиться. Тільки очима твоїми будеш дивитися і помсту над беззаконними бачити. Бо ти сказав: «Господь — надія моя», і Всевишнього ти обрав за оборонця собі. Отже, не прийде до тебе лихо, і пошесть не наблизиться до оселі твоєї. Бо Він ангелам Своїм звелить, щоб охороняли тебе на всіх стежках твоїх. На руках вони понесуть тебе, щоб нога твоя не спіткнулася об камінь. На гаспида й василиска ти наступатимеш і потопчеш лева й змія. Бо скаже Господь: «За те, що він надію поклав на Мене, Я визволю його і захищу його, бо він знає ім’я Моє. Буде кликати Мене, Я почую його; буду з ним у скрутний час, визволю його і прославлю його. Довгим життям обдарую його і дам йому спасіння Моє».

Псалми, щоб Господь врозумив правителів робити те, що потрібно народу
\subsection{Псалом 137}

Прославляю Тебе, Господи, від усього серця мого. Перед ангелами співатиму похвалу Тобі за те, що Ти почув усі слова уст моїх. Буду поклонятися перед святим храмом Твоїм і славити ім’я Твоє за милість Твою і правду Твою, бо Ти звеличив слово Твоє над усе. В який тільки день я взиватиму до Тебе, Ти скоро почуй мене, і Твоєю силою збільши силу душі моєї. Будуть прославляти Тебе, Господи, всі царі землі, коли почують слова уст Твоїх. Будуть в піснях прославляти путі Господні, бо велика слава Господня, бо високий Господь: бачить упокореного, і гордого пізнає здалека. Коли прийдуть на мене напасті, Ти відживиш мене проти лютости ворогів моїх; простягнеш руку Твою, і спасе мене правиця Твоя. Господь віддасть за мене; Господи, милість Твоя повік, не відкинь творіння рук Твоїх.

\subsection{Псалом 51}

Чого хвалишся злодійством, сильний, зловживаєш милосердям Божим кожного дня? Неправду висловлює язик твій; як бритва вигострена, така твоя лжа. Ти полюбив зло більше, ніж добро, більше неправду, ніж говорити правду. Ти полюбив слова погибелі, розмову улесливу. За те знищить тебе Бог до кінця. Викине тебе й видворить тебе з оселі твоєї, викорінить тебе із землі живих. Побачать це праведні і злякаються, посміються над тобою і скажуть: «Ось чоловік, що не в Бога шукав собі сили, а надіявся на силу багатства свого та підсилював себе злочинством своїм». А я — мов оливне зелене дерево в домі Божім, я надіюся на милість Божу повік. Буду вічно славити Тебе, Боже, за все, що Ти вчинив; буду надіятися на ім’я Твоє, бо Ти ласкавий до праведних Твоїх.

\section{Псалом за мир у країні}
\subsection{Псалом 28}

Віддавайте Господеві, сини Божі, віддавайте Господеві славу і честь. Віддавайте Господеві славу імени Його, поклоняйтеся Господеві в дивній святині Його. Голос Господній над водами. Бог слави загримів. Загримів Господь над водами великими. Голос Господа могутній, голос Господа величний. Голос Господа ламає кедри, Господь трощить кедри ливанські. Від голосу Його гори скачуть, як телята, і Сіріон-гори, як молоді буйволи. Голос Господа викрешує полум’я вогню; від голосу Господа трясеться пустиня. Потрясає Господь пустиню Кадеську. Від голосу Господнього падають лані і оголяються ліси. В храмі (всесвіту) Його все говорить про Його величність. Господь возсів на престолі над водами великими і, як цар, буде над ними повіки. Господь дасть силу народові Своєму, Господь миром благословить людей Своїх.

\section{Молитва прп. Єфрема Сиріна за повернення мирних часів}
Куди втечу від Тебе, Господи наш? У якому краї сховаюся від лиця Твого? Небо — престол Твій, земля — підніжжя Твоє, у морі шляхи Твої, у пеклі влада Твоя. Якщо близький уже кінець світу, то нехай він буде не без Твоєї милости.

Знаєш Ти, Господи, що неправди наші великі, та ми знаємо, що безмежна милість Твоя. Коли не вмилосердить Тебе милість Твоя — пропали ми за беззаконня наші. Не покинь нас, Господи, Господи, бо ми споживали Тіло і Кров Твою.

Коли діла кожного постануть на суд перед Тобою, Господи всіх, у цей останній час не відверни лиця Твого від тих, хто сповідав ім’я Твоє. Отче, Сину і Душе Святий, Утішителю, спаси нас і збережи душі наші.

Благаємо благість Твою, Господи, відпусти нам провини наші, відкинь беззаконня наші, відкрий нам двері милосердя Твого, Господи, щоб прийшли до нас часи мирні. З любові Твоєї прийми молитву нашу, бо тим, хто кається, Ти відчиняєш двері. Амінь.
\end{document}