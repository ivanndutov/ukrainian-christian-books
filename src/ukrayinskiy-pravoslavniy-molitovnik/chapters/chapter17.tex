\documentclass[chapters.tex]{subfiles}

\begin{document}
\chapter{Молитви про Царство Небесне}
\section{Молитва про щасливу смерть}
Пам’ятаю я, Господи, що коли послав Ти нас на світ, як слуг Своїх, то прийде час — і Ти покличеш нас знову до Себе. Але від очей наших закритою є Воля Твоя, для того, щоб ми не знали часу нашої смерті і тому кожної хвилини готувалися до Твого поклику. Але, щоб ця велика хвилина не застала мене непідготовленим, Ти, Господи, дай мені силу провадити дні життя мого гідно, пам’ятаючи Закони Твої і думаючи про Тебе, Творця й Отця нашого. А коли прийде хвилина, коли буде потрібно покидати цей світ, Ти, Господи, покрий мене Своєю Святою Опікою, дай мені зі свідомістю та спокоєм тіла і душі відійти до Твого Престолу і почути слова Твої: «Слуго мій благий, увійди в радість Господа Твого». Амінь.

\section{Молитва при помираючих}
Всемогутній і Милосердний Боже, Ти подав людському родові засоби для спасіння, а вірним Твоїм слугам пообіцяв, в нагороду, вічне життя. Поглянь ласкаво на Твого слугу (ім’я), який страждає у важкій недузі. А якщо його життя наближується до свого кінця, до смерті, і Ти відкликуєш його з землі у вічність, то збережи його душу, Тобою створену. Нехай у ту хвилину, коли його душа буде розлучатися з тілом, прийде по неї Твій Ангел, і нехай вона стане перед Тобою, Спасителем і Суддею, очищеною від всякої скверни гріха та нехай сподобиться почути радісні слова: «Вірний і добрий слуго, увійди в радість Небесного Царства». Амінь.

\section{Молитва під час смерті}
Премилосердний Боже! Поглянь ласкаво на слугу Свого і вислухай ті сердечні молитви, які до Тебе возносимо. Не пам’ятай гріхів його молодості і несвідомості, лише згадай про Своє безмежне Милосердя. Він щиро визнав ті гріхи і шкодує, що зробив їх, та просить у Тебе Милосердя.

Звільни його душу від усіх мук пекельних, від кайданів гріховних, та від вічного терпіння. Відкрий йому Небо й візьми слугу Свого в хороми вічного щастя. Амінь.

\section{Молитва за батьків, у яких померла дитина}
Господи Боже наш! Ти є єдиним Джерелом потіхи для засмученого серця батьків, у яких померла дитина. До Тебе прибігаємо і до Тебе молимося: потіш і скріпи їх, Боже, щоб вони мужньо, в повній відданості Твоїй Пресвятій Волі, несли тяжкого хреста, якого Ти сподобився на них покласти. Ти, Боже, дав їм цю дитину, Ти її потім і забрав, нехай буде благословенним ім’я Твоє. Ми знаємо, що всі Твої зарядження, Боже наш, навіть тоді, коли вони тяжко ранять людське серце, мають на меті лише добро. Тому можемо лише догадуватися, що Ти, найкращий Отче, покликав цю дитину до себе, щоб вона не впала жертвою зіпсуття цього світу, не втратила вічного щастя, для якого Ти, Боже наш, її створив. Просимо тебе, Господи, наділи цих батьків Твоєю Святою Ласкою, щоб вони служили Тобі до кінця їхнього життя і заслужили для себе такого часу, щоб разом зі своєю дитиною прославляти і величати Тебе в Твоєму Царстві на віки вічні. Амінь.

\section{Молитва вдівця за померлу дружину}
Христе Ісусе, Господи і Вседержителю! Зі щирістю серця мого молюся Тобі: упокой, Господи, душу померлої раби Твоєї (ім’я) в Небесному Царстві Твоєму. Владико Вседержителю! Ти благословив подружню єдність чоловіка і жінки, коли сказав: недобре бути чоловікові самому, створимо йому помічника, подібного до нього. Ти освятив цей союз на образ духовного союзу Христа з Церквою. Вірую, Господи, і визнаю, що Ти благословив з’єднати і мене святим союзом цим з однією зі слуг Твоїх. З Твоєї ж милостивої та премудрої волі взято в мене цю рабу Твою, яку Ти дав мені за помічницю й супутницю життя мого. Схиляюся перед цією волею Твоєю і молюся Тобі від усього серця мого, прийми це благання моє за рабу Твою (ім’я) і прости їй, коли згрішила вона словом, ділом, думкою, свідомо чи несвідомо; коли земне полюбила більше від небесного, коли про одяг та прикрашання тіла свого дбала більше, ніж про просвітлення одягу душі своєї, або коли не дбала про дітей своїх; коли завдала кому жалю словом або вчинком; коли нарікала в серці своїм на ближнього свого або осудила кого чи інше щось подібне вчинила. Все це прости їй, як Милостивий та Чоловіколюбний, бо нема людини, що жила б і не згрішила. Не судися з рабою Твоєю, як з творінням Твоїм, не осуди її за гріхи її на вічні муки, а зглянься й помилуй з великої милости Твоєї. Благаю і прошу Тебе, Господи, дай мені силу в усі дні життя мого не переставати молитися за спочилу рабу Твою і аж до кінця життя мого просити для неї в Тебе, Судді всього світу, прощення провин її. Ти, Боже, поклав на голову її вінець з каменя дорогого, вінчаючи її тут, на землі, так увінчай її вічною Твоєю славою в Небесному Царстві Твоєму, з усіма святими, що там торжествують, нехай разом з ними і вона вічно оспівує всесвяте ім’я Твоє з Отцем і Святим Духом. Амінь.

\section{Молитва вдови за померлого чоловіка}
Христе Ісусе, Господи і Вседержителю! Ти — втіха тих, що плачуть, сиріт і вдів захист. Ти сказав: «Поклич Мене в день журби твоєї, і врятую тебе». В дні журби своєї вдаюся до Тебе і молюся Тобі: не відверни лиця Твого від мене і вислухай благання моє, яке приношу Тобі зі сльозами. Ти, Господи і Владико всього, благозволив одружити мене з одним з рабів Твоїх, щоб нам бути одним тілом і одним духом. Ти дав мені його за чоловіка й захисника. Твоїй милостивій і премудрій волі вгодно було взяти від мене цього раба Твого і залишити мене одну. Схиляюся перед цією Твоєю волею і до Тебе вдаюся в дні журби моєї: заспокой смуток мій з приводу розлуки з рабом Твоїм, другом моїм. Коли Ти взяв його від мене, не візьми від мене Своєї милости. Як колись прийняв Ти дві лепти вдовиці, так прийми і це благання моє. Пом’яни, Господи, душу спочилого раба Твого (ім’я), прости йому всі провини його, вільні й невільні, чи то думкою, чи словом, чи ділом, свідомі чи несвідомі, але з великої милости Твоєї та безлічі щедрот Твоїх полегши й прости всі гріхи його і осели його з святими Твоїми, де немає ні хвороби, ні печалі, ні зітхання, а життя безконечне. Благаю і прошу Тебе, Господи, дай мені всі дні життя мого не переставати молитися за померлого раба Твого (ім’я) і аж до кінця мого просити в Тебе, Судді всього світу, відпущення всіх гріхів його і вселення його в небесні оселі, що їх Ти наготував тим, що люблять Тебе. Коли ж він і згрішив, але не відступив від Тебе і безсумнівно Отця, і Сина, і Святого Духа православно аж до останнього свого подиху визнавав, тому віру в Тебе замість діл прийми, бо нема чоловіка, що жив би і не згрішив. Ти один без гріха, і правда Твоя — правда вічна. Вірую, Господи, що Ти почуєш благання моє і не відвернеш лиця Твого від мене. Побачивши вдовицю, що тяжко плакала, Ти змилосердився і сина її, коли вже несли ховати, воскресив: так, змилосердившись, заспокой і мою журбу. Як Ти відчинив двері милосердя Твого для раба Твого Феофіла, що відійшов до Тебе, і простив йому всі провини його за молитви святої Церкви Твоєї, зважаючи на молитви й милостині дружини його, так і я благаю Тебе, прийми і моє благання за раба Твого (ім’я) і введи його в життя вічне. Ти бо єси уповання наше. Ти Бог милости і спасіння, і Тобі славу возсилаємо з Отцем, і Святим Духом нині, й повсякчас, і на віки віків. Амінь.

\section{Молитва за померлих}
Боже духів і всякої плоті, Ти смерть подолав і диявола знищив і життя світу Твоєму дарував! Сам, Господи, упокой душі спочилих рабів Твоїх: святійших патріархів, первосвященних митрополитів, архієпископів і єпископів, що у священничому, церковному і чернечому чині Тобі послужили; будівничих святого храму цього, православних праотців, отців, братів і сестер наших, що тут лежать і всюди; вождів і воїнів, що за віру і вітчизну життя своє поклали, вірних убієнних в міжусобицях, утоплих, згорілих, на морозі замерзлих, роздертих звірами, від голоду померлих, без покаяння несподівано померлих і тих, що не встигли примиритися з Церквою і зі своїми ворогами; самогубців, позбавлених розуму, всіх, що заповідали нам молитися, і вірних, що християнського поховання були позбавлені, у місті світлому, у місті квітучому, у місті спокою, де немає ні хвороб, ні печалі, ні зітхання. Всякий гріх, вчинений ними словом чи ділом, чи думкою, як благий Чоловіколюбець Бог прости, бо немає людини, яка б жила і не згрішила. Ти бо єдиний без гріха, правда Твоя — правда повіки, і слово Твоє — істина.

Бо Ти є воскресіння, життя і спокій спочилих рабів твоїх (імена), Христе Боже наш, і Тобі славу возсилаємо з безначальним Твоїм Отцем і Пресвятим, і Благим, і Животворчим Твоїм Духом нині, і повсякчас, і навіки-віків. Амінь.

\section{Молитва батьків за померлу дитину}
Господи, Ісусе Христе, Боже наш, Владико життя і смерті, Утішителю скорботних! Із засмученим і зворушеним серцем вдаюся до Тебе і молюся Тобі: пом’яни, Господи в Царстві Твоєму померлу дитину мою (ім’я) і сотвори їй вічну пам’ять. Ти, Владико життя і смерті, дав мені цю дитину. Твоїй же милосердній і премудрій волі вгодно було і взяти її від мене. Нехай буде благословенне Ім’я Твоє, Господи! Благаю Тебе, Суддю неба і землі, з безмежної любові Твоєї до нас, грішних, прости, Милостивий і наші батьківські гріхи, щоб не залишалися вони на дітях наших, бо знаю, що багато грішили ми перед Тобою, багато чого не дотримали, не вчинили, як Ти заповідав нам. Коли ж померла дитина наша з нашої чи своєї провини, в житті своєму служила більше світові і тілу своєму, а не Тобі, Господу й Богові своєму, коли полюбила вона принади світу цього більше, ніж слово Твоє і заповіти Твої, коли вона, віддавшись насолоді життєвій, а не жалю за гріхи свої, в нестриманості занедбала поміркованість, піст і молитву, — щиро благаю Тебе, прости, премилостивий Отче, дитині моїй усі такі провини її, прости й полегши, коли й інше зло вчинила вона в житті цьому. Христе Ісусе! Ти воскресив дочку Яірову за віру й молитву батька її, Ти зцілив дочку жінки-хананеянки за віру і благання матері її. Вислухай же і мою молитву, не відкинь і мого благання за дитину мою. Прости, Господи, всі гріхи її та, простивши і очистивши душу її, визволи з муки вічної та всели її з усіма святими Твоїми, що від віку Тобі догодили там, де нема хвороби, ні печалі, ні зітхання, а життя безкінечне, бо нема людини, щоб жила і не згрішила — Ти один без гріха. Коли будеш судити світ, нехай почує дитина моя найсолодший голос Твій: «Прийдіть, благословенні Отця Мого наслідуйте Царство, приготоване вам від створення світу». Бо Ти Отець милості і щедрот. Ти — життя й воскресіння і Тобі славу возсилаємо з Отцем і Святим Духом нині і повсякчас, і на віки віків. Амінь.

\section{Молитва дітей за померлих батьків}
Господи Ісусе Христе, Боже наш! Ти — сиріт Охоронитель, скорботних Пристановище і тих, що плачуть, Утішитель. Припадаю до Тебе я, сирота, із плачем і зітханням, і молюся Тобі: вислухай благання моє і не відверни лиця Твого від стогону серця мого та від сліз очей моїх. Молюся Тобі, Милосердний Господи, заспокой скорботу мою через розлуку з батьком моїм (ім’я) (матір’ю моєю; батьками моїми — імена), що породив і виховав мене, а душу його, що відійшла до Тебе з правдивою вірою в Тебе і твердою надією на Твою чоловіколюбність і милість, прийми в Царство Твоє Небесне. Схиляюся перед Твоєю святою волею, яка взяла його від мене, і прошу Тебе: тільки не відійми від нього Твоєї милости і добросердечности. Знаю, Господи, що Ти, Суддя світу цього, караєш гріхи та нечестя батьків на дітях, онуках і правнуках аж до третього й четвертого роду; але Ти й милуєш батьків за молитви та чесноти дітей їхніх, онуків і правнуків. Зі скорботним і зворушеним серцем благаю Тебе, милостивий Судде, не карай вічною карою спочилого, незабутнього для мене раба Твого, батька мого (ім’я), але відпусти йому всі провини його вільні й невільні, які він учинив думкою, словом чи ділом, свідомо чи несвідомо, за життя свого тут, на землі, і з милосердя та чоловіколюбности Твоєї, молитвами Пречистої Богородиці і всіх святих помилуй його і від вічної муки визволи.

Милосердний Отче отців і дітей! Дай мені всі дні життя мого, до останнього мого зітхання не переставати пам’ятати про спочилого батька мого в молитвах моїх і благати Тебе, праведного Суддю, щоб Ти оселив його на місці світлому, в місці втіхи, у місці спокою, з усіма святими, де немає ні недуги, ні смутку, ані зітхання.

Милостивий Господи! Прийми нині щиру молитву цю за раба Твого (ім’я) і віддай йому нагородою Твоєю за працю й піклування у вірі та християнській побожності, бо він навчив мене знати Тебе, свого Господа, благоговійно молитися Тобі, лише на Тебе уповати в біді, скорботах і немочах, навчав виконувати заповіді Твої. За добре піклування його про моє духовне зростання, за теплі молитви, що він приносив за мене Тобі, і за всі дари, які він виблагав мені від Тебе, віддай йому Своєю милістю, Своїми небесними благами та радостями у вічному Царстві Твоєму. Бо Ти — Бог милости, щедрот і чоловіколюбства. Ти — і радість вірних слуг Твоїх, і Тобі славу возсилаємо з Отцем і Святим Духом, нині, і повсякчас, і на віки віків. Амінь.

\section{Молитва сироти за покійних батьків}
Єдиний мій Боже, до Тебе одного я припадаю в моїх потребах, бо хто ж вірніше допомагатиме мені, як не Ти, Господи, на якого мене батьки залишили. До Тебе єдиного я взиваю в недолі й смутку, бо хто ж втішить мене, як не Ти, Боже, Отче мій Небесний. Ти моя єдина Надія. Ти ніколи мене не залишав. Вислухай тепер мою молитву, та прийми моїх покійних батька й матір до Твого Небесного Царства, щоб через їхні молитви міг я згідно Святої Волі Твоєї жити, а після земного життя, разом з ними Славу Твою оглядати. Амінь.

\section{Молитва за всякого померлого}
Пом’яни, Господи, Боже наш, у вірі й надії на життя вічне спочилого раба Твого, брата нашого (ім’я) як Милосердний і Чоловіколюбний, відпускаючи гріхи та згладжуючи неправди, полегши, даруй і прости всі його провини вільні й невільні, визволи його від вічної муки та вогню геєнського і дай йому вічні Твої блага, наготовані для тих, що люблять Тебе. Коли ж і згрішив він, та не відступив від Тебе і безсумнівно в Отця, і Сина, і Святого Духа, Бога, Тебе, в Тройці славимого, вірував і Одиницю в Тройці і Тройцю в Одиниці православно аж до останнього свого подиху визнавав. Будь милостивим до нього і віру в Тебе замість діл прийми і з святими Твоїми, як Щедрий, упокой: нема бо чоловіка, що жив би і не згрішив. Ти один тільки без усякого гріха, і правда Твоя — правда вічна, і Ти один Бог милости і щедрот, і чоловіколюбства, і Тобі славу возсилаємо, Отцю, і Сину, і Святому Духові, нині, й повсякчас, і на віки віків. Амінь.

\section{Молитва за всіх померлих}
Упокій, Господи, душі спочилих слуг Твоїх: батьків моїх (імена), родичів моїх (імена), благодійників моїх (імена), і всіх тих, хто положив життя своє за волю України (імена), і всіх християн, і всяку душу, яка спочила і Твоєї Милості потребує, і прости їм всі провини їхні, вільні й невільні, і подаруй їм Царство Небесне, і створи їм вічну пам’ять. Амінь.
\end{document}