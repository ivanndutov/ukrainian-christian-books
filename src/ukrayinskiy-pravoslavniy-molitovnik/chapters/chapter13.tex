\documentclass[chapters.tex]{subfiles}

\begin{document}
\chapter{Родинні молитви}
\section{Молитва за родину}
Ласкавий Господи, Ісусе Христе, який захотів провести довгі роки Свого земного життя в назаретській Родині, благаю Тебе, поблагослови мою дорогу родину: батька, матір, братів, сестер, всіх свояків і кровних. Пошли їм Ласку, щоб всім серцем любили Тебе, охорони від нещасть, диявольських спокус і всякого лиха. Подай згоду моїй родині, бо згода це великий Твій дар. Допоможи її членам щиро любитися надприродною Божою любов’ю, яка осолоджує прикрощі земного життя. Будь з нами в години смерті, щоб по Твоїй Ласці вийшли ми з цього світу та пішли до Неба навіки прославляти Тебе. Маріє, найкраща наша Мати і взірець для кожної родини, будь завжди з моєю дорогою родиною. Амінь.

\section{Молитва за українську родину}
Боже великий, Боже отців наших! Дай нашому народові якнайбільше добрих, святих християнських родин.

Дай нам таких батьків, які голосно й відверто признавалися б до Божого Твого Євангелія і до Твоєї служби. Дай нам батьків, які для своїх дітей були б прикладом християнського життя, правдивими опікунами та добрими провідниками у житті.

Дай нам таких матерів, які вміли б добре, по-християнськи виховувати своїх дітей, а для своїх чоловіків були б допомогою, потіхою та доброю порадою.

Дай нам таких дітей, які були б потіхою для своїх батьків і славою та красою свого народу.

Благослови, Всемогутній Боже, український народ. Подаруй йому ласку, вірно Тобі служити і доступити колись вічної нагороди в Небі, бо Тобі, Боже, у Святій Трійці Єдиний, Отче, Сину і Духу Святий, належить вся слава, честь і поклоніння на віки вічні. Амінь.

\section{Перша молитва на благословення сім’ї}
Від Тебе, Господи, одержуємо ми всі дари благі. Благослови нашу сім’ю, нехай будуть усі члени нашої сім’ї до вподоби Тобі, нехай царює в ній мир і християнська любов. Амінь.

\section{Друга молитва на благословення сім’ї}
Боже, Творцю людей та їхній милосердний Цілителю, Ти благоволив, щоб сім’я, створена подружнім союзом, стала образом союзу Христа й Церкви. Молимо Тебе: подаруй Своє благословення цій сім’ї, яка зібралася в ім’я Твоє. Нехай у ній перебувають любов, єдність, палання духу, ревність у молитві, турбота одне про одного та про всіх нужденних братів. Господь Ісус Христос, який жив у Назареті разом зі своєю Родиною, нехай завжди перебуває в цій сім’ї та береже її від усякого зла та допомагає нам бути єдиним серцем і єдиною душею на віки вічні. Амінь.

\section{Молитва на благословення наречених}
Прославляємо Тебе, Боже, Джерело любові. Завдяки Твоєму Провидінню ці двоє молодих людей (імена) зустрілися в житті і Ти спонукаєш цих Своїх дітей до взаємної любові. Вчини ласкаво, щоб ті, хто готуються до спільного життя і просять Твоєї благодаті, зміцнені Благословенням з Неба, завжди поважали і щиро любили одне одного. Просимо Тебе, зміцни їхні серця, щоб, перебуваючи у вірі і роблячи те, що подобається Тобі, вони щасливо дійшли до прийняття Тайни Вінчання, через Христа, Господа нашого. Амінь.

\section{Молитва дівчини про заміжжя}
О, Всемилостивий Господи! Я знаю, що велике моє щастя залежить від того, як любитиму я Тебе всією душею і всім серцем моїм і виконуватиму у всьому Святу волю Твою. Керуй же Сам, о Боже мій, душею моєю і наповнюй любов’ю серце моє. Я хочу догоджати Тобі Єдиному, бо Ти Творець і Бог мій. Охорони мене від гордості й самолюбства. Розум, скромність і доброчесність нехай прикрашають мене. Лінощі противні Тобі і породжують пороки, подай же мені бажання працювати і благослови труди мої.

Оскільки ж закон Твій Святий наказує людям жити в чесному подружжі, то приведи мене, Отче Святий, до цього освяченого Тобою стану, не для догоджання бажанню моєму, а для виконання призначення Твого, бо Ти сам сказав: не добре чоловікові бути одному і, створивши йому жінку на помічницю, благословив їх рости, множитися і населяти землю.

Вислухай смиренну молитву мою, з глибини дівочого серця Тобі принесену, дай мені чоловіка чесного й побожного, щоб ми з ним в любові і згоді прославляли Тебе, Милосердного Бога, Отця, і Сина, і Святого Духа сьогодні, і повсякчас, і на віки вічні. Амінь.

\section{Молитва християнського подружжя}
Господи Боже наш! Піклуючись про спасіння людей, Ти завітав до Кани Галілейської і цим показав чистоту шлюбу. Сам і нині рабів Твоїх цих (імена), яким Ти дозволив з’єднатися одне з одним, збережи в злагоді та однодумності, чесним шлюб їх яви, ложе їхнє неоскверненим збережи, допоможи їм провадити життя в непорочності і сподоби їх дожити до глибокої старості та з чистим серцем виконувати заповіді Твої. Бо Ти єси Бог наш, Бог Милості і Спасіння, і Тобі ми славу возсилаємо з Предвічним Твоїм Отцем, і Всесвятим, і Благим, і Животворчим Твоїм Духом нині, і повсякчас, і на віки віків. Амінь.

\section{Молитва чоловіка}
Небесний Отче, Ти сказав у раю, що не добре чоловікові самому жити на світі. Ти створив і дав йому помічницю — першу жінку, праматір людського роду — Єву. З Твоєї волі я вибрав дружину, щоб служити Тобі на землі. Дай мені ласку бути добрим чоловіком для товаришки мого життя, яку Ти дав мені під час шлюбу. Допоможи мені щиро та до смерті любити її, не дозволь, щоб коли-небудь я добровільно вчинив прикрість або скривдив її. Допоможи мені панувати над собою та моїми пристрастями, щоб наше подружжя було святим та милим Тобі. Дай мені мудрість розуміння її недосконалостей, дай силу працювати для нашого спільного добра, щоб моя товаришка мала все те, що добрий чоловік повинен дати своїй дружині, щоб ми одним серцем і однією душею служили Тобі на віки віків. Амінь.

\section{Молитва дружини}
Ласкавий Творче першої жінки, Єви, яку Ти Сам дав Адамові товаришкою життя та помічницею в його мандрівці до вічності. Допоможи мені бути гідною подругою цьому чоловікові, з яким Ти Сам з’єднав мене в Тайні Подружжя. Хочу бути доброю дружиною та завжди чинити приємність своєму чоловікові в тих речах, які подобаються Тобі. Дай мені мудрість розуміти його потреби, вади і недосконалості, спокійно нести тягар подружнього життя. Бережи мене від духу світу, який затроює подружнє життя, дай мені ласку бути доброю дружиною, матір’ю та господинею, щоб через мене чоловікові було легше прославляти Тебе. Маріє, Ти була взірцем для подруг, дай мені ласку наслідувати Тебе в моєму подружжі та щиро й вірно любити свого чоловіка аж до смерті. Амінь.

\section{Молитовне зітхання жінки-християнки, коли вона сподівається дитини}
О, Преславна Мати Божа! Змилуйся наді мною, бідною рабою Твоєю, і прийди мені на допомогу в час моїх недуг і небезпек, з якими народжують дітей всі бідні Євині дочки. Згадай, о благословенна в жонах, з якою радістю й любов’ю Ти поспішала в нагірню сторону відвідати родичку Твою Єлисавету, коли вона сподівалася дитини, і як чудесно позначилося благодатне відвідання Твоє на матері й дитині. З невимовного благосердя Твого дай і мені, негідній служниці Твоїй, щасливо дочекатися дитини. Даруй мені цю ласку, щоб дитина, яка зараз перебуває під моїм серцем, прийшовши до пам’яті, з радістю, подібно до святого немовляти Іоана, поклонялася Божественному Господеві Спасителю, що з любови до нас грішних смирився і Сам був Дитиною. Невимовна радість, якою переповнювалося непорочне серце Твоє при погляді на новонародженого Твого Сина і Господа, нехай усолодить скорботу, що чекає мене серед недуг народження. Життя світу, мій Спаситель, народжений Тобою, нехай врятує мене від смерти в час розродження і нехай прилучить плід утроби моєї до числа обранців Божих. Вислухай, Преславна Царице Небесна, смиренне благання моє і зглянься на мене, бідну грішницю, оком милости Твоєї; не посором моєї надії на Твоє велике милосердя і осіни мене, помічнице християн, зцілителько хворих, Твоїм материнським покровом у час моїх мук і хвороби, щоб сподобилась і я відчути на собі, що Ти — Мати милосердя, і щоб я завжди прославляла Твою милість, яка ніколи не відкидає молитви бідних і визволяє всіх, хто звертається до Тебе в час скорботи й хвороби. Амінь.

\section{Молитва благословення вагітних жінок}
Господи Боже наш, Сотворителю всілякого життя! Ти з допомогою Святого Духа підготував преславну Діву Марію на гідне помешкання для Свого Сина. Ти наповнив Святим Духом Івана Хрестителя і наказав йому радіти в лоні матері його. Поглянь на цих жінок, які просять Твого Благословення для себе і своїх дітей. Наповни їх глибокою радістю, бо на їхніх очах звершується чудо початку нового життя. Прийми гаряче прагнення тих, хто покірно Тебе благає зберегти їхнє потомство, яке Ти покликав до існування. Нехай з Твоєю допомогою їхні діти щасливо побачать світло дня й сподобляться Благодаті Святого Хрещення. Нехай вони завжди вірно служать Тобі й досягнуть вічного життя на Небі. Амінь.

\section{Молитва вагітних жінок за успішні пологи}
Всемогутній Боже, Творче всього видимого і невидимого! До Тебе, улюбленого Отця прибігаємо ми, обдаровані розумом пізнання і творчою волею, бо Ти за власною волею створив наш рід, з невимовною мудрістю створивши наше тіло із землі і вдихнувши в нього душу зі Свого Духа, щоб ми були Твоєю подобою. І хоча Ти міг за одним лише Твоїм бажанням створити нас, як і ангелів, відразу, але Твоїй Премудрості було завгодно, щоб людський рід збільшився через чоловіка і жінку в установленому Тобою порядкові шлюбу. Ти захотів благословити людей, щоб вони росли і розмножувалися, і наповнювали не лише землю, але й ангельські сонми.

О Боже і Отче, нехай буде навіки славлене і прославлене ім’я Твоє за все, що Ти для нас зробив!

Дякую Тобі також за Твоє милосердя, що з Твоєї волі внаслідок Твого чудесного творіння не лише я виникла і поповнюю число вибраних, але й що Ти благословив мене у шлюбі і послав мені плід черева. Це Твій дар, Твоя, Боже, створена милість, о Господи і Отче душі й тіла! Тому звертаюся до Тебе одного і зі смиренним серцем молю Тебе про милість та допомогу, щоб те, що Ти твориш в мені Своєю силою, було збережене і приведене до щасливого народження. Бо я знаю, о Боже, що людина не має влади і сили на те, щоб вибирати свій шлях. Ми надто слабкі і схильні до падіння, щоб оминути всі ті тенета, які, за Твоїм попущенням, приготував нам злий дух, і уникнути тих нещасть, в які вкидає нас наша легковажність. Твоя ж мудрість безмежна. Кого забажаєш, того Ти неушкоджено через Свого Ангела збережеш від усякої напасті. Тому і я, Милосердний Отче, передаю себе в моїй журбі до Твоїх рук і молю, щоб Ти подивився на мене оком милосердя і зберіг від усякого страждання. Пошли мені і моєму чоловікові відраду, о Боже, Владико всякої радості! Щоб ми, бачачи Твоє благословення, від усього серця в дусі радості Тобі служили і поклонялися. Я не хочу бути вилученою з того, що Ти наклав на весь наш рід, звелівши у хворобах народжувати дітей. Але смиренно Тебе прошу, щоб Ти допоміг мені перенести страждання і послав успішне завершення.

І якщо Ти почуєш цю нашу молитву і пошлеш нам дитину, то обіцяємо, що знову приведемо її до Тебе і Тобі посвятимо, щоб Ти залишився для нас та нашого сімені Милосердним Богом і Отцем, як і ми обіцяємо завжди разом з нашою дитиною бути Твоїми вірними слугами. Почуй, Милосердний Боже, молитву немічної Твоєї раби, виконай прохання нашого серця, заради Ісуса Христа, нашого Спасителя, Який ради нас втілився, нині ж перебуває з Тобою і Святим Духом і керує у вічності. Амінь.

\section{Молитва допомоги жінкам під час пологів до Богородиці}
Пресвята Діво, Мати Господа нашого Ісуса Христа, Ти знаєш про важкий стан матерів під час народження їхніх дітей, помилуй слугу Твою (ім’я) і допоможи їй народити дитину без труднощів. О, Всемилостива Владичице Богородице, Ти не потребувала допомоги при народженні Сина Божого, але подай допомогу слузі Твоїй, яка потребує її від Тебе.

Подаруй їй благодатну силу в цей час, а немовляті, яке повинне прийти на цей світ, дозволь вчасно отримати світло розуму у Святому Хрещенні. До Тебе припадаємо, Матері Бога Всевишнього, і молимося: будь милостивою до цієї матері, коли прийде їй час народжувати, і випроси у народженого від Тебе, Христа Бога нашого, щоб укріпив її Силою Своєю з Неба. Амінь.

\section{Молитва благословення жінки після пологів}
Боже, від Тебе походять всі Благословення, тому вчини так, щоб ця мати, укріплена Твоїм Благословенням, завжди належно дякувала Тобі й разом зі своїм дитям завжди перебувала під Твоїм захистом. Ти, Господи, подарував їй радість материнства, то благослови її, щоб вона невпинно дякуючи за своє потомство, разом з ними досягнула вічного блаженства, через Ісуса Христа, Господа нашого. Амінь.

\section{Молитва за жінку, яка народила дитину}
Господи Боже наш, Ти, що заради спасіння нас, грішних, зволив зійти з небес і народитися від святої Богородиці і Приснодіви Марії, що знаєш неміч людської природи, через безліч Твоїх щедрот прости Твоїй рабі (ім’я), що нині народила. Бо Ти Господи казав: плодіться і розмножуйтесь, і наповнюйте землю, і володійте нею. Тому й ми, раби Твої, молимося і підбадьорені Твоїм незлобливим чоловіколюбством зі страхом взиваємо до Царства Твого святого імені: зглянься з небес і побач неміч нас, гідних осудження, і прости цій Твоїй рабі (ім’я), і всьому дому, в якому народилося немовля, і всім, що до неї доторкалися і тут перебувають, як благий і чоловіколюбний Бог прости. Бо Ти єдиний маєш владу відпускати гріхи, молитвами Пресвятої Богородиці і всіх святих Твоїх. Амінь.

\section{Молитва подружжя, що не має дітей}
Вислухай нас, Милосердний і Всемогутній Боже, щоб за благання наше послана була благодать Твоя. Будь милостивим, Господи, до молитви нашої, згадай закон Твій про примноження роду людського і будь Милостивим Покровителем нашим, щоб за Твоєю допомогою зберігалося Тобою ж установлене. Могутньою силою Твоєю Ти з нічого все створив і поклав початки всього, що існує в світі, створив і чоловіка за образом Своїм і високою тайною освятив союз подружжя як образ тайни єднання Христа з Церквою. Зглянься, Милосердний, на рабів Твоїх цих, союзом подружнім сполучених, коли вони благають допомоги Твоєї, нехай буде на них милість Твоя, нехай будуть плодовитими і побачать сини синів своїх аж до третього й четвертого роду, і до бажаної старості доживуть, і ввійдуть у Царство Небесне через Господа нашого Ісуса Христа, Якому належить всяка честь, слава і поклоніння із Святим Духом на віки віків. Амінь.

\section{Молитва за матерів}
Боже, Ти вибрав Непорочну Діву Марію Матір’ю Свого Сина, благослови всіх матерів, обдаруй їх здоров’ям, наповни їх спокоєм і радістю, а після довгого життя земного приведи до життя вічного. Просимо Тебе: вчини так, щоб людський рід пишався святістю своїх матерів. Допоможи їм виховати своїх дітей на славу Тобі і для добра Церкви, через Ісуса Христа, Господа нашого. Амінь.

\section{Молитва за неохрещену дитину}
Боже, Отче Всемогутній, Джерело всілякого благословення! Ти увінчав подружній союз даром і радістю народження потомства. Поглянь милостиво на цю дитину, і сподобися відродження її Водою і Святим Духом, щоб вона була співпричетною до Твого народу. Сподоби її прийняти дар Хрещення, щоб вона стала співучасницею Твого Царства і разом з нами в Церкві славила Тебе, через Ісуса Христа, Господа нашого. Амінь.

\section{Молитва за охрещену дитину}
Господи Ісусе Христе, Ти так полюбив дітей, що промовив: «Хто прийме одне з таких дитят в ім’я Моє, — Мене приймає». Почуй наші молитви за цю дитину, яку Ти вдосконалив Благодаттю Хрещення, щоб, коли вона виросте, могла сміливо визнавати віру в Тебе, щоб її серце палало любов’ю і щоб вона з надією покладалася на Твоє Милосердя, бо Ти живеш і Царюєш на віки вічні. Амінь.

\section{Коротка молитва за дітей}
Владико Господи Вседержителю, будь милостивим до дітей моїх, приведи їх до віри і спасіння, збережи їх під захистом Твоїм, захисти їх від усілякої похітливості лукавої, віджени від них усякого ворога і супротивника, відкрий їхні вуха й очі сердечні, подаруй доброту і співчутливість серцям їхнім. Амінь.

\section{Перша молитва батьків за дітей}
Отче Небесний! Твоїй невимовній благості та милості ми доручаємо дітей, якими Ти зволив нас благословити. Твій Син, Господь наш Ісус Христос, викупив їх Своєю найдорожчою Кров’ю, а Святий Дух освятив їх у Святому Таїнстві Хрещення. Ти, Боже, зажадаєш відповіді від нас на Страшному Суді за душі дітей наших. Ми відчуваємо весь тягар відповідальності за їхні душі. Тому молимо Тебе, Предвічний Отче, подай нам світло розуму, силу і терпеливість, щоб змогли ми дітей виховати згідно зі Святою Твоєю волею. Наділи нас, Отче Небесний, Своєю Милістю, щоб ми зуміли зберегти невинність дітей наших, навчати їх власним добрим прикладом, подбати про їхнє добро душевне та тілесне. І даруй нам ласку, Боже, щоб ми Тебе славили разом з нашими дітьми у Царстві Небесному. Амінь.

\section{Друга молитва батьків за дітей}
Отче Небесний! Ти благословив мене дітьми і поклав на мене обов’язок трудитися над їхнім вихованням. Знаю, Господи, що колись Ти будеш вимагати від мене рахунку з цього важливого обов’язку. Я свідомий (свідома) того, тому вдаюся до Тебе і благаю в покорі: просвіти мій розум, щоб я своїх дітей якнайкраще провадив (провадила) згідно з Твоїми Святими Заповітами. Викоріни з мого серця надмірне прив’язування до них, яке є виявом самолюбства, щоб я не був (була) сліпим (сліпою) на їхні помилки. Охорони мене, Господи, щоб я не був (була) для них згіршенням. Дай мені так виховати моїх дітей, щоб вони все більше Тебе пізнавали, любили і ціле своє життя прожили згідно з Твоєю Святою Волею.

Пречиста Діво Маріє, будь моїм дітям Матір’ю і Заступницею. Святий Обручнику Йосифе, заопікуйся моїми дітьми. Амінь.

\section{Молитва на благословення дітей, яким виповнився один рік}
Господи Ісусе Христе, Сину Бога Живого, предвічно народжений! Ти в тілі забажав бути дитиною і любиш невинність земного життя. Ти з любов’ю пригортав і благословляв дітей, яких до Тебе приносили і приводили. Дітей цих Своїм благословенням оточи і вчини так, щоб злість не заволоділа їхнім розумом та дозволь, щоб, зростаючи роками, мудрістю й ласкою, вони завжди Тобі подобалися. Нехай цих дітей благословить Всемогутній Бог Отець, і Син, і Святий Дух. Амінь.

\section{Молитва матері за дітей до Пречистої Діви Марії}
Пречиста Діва Маріє, Мати Божа, яка непорочно народила, і на благо всього світу Свого Єдинородного Сина виховала. Вислухай, моя Свята Заступнице й Покровителько, мою тиху та щиру молитву і дай, мені сили й терпеливість виплекати й виховати моїх дітей чесними та богобоязними людьми, щоби вони були втіхою і підтримкою мені на старість, і всьому народові українському окрасою, і щоб чистим серцем славили ім’я Твоє, Богородице. Амінь.

\section{Молитва матері про терпеливість у вихованні дітей}
О, Боже! Ти знаєш, як легко мені уривається терпець, коли мене мучать і гноблять труди й страждання у вихованні моїх дітей. Тому дай мені силу, щоб я охоче приймала і зносила тягарі мого покликання. Навчи мене приймати Хреста, якого Твоїй Святій Волі подобалося на мене покласти. Щоб я приймала його щодня і несла за моїм Спасителем, який незрівнянно більше страждав заради спасіння душ наших. Зроби так, щоб я робила для моїх дітей все те, що потрібне й корисне для добра їхнього виховання. Святий Духу, Святий Кріпкий, дай мені витримати в добрі. Амінь.

\section{Молитва за дітей і внуків до Ангела-Охоронителя}
Святий Ангеле-Охоронителю моїх дітей (імена) та внуків (імена), покрий їх Твоїм покровом від стріл демона та від очей спокусника і збережи їхнє серце в ангельській чистоті. Амінь.

\section{Молитва родичів за нащадків}
Господи і Боже наш! Ти поблагословив нащадками наше подружнє життя і поставив нас учителями і провідниками дітей. Ти віддав нам у руки виноградник Свій, який ми маємо доглядати. Тому молимо Тебе, Господи, дай нам розум ясний і волю непохитну, щоб ми могли розумно і вміло виховати дітей, яких Ти нам дав, людьми гідними і праведними. Благослови, Отче Небесний, дітей наших, влий у їхні серця добру волю, добрі бажання, добрі наміри і любов до всього, що праведне і святе. Покрий нас і їх Твоїм Святим Благословенням, а ми славити будемо Тебе на протязі усіх днів життя нашого. Амінь.

\section{Молитва літніх і самотніх людей}
Господи Ісусе Христе, Ти — наша сила і надія в різних життєвих нещастях, будь нам завжди підтримкою і опікою. Навчи нас з терпеливістю і без нарікань переносити недуги літнього віку. Дай нам потрібної сили, щоб ми могли належно користуватися Твоїми дарами земними і вічними, бо Ти живеш і царюєш на віки вічні. Амінь.

\section{Молитва в день іменин до святого}
З цілого серця дякую Тобі, Господи, що Ти вчинив мене гідним прийняти Святу Тайну Хрещення і стати Членом Твоєї Церкви.

Допоможи мені, Господи, щоб я завжди був вірним і чесним сином Твоїм і мужнім борцем за справи Христові й на славу Його Святого Імені поміж людьми. Сьогодні, в день мого Небесного Заступника, перед Тобою, Боже, я відновлюю мій обіт Хрещення і, обіцяю вірити в Христа і вірно Йому служити. Вірю у все те, що Ти, Господи, нам об’явив, і постановляю жити так, як це належить жити вірному синові Твоєї Церкви. Святий (ім’я святого), мій Небесний Опікуне, заступися за мене перед Престолом Всевишнього і випроси Ласку, щоб я міг тебе наслідувати, а після смерті разом з тобою оглядати Славу Божу в Небі на віки вічні. Амінь.

\section{Молитва дітей за батьків}
Всемилостивий Боже! Дякую Тобі за моїх батьків та за всі ті добра, яких Ти мені подав через них. Вони виховали мене, навчили любити Твої Святі Заповіти та остерігатися гріхів. Я люблю їх і хочу завжди шанувати і слухати.

Господи, допоможи мені Твоєю Ласкою, щоб я цю постанову виконав (виконала). Благослови їх, Господи, всяким добром, здоров’ям та довгим і щасливим життям, оберігай від злого, дай їм силу перемагати всі труднощі, та щоб на старості літ вони дочекалися потіхи від мене, а по смерті прийми їх і мене до Небесного Царства. Амінь.

\section{Молитва дитини}
Ласкавий Отче, Творче Неба і землі, щиро дякую Тобі за те, що Ти дав мені батька і матір, щоб вони опікувалися мною та провадили мене дорогою Твоїх Святих Заповітів. Дай мені ласку бачити в них Твоїх представників на землі, любити їх і бути вдячним (вдячною) за дар життя, якого вони дали мені. Не дозволь, щоб я коли-небудь учинив (учинила) їм добровільну прикрість. Небесний Отче, ради заслуг Ісуса Христа та Пречистої Діви Марії, благаю Тебе, заопікуйся моїми дорогими батьками і нагороди їх Вічним Щастям у Небі за все те, що вони вчинили або чинять для мене.

Маріє, Покровителько всіх родин, благослови моїх батьків і випроси їм Ласку, щоб вони після смерті навіки оглядали Бога посеред Ангелів і Всіх Святих. Амінь.

\section{Молитва дитини за батька}
Ісусе Христе, Милостивий Боже наш, Ти прихильно приймаєш кожну щиру молитву. Тож молю й благаю Тебе, Ісусе Христе, зішли моєму батькові міцне здоров’я і дай йому силу старатися й добре дбати про мене. Благослови його чесну працю, нехай сповняться його надії, які він на мене покладає. І дай йому діждатися від мене потіхи й радості. Амінь.

\section{Молитва дитини за матір}
Господи Ісусе Христе, Ти пригортав малих дітей до Свого серця, я знаю, що Ти вислухаєш і мою щиру молитву. Я не благаю Тебе, Христе Благий, ні за що більше, тільки за свою рідну матір.

Зішли їй, Преблагий Господи, здоров’я й силу, щоб вона могла працювати для мене й опікуватися мною. Дай мені, Ісусе Христе, щоб я виріс (виросла) моїй любій матері на потіху. Це буде моїм найбільшим щастям. А я буду славити Тебе на віки вічні. Амінь.

\section{Молитва за батьків, рідних і друзів}
Спаси, Господи, і помилуй батьків моїх (імена), братів, сестер і рідних моїх по крові (імена), і всіх близьких мені людей з мого роду, і друзів (імена), і дай їм Твоїх земних і Небесних Благ. Амінь.

\section{Молитва за родичів та близьких нам людей}
Благослови, Господи, родичів моїх, братів і сестер, і всю родину мою. Обдаруй їх здоров’ям і добробутом і просвіти їхні серця, щоб вони могли ступати дорогою Заповітів Твоїх. Бережи під Твоєю Святою Охороною провідників, добродіїв і всіх тих, хто щиро працює для піднесення Слави Твоєї і для добра народу Твого. Спаси, Господи, і помилуй слуг Твоїх: отця мого духовного (ім’я), батьків моїх (імена) і всю родину мою (імена), вчителів (імена), добродійників (імена), прихильників (імена), і ненависників моїх (імена), і всіх християн, і всяку душу християнську, що Ласки і допомоги Твоєї потребує. Амінь.

\section{Молитва перша жінки, яка згубила немовля у своїй утробі}
О, Владико, Господи Ісусе Христе, Сину Божий! Заради великої благодаті твоєї, нас ради людей і нашого ради спасіння Ти у плоть одягнувся, і розіп’явся, і був похований, і Твоєю кров’ю оновив зітліле єство наше. Прийми моє покаяння у гріхах і вислухай слова мої. Згрішила, Господи, на небо і перед Тобою словом, ділом, душею і тілом, і думками розуму мого, заповіді Твої порушила, не послухала Твого повеління, проживала благодать Твою, Боже мій, але, як Твоє творіння, не зневіряюся у спасінні, а до безмежного Твого благосердя насмілююся прийти і молюся Тобі: Господи, в покаянні дай мені сокрушене серце і прийми мене, благаючу, і дай мені думку благу, дай мені сльози зворушення, Господи, дай мені з благодаті Твоєї покласти початок благий. Помилуй мене, помилуй мене, упалу, і пом’яни мене, грішну рабу Твою, у Царстві Твоїм нині, і повсякчас, і навіки-віків. Амінь.

\section{Молитва друга жінки, яка згубила немовля у своїй утробі}
Боже, премилосердний Христе Ісусе, Спасителю грішних, ради спасіння роду християнського Ти залишив, Всемилостивий, преславні небеса і оселився в юдолі цій плачевній і багатогрішній; Ти прийняв на Твої Божественні рамена наші немочі і поніс наші хвороби; Ти, о Страждальнику Святий, поранений був за гріхи наші, а тому і ми до Тебе, Чоловіколюбче, підносимо наші смиренні благання: прийми їх, о Преблагий Господи, і зійди до немочей наших і не пом’яни гріхів наших, і гнівний намір помститися за гріхи наші відверни від нас.

Кров’ю Твоєю всечесною Ти оновив упале єство наше, онови, Господи Ісусе Христе, Спасителю наш, і нас, сущих у тлінні гріхів, і втіш серця наші радістю Твого всепрощення. З воланням і гарячими сльозами розкаяння припадаємо до ніг Твого Божественного милосердя: очисти нас усіх, Боже наш, Твоєю Божественною благодаттю від усіх неправд, беззаконь нашого життя, щоб у святині Твого Чоловіколюбства ми прославляли всесвяте ім’я Твоє, з Отцем і Преблагим і Животворчим Духом, нині, і повсякчас, і навіки-віків. Амінь.

\section{Молитви за мертвонароджених та нехрещених немовлят}
Пом’яни, Чоловіколюбче, Господи, душі рабів Твоїх, що відійшли — немовлят, які в утробі їхніх матерів померли неждано від несвідомих дій або від трудного народження, або від якоїсь необережності і тому не прийняли святої тайни Хрещення. Охрести їх, Господи, в морі щедрот Твоїх і спаси невимовною Твоєю благодаттю. Амінь.
\end{document}