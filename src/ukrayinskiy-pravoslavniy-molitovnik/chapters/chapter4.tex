\documentclass[chapters.tex]{subfiles}

\begin{document}
\chapter{Тропарі й величання святим загальні}
\section{Пресвятій Богородиці}
\emph{Тропарі, глас 4}

До Богородиці щиро нині звернімося ми, грішні й смиренні, і припадім, у покаянні з глибини душі взиваючи: Владичице, поможи, змилосердившись над нами, поспіши, бо гинемо від безлічі провин; не відпусти рабів Твоїх ні з чим, бо Тебе за єдину надію нашу маємо.

\emph{Слава:} І нині, глас той же.

Не замовкнемо ніколи, Богородице, про силу Твою говорити ми, недостойні, бо коли б Ти не предстояла молячись, хто нас врятував би від стількох бід? Хто ж охоронив би нас донині вільними? Не відступимо від Тебе, Владичице: Твоїх бо рабів спасаєш завжди від усякої біди.

\emph{Величання}

Величаємо Тебе, Пресвятая Діво, Богом обрана Отроковице, і шануємо образ Твій святий, ним же подаєш зцілення всім, хто з вірою звертається до Тебе.

Достойно є величати Тебе, Богородице, чеснішу від херувимів і незрівнянно славнішу від серафимів.

\section{Безплотним Силам}
\emph{Тропар, глас 4}

Небесних воїнств архистратиги, молимо вас завжди ми, недостойні, щоб вашими молитвами оберігали нас покровом крил духовної вашої слави, охороняючи нас, що старанно припадаємо до вас і взиваємо: «Від бід визволіть нас, як чиноначальники Небесних Сил».

\section{Величання}

Величаємо вас, архангели і ангели і всі воїнства, херувими і серафими, що прославляють Господа.

\section{Пророкові}
\emph{Тропар, глас 2}

Пророка Твого \emph{(ім’я)} пам’ять, Господи, святкуючи, ним Тебе молимо, спаси душі наші.

\section{Величання}

Величаємо тебе, пророче Божий \emph{(ім’я)}, і шануємо святу пам’ять твою, бо ти молиш за нас Христа Бога нашого.

\section{Апостолу}
\emph{Тропар, глас З}

Апостоле святий \emph{(ім’я)}, моли милостивого Бога, щоб відпущення гріхів подав душам нашим.

\section{Величання}

Величаємо тебе, апостоле Христів \emph{(ім’я)}, і шануємо болісті і труди твої, ними ж ти трудився єси у благовісті Христовім.

\section{Святителеві}
\emph{Тропар, глас 4}

Правилом віри й образом лагідності, стриманости вчителем явив тебе стаду твоєму Той, Хто є Істиною всіх речей. Ради цього придбав ти смиренням — високе, убогістю — багатство. Отче святителю \emph{(ім’я)}, моли Христа Бога, щоб спастися душам нашим.

\section{Величання}

Величаємо тебе, святителю отче \emph{(ім’я)}, і шануємо святу пам’ять твою, бо ти молиш за нас Христа Бога нашого.

\section{Преподобному і преподобномученику}
\emph{Тропар, глас 8}

У тобі, отче, вповні спаслася душа, створена за образом Божим; взявши бо хрест, пішов ти за Христом і ділом навчав не про тіло дбати, бо воно тимчасове, а про душу — єство безсмертне; тому, преподобний \emph{(ім’я)}, разом з ангелами й радіє дух твій.

\section{Величання преподобному}

Ублажаємо тебе, преподобний отче \emph{(ім’я)}, і шануємо святу пам’ять твою, наставнику ченців і співбесіднику ангелів.

\section{Величання преподобномученику}

Ублажаємо тебе, преподобномученику \emph{(ім’я)}, і шануємо чесні страждання твої, що за Христа витерпів єси.

\section{Преподобномученикам}
\emph{Тропар, глас 4}

Боже отців наших, Ти завжди до нас Милосердний; не віддаляй милости Твоєї від нас, але молитвами їх у спокої направ життя наше.

\section{Величання}

Ублажаємо вас, преподобномученики (імена), і шануємо чесні страждання ваші, що за Христа витерпіли єси.

\section{Преподобній}
\emph{Тропар, глас 8}

У тобі, мати, вповні спаслася душа, створена за образом Божим; взявши бо хрест, пішла ти за Христом і ділом навчала не про тіло дбати, бо воно тимчасове, а про душу — єство безсмертне; тому, преподобна \emph{(ім’я)}, разом з ангелами і радіє дух твій.

\section{Величання}

Ублажаємо тебе, преподобна мати \emph{(ім’я)}, шануємо святу пам’ять твою, бо ти молиш за нас Христа Бога нашого.

\section{Преподобним жонам}
\emph{Тропар, глас 2}

З бажанням правдивим заручившись, Христославнії, і з’єднання з женихом дочасним зрікшись, і в подвигах доброчесних доспівши, зійшли ви на висоту нетління, прекраснодушні і пребагаті, стовпи черниць і взірець; тому за нас, що пам’ять вашу з любов’ю святкуємо, молітесь безперестанно.

\section{Величання}

Ублажаємо вас, преподобні матері (імена), і шануємо святу пам’ять вашу, бо ви молите за нас Христа Бога нашого.

\section{Мученикові}
\emph{Тропар, глас 4}

Мученик Твій, Господи, \emph{(ім’я)}, в стражданні своїм вінець нетлінний прийняв від Тебе, Бога нашого; маючи бо силу Твою, мучителів подолав, сокрушив і демонів немічні спокуси. Його молитвами спаси душі наші.

\section{Величання}

Величаємо тебе, страстотерпче святий \emph{(ім’я)}, і шануємо чесні страждання твої, їх же за Христа перетерпів єси.

\section{Священномученикові}
\emph{Тропар, глас 4}

І співучасником звичаїв, і спадкоємцем влади апостольської бувши, ти, богонатхненний, обрав собі діяльність на путі дослідження. Тому, правдиво навчаючи слова істини, ти й до крови трудився у вірі, священномученику \emph{(ім’я)}; моли Христа Бога, щоб спастися душам нашим.

\section{Величання}

Величаємо тебе, святий священномученику \emph{(ім’я)}, і шануємо святу пам’ять твою, бо ти молиш за нас Христа Бога нашого.

\section{Священномученикам}
\emph{Тропар, глас 4}

Боже отців наших, Ти завжди до нас Милосердний; не віддаляй милости Твоєї від нас, але молитвами їх у спокої направ життя наше.

\section{Величання}

Величаємо вас, святі священномученики (імена), і шануємо святу пам’ять вашу, бо ви молите за нас Христа Бога нашого.

\section{Мучениці}
\emph{Тропар, глас 4}

Агниця Твоя, Ісусе, \emph{(ім’я)}, взиває великим гласом: «Тебе, Жениху мій, люблю і, Тебе шукаючи, страждаю, і співрозпинаюся, і співпогрібаюся в хрещенні Твоїм, і страждаю Тебе ради, щоб царювати з Тобою, і вмираю за Тебе, щоб жити з Тобою; прийми ж мене, як жертву непорочну, що з любов’ю принесла себе в жертву Тобі». Її молитвами, яко Милостивий, спаси душі наші.

\section{Величання}

Величаємо тебе, страстотерпице Христова \emph{(ім’я)}, і шануємо чесні страждання твої, їх же за Христа перетерпіла єси.

\section{Преподобномучениці}
\emph{Тропар, глас 4}

Агниця Твоя, Ісусе, \emph{(ім’я)} взиває великим гласом: «Тебе, Жениху мій, люблю і, Тебе шукаючи, страждаю, і співрозпинаюся, і співпогрібаюся в хрещенні Твоїм, і страждаю Тебе ради, щоб царювати з Тобою, і вмираю за Тебе, щоб жити з Тобою; прийми ж мене, як жертву непорочну, що з любов’ю принесла себе в жертву Тобі». Її молитвами, яко Милостивий, спаси душі наші.

\section{Величання}

Величаємо тебе, страстотерпице свята \emph{(ім’я)}, і шануємо чесні страждання твої, їх же за Христа перетерпіла єси.

\section{Сповідникові}
\emph{Тропар, глас 8}

Православ’я наставнику, побожности і чистоти вчителю, вселенної світильнику, богонатхненна архиєреїв оздобо, \emph{(ім’я)} премудрий, наукою твоєю все просвітив єси, сопілко духовна; моли Христа Бога, щоб спаслися душі наші.

\section{Величання}

Величаємо тебе, святителю отче \emph{(ім’я)}, і шануємо святу пам’ять твою, бо ти молиш за нас Христа Бога нашого.

\section{Безсрібникам}
\emph{Тропар, глас 8}

Святі безсрібники й чудотворці, згляньтесь на немочі наші: величання задарма ви одержали, задарма і подавайте нам.

\section{Величання}

Величаємо вас, чудотворці славні \emph{(імена)}, і шануємо чесні страждання ваші, їх же за Христа перетерпіли ви.

\section{Христа ради юродивому}
\emph{Тропар, глас 1}

Глас апостола Твого Павла почувши, який промовляв: «Ми юродиві Христа ради», раб Твій, Христе Боже, \emph{(ім’я)}, юродивим став на землі Тебе ради. Тому, пам’ять його шануючи, Тебе молимо, Господи, спаси душі наші.

\section{Величання}

Ублажаємо тебе, святий праведний \emph{(ім’я)}, і шануємо святу пам’ять твою, бо ти молиш за нас Христа Бога нашого.

\section{Соборам святих отців}
\emph{Тропар, глас 8}

Препрославлений єси, Христе Боже наш, що отців наших, наче світила, на землі поставив і через них усіх до віри правдивої привів, Многомилостивий, слава Тобі.

\section{Величання}

Величаємо вас, святителі отці \emph{(імена)}, і шануємо святу пам’ять вашу, бо ви молите за нас Христа Бога нашого.

\section{Усім святим}
\emph{Тропарі, глас 3}

У всьому світі мучеників Твоїх кров’ю, як багряницею і виссоном, Церква Твоя прикрасилася і через них взиває до Тебе, Христе Боже: «Людям Твоїм щедроти Твої зішли, мир громаді Твоїй даруй і душам нашим велику милість».

\section{Величання}

Величаємо вас, апостоли, мученики, пророки і всі святі, і шануємо святу пам’ять вашу, бо ви молите за нас Христа Бога нашого.

\section{Молитви перед сповіддю}
Всі тайни свого серця я розкриваю перед Тобою, Суддею моїм. Отож, зглянься на покору мою, зглянься на смуток мій, зглянься на каяття моє і помилуй мене, Милосердний, Боже отців наших.

Приступивши до священика, перехрестись, поцілуй святий хрест, поклади два пальці правої руки на Євангеліє і сповідайся перед Спасителем твоїм Ісусом Христом.

Я, грішна людина, сповідаюсь Господу Богу Всемогутньому, в Тройці Святій Єдиному, Пречистій Діві Марії, святому Ангелові-Охоронителеві моєму, всім святим і тобі, отче духовний, в усіх гріхах своїх, якими я прогнівив Господа Бога мого від попередньої моєї сповіді: ділом, словом, думкою, всіма моїми почуттями, вільно і невільно, свідомо і несвідомо.

Прошу, отче духовний, твоїх молитов перед Христом Спасителем моїм за всі гріхи мої.

Тут можеш сказати отцеві духовному все те, що непокоїть тебе, затуманює розум твій і думки чисті, християнські, і те, що тягне тебе до гріхів і до діл лукавих.

Коли отець духовний відпускає гріхи твої, тоді поклади праву руку твою до серця свого і промов:

Боже, будь милостивий до мене, грішного, і прости мені всі гріхи мої, як простив Ти, з Свого чоловіколюбства: митареві, фарисеєві, блудниці і Савлові, що гнав Тебе.

\section{Молитва після сповіді}
Слава, честь і поклоніння Тобі, безконечно Милосердний Боже, що зглянувся ласкаво на мої сльози каяття і не погордував моїм розкаяним серцем. Ти очистив мою душу і прийняв мене негідного за Свою дитину.

Але Ти, Боже, знаєш мою нестійкість, знаєш також силу та завзятість ворогів душі моєї, тому благаю Тебе: подай мені силу витримати спокуси світу цього, подай моєму розумові світло, щоб я вже ніколи не зійшов з дороги святих заповітів Твоїх, щоб я лише Тебе одного любив і Тобі єдиному служив у всі дні життя мого. Амінь.
\end{document}