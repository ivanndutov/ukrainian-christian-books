\documentclass[chapters.tex]{subfiles}

\begin{document}
\chapter{Молитви до Чесного Хреста}
\section{Молитва до Чесного Хреста перша}
\emph{(з акафіста Хресту Господньому)}

Хресте Чесний, охоронителем душі і тіла будь мені: образом своїм бісів подолай, ворогів віджени, пристрасті приборкай і благоговіння даруй мені, і життя, і силу, і сприяння Святого Духа чесними Пречистої Богородиці молитвами. Амінь.

\section{Молитва до Чесного Хреста друга}
О, пречесний і животворчий Хресте Господній! Колись ти був знаряддям ганебної страти, нині ж ти знамення спасіння нашого, що завжди шанобливо прославляється! Як достойно зможу я, недостойний, оспівати тебе і як насмілюся схилити коліна серця мого перед Спасителем моїм, сповідуючи гріхи свої! Але милосердя і невимовне чоловіколюбство Розіп’ятого на тобі смиренну сміливість подає мені, щоб відкрити вуста мої і славити тебе; заради цього співаю тобі: радуйся, Хресте, Церкви Христової окраса й основа, всієї вселенної — утвердження, християн усіх — уповання, царів — держава, вірних — пристановище, ангелів — слава і оспівування, демонів — страх, знищення і відігнання, нечестивих і невірних — посоромлення, праведних — насолода, обтяжених — послаблення, сущих у бурі — пристановище, одержимих пристрастями — розкаяння, убогих — збагачення, плаваючих — керманич, слабких — сила, на війні — перемога і подолання, сиріт — вірний покров, дів — заступник, дівственників — цнотливості охорона, безнадійних — надія, недужих — лікар, мертвих — воскресіння! Ти прообразований чудотворним жезлом Мойсея, життєдайне джерело, що напоює спраглих духовним життям і втішає нашу скорботу; ти — ложе, на якому царствено спочивав триденний Воскреслий Переможець пекла. Заради цього і вранці, і ввечері, і в обід прославляю Тебе, триблаженне древо, і благаю: волею Розіп’ятого на тобі нехай просвітить Він і зміцнить тобою мій розум, нехай відкриє в серці моєму джерело любові, а всі мої вчинки і путі мої тобою осяє, щоб величати Прицвяхованого на тобі заради гріхів моїх, Господа Спасителя мого. Амінь.

\section{Молитва до Чесного Хреста}
Нехай воскресне Бог і розвіються вороги Його, і нехай біжать від лиця Його всі ненависники Його. Як щезає дим, нехай щезнуть, як тане віск від лиця вогню, так нехай згинуть біси від лиця тих, хто любить Бога і осіняє себе хресним знаменням і в радості промовляє: радуйся, Пречесний і Животворчий Хресте Господній, що проганяєш бісів силою розп’ятого на тобі Господа нашого Ісуса Христа, що до пекла зійшов, і подолав силу диявола, і дарував нам тебе, Хрест Свій Чесний, на прогнання всякого супротивника. О Пречесний і Животворчий Хресте Господній, допомагай мені зі Святою Дівою Богородицею і з усіма святими Небесними Силами завжди, нині і повсякчас, і на віки віків. Амінь.
\end{document}