\documentclass[chapters.tex]{subfiles}

\begin{document}
\chapter{Молитви вранішні}
Вставши від сну, перш будь-якого діла, стань побожно перед святою іконою і, уявляючи себе перед Всевишнім Богом, поклади на себе тричі знак хреста, промовляючи:

В ім’я Отця, і Сина, і Святого Духа. Амінь.

Спинившись, заспокой свої почуття, щоб твої думки лишили все земне, тоді сотвори три поклони, промовляючи:

Боже, будь милостивий до мене, грішного,

а потім читай наступні молитви без поспіху і з увагою.

Боже, прости мені всі провини мої перед Тобою.

Господи Ісусе Христе, Сину Божий, молитвами Пречистої Твоєї Матері і всіх святих помилуй нас. Амінь.

Слава Тобі, Боже наш, слава Тобі.

\section{Молитва до Святого Духа}

Царю Небесний, Утішителю, Душе істини, що всюди єси і все наповняєш, Скарбе добра і життя Подателю, прийди і вселися в нас, і очисти нас від усякої скверни, і спаси, Благий, душі наші.

\section{Трисвяте}

Святий Боже, Святий Кріпкий, Святий Безсмертний, помилуй нас (тричі).

\section{Мале славослів’я}

Слава Отцю, і Сину, і Святому Духові і нині, і повсякчас, і на віки віків. Амінь.

\section{Молитва до Пресвятої Тройці}

Пресвята Тройце, помилуй нас; Господи, очисти гріхи наші; Владико, прости беззаконня наші; Святий, зглянься і зціли немочі наші імені Твого ради.

Господи, помилуй (тричі).

Слава Отцю, і Сину, і Святому Духові нині, і повсякчас, і на віки віків. Амінь.

\section{Молитва Господня}
Отче наш, що єси на небесах, нехай святиться ім’я Твоє; нехай прийде Царство Твоє; нехай буде воля Твоя, як на небі, так і на землі. Хліб наш насущний дай нам сьогодні; і прости нам провини наші, як і ми прощаємо винуватцям нашим; і не введи нас у спокусу, але визволи нас від лукавого.

Бо Твоє є Царство, і сила, і слава, Отця, і Сина, і Святого Духа, нині, і повсякчас, і на віки віків. Амінь.

\section{Тропарі Троїчні}

Вставши після сну, припадаємо до Тебе, Благий, і Ангельську пісню співаємо Тобі, Всемогутній: Свят, Свят, Свят єси, Боже, молитвами Богородиці помилуй нас.

Слава Отцю, і Сину, і Святому Духові.

Ти, Господи, що від сну підняв мене, просвіти розум і серце і уста мої відкрий, щоб піснею прославляти Тебе, Свята Тройце: Свят, Свят, Свят єси, Боже, молитвами Богородиці помилуй нас.

І нині, і повсякчас, і на віки віків. Амінь.

Несподівано Суддя прийде, і вчинки кожного виявляться; тому зо страхом взиваймо опівночі: Свят, Свят, Свят єси, Боже, молитвами Богородиці помилуй нас.

Господи, помилуй (12 разів).

\section{Молитва св. Макарія до Пресвятої Тройці}

Вставши після сну, подяку приношу Тобі, Пресвята Тройце, Боже наш, що Ти з великої милости Твоєї і довготерпіння не прогнівався на мене лінивого та грішного і не дав мені загинути з моїми гріхами, але, з властивим Тобі чоловіколюбством, підняв мене від сну, щоб зранку прославляти державу Твою. І нині просвіти мої очі мисленні, відкрий мої уста, щоб навчатися слова Твого, і розуміти заповіді Твої, і чинити волю Твою, і вихваляти Тебе в сердечному прославленні, і в піснях славити всесвяте ім’я Твоє, — Отця, і Сина, і Святого Духа, — нині, і повсякчас, і на віки віків. Амінь.

\section{Поклоніння Ісусу Христу}
Прийдіть, поклонімось Цареві нашому, Богу.

Прийдіть, поклонімось і припадімо до Христа, Царя і Бога нашого.

Прийдіть, поклонімось і припадімо до Самого Христа, Царя і Бога нашого.

\section{Псалом 50}
Помилуй мене, Боже, з великої милости Твоєї і з великого милосердя Твого прости провини мої. Особливо омий мене від беззаконня мого і від гріха мого очисти мене. Бо беззаконня моє я знаю, і гріх мій повсякчас переді мною. Проти Тебе Єдиного я згрішив і лукаве перед Тобою вчинив, отже праведний Ти в слові Твоїм і справедливий у присуді Твоїм. Ось бо в беззаконні зачатий я, і в гріхах породила мене мати моя. Бо Ти істину полюбив єси, невідоме й таємне мудрости Твоєї явив Ти мені. Окропи мене ісопом — і очищуся, омий мене — і стану біліший від снігу. Дай мені почути радість і веселість — і зрадіють кістки мої упокорені. Відверни лице Твоє від гріхів моїх і прости всі беззаконня мої. Серце чисте створи в мені, Боже, і духа праведного віднови в нутрі моєму. Не відкинь мене від лиця Твого і Духа Твого Святого не відніми від мене. Поверни мені радість спасіння Твого і духом могутнім укріпи мене. Навчатиму беззаконників шляхів Твоїх, і нечестиві навернуться до Тебе. Визволи мене від вини кривавої, Боже, Боже спасіння мого, і язик мій радісно славитиме правду Твою. Господи, відкрий уста мої, і язик мій сповістить хвалу Твою. Бо коли б Ти жертви забажав, приніс би я: всепалення Ти не бажаєш. Жертва Богові — це дух упокорений, серцем скорботним і смиренним Ти не погордуєш. Ущаслив, Господи, благоволінням Твоїм Сион, і хай збудуються стіни Єрусалимські. Тоді буде угодна Тобі жертва правди, приношення і всепалення: тоді покладуть на Жертовник Твій тельців.

\section{Символ віри}
Вірую в Єдиного Бога Отця, Вседержителя, Творця неба і землі, всього видимого і невидимого. І в Єдиного Господа Ісуса Христа, Сина Божого, Єдинородного, що від Отця народився перше всіх віків.

Світло від Світла, Бога Істинного від Бога Істинного, рожденного, несотворенного, єдиносущного з Отцем, що через Нього все сталося.

Він для нас, людей, і ради нашого спасіння зійшов з небес, і воплотився від Духа Святого і Марії Діви, і став чоловіком. І розп’ятий був за нас при Понтії Пилаті, і страждав, і був похований.

І воскрес на третій день, як було написано.

І вознісся на небо, і сидить праворуч Отця.

І знову прийде у славі судити живих і мертвих, і Царству Його не буде кінця.

І в Духа Святого, Господа Животворчого, що від Отця походить, що Йому з Отцем і Сином однакове поклоніння і однакова слава, що говорив через пророків.

В Єдину, Святу, Соборну і Апостольську Церкву.

Визнаю одно хрещення на відпущення гріхів.

Чекаю воскресіння мертвих.

І життя будучого віку. Амінь.

\section{Молитва 1-а, святого Макарія Великого}
Боже, очисти мене грішного, бо я ніколи нічого доброго не вчинив перед Тобою, але Ти визволи мене від усього злого, і хай буде в мені воля Твоя, щоб неосудно я відкрив уста мої грішні і славив святе ім’я Твоє, Отця, і Сина, і Святого Духа, нині, і повсякчас, і на віки віків. Амінь.

\section{Молитва 2-а, того ж святого}
Від сну вставши, ранішню приношу Тобі, Спасе, пісню і, припадаючи, прошу Тебе: не дай мені заснути в гріховній смерті, але пожалій мене, Ти, що добровільно розп’явся; і мене, який лежить у лінощах, підніми швидко й спаси мене, в предстоянні і молитві. І по сні нічнім пошли мені світлий і безгрішний день, Христе Боже, і спаси мене.

\section{Молитва 3-я, того ж святого}
До Тебе, Владико Чоловіколюбче, вставши після сну, вдаюся і до праці Твоєї стаю з Твоєї милости, і молюся Тобі: допомагай мені повсякчасно і в усякому ділі, і захисти мене від усього лихого, що є в світі, і від диявольської спокуси, і спаси мене, і введи в Царство Твоє вічне, бо Ти мене створив і дбаєш про мене, щоб обдарувати мене всяким добром; на Тебе вся моя надія, і Тебе я щиро славлю нині, і повсякчас, і на віки віків. Амінь.

\section{Молитва 4-а, того ж святого}
Господи, що з великої милости Твоєї і щедрости Твоєї дав мені, рабу Твоєму, минулу ніч перебути без напасти від усього злого і супротивника, Ти Сам, Владико, Створителю всього, сподоби мене при світлі Твоєї істини і з просвітленим серцем творити волю Твою нині, і повсякчас, і на віки віків. Амінь.

\section{Молитва 5-а, св. Василія Великого}
Господи Вседержителю, Боже Небесних Сил і всякої плоті, що на небі живеш і нас, смиренних, доглядаєш, серця й думки випробовуєш і те, що є таємне в людині, добре бачиш; Безпочаткове і Вічне Світло, в Якому немає ані тіні зміни або перетворення, Сам, Безсмертний Царю, прийми молитви наші, які ми в цей час грішними нашими устами творимо, на безліч щедрот Твоїх сподіваючись, і прости нам гріхи наші, чи словом, чи ділом, чи думкою, свідомо чи несвідомо заподіяні, і очисти нас від усякої скверни тіла й духу; і дай, щоб ми в бадьорості серця і тверезості розуму прожили всю ніч цього життя, очікуючи світлого і явленого дня Єдинородного Сина Твого, Господа Бога і Спаса нашого, Ісуса Христа, коли Він зі славою прийде, Суддя всіх, відплатити кожному по його ділах.

Нехай не в розслабленні і лінощах, а в бадьорості і щирій праці ми з’явимось готовими і ввійдемо з Ним у Божественне Царство Його слави, де голос невмовкний тих, що святкують, і невимовна утіха всіх, що бачать неосяжну доброту лиця Твого. Бо Ти є істинне Світло, що освітлює і освячує все, і Тебе славить всяке створіння на віки віків. Амінь.

\section{Молитва 6-а, того ж святого}
Благословляємо Тебе, Небесний Боже і Милосердний Господи, за те, що виявляєш на нас повсякчас безліч великих, недосяжних славних діл Своїх, що дав нам сон на підтримку нашої немочі і відпочинок від праці знесиленого тіла. Дякуємо Тобі, що не погубив нас з нашими гріхами, а зі звичайною Твоєю любов’ю до людей підняв нас від сну, щоб прославляли ми Твою владу. Отже, молимо Твою безмірну добрість: просвіти наші думки, підведи очі і розум наш від тяжкого сну лінощів, відкрий наші вуста і наповни їх похвалою Тобі, щоб ми мали змогу непохитно визнавати і славити Тебе, Безпочаткового Отця із Єдинородним Твоїм Сином і Пресвятим, Благим і Животворчим Твоїм Духом нині, і повсякчас, і на віки віків. Амінь.

\section{Молитва 7-а, до Пресвятої Богородиці}
Оспівуючи благодать Твою, Владичице, молю Тебе, мій розум просвіти благодаттю. Чинити по правді мене настав шляхом Христових заповідей. Бадьорим до пісні укріпи, сон зневіри віджени. Молитвами Твоїми, Богоневісто, звільни з гріховних пут, що ними я скутий. Охороняй мене вдень і вночі, захищай від ворогів, що нападають на мене. Ти, що Життєдавця Бога народила, оживи мене, умертвленого пристрастями. Ти, що народила Світло не вечірнє, просвіти мою засліплену душу. О пречудна Владична палато, мешканням Святого Духа мене створи. Ти, що Цілителя народила, вилікуй душі моєї багаторічні пристрасті. Розбурханого життєвою бурею, на стезю покаяння мене спрямуй. Визволи мене від вічного вогню, від червія злого й пекла. Повинного в багатьох гріхах, не покажи мене радісним бісам. Застарілого нечутливими згрішеннями, створи мене, Пренепорочна Мати, новим. Звільни мене від усякої муки і Владику всього ублагай. Сподоби мене заслужити небесної радости зо всіма святими. Пресвята Діво, почуй голос нікчемного раба Твого. Пошли мені, Пречиста, струмінь сліз, щоб очистити скверну моєї душі. Стогін від мого серця Тобі приношу безперестанку, відгукнись, Владичице. Молебну мою службу прийми і Богові Благоутробному принеси. Вища від ангелів, мене вищим за все світське створи. Світлоносна Покрово Небесна, духовну благодать в мене направ. Руки підійму і уста, осквернені скверною, для похвали, Всенепорочна. Душорозтлінних ворогів мене позбав, Христа старанно благаючи. Йому ж честь і поклоніння належить, нині, і повсякчас, і на віки віків. Амінь.

\section{Молитва 8-а, Господу нашому Ісусу Христу}
Многомилостивий і Всемилостивий Боже мій, Господи Ісусе Христе, що з безмірної любови Ти зійшов і воплотився, щоб спасти всіх. Тебе, Спасителю, молю, спаси мене з благодаті Твоєї; бо якщо за діла мої спасеш, то це не буде благодать і дар, але більш як обов’язок. О Спасителю мій, многомилостивий і невимовний в милості! Ти Сам сказав, Христе мій: «Хто вірує в Мене, житиме і смерти не побачить повік». Коли ж насправді віра в Тебе спасає тих, що у відчаї, то я вірую, спаси мене, бо Ти Бог мій і Творець; віру цю, Боже мій, замість діл зарахуй мені, бо не знайдеш вчинків, заради яких Ти виправдав би мене. Але тієї віри моєї нехай вистачить замість діл моїх, вона нехай відповідає за мене, і нехай виправдає мене, нехай зробить мене учасником вічної Твоєї слави. Щоб сатана не полонив мене і не похвалився, Слово, що відірвав мене від Твоєї руки й захисту. Але хочу чи не хочу, спаси мене, Христе, Спасе мій; вийди мені назустріч, рятуй мене, бо гину. Ти бо єси мій Бог від утроби матері моєї. Сподоби мене, Господи, нині возлюбити Тебе, як іноді я любив той самий гріх; і ще послужити Тобі без лінощів і щиро, як служив раніше облесливому сатані. Нехай вже нікому не служитиму, тільки Тобі, Господу і Богу моєму, Ісусу Христу, в усі дні життя мого нині, і повсякчас, і на віки віків. Амінь.

\section{Молитва 9-а, до Ангела-Охоронителя}
Святий ангеле, приставлений до моєї грішної душі і до пристрасного мого життя, не покидай мене грішного і не відступи від мене за нестриманість мою. Не попусти злому духові володіти мною через пристрасті мого смертного тіла. Зміцни мої немічні сили в боротьбі з гріхами і наставляй мене на спасенну дорогу. О святий ангеле Божий, охоронителю й захиснику моєї нерозкаяної душі і тіла, прости мені все, чим образив я тебе за всі дні життя мого, і коли чим погрішив минулої ночі, захисти мене в нинішній день, і охороняй мене від усякої спокуси диявольської, щоб я ніякими гріхами не гнівив Бога мого, і молися за мене до Господа, щоб Він утвердив мене у страху Своїм і показав мене достойним рабом Своєї безмірної благости. Амінь.

\section{Молитва 10-а, до Пресвятої Богородиці}
О Пречиста Діво, Богородице, надіє всіх християн, що живуть на землі! Не відвертайся від мене грішного, бо я непохитно надіюся на Твоє милосердя. Погаси в мені полум’я багатьох пристрастей моїх; запали моє жорстоке серце і мій холодний розум любов’ю до виконання заповідей Сина Твого і Бога нашого. Викорени, Пречиста, з мого затьмареного розуму і з мого жорстокого серця безнадійність, недбалість, лінощі та безглуздя і всі темні і брудні думки та наміри. Зціли, Владичице, мої тяжкі духовні й тілесні рани, не попусти, щоб зло перемагало мене, і Твоїми материнськими молитвами прихили до мене милість Сина Твого, Господа нашого Ісуса Христа, бо Ти благословенна від усіх родів, і славиться пречисте ім’я Твоє на віки віків. Амінь.

\section{Молитва до святого, ім’я якого носиш}
Моли Бога за мене, святий угоднику Божий (ім’я святого), бо я щиро до тебе звертаюся, скорого помічника й молитвеника за душу мою.

\section{Пісня до Пресвятої Богородиці}
Богородице Діво, радуйся, Благодатна Маріє, Господь з Тобою.

Благословенна Ти в жонах, і благословен плід утроби Твоєї, бо Ти породила Спаса душ наших.

\section{Тропар до Хреста і молитва за Батьківщину}
Спаси, Господи, людей Твоїх і благослови насліддя Твоє, перемогу побожному народові нашому на супротивників подай і Хрестом Твоїм охороняй нас оселю Твою.

\section{Молитва Оптинських старців}
Отче, дай мені з душевним спокоєм зустріти все, що принесе мені цей день. Дай мені цілком віддатися волі Твоїй святій, в усякий час цього дня, в усьому настав і підтримай мене. Які б я не одержав звістки протягом дня, навчи мене сприйняти їх із спокійною душею і твердим упевненням, що на все свята воля Твоя. В усіх словах і справах моїх керуй моїми думками й почуттями. В усіх несподіваних випадках не дай мені забути, що все Ти послав. Навчи мене щиро й розумно обходитися з кожною людиною, нікого не соромлячи й не засмучуючи. Отче, дай мені сили знести втому прийдешнього дня та всі події сьогодення. Керуй моєю волею і навчи мене молитися, вірити, надіятись, терпіти, прощати й любити. Амінь.

\section{Молитви за живих}
Пом’яни, Господи Ісусе Христе, Боже наш, милості й щедроти Твої від віку сущі, заради яких Ти став людиною і розп’яття і смерть, заради спасіння православно в Тебе віруючих, перетерпіти зволив єси; і воскрес із мертвих, і вознісся на небеса, і сидиш праворуч Бога Отця, і милостиво споглядаєш на моління тих, хто смиренно всім серцем призиває Тебе: прихили вухо Твоє і почуй моє смиренне моління нікчемного раба Твого, серед пахощів духовних, що приношу Тобі за всіх людей Твоїх. І найперше пом’яни, Господи Ісусе Христе, Боже наш, Церкву Твою Святу, Соборну й Апостольську, що чесну Свою Кров за неї пролив єси, і утверди, і зміцни, і пошир, збагати, замири, і непереможною від ворогів пекельних повік збережи; розбрат Церков втихомир, думки поганські вгаси, єресі знищи і скорени, безбожництво вигуби і на ніщо силою Святого Твого Духа перетвори (поклін). Владу нашу миром оточи і від усякого ворога й супротивника охорони. Вклади в серце їм усяку добру і спокійну думку про Церкву Твою Святу і про всіх людей Твоїх, щоб і ми в спокої тихе й безтурботне життя провадили в істинній вірі, у всякому благочесті та чистоті (поклін).

Спаси, Господи, і помилуй Святійших Вселенських Патріархів православних, Святійшого отця нашого (ім’я), Патріарха Київського і всієї Руси-України, преосвященнійших митрополитів, архиєпископів, єпископів, пресвітерів, дияконів, весь чернечий чин і весь причет церковний, що поставив єси їх пасти Твоє словесне стадо, і молитвами їх помилуй і спаси мене, грішного (поклін). Спаси, Господи, і помилуй отця мого духовного (ім’я) і святими його молитвами прости гріхи мої (поклін).

Спаси, Господи, і помилуй батьків моїх, братів, сестер і рідних моїх, і всіх родичів моїх та друзів і даруй їм мир Твій та все добре (поклін).

Спаси, Господи, і помилуй старих і молодих, бідних, сиріт, вдовиць, хворих, засмучених, тих, що в біді, в скорботі, у злиднях, в неволі, у в’язницях, на засланні; особливо ж тих, що за Тебе і віру православну безбожниками, відступниками та єретиками переслідувані, пом’яни їх, відвідай, укріпи, заспокой і скоро силою Твоєю полегшення, волю і визволення їм подай (поклін).

Спаси, Господи, і помилуй доброчинців наших, що допомагають нам, жаліють і доглядають нас, дають нам милостиню і просять за нас, недостойних, молитися за спасіння і вічних благ Твоїх отримання (поклін). Спаси, Господи, і помилуй посланих на службу, тих, що подорожують, отців, братів і сестер наших і всіх православних християн (поклін).

Спаси, Господи, і помилуй тих, кого безумством своїм я спокусив і з правдивої дороги звів, а на погані й негідні діла напровадив їх; Божим Твоїм промислом на істинну путь знову їх настанови (поклін).

Спаси, Господи, і помилуй тих, що ненавидять і кривдять нас і роблять нам напасть; не допусти їх до загибелі через мене, грішного (поклін).

Тих, що від православної віри відступили, погибельними єресями засліплені, і від Церкви нашої відкололися, світлом Твого пізнання просвіти й до Святої Твоєї Соборної і Апостольської Церкви знову приєднай (поклін).

Мерзенну безбожну богохульну владу скорени, а правовірну утверди, силу християнську піднеси і милості Твої багаті нам пошли (поклін).

\section{Молитва за померлих}
Пом’яни, Господи, спочилих Святійших Патріархів православних, преосвященних митрополитів, архиєпископів, єпископів, священичий і чернечий чин, причет церковний, що Тобі послужили, благочестивих і повік незабутніх фундаторів святих храмів й у вічних Твоїх оселях зі святими упокой (поклін).

Пом’яни, Господи, душі спочилих рабів Твоїх, батьків моїх (їх імена) й увесь мій рід і прости їм всі провини вільні й невільні, даруй їм Царство й причастя вічних Твоїх благ і Твого безконечного й блаженного життя насолоду (поклін). Пом’яни, Господи, і всіх в надії на воскресіння і життя вічне спочилих отців, братів і сестер наших, православних християн, що тут і повсюди лежать, і зі святими Твоїми, де сяє світло лиця Твого, всели, і нас помилуй, як Благий і Чоловіколюбний. Амінь (поклін).

Подай, Господи, відпущення гріхів усім раніше спочилим у вірі й надії на воскресіння, отцям, братам і сестрам нашим, і створи їм вічную пам’ять (тричі).

\section{Достойно є; закінчення молитов}
Достойно є, і це є істина, славити Тебе, Богородицю, Присноблаженну і Пренепорочну і Матір Бога нашого. Чеснішу від херувимів і незрівнянно славнішу від серафимів, що без істління Бога Слово породила, сущу Богородицю, Тебе величаємо.

Слава Отцю, і Сину, і Святому Духові нині, і повсякчас, і на віки віків. Амінь.

Господи, помилуй (тричі).

Господи Ісусе Христе, Сину Божий, молитвами Пречистої Твоєї Матері і всіх святих помилуй нас. Амінь.

\emph{Під час Великого посту промовляємо цю молитву}:

\section{Молитва святого Єфрема}
Господи і Владико життя мого, духа лінивства, безнадійности, владолюбства і пустомовства не дай мені (доземний поклін).

Духа ж чистоти, смиренномудрости, терпеливости й любови дай мені, рабові Твоєму (доземний поклін).

Так, Господи Царю! Даруй мені бачити провини мої і не осуджувати брата мого, бо Ти благословенний навіки-віків. Амінь (доземний поклін).

\emph{Після цього 12 малих поклонів, молячись:}

Боже, будь милостивий до мене грішного. Боже, очисти мої гріхи і помилуй мене. Без числа нагрішив я, Господи, прости мені.

\emph{Після цього всю молитву:}

«Господи і Владико життя мого…» (доземний поклін).

У Світлий тиждень, замість ранішніх і вечірніх молитов, читаються Пасхальні часи.

\emph{Від Великодня до Вознесіння, замість молитви до Святого Духа «Царю Небесний», співаємо тропар Пасхи:}

Христос воскрес із мертвих, смертю смерть подолав і тим, що в гробах, життя дарував (тричі).  
\end{document}