\documentclass[chapters.tex]{subfiles}

\begin{document}
\chapter{Патріотичні молитви}
\section{Молитва за Україну}
Боже Великий, Боже Всесильний! Ми, грішні діти Твої, у покорі сердець наших приходимо до Тебе і схиляємо голови наші. Отче! Прости провини наші та провини батьків, дідів і прадідів наших. Прийми нині, благаємо Тебе, щиру молитву нашу і подяку нашу за безмежне милосердя Твоє до нас. Вислухай наші молитви і прийми благання сердець наших. Благослови нашу Батьківщину Україну, долю та щастя їй дай.

Премилосердний Господи, усім, хто вдається до Тебе з благанням, ласку Твою подай.

Благаємо Тебе, Боже, за братів і сестер наших, за вдовиць, за сиріт, за калік і немічних, і за тих, що Твого милосердя та допомоги Твоєї потребують.

З’єднай нас усіх в єдину велику Христову сім’ю, щоб усі люди, як брати, славили величне ім’я Твоє завжди, і нині, і повсякчас, і на віки віків. Амінь.

\section{Молитва за Українську Православну Церкву}
Господи Ісусе Христе, наш Спасителю і Найвищий Учителю! Щиро благаємо Тебе за увесь світ та за народи світу цього. Нехай не буде вже серед нас неправди, ворожнечі, кровопролиття, а нехай запанує в усьому світі Твоя свята правда та наука в заклику до згоди, рівности, братерства та єднання.

Згадай, Спасителю наш, і про Церкву Святу Твою, яку Ти Своєю Святою Кров’ю Непорочною здобув і освятив, а князі віку цього свавільно розділили, роз’єднали та зневажили. Нехай скоро об’єднаються в Єдине Святе Тіло Твоє всі Церкви всіх народів світу, що в Тебе, Єдиного Бога, вірують, молитвами і побожними ділами Тебе прославляють.

Церкву нашу Святу Українську Православну у славі піднеси та в одне тіло з Церквами всіх народів сполучи, щоб усіма мовами славили всесвяте ім’я Твоє, в Тройці Єдиного Бога, Отця, і Сина, і Святого Духа, нині, і повсякчас, і на віки віків. Амінь.

\section{Боже великий, єдиний}

Боже великий, єдиний,

Нам Україну храни,

Волі і світу промінням

Ти її осіни.

Світлом науки і знання

Нас, дітей, просвіти,

В чистій любові до краю

Ти нас, Боже, зрости.

Молимось, Боже єдиний,

Нам Україну храни,

Всі свої ласки, щедроти

Ти на люд наш зверни.

Дай йому волю, дай йому долю,

Дай доброго світу, щастя,

Дай, Боже, народу

Многая, многая літа.

\section{Молитва за український народ}
Всемогутній Боже і Царю Всесвіту, Спасителю наш, Ісусе Христе, що любиш увесь людський рід і своїм Безмежним Провидінням опікуєшся кожним народом зокрема! Поглянь милосердно і на наш український народ, який з повною надією припадає до Тебе, як до свого найкращого Отця і Премудрого Царя. Ми, діти цього народу, є покірно послушними Твоїй Святій Волі, ми любимо всіх людей, яких Ти відкупив Своєю Святою Кров’ю на Хресті, а найперше любимо щирою християнською любов’ю наш український народ. Тому з любові до нього, а вірніше з любові до Тебе, наш Боже, благаємо: прости йому всі провини; виправ всі його злі нахили, а вкоріни добрі нахили; змилосердься над ним у всіх його потребах. Оберігай його перед усякою кривдою й несправедливістю ворогів. Подавай йому повсякчас Твоє щедре Благословення.

Благаємо Тебе, наш Боже, про особливу опіку й поміч для нашого народу, щоб серед усіх переживань і спокус з боку світу, диявола і його слуг, він міг завжди зберегти Небесне Світло Віри, щоб перемагав всілякі труднощі, перебуваючи в добрі, і завжди належав до Благословенного Твого Божого Царства, — і тут на цьому світі, і в Небесній Батьківщині. Дай нам Ласку, щоб ми всі до одного, з’єднані вірою і союзом любові під Твоїм проводом і проводом Святої Вселенської Церкви, ішли завжди дорогами правди і справедливості, любови та спасіння. Пошли українському народові святих, великих Твоїх слуг, щоб прикладом і словом були його мудрими провідниками у всіх царинах народного, суспільного і громадського життя. Провідникам нашого народу дай Світло Твоєї Премудрості з Неба. Дай нам численне й добре та святе духовенство! Заопікуйся його молоддю, щоб вони не розтратили Ласки Святого Хрещення, щоб одержували в родині і школі основне християнське виховання і стали добрими синами і дочками свого народу. Благослови всі наші родини, щоб батьки були зразковими і ревними християнами, а матері визначалися мудрістю, побожністю і дбайливістю у вихованні дітей. Заохочуй багатьох з нашого народу до життя досконалішого, до святості, до геройських жертв за справи Церкви й народу.

Просвіти всіх нас, нахили наші серця, щоб усі ми якнайкраще пізнавали й цінили святу нашу віру і, визнаючи її, відчували себе щасливими, були у вірі твердими, хоч би треба було понести й мученичеську смерть, та щоб за законами Святої Віри упорядковували своє життя. Благослови також дочасне добро нашого народу. Дай йому волю, щоб він міг вільно розвивати Тобою дані таланти. Обдаруй його правдивою незіпсованою просвітою. Благослови його працю на всіх ділянках розвитку: науки, мистецтва й добробуту.

Та благослови всіх і все, щоб наш народ, живучи мирно та щасливо, міг добре Тобі служити, а з Твоєю допомогою одержати Вічну Небесну Батьківщину. Пресвята Богородице, Мати-Царице України, святий Обручнику Йосифе, Покровителю Вселенської Церкви, святий архангеле Михаїле, і всі Покровителі українського народу, опікуйтеся завжди цим народом, щоб він став народом святим, щоб виконав своє Боже призначення, щоб навернув усіх, хто ще є поза Церквою, до світла віри, щоб прихилився до світлого добра людського роду, щоб був поміччю і потіхою Святої Вселенської Церкви та щоб приносив Вічному Цареві безперестанну славу, честь і поклін на віки вічні. Амінь.

\section{Молитва за кращу долю українського народу}
Всемогучий, Безсмертний Боже, Отче Господа нашого Ісуса Христа! Ти страшний у Твоєму Гніві, але безконечно добрий у Твоєму Милосерді! Ти посилаєш на наш грішний людський рід тяжкі випробовування і тяжкі терпіння. Ти і на наш народ послав випробовування, яких зазнали наші батьки. Ми всі мусимо важко трудитися, щоб здобути для себе хліб щоденний. Віримо і надіємося, що ці тяжкі допусти ти дав нашому народу як випробовування, а не як покарання.

Грішних Твоїх синів Ти не відкидаєш, не відвертаєш Твого Лиця від їхньої долі, бо не хочеш смерті грішника, а лише його навернення та спасіння. Тому надіємося, що всі терпіння, яких доводиться нам у житті витерпіти з Милосердної Твоєї Волі, є для нашого ж добра. Ми віримо і надіємося, що покаянням, щирою молитвою і щирим Причастям Найсвятіших Тайн, ми зможемо заслужити собі зменшення випробовувань і терпінь, що Милосердним Оком Ти поглянеш на Твоїх дітей, що дозволиш їм витримати з честю всі злидні теперішнього життя і Всемогучою Твоєю Волею положиш край нашим терпінням.

Преблагословенна Діво Маріє! Дорогоцінною кров’ю Твого Сина бажаємо випросити пробачення в Тебе за всі наші гріхи й за гріхи наших батьків і братів. Через наших священиків приносимо Тобі Безкровну Жертву Пресвятої Євхаристії, яка є повторенням і відновленням Хрестової Жертви Твого Сина і Бога, Спасителя нашого Ісуса Христа.

Прийми, Господи, цю нашу жертву, як надолуження за всі гріхи людей, прийми наші благальні прохання про Милосердя і Твою Святу Благодать. Будь милостивим до всіх тих, хто терпить. Даруй їм Ласку, терпеливо зносити гірку долю і через терпеливість заслужити на пільги в терпіннях і перемогу в цьому смертному житті та на блаженство у Вічності. Амінь.

\section{Молитва за єдність церков}
Господи, Ти полюбив Церкву і віддав за неї Своє життя, освятив її і очистив Своїм Словом. Відверни від неї всілякі розколи, звільни її від духу суперництва і заздрості. Збережи нас від осудження без любові та з’єднай нас у Твоїй Святій Справі. Благослови всіх людей, які люблять Тебе, Господа нашого Ісуса Христа, хоч би як вони себе називали. Веди всіх до міцнішого з’єднання з Тобою. Збережи Твою Церкву в єдності Пастирю Душ, прожени темряву з їхнього розуму, покажи марність їхніх земних мрій і розкоші. Зворуш ласкою їхні затверділі серця та витягни їх з гріховної прірви, до якої запровадив їх гріх і підступ сатани. Ісусе, Ти можеш спасти їх, бо Милосердю Твого Люблячого Серця немає межі.

Пресвятая Богородице Маріє! Наша Люба Мати і Заступнице бідних грішників, молися за них, щоб пізнали свій нужденний стан, навернулися до Бога та спасли свою душу. Амінь.
\end{document}