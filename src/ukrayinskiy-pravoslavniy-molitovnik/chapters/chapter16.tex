\documentclass[chapters.tex]{subfiles}

\begin{document}
\chapter{Молитви-прохання}
\section{Молитва на всяке прохання}
Споглянь, Владико Чоловіколюбче, милостивим оком на рабів Твоїх і почуй благання наші, що з вірою просимо, як і Сам Ти сказав: «Усе, що в молитві проситимете, віруйте — одержите, і буде вам, і знов: просіть, і дасться вам». Тому і ми, що надіємось на милість Твою, хоч і недостойні, просимо: подай милість Твою рабові твоєму (ім’я) та задовольни добре бажання його, в мирі та спокої, у здоров’ї та довговічності все життя його збережи. Тобі славу возсилаємо, Отцю, і Сину, і Святому Духу, і нині, і повсякчас, і на віки віків. Амінь.

\section{Молитва друга на всяке прохання}
Господи Боже наш, вислухай голос моління нас грішних і помилуй рабів Твоїх (імена), милістю і щедротами Твоїми виконай усі прохання їхні і пробач їм гріхи вільні і невільні, прими їхні молитви та пожертви перед Престолом Владицтва Твого. Охорони їх від усіх ворогів видимих і невидимих, від усякої напасті, біди, смутку і немочі звільни, дай їм здоров’я та довголіття. Поглянь, Владико Чолоковіколюбче, милостивим оком Твоїм на рабів Твоїх, і почуй молитви наші, які ми з вірою приносимо, бо Ти Сам сказав: усе що в молитві просите, віруйте — і буде вам, і знову: просить, і дасться вам. Тому й ми, хоч і недостойні, надіючись на милість Твою, просимо: подай милість Твою рабам Твоїм і виконай добрі бажання їх, у мирі, спокої, здоров’ї та довголітті всі дні їх збережи, просимо Тебе, скоро вислухай і помилуй. Амінь.

\section{Терпіння і боротьба зі злом. Молитва в терпінні}
Всемогучий Боже! Твоє Провидіння опікується всім створінням, і нічого в світі не діється без Твоєї Святої Волі чи Твого допусту. Ти, Господи, знаєш мій біль і мої терпіння, пішли мені Твою Святу Ласку, щоб я витримав у надії і терпеливості так довго, доки це Тобі подобається. Прийми ці терпіння як покуту за мої гріхи, нехай вони очистять мою душу і полегшать мою дорогу до Неба. Найсолодший Ісусе, допоможи мені нести мого хреста, навчи мене, як я маю наслідувати Тебе і з Тобою разом терпіти. Ти сказав: «Прийдіть до мене всі струджені і обтяжені, і я заспокою вас». Ісусе, потіш мене, навчи в кожній хвилині мого життя бачити Божу Волю і вірно виконувати її. Ісусе, укріпи мене в моєму терпінні. Мати Божа, випроси для мене у Твого Сина Ласку терпеливості. Амінь.

\section{Молитва за тих, хто у в’язниці}
Господи Ісусе Христе, Боже Наш, святого апостола Твого Петра від кайданів і тюрми без всякої шкоди звільнивший, смиренно молимося до Тебе, прийми милостиво цю жертву, щоб позбавилися гріхів слуги Твої (слуга Твій, слуга Твоя) (ім’я) в тюрму посаджені і молитвами їхніми та нашими, як Чоловіколюбець, всесильною Твоєю Правицею від всіляких злих обставин охорони і на свободу щасливо виведи. Амінь.

\section{Молитва про перемогу над спокусами}
Господи Милосердний! Ти створив людину, щоб душею і тілом прославляла Тебе і вірно Тобі служила. Ти вчинив моє тіло храмом Святого Духа й освячуєш мене щоразу, коли в Пресвятій Тайні Євхаристії я приймаю Найсвятіше Тіло і Кров Твого Єдинородного Сина Ісуса Христа.

Дай мені, Господи, силу, щоб я міг мужньо й успішно переборювати спокуси злого духу та всі мої низькі пристрасті й пожадання. Зачини двері мого серця перед жаданням земної слави та скороминущого майна, не дай доступити до мене ненависті й заздрощам, віджени від мене гріховні думки та бажання і дай мені силу подолати їх. Дозволь, щоб я через щоденні перемоги, віддав усі мої сили душі й тіла Тобі на служіння.

Укріпи, Господи, Твоєю Милістю мою слабку людську природу, допоможи мені, щоб я щораз більше освячував себе, а колись у майбутньому житті прийми мене до Твоєї Небесної Слави разом з усіма Святими й тими, хто вірно служив Тобі.

Пресвята Богородице, Мати моя Небесна, зглянься на мене, бо я віддався Тобі в опіку й Тобі хочу належати, рятуй і підтримуй мене під час моїх спокус. Амінь.

\section{Молитви при спокусі, на зміцнення}
1. Свята страшна сило Животворчого Хреста, просвіти мене! Навчи мене, Господи, молитися і вірити Тобі, пошли ангела-охоронителя на зміцнення сил і освячення душі. Світло небесне від престолу Правди, зійди і зміцни мене.

2. Зміцни мене, Господи, у боротьбі моїй, не дозволь лукавому змієві мати владу наді мною, не дай загинути в безодні безчестя мого. Зміцни у несенні хреста мого. Амінь.

\section{Молитва про визволення від нечистих помислів}
Владико, Господи Боже мій, у руках Твоїх доля моя. Спаси мене Сам з милости Твоєї, не дай мені загинути у гріхах моїх і не допусти підкоритися плоті, що оскверняє душу мою. Твоє бо я творіння, не зневажай діло рук Твоїх, не віддаляйся, змилосердися і не посором, не покинь мене, Господи, бо я немічний і до Тебе, Покровителя мого, Бога, вдаюся: зціли душу мою, бо я згрішив перед Тобою. Спаси мене з милости Твоєї, бо Ти піклувався про мене від юности моєї, — нехай посоромляться ті, що хочуть віддалити мене від Тебе через грішні вчинки, помисли непристойні, спомини нечисті; віддали від мене всяку розпусту й великі пороки. Бо Ти один тільки Святий, один Кріпкий, один Безсмертний, у всьому незрівнянну могутність маєш, і від Тебе одного подається всім сила проти диявола і воїнства його. Бо Тобі належить всяка слава, честь і поклоніння, Отцю, і Сину, і Святому Духові, нині, і повсякчас, і на віки віків. Амінь.

\section{Молитва від чаклунства}
Господи, Ісусе Христе, Сину Божий, охорони мене заступництвом святих Твоїх ангелів, молитвами Всепречистої Владичиці нашої Богородиці і Вседіви Марії, силою Чесного і Животворчого Хреста, святого архістратига Божого Михаїла та інших небесних сил безтілесних, святого пророка Предтечі і Хрестителя Господнього Іоана, святого апостола і євангеліста Іоанна Богослова, священномученика Кіпріана і мучениці Юстини, святителя Миколая, архієпископа Мирлікійського, чудотворця, святителя Лева, єпископа Катанського, святителя Микити Новгородського, святителя Іоасафа Білгородського, святителя Митрофана Воронезького, преподобного Сергія, ігумена Радонезького, преподобних Зосими і Савватія Соловецьких, преподобного Серафима Саровського, чудотворця, святих мучениць Віри, Надії і Любові та матері їхньої Софії, святого мученика Трифона, святих і праведних богоотців Якима і Анни і всіх святих Твоїх, допоможи мені, недостойному рабу Твоєму (ім’я), визволи мене від усіх підступів ворожих, від усякого зла, чаклування, чарівництва і від людей лукавих, щоб не змогли вони вчинити мені що-небудь зле.

Господи, світлом Твого сяяння охорони мене зранку, в полудень, увечері й під час сну та силою благодаті Твоєї відверни, віддали всякі злі нечисті сили, що діють за намовою диявола. А якщо задумане зло почало свою дію, зупини його та поверни в безодню пекельну, бо Твоє є царство, і сила, і слава Отця, і Сина, і Святого Духа. Амінь.

\section{Молитва про дарування молитви}
Навчи мене, Господи, усердно молитися до Тебе з увагою і любов’ю, без яких молитва не буває почутою! Нехай не буде в мене недбалої молитви, яка приводить до гріха. Амінь.

\section{Молитва про дарування розуміння}
Даруй, Господи, мені, недостойному, благодать розуміння, щоб розпізнавати все те, що є Тобі приємним, і не тільки розпізнавати, але і виконувати його та не приставати до порожнього або гріховного, щоб співстраждати зі страждаючими і бути милосердним до всіх грішників. Амінь.

\section{Молитва про дарування терпіння}
О, дивний Сотворителю Чоловіколюбний Владико, Багатомилостивий Господи! З серцем засмученим і смиренним молюся до Тебе: не погорди молитвами мене грішного, не відкинь моїх сліз і ридання, почуй мене, як жінку-хананеянку, і прийми як блудницю, прояви і на мені, грішному, велику Милість Чоловіколюбства Твого. Ризою Твоєю Чесною захисти мене, помилуй і підкріпи мене, щоб всі біди і напасті, які посилаються від Тебе, я перетерпів з подякою, в надії на вічне блаженство. Найперше смуток мій на радість перетвори, щоб не впав я у відчай і не загинув нерозкаяним. Бо Ти є Джерелом милості і Надією спасіння нашого, Христе Боже наш, і Тобі славу возсилаємо з Безначальним Твоїм Отцем і з Пресвятим, Благим і Животворним Твоїм Духом, нині, і повсякчас, і на віки вічні. Амінь.

\section{Молитва на відречення від гріховних зв’язків}
Відсічи, Господи, таємні пута, що зв’язують мене з життям колишнім і недостойним, від якого відрікаюся, як від сатани. Ім’я Твоє прикликаю, Світильника життя істинного і Милосердного Спасителя. Амінь.

\section{Молитва під час боротьби тілесної}
О Мати Господа, мого Творця, Ти — корень дівства і нев’янучий цвіт чистоти. Допоможи мені немічному, бо я страждаю тілесною пристрастю, а для захисту маю тільки Твоє заступництво і допомогу Сина Твого, Ісуса Христа. Амінь.

\section{Молитва дружини за навернення чоловіка на добру дорогу}
Милосердний Боже, молю Тебе, Господи наш Ісусе Христе, і Тебе, Пречиста Діво, Мати Божа, і всіх вас, Святі мученики, прихиліться до смиренного благання зажуреної жінки й наверніть мого чоловіка (ім’я) на добру дорогу чесного життя. Просвітіть його розум і його серце, нехай він, нещасний, покине свої злі звички й погані грішні намагання, нехай стане знову прихильним до мене й дітей і нехай почне жити чесно й тверезо. Згляньтеся на мене, засмучену й у розпачі, згляньтеся на моїх невинних дітей, поверніть через свою доброту мені чоловіка, а дітям батька, щоб ми разом жили чесно й у злагоді і прославляли імена Ваші. Амінь.

\section{Молитва за примирення людей, що ворогують, перед образом «Пом’якшення злих сердець»}
Пом’якши злі серця наші, Богородице, і згаси напади тих, що ненавидять нас, і всяку тугу душі нашої розріши. На Твій святий образ споглядаючи, Твоїм стражданням і милосердям до нас зворушені, рани Твої цілуємо і стріл наших, що зранюють Тебе, жахаємося. Не дай нам, Мати милосердна, у жорстокосерді нашому й від жорстокосердя ближніх загинути, Ти бо воістину злих сердець пом’якшення.

\emph{(Молитву читають при нападах роздратування чи неприязні до нас ближніх)}

\section{Молитва за наших ворогів та недоброзичливців}
Спаси, Господи і помилуй тих, хто ненавидить і кривдить нас, і спричиняє мені нещастя, і не дай їм загинути через мене грішного. Амінь.

\section{Молитва на припинення ворожнечі та примноження любові}
Владико любові, Чоловіколюбче, Царю віків і Подателю добра, що зруйнував стіни ворожнечі і мир подав рабам Твоїм. Укорени в нас страх Твій і любов один до одного утверди; вгаси усяку міжусобицю, відійми всі спокуси суперечок, бо Ти єси мир наш, і Тобі славу віддаємо, Отцю і Сину, і Святому Духові нині, і повсякчас, і на віки віків. Амінь.

\section{Молитва про примноження любові та викорінення ненависті}
Ти, Ісусе Христе, зв’язав Апостолів Твоїх Союзом любові, тому і нас вірних слуг Твоїх до Себе цим Союзом прив’яжи, щоб ми виконували Заповіти Твої, щоб ми один одного нелицемірно любили, через наші молитви до Богородиці і Всіх Святих. Амінь.
\end{document}