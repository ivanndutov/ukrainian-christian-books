\documentclass[chapters.tex]{subfiles}

\begin{document}
\chapter{Вибрані «Давидові псалми»}

\section{Псалом 22}
Господь пасе мене, і нічого мені не бракуватиме. На місці плодючім, там Він оселив мене; біля води спокійної виховав мене. Душу мою навернув; навів мене на стежки правди задля імені Свого. Коли піду, навіть у тіні смертній, не боятимуся зла, бо Ти зі мною; жезл Твій і палиця Твоя — вони потішили мене. Ти наготував переді мною трапезу на очах гнобителів моїх; Ти намастив оливою голову мою, і чаша Твоя, що напоює, найкраща. І милість Твоя буде йти слідом за мною по всі дні життя мого, й оселюся в домі Господнім на довгі дні.

\section{Псалом 26}
Господь — Просвічення моє і Спаситель мій, кого убоюся? Господь — Захисник життя мого, кого устрашуся? Коли нападають на мене противники й вороги мої, щоб пожерти тіло моє, то вони спотикаються і падають. Коли й військо проти мене стане, не злякається серце моє; коли й ціла війна стане проти мене, і тоді я буду надіятися на Нього. Одного прошу в Господа, одного бажаю: щоб жити мені в домі Господньому всі дні життя мого, дивитися на красу Господню, щоранку молитися в храмі Його. Бо Він сховає мене в домі Своїм у день біди, сховає в оселі Своїй, на скелю піднесе мене. Я піднесу голову мою вище за ворогів моїх, що оточують мене. Принесу в домі Його жертву хвали. Буду співати й прославляти Господа. Почуй, Господи, голос мій, яким я взиваю до Тебе; зглянься на мене і помилуй мене. Від Тебе говорить серце моє: «Шукайте Господа!» Я шукаю лиця Твого, Господи. Не відвертай же від мене лиця Твого і не відкидай у гніві раба Твого. Ти — поміч моя, не відкинь мене і не покинь мене, Боже, Спасителю мій. Бо коли батько й мати покинули мене, то Ти, Господи, прийняв мене. Навчи ж мене, Господи, путей Твоїх і настанови на стежку правди, задля ворогів моїх. Не віддавай мене в руки ворогам моїм, бо повстали проти мене лжесвідки і злобою палають на мене. Але я вірю, що побачу милість Господню на землі живих. Надійся ж на Господа і будь мужнім. Нехай кріпиться серце твоє. Чекай допомоги від Господа.

\section{Псалом 34}
Господи! Будь Суддею тих, що змагаються зі мною, і побори тих, що бажають побороти мене. Візьми зброю Твою і щит і стань на поміч мені. Вийми меч і загороди дорогу тим, що переслідують мене. Скажи душі моїй: «Я твоє спасіння». Нехай осоромляться ті, що шукають душу мою. Нехай повернуть назад і покриються безчестям ті, що задумали зло проти мене. Нехай стануть вони як той порох перед вітром, і ангел Господній нехай прожене їх. Нехай дорога їхня буде темна і слизька, і ангел Господній нехай переслідує їх. Бо вони без вини таємно поставили сітки свої на мене, викопали яму для душі моєї. Нехай же несподівано прийде загибель на нього. І сітка, що він таємно на мене поставив, нехай уловить його, і в ту яму нехай сам упаде, на свою погибель. А душа моя буде радіти у Господі, буде втішатися спасінням від Нього. Всі кості мої скажуть: «Господи, хто подібний до Тебе? Ти визволяєш слабкого від сильного, бідного і вбогого — від гнобителів його». Стали проти мене свідки неправедні. Про те, чого не знаю я, вони допитують мене. Віддають мені злом за добро, осиротили душу мою. Я ж, коли хворіли вони, одягався в одежу скорботи, смиряв постом душу мою, і молитва моя від серця мого не спинялася. Наче за приятелем або за братом своїм я побивався, ходив, сумуючи та плачучи, наче за матір’ю. А вони з нещастя мого радіють, збираються та радяться, як збільшити рани мої, ганьблять мене, не знаю, за що; спокушаючи, насміхаються, скрегочуть зубами. Господи, чи довго будеш дивитися на це? Одведи душу мою від злочинства їхнього, від левів — самотню мою. Я буду прославляти Тебе на великих зібраннях, серед численних народів буду хвалити Тебе. Щоб не торжествували наді мною ті, що ворогують проти мене даремно, не переморгувалися очима своїми ті, що ненавидять мене без вини. Бо не про мир говорять вони, а проти мирних землі складають свої замисли. Розширюють на мене уста свої і кажуть: «Добре, добре, бачили очі наші». Ти бачив це, Господи, не мовчи. Господи, не віддаляйся від мене. Встань, Господи, і заступись за правоту мою. Розбери тяжбу мою, Господи мій і Боже мій! Розсуди мене, Господи, за правдою Твоєю, щоб не величалися вони наді мною. Щоб не говорили в серці своїм: «Добре, добре, цього ми бажали». Щоб не сказали: «Ми поглинули його». Нехай постидяться й осоромляться ті, що радіють з мого нещастя. Нехай покриються соромом і ганьбою ті, що вихваляються проти мене. І нехай радуються і веселяться ті, що співчувають правді моїй. І нехай завжди кажуть: «Який величний Господь, що бажає миру рабові Своєму!» А язик мій буде звіщати правду Твою та прославляти Тебе по всі дні.

\section{Псалом 102}
Благослови, душе моя, Господа і вся істото моя — ім’я святеє Його. Благослови, душе моя, Господа і не забувай усіх добродійств Його. Він очищає всі беззаконня твої, зціляє всі недуги твої. Він звільняє від тління життя твоє, вінчає тебе милістю і щедротами. Він виконує благі бажання твої: оновиться, подібно орляті, юність твоя. Господь творить справедливість і суд усім покривдженим. Показав путі Свої Мойсеєві, синам Ізраїлевим — хотіння Свої. Щедрий і милостивий Господь, довготерпеливий і многомилостивий. Не до кінця прогнівається і повік не ворогуватиме. Не за беззаконнями нашими вчинив нам і не за гріхами нашими воздав нам. Бо як високо небо над землею, так утвердив Господь милість свою над тими, що бояться Його. Як далеко схід від заходу, так віддалив Він від нас беззаконня наші. Як отець милує дітей, так милує Господь тих, що бояться Його. Бо Він знає сутність нашу, пам’ятає, що ми — порох землі. Людина — як трава, дні її — немов цвіт польовий цвіте й відцвітає. Повіє вітер над нею, і не стане її: не знайти вже й місця по ній. Милість же Господня від віку й до віку на тих, що бояться Його. І правда Його на синах синів, що бережуть завіти Його і пам’ятають заповіді Його, щоб виконувати їх. Господь на небесах уготував Престол Свій, і Царство Його усім володіє, благословіть Господа всі ангели, сильні міцністю, що виконуєте слово Його, слухаючи голосу слів Його, благословіть Господа всі Сили Його, слуги Його, що творите волю Його. Благословіть Господа всі діла Його. На всіх місцях володіння Його благослови, душе моя, Господа!

\section{Псалом 129}
Коли не Господь будує дім, даремно трудяться будівничі; коли не Господь береже місто, даремно пильнує сторожа. Даремно встаєте ви рано і лягаєте пізно, їсте хліб тяжко здобутий, тоді як Господь дає улюбленим Своїм спокійний сон. Ось діти — насліддя від Господа; нагорода від Нього — плід утроби. Як стріли в руках сильного, так і сини молоді. Блаженний, хто здобув таку поміч собі. Не осоромляться вони, коли біля брами міста свого говоритимуть з ворогами.

\section{Псалом 131}
Пом’яни, Господи, Давида і всю лагідність його. Як він присягнувся Господу, давав обітниці Богу Якова: «Не ввійду в світлицю дому мого, не ляжу на постіль мою, не дам заснути очам моїм і задрімати повікам моїм; не заспокоюся, поки не знайду оселі Господа, дому для Бога Якова». Ось ми чули про Нього в Єфрафі, знайшли Його на полях Нарима. Ходімо в оселі Його, поклонімося підніжжю ніг Його. Воскресни, Господи, у спокій Твій, Ти і кивот святині Твоєї. Священики Твої зодягнуться в правду, і преподобні Твої зрадіють. Ради Давида, раба Твого, не відверни лиця від помазаника Твого. Клявся Господь Давидові істиною і не зречеться її: «Із синів роду твого посаджу на престолі твоїм. А коли сини твої зберігатимуть завіт Мій і повеління Мої, яких Я навчу їх, то й сини їхні навіки сидітимуть на престолі твоїм». Вибрав Господь Сіон, забажав Його на оселю для Себе. «Це місце спокою Мого повік віку, — сказав Господь. — Тут оселюся, бо Я полюбив його. Поживу його Я, благословляючи, благословлю, і вбогих його Я нагодую хлібом. Священиків його Я зодягну в спасіння, і праведні його будуть радіти радістю. Там Я вирощу спасіння Давидові, поставлю світильник помазаникові Моєму. Ворогів його осоромлю Я, на ньому ж розцвіте святиня Моя».

\section{Псалом 138}
Господи, Ти випробував мене і знаєш мене. Ти знаєш, коли я сяду і коли я встану. Ти наперед знаєш думки мої. Стежку мою і місце перебування мого Ти визначив і всі путі мої відомі Тобі. Ще нема слова на язиці моїм, а вже Ти, Господи, все знаєш. Ти ствердив мене і поклав на мене руку Твою. Дивне для мене всевідання Твоє; високе воно для мене, і я не можу збагнути його. Куди піду я від Духа Твого і від лиця Твого куди втечу? Зійду на небо — Ти там перебуваєш; зійду в пекло — і там Ти. Чи візьму крила в ранньої зорі і переселюся на самий край моря — і там рука Твоя поведе мене, і правиця Твоя триматиме мене. Сказав би я: може темрява сховає мене, то темрява стане світлою перед Тобою. Не сховає від Тебе й темрява, бо й ніч перед Тобою, як день; і пітьма перед Тобою як світло. Ти створив усе нутро моє, витворив мене в утробі матері моєї. Прославляю Тебе за те, що Ти так дивно створив мене. Дивні діла Твої, і душа моя добре це знає. Не були втаєні від Тебе кості мої, коли в тайні зачався я, коли в утробі творилося тіло моє. Очі Твої бачили зародок мій, і в книзі Твоїй були записані всі дні, призначені для мене, коли ще й одного з них не було. О, які величні для мене замисли Твої, Боже! І яка велика кількість їх! Став би лічити їх, та їх більше, ніж піску; коли я пробуджуюсь, я все ще перед Тобою. О, коли б Ти, Боже, знищив нечестивого! Відійдіть від мене, кровожерні! Вороги Твої, Господи, говорять проти Тебе зневажливо; марне замишляють вороги Твої. Чи ж мені не мати ненависті проти тих, що Тебе, Господи, ненавидять? І чи не цуратися тих, що повстають проти Тебе? Повною ненавистю ненавиджу їх; ворогами моїми стали вони. Випробуй мене, Боже, і побач серце моє, досліди думки мої. Подивись, чи не на шляху я беззаконня і чи на добрій я дорозі? Якщо ні, то настанови мене на путь вічну.
\end{document}