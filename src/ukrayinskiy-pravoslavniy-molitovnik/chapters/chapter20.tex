\documentclass[chapters.tex]{subfiles}

\begin{document}
\chapter{Благословення різних предметів}
\section{Молитва на освячення всякої речі}
Творче і Будівничий людського роду, Дарувателю духовної Ласки, подателю вічного Спасіння! Сам, Господи, пішли з висоти Духа Твого Святого і благослови цю річ (назва речі), щоб вона, озброєна силою Твого Небесного заступництва, була для всіх, хто бажатиме вживати її, корисною для спасіння і сприятливою та допоміжною на все добре в ім’я Христа, Господа нашого. Амінь.

\section{Молитва на благословення будь-яких предметів}
Всемогутній, Вічний Боже! Всяке Твоє створіння є добрим, і немає нічого, що саме по собі було б поганим, оскільки ми приймаємо його з подякою Твоєї руки. Бо все освячує Твоє слово і наша молитва. Славимо Тебе, подяку Тобі складаємо, бо великою є Твоя слава, сила й доброта. Поблагослови цей предмет (називаємо предмет), щоб усі, хто згідно з Твоєю Волею буде ним користуватися, зростали у вірі, а від Тебе отримували допомогу й опіку, бо просимо цього через Ісуса Христа, Господа нашого. Амінь.

\section{Молитва на освячення пам’ятної дошки}
Господи, Всемогутній Боже, освячуємо цю пам’ятну дошку на честь (з нагоди). Хочемо цією пам’ятною дошкою нагадати про заслуги наших померлих. Нехай приклад їхнього життя заохотить нас краще шанувати Господа Бога і сумлінніше виконувати наші обов’язки перед Богом, Церквою, Вітчизною, через Христа, Господа нашого. Амінь.

\section{Молитва на освячення прапора}
Благословенний будь, Господи Боже наш! Ти через Твого Сина, Ісуса Христа, покликав до життя Твою Святу Церкву як знамено народам, щоб вона позбирала розсіяних Божих дітей разом. Поблагослови цей прапор, щоб ті, хто навколо нього зберуться і кого він вестиме за собою, через заступництво Пресвятої Діви Марії і всіх Святих, завжди визнавали Христа, і щоб ніхто з них, не загинув, але щоб усі дійшли до вічного щастя, через Христа, Господа нашого. Амінь.

\section{Молитва на освячення нової криниці}
Господи і Творче Всесвіту! Спочатку Ти створив Небо і землю і води, які були розпорошеними по піднебессі і Дух Божий словом Своїм зібрав їх разом, щоб утворити джерела по всій землі. Ти, Боже, утворив з них струмки і ріки, щоб на Землі появилося життя, щоб водою могли користуватися рослини і тварини. Ти в пустині гірку, шкідливу воду, деревом перетворив на здорову, через посередництво Мойсея, а сіллю оздоровив воду, через посередництво Єлисея. Тому і ми сьогодні ласкаво просимо Тебе: поглянь, Господи, на недостойних слуг Твоїх, змилосердися над ними і пішли нам благословення Своє на цю криницю і воду в ній. Міцною правицею Своєю віджени від неї всілякі ворожі підступи, пакості злих людей і вплив нечистих духів. Нехай води в цій криниці завжди буде в достатку, а всі ті, хто буде споживати її для їжі чи на інші потреби, отримає добре здоров’я, силу проти різних недугів та спокус. Бо Ти, Господи, все благословляєш і освячуєш, і Тобі ми славу возсилаємо Отцю, і Сину, і Святому Духові, сьогодні і повсякчас, і на віки вічні. Амінь.

\section{Молитва на благословення рибальських сіток}
Господи Ісусе Христе, Боже наш, Ти п’ятьма хлібами і двома рибами п’ять тисяч присутніх людей нагодував і ще повелів зібрати залишки після трапези, яких залишилося багато. Сам, Владико Всесильний, і ці рибальські сітки благослови, через заступництво і молитви Преблагословенної, Славної Владичиці нашої Богородиці і Вседіви Марії і святого славного і всехвального першоапостола Петра, бо через них люди споживали хліб і рибу. В мирі і здоров’ї душі і тіла рибаків збережи, які будуть трудитися разом зі своїми сітками. Бо Ти подаєш всім нам блага, якими ми користуємося, тому славимо Тебе з Безначальним Твоїм Отцем, і Пресвятим, і Благим і Животворним Твоїм Духом, сьогодні і повсякчас, і на віки вічні. Амінь.

\section{Молитва на благословення нової пасіки}
Боже Предвічний, Всемогутній! Ти є Владикою Неба і Землі і управителем всякого створіння. Ти обіцяв вибраному народові Землю Обіцяну, де ріками тече мед і молоко, Ти в пустині Івана Хрестителя, Твого Предтечу, харчуватися медом від диких бджіл навчив. Так само і сьогодні, Владико, заопікуйся і нашим харчуванням, поблагослови цю пасіку зі всіма вуликами і бджолиними сім’ями. Зішли Свою Ласку на них, і зроби так, щоб ці вулики завжди були наповнені стільниками з медом. Захисти їх від всякого лукавства, заздрощів, ворожбитства і злих духів. Бо Ти маєш владу милувати і спасати, Христе Боже наш, і Тобі ми славу возсилаємо, з Отцем, і Святим Духом, сьогодні, і повсякчас, і на віки вічні. Амінь.

\section{Молитва на благословення меду}
Всемогутній, Вічний Боже, благословенним є все те, що Ти створив, і немає нічого поганого в тому всьому, що ми вдячно приймаємо з Твоїх рук, бо Твоє Слово й молитва Церкви все освячують.

Ми любимо Тебе, славимо Тебе, поклоняємося Тобі, подяку складаємо Тобі, бо великою є слава, сила і доброта Твоя. Через заступництво святого Климента, якого Церква визнає покровителем бджільництва, благослови цей мед. Щоб усі, хто з Твоєї Волі його споживатиме, зростали у вірі та отримували Твою допомогу й підтримку, через Ісуса Христа, Господа і Спасителя нашого. Амінь.

\section{Молитва на благословення хліба}
Владико Святий, Отче Вседержителю, Предвічний Боже! Благоволи освятити хліб цей Твоїм святим Духовним благословенням, щоб він був усім, хто споживає його, на спасіння душі, на здоров’я тілесне і захист проти недугів та усякого ворожого підступу — через Господа нашого Ісуса Христа, який з Неба зійшов, дає життя й спасіння світові, з Тобою живе і Царює в єдності Святого Духа сьогодні, і повсякчас, і на віки вічні. Амінь.

\section{Молитва на благословення вина}
Благословенний Ти, Господи, Боже Всесвіту, бо з Твоєї щедрості ми отримали вино, плід виноградної лози й праці людських рук. Твій Єдинородний Син, наш Господь, Ісус Христос, установив вино знаком Нового Заповіту, через перетворення його в Свою Кров, і споживання нами під час Причастя. Вчини так, щоб, як святому Іванові не зашкодило отруєне вино, так і всім тим, хто сьогодні споживає освячене вино, не зашкодили небезпеки та хвороби для душі й тіла.

Вчини також так, щоб ми любили один одного й завжди жили в мирі та злагоді, через Ісуса Христа, Господа нашого. Амінь.

\section{Молитва на освячення солі}
Боже Спасителю наш, який з’явився через пророка Єлисея в Ієрихоні і там за допомогою солі зробив здоровою шкідливу воду! Ти наказав пророкові Іллі вкинути солі до води, щоб вона знову стала здоровою. Всемогутній Боже! Просимо Тебе, Сам поблагослови цю сіль і зроби її приношенням радості, бо Ти є Бог наш, і Тобі славу возсилаємо, Отцю, і Сину, і Святому Духові, сьогодні, і повсякчас, і на віки вічні. Амінь.

\section{Молитва на благословення заводу (фабрики, майстерні)}
Господи, Боже Вседержителю, Ти створив Ангелів і людей для Слави і служби Своєї, щоб вони допомагали Тобі управляти іншими створіннями. Ти помістив першу людину в розкішному раю, щоб доглядала його, а коли вона згрішила, Ти її з нього прогнав і наклав на неї працю, як кару за гріх, кажучи: «У поті лиця свого їстимеш хліб твій» (Бут. 3,19). Ти, Господи, звелів Ноєві спорудити ковчег з соснового дерева, щоб урятувати від потопу людський рід, тварин, звірів і птахів. За царя Соломона Ти прийняв працю робітників, які будували єрусалимський храм для звеличення і прославлення пресвятого Імені Твого (1 Цар. 5,27-32). Ти поблагословив його кажучи: «Я освятив цей храм, що ти збудував, і оселив там на віки Моє Ім’я» (1 Цар. 9,3). — Отче Небесний, Ти для добра людей установив усілякі ремесла і праці, щоб одні другим допомагали у виконуванні Пресвятої Твоєї Волі і збагачували себе та при цьому шукали дорогу до Царства Небесного. Прийди сьогодні і поблагослови цей завод (фабрику, майстерню), щоб він був джерелом праці на Твою Славу, на добро душевне і тілесне для всіх тих, що будуть тут працювати. Охороняй його від усіляких диявольських затій і лиха. А Святим Своїм Ангелам повели, щоб тут перебували, охороняли й провадили всіх дорогою чеснот. Бо Ти є Бог Милості і Податель всілякого добра, і ми Тобі Славу віддаємо: Отцю, і Сину, і Святому Духові, сьогодні і повсякчас, і на віки вічні. Амінь.
\end{document}