\documentclass[chapters.tex]{subfiles}

\begin{document}
\chapter{Молитви до Господа}
\section{Молитва до Пресвятої Тройці}
Пресвята Тройце, єдиносущна державо, всіх благ причино, яку подяку принесемо Тобі за все, що Ти сотворила для нас, грішних і недостойних, раніше ніж ми народилися на світ; за все, що Ти подаєш кожному із нас по всі дні і що Ти приготувала нам у грядущому віці. Подобає бо за такі благодіяння і щедроти дякувати Тобі не тільки словами, але більше ділами і виконанням заповідей Твоїх; ми ж, підкорившись нашим пристрастям і злим звичаям, від юності впадаємо в незліченні гріхи і беззаконня. Тому, як нечистим і оскверненим, не тільки перед Твоє пресвітле лице соромно явитися, але і промовляти Твоє трисвяте ім’я не подобає нам, якби Ти Сама не благозволила, для нашої утіхи, сповістити, що Ти любиш чистих і праведних, а грішників, що каються, милуєш і благосердно приймаєш. Зглянься, о Пребожественна Тройце, з висоти святої слави Твоєї на нас, багатогрішних, і благий намір наш, замість добрих діл, прийми, і подай нам дух істинного покаяння, щоб, зненавидівши всякий гріх, пожили ми в чистоті і правді до кінця наших днів, виконуючи пресвяту волю Твою і прославляючи чистими помислами найсолодше і величне ім’я Твоє. Амінь.

\section{Молитва до Святої Живоначальної Тройці}
Всесвята Тройце, Боже і Творче всього світу! Підкріпи і направ серце моє почати з розумінням і скінчити добрими ділами богонадхненні оці книги, що їх Дух Святий висловив устами Давида, бо я хочу ними Тебе славити. Але, знаючи неміч мою, я припадаю до Тебе й прошу: Господи! Розкрий розум мій і укріпи серце моє, щоб я не втомлявся, але відчував насолоду від святих пісень та щоб навчився добрих діл словами їх. Благозволь, Господи, ними просвітити мене, щоб я став спільником Святих Твоїх. І ось тепер, Владико, благослови від усього серця співати Тобі:

Прийдіть, поклонімось Цареві нашому, Богу.

Прийдіть, поклонімось і припадімо до Христа, Царя і Бога нашого.

Прийдіть, поклонімось і припадімо до Самого Христа, Царя і Бога нашого.

\section{Молитва до Господа Ісуса Христа}
Прещедрий і премилосердний Господи, Спасителю наш, Ти просвітив сяйвом Твоїм краї світу і покликав нас до єднання в Церкві Твоїй святій, обіцяючи спадок нетлінних і вічних благ. Зглянься милостиво на нас, недостойних рабів Твоїх, і не згадуй беззаконня наші, але з невичерпного благосердя Твого прости нам усі провини наші. Бо хоч переступаємо ми святу волю Твою, але не відходимо від Тебе, Бога і Спаса нашого. Перед Тобою ми згрішаємо, але Тобі єдиному служимо, в Тебе єдиного віруємо, до Тебе єдиного вдаємося, тільки Твоїми рабами бути бажаємо. Ти знаєш неміч нашу і підступи, якими звідусіль ворог оточує нас, і приваби та спокуси світу цього, через які без Тебе, як казав Ти, не можемо нічого творити. Сам, Господи, очисти нас і спаси нас. Сам просвіти наш розум, щоб непохитно вірували Тобі, єдиному Спасу і Визволителю нашому. Сам зворуши серця наші, щоб любити Тебе, Бога і Творця нашого. Сам направ ступні наші, щоб не спотикаючись ходили при світлі заповідей Твоїх. Покажи нам, Господи, велику милість Твою, щоб, догоджаючи Тобі у житті нашому, сподобились і в Царстві Небесному славити Тебе з Предвічним Отцем і Пресвятим Духом навіки-віків. Амінь.

\section{Молитва до Господа Ісуса розіп’ятого}
На Хресті розіп’ятий за нас, Ісусе Христе, Єдинородний Бога Отця Сину, милості, любові і щедрот невичерпне джерело! Знаю, що ради моїх гріхів від невимовного чоловіколюбства Кров Свою пролити на хресті зволив Ти, яку я, окаянний і невдячний, дотепер скверними моїми ділами зневажав і ні в що ставив, тому із глибини беззаконня і нечистоти моєї очима серця на Тебе, розіп’ятого на Хресті Визволителя нашого, поглянувши, зі смиренням і вірою в глибину ран, Твого милосердя сповнених, себе віддаю, гріхів прощення і скверного життя мого виправлення благаю. Милостивим будь до мене, Владико і Суддя мій, не відкинь мене від лиця Твого, але всесильною Твоєю рукою Сам мене з Себе наверни і на шлях істинного покаяння направ, щоб віднині поклав спасіння мого початок. Божественними стражданнями Твоїми втихомир мої плотські пристрасті; пролитою Твоєю Кров’ю очисти мої душевні скверни; розп’яттям Твоїм розіпни мене для світу цього з його спокусами та похотями; хрестом Твоїм охорони мене від невидимих ворогів, що шукають душу мою; проколеними ногами Твоїми від усякої дороги лукавої збережи ноги наші, проколеними руками Твоїми руки мої від усякого невгодного Тобі діла утримай; розіп’ятою плоттю розіпни страхові Твоєму плоть мою, щоб, ухилившись від зла, творив благо перед Тобою. Ти, що голову на хресті схилив, схили до смирення вознесену мою гординю. Вінцем Твоїм терновим охорони вуха мої, щоб вони не слухали непотребного. Ти, що жовч устами спожив, постав охорону нечистим устам нашим. Проколене списом маючи серце, серце чисте сотвори в мені; всіма Твоїми ранами всього мене любов’ю Твоєю порань, щоб Тебе, Господа мого, полюбити всією душею, всім серцем, усією силою і всіма помислами. Дай мені Себе подорожнього, що не маєш де голови прихилити, дай мені Себе всеблагого, що визволяєш душу мою від смерті, дай мені Себе найсолодшого, що насолоджуєш мене в скорботах і напастях Своєю любов’ю. Кого спочатку ненавидів, прогнівляв, від себе відганяв і на хресті розпинав, Того нині полюблю, з радістю прийму і з насолодою хрест Його до кінця життя понесу. Не дай віднині, о всеблагий Визволителю мій, жодній моїй волі звершитися, бо вона зла і непотребна, щоб знову не впав я в рабство пануючого в мені гріха. Але Твоя воля блага, спасти мене бажаюча, нехай звершується в мені завжди. Їй же себе вручаючи, Тебе, розіп’ятого Господа мого, розумними очима серця мого уявляю і молюся із глибини душі, щоб і в час розлучення мого з тілом Тебе єдиного на хресті бачив, що в руки захисту Твого мене приймаєш і від піднебесних духів злоби оберігаєш, оселяєш же з грішниками, які покаянням Тобі благовгодили. Амінь.

\section{Молитва інша до Господа Ісуса розіп’ятого}
Господи Ісусе Христе, Сину Бога Живого, Творче неба і землі, Спасителю світу, це я, недостойний і більш за всіх грішний, смиренно коліна серця мого перед славою величі Твоєї схиливши, оспівую хрест і страждання Твої і подяку Тобі, Царю всіх і Богу, приношу; бо зволив Ти всі труди і всілякі біди, напасті і муки як людина перетерпіти, щоб усім нам у всяких печалях, нужді і озлобленнях помічником і спасителем бути. Знаю, всесильний Владико, що все це Тобі не було потрібно, але заради людського спасіння, щоб усіх нас викупити від лютого рабства ворожого, хрест і страждання перетерпів Ти. Чим віддячу Тобі, Чоловіколюбче, за все, що зробив Ти для мене грішного? Не знаю. Душа, тіло і все добре від Тебе є, і все моє Твоє, і я Твій. Тому, на безмірне Твоє благоутробне милосердя, Господи, надіючись, оспівую Твоє невимовне довготерпіння, величаю незбагненне приниження, славлю Твою безмірну милість, поклоняюся пречистим страстям Твоїм і з любов’ю, цілуючи рани Твої, взиваю: помилуй мене грішного і зроби, щоб не безплідним був у мені хрест Твій святий, щоб, причащаючись із вірою стражданням Твоїм, сподобився я бачити і славу Царства Твого на небі. Амінь.

\section{Молитва до Господа Ісуса Христа людини, що кається}
\emph{(з акафіста до святого Причастя)}

Господи, згрішили ми на небо і перед Тобою! Прийми навернення і покаяння наше, прийми наші стогони і сльози, прийми покаяння нас грішних, прийми ридання і волання наше аж до смерті. Прийми нас, окаянних, що живемо безсоромно, прийми нас, Чоловіколюбче, що дуже прогнівали Тебе; прийми нас, Владико, що провели все життя в розпусті, у всякому лукавстві і нечистоті; прийми нас, Господи Боже, що переступили заповіді Твої, і не воздай нам за діла рук наших. Сповідуємо Тобі, Владико, що за гріхи наші ми недостойні навіть подивитися на це сонце, тому що немає такого гріха, немає жодного такого злодіяння, якого б ми не вчинили, окаянні, але прийми нас, Господи, як блудного сина, прийми як розбійника, прийми як блудницю і митаря.

Господи, наверни нас! Господи, надоум нас і не прогнівайся на нас, але будь милостивим до гріхів наших! Бо Ти є Бог наш, і іншого крім Тебе ми не знаємо. Господи, вирви нас від ворогів наших і не ввійди в суд з рабами Твоїми. Господи, Ти є Бог наш, і ми люди Твої. Згрішили ми, вчинили беззаконня і неправду, сотворили зло, згрішили у всьому, не послухали заповідей Твоїх і тому далеко ухилилися від Тебе. Але помилуй діла рук Твоїх; помилуй, Владико, тих, що хитрістю диявола були вигнані з раю; помилуй нас і одягни нас в одежу радості і спасіння; помилуй нас, кого диявол позбавив Твоєї допомоги! Милосердний, помилуй тих, хто залишив Тебе і служив йому; помилуй заблудлих; помилуй тих, хто не зберіг заповідей Твоїх і пішов за злочинністю демонів; помилуй тих, що осквернили себе гріхами; помилуй тих, що перебувають під владою диявола; помилуй тих, що осквернили себе розпустою. Помилуй, Милосердний; помилуй, Благий; помилуй, Довготерпеливий, бо руки наші вчинили усяке зло, всяке осквернення, користолюбство і неправду. Осквернили душу, яку Ти створив за образом Твоїм, осквернили тіло, осквернили почуття. Язик наш був гострим, як меч, спрямований проти ближнього, очі наші випромінюють полум’я, руки наші повні крові, ноги наші швидкі на звершення зла, вуста наші осквернені злослів’ям, і, коротко кажучи, осквернили ми землю і повітря. Воістину неправедні діла наші дійшли навіть до небес; грабіжництво наше перевищило гори, користолюбство наше перевищило хмари, пороки наші досягли до неба, гріхи наші не підлягають прощенню, злочини наші не мають виправдання, гріхи наші непоправні, і ось земля не може нести наших злочинів. Тому, Господи, потребу маємо у Твоєму милосерді. Ти знаєш неміч єства нашого; помилуй, Господи, творіння рук Твоїх. Ось, взиваємо до милосердя Твого: не позбав нас Твоєї допомоги, пошли нам недостойним милість Твою, даруй нам грішним заступництво Твоє, яви нам лице Твоє, і спасемося. Амінь.

\section{Молитва до Господа перша}
\emph{(з акафіста всемогутньому Богові в нашесті печалі)}

Господи, спаси мене, що гину! Ось корабель мій терпить біду від хвиль житейських, і близьке потоплення моє; але Ти, як Бог милосердний і співстраждальний немочам нашим, владою Твоєю всесильною втихомир стихію, що хоче занурити мене в глибину зла; і нехай буде тиша, бо море і вітри слухають Тебе. Амінь.

\section{Молитва до Господа друга}
Спаси мене, Спасе мій, з милості Твоєї, а не за діла мої. Ти хочеш спасти мене. Ти знаєш, якими путями спасти мене. Спаси мене, як Ти хочеш, як можеш, як знаєш. Як Той, що знає долю всіх людей, спаси мене. Я на Тебе, Господа мого, надіюся, і Твоїй волі святій себе вручаю. Твори зі мною, що бажаєш: якщо хочеш, щоб я був у світлі, будь благословенний; якщо хочеш, щоб я був у темряві, будь знову благословенний. Якщо відкриєш мені милосердя Твого двері, — добро це і благо; якщо закриєш для мене двері милосердя Твого, то благословенний Ти, Господи, що вчинив зі мною по правді. Якщо не погубиш мене з беззаконнями моїми, слава безмірному милосердю Твоєму; якщо погубиш мене з беззаконнями моїми, слава праведному судові Твоєму; і як бажаєш, так зі мною чини. Амінь.

\section{Молитва до Господа нашого Ісуса Христа}
Владико Господи Ісусе Христе Боже мій, Який з невимовного Свого чоловіколюбства в кінці віків тіло прийняв від Приснодіви Марії. Я, раб Твій, Владико, дякую за Твою спасенну до мене турботу. Оспівую Тебе, бо завдяки Тобі я Отця пізнав. Благословлю Тебе, заради Якого і Дух Святий від Отця у світ приходить. Поклоняюся перед Твоєю за плоттю Пречистою Матір’ю, яка стала причетною до такого великого таїнства. Вихваляю Твої Ангельські сили, які оспівують і служать Твоїй величі. Прославляю Предтечу Івана, який Тебе, Господи, хрестив. Вшановую пророків, які звіщали про Тебе, прославляю апостолів Твоїх святих, мучеників і священиків Твоїх славлю, вшановую всіх Твоїх преподобних та праведних. Так багато всіх незбагненно прославлених Твоїм Божеством на молитву до Тебе, всещедрого Бога, я, раб Твій, закликаю. Прошу, прости провини мої і, заради всіх і святих Твоїх, щедро наділи мене святими Твоїми щедротами. Бо Ти благословенний єси навіки. Амінь.

\section{Молитва до Господа}
\emph{(з Акафіста покаянного на основі Великого канону преподобного Андрія Критського)}

Господи, Господи, ось я стою перед Тобою, як блудний син, сповідуючи гріхи свої Твоїй благості. Згрішив перед Тобою, Спасителю мій, згрішив. Від юності моєї гріх вчинив я і осквернив душу і тіло лютими діяннями. Згрішив перед Тобою, як Адам і Єва, не послухав заповіді Твоєї, згрішив, як Каїн і як усі грішники Старого і Нового Завіту. Розум затьмарив помислами неподобними, серце наповнив нечистотою і всякими гріховними почуттями; ослабив волю лінивством і схильністю до гріха. Увесь я покритий ранами гріховними, як той, що попав до розбійників; усі рани душі моєї виточують гній. Хто зцілить мене, або хто підніме мене із глибини гріховного життя, якщо не Ти, Лікарю благодатний? Заради цього до Тебе вдаюся, Царю мій і Боже мій, зі сльозами молюся. Вилікуй і зціли мене; розум мій очисти від нечистих помислів, визволи серце від гріховної нечисті, щоб я зміг чистим серцем і розумом прославляти Тебе, Сотворителя мого і Бога. Волю ж укріпи на діла благі, щоб я творив тільки те, що угодно Тобі. Прости мені всі гріхи вільні і невільні, і все, що свідомо і несвідомо вчинене мною. Ти є весь благодать і любов, і маєш владу відпускати гріхи, і Тобі славу возсилаємо з Отцем і Святим Духом нині, і повсякчас, і навіки-віків. Амінь.

\section{Молитва до Святого Духа}
\emph{(з акафіста Святому і Животворчому Духові)}

Душе Святий, Ти весь всесвіт Собою наповнюєш і всім життя подаєш, від скверних же людей віддаляєшся, смиренно благаю Тебе: не погордуй нечистотою душі моєї, але прийди і вселися в мені і очисти мене від усякої гріховної скверни, щоб з Твоєю допомогою залишок життя мого я прожив у покаянні і творенні добрих діл, і так прославлю Тебе з Отцем і Сином навіки-віків. Амінь.

\section{Молитва покаяння}
Полегши, відпусти, прости, Боже, гріхи наші вільні й невільні, чи словом, чи ділом, свідомі чи несвідомі, вдень чи вночі, чи в думках, чи в помислах заподіяні — все нам прости, бо Ти Милосердний і Чоловіколюбець.

\section{Молитва покаянна до Господа нашого Ісуса Христа}
Владико Христе Боже, Ти, що Своїми стражданнями мої страждання зцілив і ранами Своїми мої рани вилікував. Даруй мені, який багато перед Тобою нагрішив, сльози покаяння; сподоби, щоб тіло моє було причасне до пахощів Животворчого Тіла Твого, і душу мою, що її супротивник гіркотою напоїв, насолоди Кров’ю Твоєю Чесною; розум мій, що донизу схилився, до Тебе піднеси, і від безодні погибелі відведи: бо не маємо покаяння, не маємо жалю за гріхи, не маємо сльози утіхи, яка дітей до Твого спадку приводить. Духовно уражений життєвими турботами не можу споглянути на Тебе у хворобі моїй, не можу зігрітися слізьми любові до Тебе. Але, Владико Господи Ісусе Христе, скарбе добра, даруй мені цілковите покаяння і старанність серця у пізнанні Тебе, даруй мені благодать Твою і віднови в мені ознаки Твого образу. Не залиши мене, що залишив Тебе, відгукнися на пошук мій, приведи мене до пасовища Твого і долучи мене до овець вибраного Твого стада, виховай мене разом з ними на посівах Божественних Твоїх Таїнств. Молитвами Пречистої Твоєї Матері й усіх святих Твоїх. Амінь.

\section{Молитва «Владико Вседержителю»}
Владико Вседержителю, лікарю душ і тіл, що упокорюєш і піднімаєш, кидаєш у хворість і знову зціляєш! Навісти милосердям Твоїм хворого брата нашого \emph{(сестру нашу — ім’я)}; простягни всемогутню руку Твою, повну зцілення і лікування, уздоров його \emph{(її)}, підійми з ложа і немочі, заборони духові хвороби, відверни від нього \emph{(неї)} всяку язву, всяку недугу, всяке страждання, всяку огневицю і трясовицю. А як є на ньому \emph{(ній)} провини або беззаконня, — полегши, залиши, прости ради Твого чоловіколюбства.

Так, Господи, змилосердься над створінням Твоїм в Ісусі Христі, Господі нашім, що з Ним благословен єси з Пресвятим і Благим і Животворчим Твоїм Духом нині, і повсякчас, і на віки віків. Амінь.

\section{Молитва «Кожного часу»}
Кожного часу і кожної години, на небі й на землі поклоняємий і славимий, Христе Боже, Довготерпеливий і Премилосердний, що праведників любиш і грішників милуєш, що всіх кличеш до спасіння обітницею прийдешніх благ! Прийми, Господи, в цю годину й наші молитви і направ життя наше до заповідей Твоїх. Душі наші освяти, тіла очисти, помисли направ, думки очисти, і визволи нас від усякої скорботи, біди і страждання. Оточи нас святими Твоїми Ангелами, щоб ми, бережені і проваджені ними, прийшли до єдності віри і до пізнання неприступної Твоєї слави, бо Ти благословен єси на віки віків. Амінь.

\section{Молитва до Господа свт. Філарета (Дроздова)}
Господи! Не знаю, чого просити в Тебе. Ти один знаєш, що мені потрібно. Ти любиш мене більше, ніж я вмію любити себе. Отче! Дай рабові Твоєму, чого сам я просити не вмію. Не насмілююся просити ані хреста, ані втіхи. Тільки стою перед Тобою, серце моє відкрите. Ти бачиш потреби, яких я не бачу. Поглянь і вчини зі мною за Твоєю милістю. Урази і зціли, скинь і підведи мене. Благоговію перед святою Твоєю волею і незбагненними для мене Твоїми шляхами. Приношу себе в жертву Тобі. Віддаю себе Тобі. Нема в мене іншого бажання, крім чинити волю Твою. Навчи мене молитися. Сам у мені молись. Амінь.
\end{document}