\documentclass[chapters.tex]{subfiles}

\begin{document}
\chapter{Молитви за недужих}
\section{Молитва за медичних працівників}
Господи, Ісусе Христе, Ти на землі перебував і всім чинив добро. Ти зцілював хворих, щоб зміцнити нашу віру. Вчини так, щоб кожен лікар своєю дбайливістю та щирістю давав хворим надію на одужання і кожен лікар пам’ятав, що його праця — це служіння Тобі. Обдаруй наших лікарів здоров’ям, терпеливістю, безкорисливістю та милосердям, щоб з їхньою допомогою хворі пізнавали Твою доброту. Господи, Ти похвалив милосердного самарянина за те, що він не обминув потерпілого. Вчини так, щоб усі медичні працівники з любов’ю турбувалися про хворих. Нехай у кожному недужому вони бачать Самого стражденного Христа. Обдаруй їх Своєю милістю, щоб вони вкладали душу у працю свою. Подай їм лагідне й добре слово, терпеливість і доброзичливість у труднощах з важкими пацієнтами, щоб за їхнє милосердя до хворих вони дістали й Твого милосердя. Амінь.

\section{Молитва про те, щоб доглядати з любов’ю за недужими}
Господи Ісусе Христе, Сину Бога Живого, Агнче Божий, що береш гріхи світу, Пастирю добрий, що поклав душу Свою за вівці Свої, Небесний Лікарю душ і тіл наших, що зціляєш усяку недугу та хворобу в людях Твоїх! Припадаю до Тебе, допоможи мені, недостойному слузі Твоєму. Споглянь, Многомилостивий, на обов’язок і служіння моє і дай мені бути вірним у малому: послужити хворим заради Тебе, носити немочі немічних, і не собі, а Тобі Єдиному догоджати в усі дні життя мого. Бо Ти сказав, Найсолодший Ісусе: що зробите одному з братів Моїх менших, те зробите Мені. Так, Господи, дай мені грішному за словом Твоїм, щоб я сподобився творити волю Твою благу на втіху й розраду хворих слуг Твоїх, яких Ти викупив Своєю чесною Кров’ю. Пошли мені благодать Твою, що спалює в мені терня пристрастей, бо Ти покликав мене, грішного, служити Тобі. Без Тебе ми не можемо творити нічого: відвідай серце моє, щоб бути мені завжди коло хворих і відкинутих; наповни мою душу любов’ю Твоєю, яка все терпить і ніколи не згасає. Тоді зможу, Тобою зміцнений, аж до останнього свого подиху добрим подвигом позвизатися та віру зберегти. Бо Ти — Джерело зцілення душ і тіл наших, Христе Боже наш, і Тобі славу, подяку та поклоніння возсилаємо нині, і повсякчас, і на віки вічні. Амінь.

\section{Перша молитва за недужого}
Ісусе Христе, Боже наш! Коли хвора жінка з великою вірою доторкнулася до краю одежі Твоєї, вона стала здоровою. За вірою її Ти дав їй те, чого вона хотіла. З такою ж вірою і я вдаюся до Тебе, Лікарю всіх, допоможи хворому слузі Твоєму (ім’я) в його недузі. Дай йому спокій душевний і силу тілесну, благослови його сном спокійним, а з ним і здоров’ям, і силою, та підійми його з постелі недуги. Милість Твоя, Господи, хай буде над нами, бо ми надіємось на Тебе. Амінь.

\section{Друга молитва за недужого}
Владико Вседержителю i Святий Царю, що всiх караєш, але не нищиш, змiцнюєш пiдупалих i пiдносиш повалених, печалi життя людського розвiюєш. Молимось Тобi, Боже наш, раба Твого хворого (ім’я) вiдвiдай милiстю Твоєю, прости йому всякий грiх вiльний чи невiльний. Так, Господи, Твою цiлющу силу з неба зiшли, торкнися тiла, погаси гарячку, вгамуй страждання i всяку таємну немiч. Будь лiкарем рабовi Твоєму, пiднiми його з постелi недуги i вiд ложа болю цiлим i здоровим, дай йому бути Церквi Твоїй послушним i волi Твоєї виконавцем. Бо Тобi належить милувати i спасати нас, Боже наш, i Тобi славу возсилаємо, Отцю, i Сину, i Святому Духовi, нинi i повсякчас, i на вiки вiкiв. Амiнь.

\section{Третя молитва за недужого}
О премилосердний Боже, Отче, Сине i Святий Душе, що Тобi в Нероздiльнiй Тройцi поклоняємось i Тебе славимо! Зглянься милостиво на раба Твого (ім’я) недужого, вiдпусти йому всi провини його, подай йому зцiлення вiд хвороби, поверни йому здоров’я i сили тiлеснi, подай йому довге й щасливе життя, земнi й духовнi Твої блага, щоб вiн разом з нами приносив подячнi молитви Тобi, всещедрому Богу i Творцевi нашому.

Пресвята Богородице, всесильним заступництвом Твоїм допоможи менi ублагати Сина Твого, Бога нашого, за зцiлення раба Божого (ім’я).

Усi святi й ангели Господнi, молiте Бога за недужого раба Його (ім’я). Амiнь.

\section{Четверта молитва за недужого}
Лiкарю душ i тiл наших, Джерело життя нашого, Христе Iсусе, Господи i Спасителю наш! Споглянь милосердним оком на раба Твого в тяжкiй недузi, прихились до наших слiзних благань, перстом милосердя Твого доторкнись до немiчного тiла хворого брата нашого, вгаси вогонь тiла його, зменши бiль недуги його, поверни здоров’я йому, пiднеси його з постелi немочi, продовж днi життя його, щоб на землi вiн послужив Тобi, ходив по стежках заповiдей Твоїх та удостоївся Твого Небесного Царства, i разом зi святими славив Тебе, Бога Милостивого, з Отцем i Духом на вiки вiкiв. Амiнь.

\section{Перша молитва під час недуги}
Господи Боже, Владико життя мого! Ти з милості Своєї сказав: «Не хочу Я смерти грішника, але щоб він навернувся і живим був». Я знаю, що ця хвороба, яку я терплю, — це спокута за мої гріхи та беззаконня, знаю, що ділами своїми я заслужив значно більшої кари, але, Чоловіколюбче, вчини зі мною не за моїми гріхами, а за безмежним Милосердям Своїм. Дай мені сили, щоб я терпляче й мужньо переносив хворобу, а після одужання навернувся всім серцем, усією душею та всіма почуттями до Тебе, Господа Бога і Творця мого, та жив за святими Твоїми заповідями задля блага ближніх і задля свого спасіння. Амінь.

\section{Друга молитва під час недуги}
Господи Ісусе, Сину Божий, милосердний Лікарю душ і тіл наших, зглянься на мене, негідного, немічного та грішного раба Твого. Ти уздоровлював калік і немічних за вірою їхньою в Божество Твоє, кажучи: «За вірою вашою нехай буде вам». Я знаю, Господи, що віра моя слабка, але благаю Тебе, Джерело життя, і взиваю: «Вірую, Господи, допоможи моєму невірству!». Ти, Господи, приймав малу віру сліпих, калік і немічних і давав їм зцілення. Не відкинь, Милосердний, і мене немічного. Я знаю, я грішний і негідний милости Твоєї, але Ти Сам сказав, що прийшов для грішників, і це дає мені надію звертатися до Тебе, Милосердний Господи. Піднеси мене, Спасителю мій, з постелі недуги моєї, продовж дні життя мого, щоб я міг у покаянні скінчити його та сподобитися вічного Твого Царства, де в сяйві слави лиця Твого вже не буде ані недуги, ні журби, ні зітхання. Помилуй мене, Господи, з великої милости Твоєї, зглянься наді мною немічним, дай мені силу славити Тебе, Єдиного і Милосердного, з Отцем, і Святим, і Животворчим Твоїм Духом, нині, і повсякчас, і на віки вічні. Амінь.

\section{Молитва над багатьма недужими}
Господи, Боже наш, Ти послав у світ Свого Сина, щоб він обтяжив Себе нашими стражданнями і поніс на Собі наші недуги, чинив добро та зціляв усіх. Смиренно молимо Тебе: поблагослови цих хворих людей, дай їм тілесну силу, зміцни дух і терплячість у хворобі. Верни їм здоров’я, щоб вони подолавши слабкість і підтримані Твоєю допомогою, втішилися здоров’ям і з радістю прославляли Тебе у громаді вірних в ім’я Отця, і Сина, і Святого Духа, завжди, нині, і повсякчас, і на віки вічні. Амінь.

\section{Молитва над одним недужим}
Господи, Святий Отче, Усемогутній вічний Боже, Ти Своїм благословенням підтримуєш і зміцнюєш нашу спотворену людську природу. Споглянь милостиво на Свого хворого слугу (ім’я) і подай йому перемогу над хворобою, щоб до нього вернулося здоров’я та щоб він, пам’ятаючи про Твоє милосердя, вдячно прославляв святе ім’я Твоє на віки вічні. Амінь.

\section{Молитва до Спасителя, що дає благодать зцілення від недуги пияцтва}
Господи, Спасителю наш, Ти перетерпів великі спокуси в пустелі від диявола, не дай нам стати його покірними рабами. Благодаттю Свою допоможи нам позбавитися ганебної пристрасті. Врозуми нас, щоб віднині всі дні нашого земного життя ми перебували тверезими. Спрямуй ноги наші, щоб без блукань перейти нам шлях, який Ти, Всеблагий, заповідав. Господи, Боже наш, Щедрий і Всемилостивий, Ти не бажаєш смерти грішника, але його навернення та успадкування вічного життя. Зглянься милостивим оком Твоїм на рабів Твоїх, одержимих недугою пияцтва. Ти відаєш нашу скорботу від цієї недуги, знаєш немічність природи нашої, бачиш силу великого й лукавого спокусника. Врозуми затьмарених розумом, вилікуй одержимих, утверди в поміркованості, зміцни у стриманості, щоб не впасти нам у безнадію. Нехай відродиться в нас турбота про спасіння душі та про здоров’я тіла на славу Твою, про добродійне життя на власну користь і на благо ближніх, щоб тверезим життям своїм прославити пресвяте ім’я Твоє навіки. Амінь.

\section{Молитва прав. Іоана Кронштадського за одержимих недугою пияцтва}
Господи, зглянься милостиво над рабом Твоїм (ім’я), обманутим підступністю черева та плотськими веселощами. Сподоби його (ім’я) пізнати насолоду стримання в пості та плодів його. Амінь.

\section{Молитва над хворими дітьми}
Господи Ісусе Христе, Ти приймав дітей і говорив, що їм належить Царство Небесне. Ти сказав, що тайни Царства Божого не відкрилися мудрим і розумним, а лише немовлятам. Ти, входячи в Єрусалим на осляті, з радістю прийняв славослів’я дітей, які співали Тобі: «Осанна!». Тому благослови всіх, хто старається вернути здоров’я цим хворим дітям і простягни ласкаво руку Свою над цими дітьми. Нехай Церква Твоя та батьки цих дітей зрадіють поверненню до них повноти сил та вдячним серцем прославляють Тебе на віки вічні. Амінь.

\section{Молитва від страху}
Всемогутній Боже! Час Твоєї Слави настав: змилосердься наді мною і позбав мене великого нещастя. На Тебе покладаю мої надії. Бо сам я немічний і грішний. Допоможи мені, Боже, і визволи мене від страху. Господи, помилуй мене. Амінь.

\section{Молитва від розсіяння}
Розсіяний розум мій збери, Господи, і застигле серце моє очисти, як апостолу Петрові дай покаяння, як митареві — молитву, як блудниці — сльози, щоб голосно взивати до Тебе: Боже, спаси мене, бо Ти Єдиний і Чоловіколюбний. Амінь.

\section{Читання Євангелiя за зцiлення хворого, під час операції, обстеження}
Спаси, Господи, i помилуй раба Твого (ім’я) словами Божественного Євангелiя Твого, прочитаного за спасiння раба Твого. Попали, Господи, терна всiх його грiхiв i нехай вселиться в нього благодать Твоя, що опалює, очищає й освячує всю людину навiки. Амiнь.

\emph{(I прочитати, стоячи, главу з Євангелiя. Пiд час операцiї чи важких обстежень читають глави одну по однiй, скiльки стане сил).}

\section{Молитва перед операцією}
Господи, Ісусе Христе, Боже наш, Ти терпляче переніс побиття, катування, страждання та біль від ран на пресвятому тілі Твоєму, лише для того, щоб спасти душі й тіла людей Твоїх. Зглянься сьогодні милостиво на стражденне тіло слуги Твого (ім’я) і дай йому сили перенести операцію, що на нього чекає. Благослови, Господи, лікарів та їхніх помічників і всі їхні засоби, які вони використовуватимуть, аби вернути йому здоров’я. І зроби так, щоб він прийняв тілесне терпіння, як необхідність задля спасіння своєї душі. Бо Ти, Христе Боже наш, Милосердний, і ми Тобі славу возсилаємо, з Предвічним Твоїм Отцем, і Пресвятим, і Благим, і Животворчим Твоїм Духом, нині, і повсякчас, і на віки вічні. Амінь.

\section{Молитва подяки після видужання}
Милосердний Господи, Ти заради нас, задля нашого спасіння, як людина, переніс страшні терпіння, наругу, смерть і похорон, а по воскресінні Твоєму Ти всім, хто з вірою до Тебе вдається, подаєш Своє безмежне милосердя, Свою допомогу в години смутку та немочі душевної й тілесної. Ти по всі дні життя нашого тримаєш над нами правицю Свою святу. Прийми ж, Милосердний Спасителю мій, Джерело життя мого, щиросердну і безмежну подяку мою за Твоє милосердя до мене, негідного та грішного раба Твого, бо Ти вернув мені здоров’я, зміцнив мої сили, підняв мене з постелі недуги та благословив далі ходити по цій грішній землі. Прийми, Сину Божий, оцей не лукавий, а щирий мій дар молитви й подяки Тобі за Твою милість до мене, і благаю Тебе: не покидай мене без опіки Твоєї, а керуй мною і життям моїм, щоб я жив заповідями Твоїми й чинив святу волю Твою, прославляв Тебе і сподобився досягти вічного і блаженного життя у Небеснім Царстві Твоїм. Благослови, щоб і там, і тут, на землі, я приносив поклоніння і славлення Тобі з Безпочатковим Твоїм Отцем і Пресвятим, і Животворчим Твоїм Духом на віки вічні. Амінь.

\end{document}