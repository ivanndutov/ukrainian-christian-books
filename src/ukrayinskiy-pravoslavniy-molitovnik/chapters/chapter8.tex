\documentclass[chapters.tex]{subfiles}

\begin{document}
\chapter{Молитви до ангелів та архангелів}
\section{Молитва до Ангела-Охоронителя}
\emph{(з акафіста святому Ангелу-Охоронителю)}

О, святий Ангеле, Охоронителю і покровителю мій благий! Сокрушеним серцем і болісною душею стою перед тобою, благаючи: почуй мене, грішного раба твого (ім’я), що з гучним воланням і гірким плачем взиваю; не пом’яни беззаконь і неправд, якими я окаянний прогнівляю тебе в усі дні і години, і творю себе мерзенним перед Сотворителем нашим Господом; будь до мене милосердним і не відлучайся від мене, скверного, навіть до кончини моєї; розбуди мене від сну гріховного і сприяй твоїми молитвами останок життя мого без пороку пройти і створити плоди, достойні покаяння; найбільше ж збережи мене від смертельних гріховних падінь, щоб я не загинув у відчаї і щоб не радувався ворог через мою загибель. Знаю воістину і вустами сповідую, що немає такого друга і предстоятеля, захисника і поборника, як ти, святий ангеле; стоячи перед престолом Господнім, молишся за мене негідного і більше за всіх грішного, щоб не взяв преблагий Господь душі моєї в день, несподіваний для мене, і в день творіння зла. Не переставай же умилостивляти премилосердного Господа і Бога мого, щоб Він простив мої гріхи, які я вчинив у всьому житті моєму ділом, словом і всіма моїми почуттями, і якими Він знає путями спас мене; нехай покарає мене тут з великого Свого милосердя, але нехай не викриває і не карає мене там за Своїм нелицемірним правосуддям; щоб Він сподобив мене принести покаяння, з покаянням же достойно прийняти божественне причастя — про це більше за все благаю і такого дару щиро бажаю. У страшну ж годину смерті будь невідступним під мене, благий охоронителю мій, проганяючи темних демонів, які мають устрашити тремтячу душу мою; захисти мене від їхнього уловлення, коли я маю проходити повітряні митарства, щоб я, охоронюваний тобою, безбідно досяг жаданого мені раю, де собори святих і величне ім’я в Тройці славленого Бога, Отця, і Сина, і Святого Духа, Йому ж подобає честь і поклоніння навіки-віків. Амінь.

\section{Інша молитва до Ангела-Охоронителя}
Ангеле Божий, охоронцю мій святий, даний мені з небес Богом для мого збереження! Щиро молюся до тебе: нині просвіти мене, від усякого зла охорони, всяке благо навчи мене чинити і направ на шлях спасіння. Амінь.

\section{Молитва до святого Архістратига Божого Михаїла}
Святий і великий Архангеле Божий Михаїле, несповідимої і пресущної Тройці перший в Ангелах предстоятелю, роду ж людського заступнику і охоронителю, що скрушив з воїнством своїм главу прегордого Денниці на небесах і посоромлюєш злобу і підступи його на землі! До тебе вдаємося з вірою і тобі молимося з любов’ю: будь щитом непорушним і огорожею твердою Святій Церкві і Православній Вітчизні нашій, захищаючи їх вогняним мечем твоїм від усіх ворогів видимих і невидимих. Не позбав же, о Архангеле Божий, допомоги і заступництва твого і нас, що прославляємо сьогодні святе ім’я твоє: ми ж хоч і багатогрішні є, але хочемо не в беззаконнях наших загинути, а навернутися до Господа і оживленими бути від Нього на діла благі. Осяй розум наш світлом лиця Божого, що завжди сяє на вогнеподібному чолі твоєму, щоб ми могли розуміти, що є воля Божа про нас блага і довершена, і бачити все, що нам подобає творити, а що ненавидіти і відкидати. Зміцни благодаттю Господньою слабку волю і немічну силу нашу, щоб, утвердившись у Законі Господньому, ми перестали перейматися земними помислами і похітливими думками подібно до нерозумних дітей, які скоро гинуть, насолоджуючись красою світу цього, і ради тлінного і земного безумства забувають про вічне і небесне. Над усім цим виблагай нам з висоти, Господній Архангеле, дух істинного покаяння, нелицемірну печаль у Бозі і скрушення за гріхи наші, щоб останок днів життя нашого ми прожили не в догоджанні почуттям і страстям гріховним, але у виправленні вчинених нами провин зі сльозами віри і сердечного покаяння, подвигами чистоти і святими ділами милосердя. А коли наблизиться час кончини життя нашого, не залиш нас, Архангеле Божий, беззахисних проти духів злоби піднебесних, які звикли перепони чинити душі людській, що сходить на небо. Щоб, захищені тобою, без перешкод досягли ми преславних осель райських, де немає ні скорботи, ні зітхання, але безкінечне життя, сподобились побачити пресвітле лице Всеблагого Господа і Владики нашого і, впавши зі слізьми біля Його ніг, у радості вигукнули: Слава Тобі, Найдорожчий Визволителю наш, що за превелику любов Твою до нас, недостойних, Ти благозволив послати нам Ангелів Твоїх для спасіння нашого. Амінь.

\section{Молитва до всіх святих і безплотних небесних сил}
Боже Святий, що у святих спочиваєш. Тебе трисвятим голосом ангели на небесах оспівують і люди на землі у святих Твоїх прославляють. Ти святим Твоїм Духом кожному благодать у міру дару Христового даєш і нею Святій Твоїй Церкві, призначивши одних апостолами, інших — пророками, інших — благовісниками, інших — пастирями і вчителями, їхніми вустами проповідував. Безліч із роду в рід удосконалених святих, які Тобі Самому в усьому та в усіх діючому Богові різними чеснотами вгодили, залишивши нам свій приклад добрих подвигів, до Тебе в радості перейшли і готові допомогти нам у напастях, що самі вони колись перетерпіли. Згадуючи всіх цих святих і вихваляючи їхнє богоугодне життя, Тебе Самого, що в них діяв, прославляю і, віруючи в те, що їхні благодіяння Твоїм даром були, щиро молю Тебе, Святий святих, сподоби мене, грішного, їхнє вчення, життя, любов, віру та довготерпіння наслідувати, і їхньою молитовною допомогою, а найбільше за Твоєю всемогутньою благодаттю, разом з ними небесної слави сподобитися, хвалячи пресвяте ім’я Твоє: Отця, і Сина, і Святого Духа навіки. Амінь.

\section{Небесним безплотним Силам}
\emph{Тропар}

Небесних воїнств архистратиги, завжди молимо вас ми, недостойні, щоб вашими молитвами оберігали нас покровом крил духовної слави, охороняючи нас, що старанно припадаємо до вас і взиваємо: «Від бід визволіть нас як чиноначальники Небесних Сил».

\emph{Кондак}

Архистратиги Божі, служителі слави Божественної, начальники ангелів і людей наставники, корисного для нас просіть і великої милості як безплотних архистратиги.
\end{document}