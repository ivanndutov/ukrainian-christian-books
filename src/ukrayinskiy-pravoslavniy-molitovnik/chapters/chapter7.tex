\documentclass[chapters.tex]{subfiles}

\begin{document}
\chapter{Молитви до Пресвятої Богородиці}
\section{Молитва до Пресвятої Богородиці}
\emph{(з акафіста Благовіщенню Пресвятої Богородиці)}

О, Пресвята Владичице Богородице! Зі страхом і любов’ю припадаючи до чесної ікони Твоєї, молимося Тобі: не відверни лиця Твого від нас, що вдаємося до Тебе; ублагай, Мати милосердна, Сина Твого, Бога нашого, Господа Ісуса Христа, щоб у мирі охоронив землю і Церкву нашу святу зберіг непорушною від зневіри, єресей і розколів; визволи всіх, що з вірою моляться Тобі, від гріха, від підступів ворожих, від усякої спокуси, скорбот, біди й печалі сердечної, подай чистоту думок, виправлення життя грішного й прощення гріхів, щоб усі ми вдячно, прославляючи велич Твою, сподобилися Царства Небесного й там з усіма святими прославляли пречисте й величне ім’я Отця, і Сина, і Святого Духа. Амінь.

\section{Молитва до Пресвятої Богородиці}
\emph{(з акафіста Успінню Пресвятої Богородиці і з акафіста Покрову Пресвятої Богородиці)}

Царице моя преблага, надіє моя Богородице, пристановище сиріт і подорожніх заступнице, скорботних радосте, покривджених покровителько, бачиш мою біду, бачиш мою скорботу, допоможи мені як немічному, управи мене як подорожнього; кривду мою знаєш, розріши її як хочеш; бо я не маю іншої помочі, крім Тебе, ні іншої заступниці, ні благої утішительки, тільки Тебе, о Богомати; охорони мене і покрий мене навіки-віків. Амінь.

\section{Молитва до Пресвятої Богородиці перед Її Іверською чудотворною іконою}
О, Пресвята Діво, Мати Господа, Царице Небес і землі! Зглянься на багатостраждальне зітхання душ наших, подивися з висоти святої Твоєї на нас, що з вірою і любов’ю поклоняємося пречистому образу Твоєму. Ось бо, в гріхи занурені і скорботами обтяжені, дивлячись на образ Твій, як на Тебе живу, що перебуваєш між нами, приносимо смиренні наші моління. Бо не маємо ні іншої помочі, ні іншого заступництва, ні втіхи, тільки Тебе, о Мати всіх скорботних і обтяжених! Поможи нам, немічним, утамуй скорботу нашу, настав на шлях праведний нас заблудлих, вилікуй і спаси безнадійних, даруй нам останок часу життя нашого у мирі і тиші проводити, подай християнську кончину, і на Страшному суді Сина Твого явися нам милосердною Заступницею, щоб завжди оспівували, величали і славили Тебе як благу Заступницю християнського роду зі всіма, що догодили Богові. Амінь.

\section{Молитва до Пресвятої Богородиці на честь Її Почаївської чудотворної ікони}
До Тебе, о Богомати, молитовно звертаємося ми, грішні, згадуючи Твої чудеса, явлені у святій Почаївській лаврі, і за свої уболіваючи гріхи. Знаємо, Владичице, знаємо, що не подобає нам, грішним, чого-небудь просити, тільки того, щоб Праведний Суддя беззаконня наші простив. Усе бо, що ми в житті перетерпіли, скорботи, і недолю, і хвороби, як плоди наших беззаконь, виросли нам; Бог попустив це для нашого виправлення. Тому все це істиною і судом Своїм навів Господь на грішних рабів Своїх, що в печалях своїх до заступництва Твого, Пречиста, вдаються і з розчуленим серцем до Тебе взивають так: гріхів і беззаконь наших, Блага, не пом’яни, але піднеси пречисті руки Твої до Сина Твого і Бога, щоб Він простив нам вчинене нами, щоб за незчисленні наші невиконання обіцяння лиця Свого від рабів Своїх не відвернув і щоб благодаті Твоєї, яка допомагає нам у спасінні, від душ наших не відняв. Так, Владичице, будь спасінню нашому заступницею і слабкодухістю нашою не погордуй, зглянься на стогони наші, що в бідах і скорботах наших перед чудотворним Твоїм образом возносимо. Просвіти зворушливими помислами розум наш, віру нашу укріпи, надію утверди, дар найсолодшої любові сподоби прийняти. Цими ж, Пречиста, даруваннями, а не хворобами і скорботами, життя наше до спасіння приводиться, але від зневіри і відчаю душі наші охорони, позбав нас, слабких, від бід, що надходять на нас, і недолі, і наклепів людських, і нестерпних хвороб. Даруй мир і благоденство Україні нашій заступництвом Твоїм, Владичице, утверди православну віру в країні нашій і в усьому світі. Церкву апостольську і соборну на приниження не віддай, устави святих отців навіки непохитними збережи і всіх, хто до Тебе вдається, від рову згубного спаси. Ще ж і єрессю зваблених братів наших або тих, що спасительну віру у гріховних пристрастях згубили, знову до істинної віри і покаяння наверни, щоб, разом з нами Твоєму чудотворному образу поклоняючись, Твоє заступництво сповідували. Сподоби ж нас, Пречиста Владичице Богородице, ще в цьому житті перемогу істини Твоїм заступництвом побачити, сподоби нас благодатну радість раніше нашої кончини сприйняти, як у давнині насельників почаївських Твоїм явленням переможцями і просвітителями агарян показала Ти, щоб усі ми, вдячним серцем разом з ангелами, і пророками, і апостолами, і з усіма святими Твоє милосердя прославляючи, віддавали славу, честь і поклоніння в Тройці славивому Богу, Отцю, і Сину, і Святому Духу, навіки-віків. Амінь.

\section{Молитва до Пресвятої Богородиці перед Її чудотворною іконою «Владимирська» (Вишгородська)}
О, Всемилостива Владичице Богородице, Небесна Царице, Всемогутня Заступнице, неосоромлене наше уповання! Дякуючи Тобі за всі великі благодіяння, що з роду в рід людям нашим від Тебе були, перед пречистим образом Твоїм благаємо Тебе: збережи Україну нашу, всяке місто, село і країну, і предстоящих рабів Твоїх, і всю землю нашу від голоду, мору, землетрусу, потопу, вогню, меча, нашестя чужинців і міжусобної боротьби. Збережи і спаси, Діво, Владику і Отця нашого (ім’я) Святійшого Патріарха Київського і всієї Руси-України, і Владику нашого (ім’я), і всіх Преосвященних митрополитів, архієпископів і єпископів православних. Дай їм Церквою Українською добре управляти, вірних овець Христових у чистоті зберегти. Пом’яни, Владичице, і весь священичий і чернечий чин, зігрій серця їх ревністю у Бозі і укріпи кожного бути достойними свого звання. Спаси, Владичице, і помилуй усіх рабів Твоїх і даруй нам путь земного життя без пороку пройти. Утверди нас у вірі Христовій і в старанні до Православної Церкви, вклади в серця наші дух страху Божого, дух благочестя, дух смирення, в напастях терпіння нам подай, у благоденстві — стриманість, до ближніх любов, до ворогів усепрощення, у добрих ділах успіх. Визволи нас від усякої спокуси і від закам’янілої нечуттєвості, в страшний же день Суду сподоби нас заступництвом Твоїм стати праворуч Сина Твого, Христа Бога нашого, Йому ж належить усяка слава, честь і поклоніння з Отцем і Святим Духом нині, і повсякчас, і навіки-віків. Амінь.

\section{Молитва до Пресвятої Богородиці перед Її чудотворною іконою «Києво-Печерською»}
О, Пресвята і Пренепорочна Діво Богородице, наша Печерська похвало і окрасо, покрове і заступництво, істинна Владичице обраного жереба Твого, прийми нас, недостойних рабів Твоїх, що приносимо Тобі убоге моління з вірою і любов’ю перед чудотворним образом Твоїм, щоб Ти відвідала наше гріховне житіє, просвіщаючи його світлом покаяння. Осіни нас, обтяжених великими скорботами, Твоєю небесною радістю, і сподоби нас завжди славити Тебе в мирі і благоденстві в обителі Печерській; і тут пройшовши путь життя нашого без пороку, з Твоєю допомогою досягнемо вічної радості у світлості святих і божественних отців наших Антонія і Феодосія та всіх преподобних Печерських, і єдиними устами і серцем оспівуємо Тебе і Предвічного Сина Твого, з безначальним Його Отцем і пресвятим і Всеблагим і Животворчим і Єдиносущним Його Духом навіки-віків. Амінь.

\section{Молитва друга до Пресвятої Богородиці}
О, Пресвята і Преблагословенна Мати Господа і Спаса нашого Ісуса Христа, Пречиста Приснодіво Богородице Маріє! Припадаємо і поклоняємося Тобі перед святою і чудотворною іконою Твоєю, яку з давніх часів благозволила Ти прославити безліччю чудес і зцілень. Смиренно благаємо Тебе, благу і милосердну Заступницю нашу: молитвами угодників Твоїх, преподобних Антонія і Феодосія Печерських, почуй голос моління нас, грішних, не зневаж зітхань від душі, подивись на скорботи і біди, що нас охопили, і як воістину любляча Мати поспіши на поміч нам, безпомічним, сумним, що впали у многі і тяжкі гріхи і завжди прогнівляємо Господа і Сотворителя нашого. Його ж ублагай, Заступнице наша, щоб Він не згубив нас з беззаконнями нашими, але щоб Він показав нам чоловіколюбну Свою милість. Виблагай, Владичице, у Його благості всім православним християнам здоров’я і душевне спасіння, благочестиве і мирне життя, землі родючість, добре поліття, своєчасні дощі і благословення звище на всі добрі діла і починання; хворих зціли, скорботних утіш, бідуючим допоможи; сприяй усім нам носити ярмо Христове в терпінні і смиренні, і сподоби нас побожно закінчити земне житіє, християнську ж і неосоромлену кончину одержати і Небесне Царство успадкувати. О Царице Всехвальна, Мати Божа всеблага! Простягни богоносні руки Твої до улюбленого Сина Твого і Бога нашого, ублагай Його спасти і визволити від загибелі вічної усіх ченців обителі Твоєї Печерської. Як у давнину на горі Печерській Ти явила милість, Владичице, так і нині покажи, Богородице, щедроти Твої над Твоєю обителлю і над нами, недостойними дітьми Твоїми; щоб і тут, у земному житті нашому, і після кончини життя нашого ми прославляли безначального Отця, Єдинородного Його Сина, і Єдиносущного і Святого Його Духа, і Твоє милостиве заступництво навіки-віків. Амінь.

\section{Молитва до Пресвятої Богородиці перед Її чудотворною іконою «Всецариця»}
О, Пречиста Богомати, Всецарице! Почуй багатостраждальне зітхання наше перед чудотворною іконою Твоєю, що з Афону до нас принесена, поглянь на дітей Твоїх, незцілимими хворобами обтяжених, які з вірою до святого Твого образа припадають. Як птах покриває крилами пташенят своїх, так і Ти нині, Приснодіво, покрий нас покровом крил Твоїх, захисти всецілющим чесним Твоїм омофором. Там, де немає надії, безсумнівною надією будь. Там, де люті скорботи панують, терпінням і ослабленням явися. Там, де морок відчаю в душу вселився, нехай засяє світло Божества! Малодушних утіш, немічних укріпи, для зачерствілих серцем пом’якшення і просвіщення даруй. Зціли хворих людей Твоїх, о Всецарице Милостива! Розум і руки тих, що лікують нас, благослови, щоб послужили вони знаряддям Всемогутнього Христа, Спаса нашого. Як живій, сущій з нами, молимося перед іконою Твоєю, Владичице! Простягни руки Твої, повні зцілень, радосте скорботних, у печалях утіхо, щоб, чудотворну допомогу скоро отримавши, ми прославляли Животворчу і Нероздільну Тройцю, Отця, Сина і Святого Духа навіки-віків. Амінь.

\section{Молитва до Пресвятої Богородиці перед чудотворною Її іконою «Утамуй мої печалі»}
\emph{Тропар, глас 3}

Утамуй болісті скорботної моєї душі, Пречиста, що всяку сльозу з лиця світу втираєш. Ти бо людей від недуг визволяєш і скорботи розвіюєш, для всіх Ти — надія й опоро, Пресвята Мати-Діво.

\section{Молитва}

Надіє всіх кінців землі, Пречиста Діво, Владичице Богородице, утіхо моя! Не відкидай мене грішного, що наважуюсь благати Тебе негідними вустами й молю: погаси в мені полум’я гріховне і покаянням окропи душевні й тілесні рани, зціли, полегши, Владичице, недугу, стримай бурю злих нападів, Пречиста, звільни від тягаря гріхів моїх і втамуй мої печалі, що зранюють серце, Преблагая. Ти бо — піднесення роду людського й у печалях скора Утішителька. Незліченні милості Твої, Всеблагословенна, і нехай славить Тебе душа моя по всі дні життя до останнього подиху. Амінь.

\section{Молитва до Пресвятої Богородиці перед чудотворною Її іконою «Цілителька»}
О, Пресвята Владичице, Царице Богородице, вища за всі небесні сили і святіша за всіх святих! Припадаємо і поклоняємося Тобі перед всечесним цілительним образом Твоїм, згадуючи дивне явлення Твоє болящому клірику Вікентію, і старанно благаємо Тебе, всесильну роду нашого Заступницю і Помічницю: як у давнину Ти подала зцілення тому клірику, так і нині зціли наші душі і тіла, що хворіють гріховними ранами і різноманітними пристрастями; визволи нас від усяких напастей, бід, скорбот і вічного осудження. Збережи від душозгубних вчень і невір’я, від улесливих і несподіваних підступів невидимих ворогів. Подай нам християнську кончину безболісну, мирну, неосоромлену, Святих Таїн причасну. Сподоби нас на непідкупному судилищі Христовому стати праворуч Всеправедного Судді і почути блаженний Його голос: прийдіть, благословенні Отця Мого, успадкуйте приготоване вам Царство від створення світу. Амінь.

\section{Молитва до Пресвятої Богородиці перед чудотворною Її іконою «Невпивана Чаша»}
Всемилостива Владичице! До Твого заступництва нині вдаємося, благань наших не зневаж, але милостиво вислухай нас: дружин, дітей, матерів — і тяжкою недугою пияцтва одержимих і заради цього від матері своєї, Церкви Христової, і спасіння відпадаючих братів, і сестер, і родичів наших зціли.

О, милостива Мати Божа, доторкнися до сердець їхніх і швидко підніми від гріховних падінь і приведи їх до спасительної стриманості.

Ублагай Сина Свого, Христа Бога нашого, щоб Він простив нам гріхи наші і не відвернув милості Своєї від людей Своїх, але укріпив нас у тверезості і цнотливості.

Прийми, Пресвята Богородице, молитви матерів, що за дітей своїх сльози проливають; дружин, що за своїх чоловіків ридають; дітей, безпомічних і вбогих, заблудлими залишених, і всіх нас, що до ікони Твоєї припадаємо. І нехай дійде це волання наше, молитвами Твоїми, до престолу Всевишнього.

Покрий і збережи нас від лукавого ловління і всіх підступів ворожих, у страшний же час виходу нашого допоможи безперешкодно пройти повітряні митарства, молитвами Твоїми позбав нас вічного осудження, щоб покрила нас милість Божа у нескінченні віки віків. Амінь.

\section{Молитва до Пресвятої Богородиці перед Її чудотворною іконою «Троєручиця»}
О, Пресвята Преблагословенна Діво Богородице Маріє! Припадаємо і поклоняємося Тобі перед святою чудотворною Троєручною іконою Твоєю, згадуючи преславне чудо Твоє через зцілення відсіченої правиці преподобного Іоанна Дамаскина від ікони цієї; заради цього знамення, яке і донині є видимим на ній в образі третьої руки, до зображення Твого прикладеної. Молимося Тобі і просимо Тебе, Всеблагу і Всещедру роду нашого Заступницю, почуй нас, і як блаженного Іоанна, який у скорботі і хворобі до Тебе взивав, почула Ти, так і нас не відкинь, скорботних і хворих від ран різних пристрастей, що до Тебе від щирої і смиренної душі старанно вдаємося. Ти бачиш, Всемилостива, немочі, озлоблення і нашу потребу у Твоїй допомозі і заступництві, бо ми з усіх сторін ворогами оточені і не маємо нікого, хто допомагав би і захищав би, якщо тільки Ти змилосердишся до нас, Владичице.

Щиро молимося Тобі: почуй наш скорботний голос і допоможи нам святоотцівську православну віру до кінця наших днів непорочно зберігати, всі заповіді Господні постійно виконувати, щире покаяння за гріхи наші завжди Богові приносити і сподобитися мирної християнської кончини життя нашого і доброї відповіді на Страшному суді Сина Твого і Бога нашого. Його ж ублагай за нас материнською Твоєю молитвою, нехай не засудить нас за беззаконня наші, але нехай помилує нас із великої і невимовної Своєї милості, о Всеблага! Почуй нас і не позбавляй нас Твоєї державної допомоги, щоб Тобою живих одержавши спасіння, ми оспівували і прославляли на землі народженого від Тебе Спасителя нашого Господа Ісуса Христа, Йому ж належить слава і держава, честь і поклоніння разом із Отцем і Святим Духом нині, і повсякчас, і навіки-віків. Амінь.

\section{Молитва до Пресвятої Богородиці перед Її чудотворною іконою «Казанська»}
О, Пресвята Діво, Владичице Богородице! Зі страхом, вірою і любов’ю припадаючи перед чесною іконою Твоєю, благаємо Тебе: не відверни лиця Твого від тих, хто вдається до Тебе, благай, милосердна Мати, Сина Твого і Бога нашого, Господа Ісуса Христа, щоб зберіг мир у країні нашій, щоб Церкву Свою святу зберіг непохитною від невір’я, єресі і розколу. Бо ми не маємо іншої допомоги, не маємо іншої надії, крім Тебе, Пречиста Діво, Ти бо Всесильна для християн Помічниця і Заступниця. Позбав усіх, хто з вірою Тобі молиться, від падінь гріховних, від наклепів злих людей, від усяких спокус, скорбот бід і від несподіваної смерті; даруй нам дух сокрушення, смирення серця, чистоту помислів, виправлення гріховного життя і прощення гріхів, щоб усі, вдячно оспівуючи велич Твою, сподобилися Небесного Царства і там зі всіма святими прославили пречесне і величне ім’я Отця, і Сина, і Святого Духа. Амінь.

\section{Молитва до Пресвятої Богородиці перед Її чудотворними іконами «Знайдення загиблих» і «Всіх скорботних радість»}
О, Пресвята і преблагословенна Діво, Владичице Богородице! Споглянь милостивим Твоїм оком на нас, що стоїмо перед святою іконою Твоєю і з розчуленням молимось Тобі; виведи нас із глибини гріховної, просвіти розум наш, затьмарений пристрастями, вилікуй рани душ і тіл наших. Бо ми не маємо іншої допомоги, не маємо іншої надії, тільки Тебе, Владичице. Ти знаєш усю неміч і гріхи наші, до Тебе вдаємося і взиваємо: не позбав нас Твоєї небесної помочі, але заступайся за нас завжди і Твоїм невимовним милосердям і щедротами спаси і помилуй нас, що гинемо. Даруй нам виправлення гріховного життя нашого і позбав нас від скорбот, бід і хвороб, від несподіваної смерті, пекла і вічної муки. Ти бо, Царице і Владичице, скора Помічниця і Заступниця всім, хто вдається до Тебе, Ти надійне пристановище грішників, що каються. Подай же нам, Преблага і Всенепорочна Діво, християнський кінець життя нашого, мирний і безгрішний, і сподоби нас Твоїм заступництвом оселитися в оселях небесних, де безперестанно лунає голос тих, що святкують і радістю прославляють Пресвяту Тройцю, Отця і Сина і Святого Духа, нині, і повсякчас, і навіки-віків. Амінь.

\section{Молитва до Пресвятої Богородиці на честь Її чудотворної ікони «Семистрільна» («Пом’якшення злих сердець»)}
О, багатостраждальна Мати Божа, вища за всіх дочок землі, за Своєю чистотою і за безліччю страждань, що перетерпіла Ти на землі, прийми багатоболісні зітхання наші і збережи нас під покровом милості Твоєї. Іншого бо пристановища і теплого заступництва крім Тебе ми не маємо, але як Та, що має сміливість перед Тобою Народженим, допоможи і спаси нас молитвами Твоїми, щоб ми безперешкодно досягли Царства Небесного, де зі всіма святими будемо оспівувати в Тройці єдиного Бога нині, і повсякчас, і навіки-віків. Амінь.

\section{Молитва до Пресвятої Богородиці перед Її чудотворною іконою «Неопалима купина»}
О, Пресвята і Преблагословенна Мати Господа нашого Ісуса Христа! Припадаємо і поклоняємося Тобі перед святою і пречесною іконою Твоєю, якою Ти дивні і преславні чудеса являєш, від вогненного спалення і блискавичного грому оселі наші спасаєш, хворих зціляєш і всі благі наміри наші добре виконуєш. Смиренно молимо Тебе, всесильна роду нашого Заступнице, сподоби і нас, немічних та грішних, Твоєї материнської участі і допомоги. Спаси і збережи, Владичице, під покровом милості Твоєї Церкву Твою (обитель цю), всю країну нашу православну і всіх нас, що припадаємо до Тебе з вірою і любов’ю та сердечно просимо зі сльозами Твого заступництва.

Владичице Всемилостива, умилосердься над нами, бо обтяжені ми багатьма гріхами і не маємо відваги звернутись до Христа Господа і просити Його про помилування і прощення, але Тебе просимо Його ублагати, Матір Його по плоті. Ти ж, Всеблага, простягни до Нього богоприємні руки Твої і заступися за нас перед благістю Його, випрошуючи нам прощення провин наших, благочестивого і мирного життя, благої християнської кончини і доброї відповіді на Страшному суді Його. Найбільше ж у час грізного відвідання Божого, коли загоряться вогнем домівки наші, чи блискавковим громом настрашені будемо, яви нам милостиве Твоє заступництво і могутню допомогу. Щоб, спасенні всесильними Твоїм до Господа молитвами, тимчасового покарання Божого тут позбавились і вічне блаженство райське успадкували та зі всіма святими оспівували пречесне і величне ім’я Пресвятої Тройці, Отця, і Сина, і Святого Духа, і Твоє велике до нас милосердя, навіки-віків. Амінь.

\section{Молитва до Пресвятої Богородиці на честь Її чудотворної ікони «Заступниця грішних»}
До кого взиватиму, Владичице, до кого звернуся у горі моєму, як не до Тебе, Цариці Небесної? Хто плач мій і зітхання прийме і благання наші швидко послухає, якщо не Ти, Всеблага Заступнице, усіх наших радостей Радість! Почуй же і нинішні піснеспіви і благання і за мене, грішного, що Тобі приносяться. І будь мені Матір’ю і Покровителькою і радості всім нам Подателькою. Влаштуй життя моє як хочеш і як знаєш. Вручаю бо себе Твоєму покрову і піклуванню, щоб я завжди радісно співав Тобі зо всіма: радуйся, Благодатна; радуйся, Обрадувана; радуйся, Преблагословенна; радуйся, Препрославлена, повіки. Амінь.

\section{Молитва до Пресвятої Богородиці на честь Її чудотворної ікони «Розчулення»}
Прийми, всесильна Пречиста Владичице Богородице, ці чесні дари, Тобі єдиній притаманні, від нас, недостойних рабів Твоїх; від усіх родів обрана, що явилася вищою за все небесне і земне творіння, бо заради Тебе Господь сил був з нами, і через Тебе Сина Божого пізнали і сподобилися святого Тіла Його і пречистої Крові Його; тому блаженна Ти в родах родів, Богоблаженна, світліша за херувимів і чесніша за серафимів. І нині, всехвальна Пресвята Богородице, не переставай молитися за нас, недостойних рабів Твоїх, щоб позбавитися нам усякої лукавої ради і всяких обставин і зберегтися неушкодженими від усякого отруйного підступу диявольського; але навіть до кінця збережи нас неосудженими молитвами Твоїми, щоб Твоїм заступництвом і допомогою спасенні, славу, хвалу, подяку і поклоніння за все в Тройці Єдиному Богу і всього Сотворителю ми возсилали нині, і повсякчас, і навіки-віків. Амінь.

\section{Молитва до Пресвятої Богородиці на честь Її чудотворної ікони «Відрада», або «Утішення»}
О, Пресвята Владичице Богородице, Відрадо і Утішення наше! Поглянь милостиво на людей, що з вірою і любов’ю поклоняються пречистому образу Твоєму; прийми наш похвальний спів і принеси Твою теплу молитву за нас грішних до Господа, щоб Він, зневаживши усі наші гріхи, спас і помилував нас. О Предивна Владичице! Покажи на нас дивні милості Твої: збережи і помилуй святійшого Патріарха (ім’я) і преосвященного (ім’я), даруй їм здоров’я і спасіння; збережи пастирів Церкви, христолюбиве воїнство і всіх людей Твоїх. Благаємо Тебе розчулено: позбав і всіх нас від усякої скорботи, настанови на путь усяких чеснот і благостині, спаси від спокус, бід і хвороб, візьми від нас наклепи і сварки, охорони від блискавки і грому, від пожежі, голоду, землетрусу, повені і смертоносної рани; подай нам Твою милостиву допомогу в дорозі: і на морі, і на суші, і в повітрі. Ось бо, Всемилостива Владичице, в несумнівній надії ми підносимо до Тебе нашу убогу молитву. Не відкинь наших сліз і зітхань, не забувай нас в усі дні життя нашого, але завжди перебувай з нами і Твоїм заступництвом і клопотанням у Господа подавай нам відраду, втіху, захист і поміч, щоб ми славили, і величали, і ублажали Тебе в усі роди навіки-віків. Амінь.

\section{Молитва до Пресвятої Богородиці перед Її чудотворною іконою «Зглянься на смирення»}
О, Пресвята Царице, Діво Богородице, вища від херувимів і незрівнянно славніша від серафимів, Богом обрана Отроковице! Поглянь з висоти небесної Твоїм милостивим оком на нас, недостойних рабів Твоїх, що розчулено зі сльозами перед Пречистим образом Твоїм молимося: не залиш нас у цьому багатостраждальному і суєтному земному житті без захисту і Твого державного покрову. Спаси нас від скорбот, бо гинемо, підніми нас із глибини гріховної, просвіти наш розум, затьмарений пристрастями. Вилікуй рани душ і тіл наших! О, Благословенна Мати Чоловіколюбного Владики! Вилий на нас велику милість Твою, зміцни нашу розслаблену волю на виконання Христових заповідей, пом’якши наші скам’янілі серця любов’ю до Бога і ближніх, даруй нам покірність сердечну й істинне покаяння, щоб, очистившись від гріховної скверни, ми сподобилися мирного християнського кінця життя і доброї відповіді на Страшному і праведному суді Господа нашого Ісуса Христа. Бо Йому належить усяка слава, честь і поклоніння з безначальним Його Отцем, Пресвятим, і Благим, і Животворчим Його Духом нині, і повсякчас, і навіки-віків. Амінь.

\section{Молитва до Пресвятої Богородиці на честь Її чудотворної ікони «Несподівана радість»}
О, Пресвята Діво, Всеблагого Сина Мати Всеблага, міста і святого храму цього Покровителька, всіх сущих у гріхах, скорботах, бідах і хворобах вірна Молитвениця і Заступниця! Прийми молебний спів цей від нас, недостойних рабів, що Тобі возносимо, і як у давнину грішника, що кожного дня молився перед чесною Твоєю іконою, Ти не зневажила, але дарувала йому несподівану радість покаяння і схилила Сина Твого великим і старанним перед Ним заступництвом до прощення цього грішника, так і нині не зневаж моління нас, недостойних рабів Твоїх, і ублагай Сина Твого і Бога нашого, щоб і всім нам, що з вірою і зворушенням поклоняємося перед Твоїм образом, несподівану радість кожному за його потребою дарував; щоб усі на небі і на землі знали Тебе як тверду і неосоромлену Заступницю роду християнського і, це знаючи, славили Тебе і через Тебе Сина Твого, з безначальним Його Отцем і Єдиносущним Його Духом нині, і повсякчас, і навіки-віків. Амінь.

\section{Молитва до Пресвятої Богородиці перед чудотворною Її іконою «Годувальниця»}
О, Пресвята Владичице наша Богородице, Небесна наша Царице, що молоком Христа Бога нашого годувала і нас, негідних і многогрішних, Твоєю багатою милістю годуєш. Спаси і визволи нас, грішних рабів Твоїх, від даремних наклепів, від усякої біди й напасті і наглої смерті. Помилуй нас у денну, ранішню й вечірню пору і в усякий час збережи нас, чи то сидимо, чи стоїмо — охорони, коли спимо вночі і коли ходимо — на всіх шляхах покрий і заступи. Захисти нас, Владичице Богородице, від усіх ворогів наших видимих і невидимих та від усякого зла. На всякому місці і на всякий час будь нам, Мати Преблага, непоборною стіною й міцним заступництвом завжди, нині, і повсякчас, і навіки-віків. Амінь.

\section{Молитва до ікони Пресвятої Богородиці «Поручителька за грішних»}
О, Владичице моя Преблагословенна, Захиснице роду людського, Помічнице і спасіння всіх, хто до Тебе приходить. Знаю, істинно знаю, що я дуже нагрішив і прогнівив Тебе, Всемилостива Владичице, та Народженого від Тебе Всеблагого Бога. Але знаю багато таких прикладів, коли ті, які ще до мене прогнівили Милостивого Бога (митарі, блудниці та інші грішники), за своє в майбутньому покаяння та благочестиве життя отримали прощення гріхів. Споглядаючи очима своєї грішної душі на ці образи, які сподобилися такого великого Божого милосердя, насмілився і я, грішний, звернутися з покаянням до Тебе, сподіваючись, що Ти, о Всемилостива Владичице, подаси мені руку допомоги і виблагаєш у Сина Твого і Бога прощення тяжких моїх гріхів. Вірую і визнаю, що Той, Якого Ти народила, Син Твій, є воістину Христос, Син Бога Живого, Суддя живих та мертвих, Який відплачує кожному по ділах його. Ще вірую і сповідую Тебе істинною Богородицею, джерелом милосердя, утіхою тих, що плачуть, віднайденням загиблих, сильною та невтомною перед Ним Заступницею, Яка дуже любить рід християнський, та Поручителькою мого покаяння. Бо воістину немає для людей іншої допомоги та захисту, крім Тебе, о Всемилостива Владичице. Ніхто з тих, що на Тебе надіється, не буде посоромленим. І ніхто з тих, що за Твоїм посередництвом благає Бога, не буде знехтуваний. Тому молю Тебе, Всеблагая: відкрий двері милосердя Твого преді мною, що заблукав і впав у глибину темряви. Не відвертайся від мене, нечестивого, не зневаж молитви мене, грішного, не залишай мене, окаянного, якого злий ворог провадить до загибелі. Але ублагай за мене, Народженого від тебе милосердного Бога, щоб Він простив великі мої гріхи і спас мене від загибелі, щоб і я разом з усіма, хто отримав прощення, оспівав та прославив безмірне милосердя Народженого від Тебе Бога і Твоє за мене бездоганне заступництво у цьому житті й у вічному віці. Амінь.

\section{Молитва 2-а до Пресвятої Богородиці}
Пресвята Владичице моя Богородице, святими Твоїми і всесильними молитвами віджени від мене, смиренного і недостойного раба Твого, безнадійність, недбалість, лінощі і безглуздя і всі темні, лукаві й брудні думки та наміри від окаянного мого серця та від затьмареного мого розуму. Погаси полум’я моїх пристрастей, бо я нещасний і негідний. Визволи мене від незчисленних нечестивих спогадів, думок та дій. Бо Ти Благословенна єси у всіх народах, і славиться пречесне ім’я Твоє на віки віків. Амінь.

\section{Молитва 3-я до Пресвятої Богородиці}
О, Пресвята Богородице! Почуй, Милосердна, молитви нас, грішних і смиренних рабів Твоїх, і ублагай Бога, Сина Твого, щоб послав нам і всім, хто припадає до Тебе, здоров’я душевне й тілесне, і все, що для вічного та земного життя потрібне. Нехай простить нам провини вільні й невільні. Нехай визволить нас від скорбот та хвороб, напасті та всякого злого випадку. Так, Царице наша преблагая, надіє наша непереможна і заступнице непоборна! Не відверни лиця Твого від нас через безліч гріхів наших, але простягни до нас руки материнського Твого милосердя і створи нам благе чудо. Яви нам щиру допомогу Твою і посприяй нам у всякій благій справі. Але від всякого гріховного починання та лукавої думки визволи нас, бо ми завжди славимо пречесне ім’я Твоє і поклоняємося чесному образу Твоєму, і величаємо Бога Отця, і Єдинородного Сина Його, Господа нашого Ісуса Христа, і Святого Духа зо всіма святими на віки віків. Амінь.

\section{Молитва 4-а до Пресвятої Богородиці}
Богорадувана Владичице, Препрославлена Мати милосердя і чоловіколюбства, всемилостива за весь світ Заступнице, ми, раби Твої, що під Твій Божественний покров з радістю приходимо і до Твого чудесного образу смиренно припадаючи, молимося: щиро молися, о Препрославлена Царице і Владичице за нас до Сина Твого і Бога нашого, щоб Він заради Тебе визволив нас від усякої хвороби, журби та гріха, визнав нас достойними наслідувати Небесне Його Царство. Бо Ти як Мати без остраху приступаєш до Нього і отримуєш все, що захочеш, єдина блага та благословенна на віки. Амінь.

\section{Пресвята Богородице, спаси нас}
Маючи безліч спокус, до Тебе з надією на спасіння припадаю. О, Мати Слова і Діво, від лихого та нечестивого спаси мене. Пресвята Богородице, спаси нас.

\section{Молитва до Пресвятої Богородиці на честь Її ікони «Тихвинська»}
О, Пресвята Діво, Мати Господа вишніх сил, Царице неба і землі, міста й країни нашої всесильна Заступнице. Прийми хвалебно-подячний спів цей від нас, недостойних рабів Твоїх, і вознеси молитви наші до Престолу Бога, Сина Твого, щоб Він помилував нас за неправди наші і надалі виявляв благість Свою до тих, що шанують всечесне ім’я Твоє і з вірою та любов’ю поклоняються чудотворному Твоєму образу. Бо ми недостойні від Нього помилування отримати, якщо Ти, о Владичице, не умилостивиш Його до нас. Бо Ти все від Нього можеш отримати. Тому до Тебе прибігаємо, як до невідмовної і скорої Заступниці нашої: почуй нас, що молимось до Тебе, покрий нас владичним покровом Твоїм і виблагай у Бога, Сина Твого, щоб пастирі наші ревно піклувалися про душі наші, владі — мудрість і силу, суддям — правду й непідкупність, наставникам — розум і смиренномудрість, подружжям — любов і згоду, дітям — послух, ображеним — терпіння, образникам — страх Божий; тим, що у скорботі перебувають — мужність і добродушність; тим, що мають радість — стриманість, а всім нам — дух розуму, благочестя, милосердя, смиренності, чистоти і правди. О, Владичице Пресвята, змилосердься над немічними людьми Твоїми: розпорошених збери, заблудлих на істинний шлях направ, старих підтримай, юних у цнотливості настав, дітей виховай, і з великого милосердя Свого заступайся за всіх нас, піднеси нас із безодні гріховної і просвіти наші очі сердечні, щоб бачити нам спасіння. Будь милостивою до нас тут і там, в краю земного перебування і на Страшному Суді Сина Твого. Тих, що відійшли від цього життя братів і отців наших сподоби життя вічного з ангелами і всіма святими. Бо Ти, Владичице, славою небесних і сподіванням земних, в Бозі єси Надією та Заступницею всіх, хто до Тебе з вірою приходить. До Тебе, отже, молимось, і Тобі, як всемогутній і Помічниці, самі себе і один одного і все життя наше віддамо нині і повсякчас, і на віки віків. Амінь.

\section{Молитва до Пресвятої Богородиці на честь Її ікони «Споручниця грішних»}
До кого піднесу я сльози мої, Владичице, до кого прийду я з горем моїм, як не до Тебе, Царице Небесна? Хто плач мій і стогін мій прийме, і молитви наші завжди вислухає, як не Ти, Всеблага Заступнице, всіх наших радощів Радість? Почуй же й нині піснеспіви та моління, що й за мене, грішного, до Тебе возносяться. Будь мені, Мати, Покровителькою і радощів всім нам Подателькою. Влаштуй життя моє, як бажаєш та знаєш. Віддаю бо себе під Твій покров та турботи, щоб радісно разом зі всіма завжди співати Тобі: радуйся, Благодатна; радуйся, Обрадувана; радуйся, Преблагословенна; радуйся, Препрославлена навіки. Амінь.

\section{Інша молитва на честь тієї ж ікони}
Царице моя Преблага, Надіє моя Пресвята, Заступнице грішних! Це бідний грішник стоїть перед Тобою! Не залиш мене, якого всі залишили; не забудь про мене, про якого всі забули; дай радість мені, який не знає радості. О, які великі мої біди і скорботи! О, якими безмірними є мої гріхопадіння! Як нічна темрява — життя моє. Немає серед синів людських жодного, який міг би мені допомогти. Ти — Єдина моя Надія. Ти — Єдиний мій Покров, Захист і Утвердження. Насмілююсь піднести до Тебе немічні мої руки і молю: змилосердься наді мною, Всеблага, зглянься, помилуй того, якого відкупив Кров’ю Своєю Син Твій, вгамуй хвороби скорботної моєї душі, вгамуй гнів тих, що ненавидять і ображають мене, зміцни немічні мої сили, віднови, як у орла, юність мою, не дай мені знесилитися у виконанні заповідей Божих. Вогнем небесним доторкнися до розбитої моєї душі і наповни її вірою бездоганною, любов’ю нелицемірною та надією непохитною. Щоб завжди я оспівував і славив Тебе, Преблагословенну Заступницю світу, Покров наш і Просительницю за всіх нас, грішних, і поклоняюся Сину Твоєму і Спасу нашому, Господу Ісусу Христу, з Безпочатковим Його Отцем і Життєподательним Духом Святим на віки віків. Амінь.

\section{Молитва до Пресвятої Богородиці на честь Її ікони «Боголюбська»}
О Пречиста Владичице Богородице, Мати Боголюбна, Надіє нашого спасіння! Зглянься милостиво на тих, що з вірою та любов’ю моляться та поклоняються до Пречистого Твого образа. Прийми цей наш хвалебний спів і пролий теплу Твою молитву за нас грішних до Господа, щоб Він, зглянувшись над нашими гріхами, спас і помилував нас! О, Всечесна Владичице! Вияви на нас дивовижні милості Твої. Молимо Тебе смиренно: визволи нас від усякої скорботи, настав на шлях всякої доброчинності й благодіяння, спаси від спокус, бід та хвороб. Припини серед нас наклепи та суперечки. Збережи нас від блискавки з неба, спалаху вогню, голоду, землетрусу, потопу і смертоносної хвороби. Подай нам Свою милостиву допомогу під час подорожі по морю та землі, щоб ми жахливо не загинули. О, Всемилостива, Боголюбива Мати, з непохитною надією возносимо до Тебе нашу смиренну молитву! Не зневаж наших сліз та уповань, не забувай про нас протягом всього нашого життя, але завжди перебувай з нами і Твоїм заступництвом та ходатайством перед Господом подай нам зміцнення, втіху, захист і допомогу, щоб ми завжди славили і величали преблагословенне та препрославлене ім’я Твоє. Амінь.

\section{Молитва до Пресвятої Богородиці на честь Її ікони «Скоропослушниця»}
Преблагословенна Владичице, Приснодіво Богородице, що Бога Слово раніше всякого слова для спасіння нашого народила і благодать Його найбільше за всіх прийняла, Море Божественних дарів і чудес; завжди жива Ріка, Яка подає благодать усім, хто з вірою до Тебе прибігає! Перед Твоїм чудотворним образом припадаючи, молимось до Тебе, Всещедрої Матері Чоловіколюбного Владики: сподоби нас всещедрої милості Твоєї і прохання наші, що ми приносимо Тобі, Скоропослушнице, якомога швидше виконай на користь, утіху і спасіння кожному. Вияви, Преблага, благодать Твою на рабах Твоїх. Подай хворим зцілення і здоров’я; тим, що хвилюються — спокій; полоненим — свободу; страждаючим — утіху. Визволи, Всемилостива Владичице, кожне місто і країну від голоду, хвороби, землетрусу, потопу, вогню, меча та іншої тимчасової та вічної кари. Материнськими Своїми молитвами гнів Божий відверни. Від душевної немочі, спокус та гріхопадінь рабів Твоїх визволи. Щоб ми, перемагаючи спокуси та живучи благочестиво у цьому віці, в майбутньому сподобилися вічних благ благодаттю й чоловіколюбством Сина Твого і Бога, Якому належить всяка слава, честь і поклоніння з Безначальним Його Отцем і Пресвятим Духом нині і повсякчас, і на віки віків. Амінь.

\section{Молитва до Пресвятої Богородиці на честь Її ікони «Владична»}
Світу Заступнице, Мати препрославлена! Зі страхом, вірою і любов’ю припадаючи перед чесною Твоєю Владичною іконою, щиро молимось до Тебе: не відвертайся від тих, що до Тебе прибігають. Ублагай, милосердна Мати світу, Сина Твого і Бога нашого, найдорожчого Господа Ісуса Христа, щоб зберіг у мирі країну нашу, утвердив державу нашу в благоденстві і від міжусобиці нас визволив, зміцнив святу нашу Православну Церкву і зберіг її від невір’я, розколу та єресей. Не маємо іншої допомоги, окрім Тебе, Пресвята Діво. Ти — християн перед Богом всесильна Заступниця. Праведний Його гнів пом’якшуючи, всіх, хто з вірою до Тебе молиться, від гріховних падінь, злих людських обмов, голоду, скорботи і хвороб визволи. Дай нам дух сокрушенний, смирення серця, чистоту думок, виправлення гріховного життя і прощення провин наших. Щоб всі ми, які вдячно велич Твою прославляємо, сподобилися Царства Небесного і там зі всіма святими прославляли всечесне та величне ім’я в Тройці Славимого Бога: Отця, і Сина, і Святого Духа. Амінь.

\section{Молитва до Пресвятої Богородиці на честь Її ікони «Федорівська»}
До кого взиватиму, Владичице, до кого удамся в скорботі моїй. До кого принесу сльози та ридання мої, якщо не до Тебе, Царице Неба і землі. Хто вийме мене з вогнища гріхів та беззаконня, як не Ти, о Мати Життя, Заступнице і Пристановище роду людського. Почуй ридання моє, утіш і помилуй мене в горі моєму, захисти під час бід і напастей; визволи від утисків та скорбот, від усяких недуг та хвороб, від ворогів видимих та невидимих; ослаб ворожнечу тих, що чинять мені утиски. Щоб я вільний був від наклепів та злоби людської. Також визволи мене від негідних прагнень власного тіла. Покрий мене покровом милості Твоєї, щоб здобув спокій, радість та очищення від гріхів. Віддаю себе під Твій материнський захист. Будь мені, Мати і надіє, покровом, допомогою, заступництвом, радістю, утіхою і швидкою у всьому помічницею. О, чудесна Владичице! Кожен, хто приходить до Тебе, не залишиться без Твоєї всесильної допомоги. Тому і я, недостойний, до Тебе вдаюся, щоб отримати визволення від несподіваної та жахливої смерті, від скреготу зубів та вічних мук, Небесного ж Царства отримати сподобився і до Тебе від щирого серця зі смиренням промовляв: радуйся, Мати Божа, наша щира Молитвениця та Заступниця, на віки віків. Амінь.

\section{Молитва до Пресвятої Богородиці на честь Її ікони «Віднайдення загиблих»}
Щира Заступнице, Милосердна Мати Господа! До Тебе прибігаю я, негідний і найгрішніший за всіх людей. Почуй голос моління, плачу та зітхання мого. Бо беззаконня мої перевершили голову мою і я, як корабель у безодню, занурююся в море гріхів моїх. Але Ти, Всеблага і Милосердна Владичице, не зневаж мене, що втратив надію і гину в гріхах. Помилуй мене, що каюся за мої злі діла, і направ на істинний шлях заблудлу та негідну мою душу. На Тебе, Владичице моя Богородице, всю мою надію покладаю. Ти, Мати Божа, збережи та захисти мене під покровом Твоїм нині і повсякчас, і на віки віків. Амінь.

\section{Молитва до Пресвятої Богородиці на честь Її ікони «Озерянська»}
Зібравши всі думки, які від суєти життя та власного розуму виникли, до Твоєї, Богородице, ікони душевні та тілесні очі наші підносимо, і до Тебе взиваємо: не дай, Пречиста, нам, немічним, у гріхах наших загинути. Не дай злому демону наші душі від Сина твого відвести і в гущавину пристрастей загнати. Але допоможи нам Його заповіді завжди пам’ятати. Щоб ми не віддавалися блудним пристрастям і між собою не ворогували, але щоб сьогодні, до Тебе, щиро молячись за виправлення нашого життя, ми з Твого храму в оселі свої повернувшись, цей добрий намір щиросердно зберігали. Після цього і тілесні наші потреби задовольни, і недуги Твоєю, Препрославлена, молитвою зціли. Знаємо, що нам за це належить більше терпіти, у скорботі плакати й Господа молити. Бо знемагають у журбі душі наші й суєтні думки нашу віру розхитують. Тому із сокрушеним серцем до Тебе взиваємо. Як безліч страждаючих від немочі і скорботи Ти зцілила, так і нас, що перед Твоєю чудотворною іконою припадаємо і в скорботі до Тебе молимось, почуй і прохання всіх щиро виконай: хворим — здоров’я подай; тим, що у скорботі перебувають, — утіху, ворогуючим — мир від Бога; тим, що мають сумніви, — утвердження у вірі. Почуй грішних рабів твоїх, Пречиста. Підведи нас для подячного Тобі славослів’я. Нехай збільшиться віра людей, нехай прославиться в них Твоє заступництво, і нехай так утвердиться в нас Царство Сина Твого. Йому ж належить всяка слава, честь і поклоніння, з Безпочатковим Його Отцем, і Пресвятим, і Благим, і Животворчим Духом, нині і повсякчас, і на віки віків. Амінь.

\section{Молитва на честь ікони «Знамення» Пресвятої Богородиці у Великому Новгороді}
О, Пресвята і Преблагословенна Мати Найдобрішого Господа нашого Ісуса Христа! Перед святою Твоєю чудотворною іконою дивне знамення заступництва Твого, під час ворожого нашестя на Великий Новгород від неї виявлене згадуючи, Тобі поклоняємось і припадаємо. Смиренно молимо Тебе, всесильна роду нашого Заступнице, як колись отцям нашим допомогу швидко надала, так і нині нас, немічних та грішних, Твого материнського заступництва та благої настанови сподоби. Церкву святу зміцни, місто Твоє і всю країну нашу православну і всіх нас, що з вірою та любов’ю до Тебе припадаємо, і з сокрушенням сердець та слізьми Твого заступництва просимо, помилуй і збережи. О, Всемилостива Владичице! Змилосердься над нами, що многими гріхами обтяжені. Простягни до Христа Бога богоприємні руки Твої і заступайся за нас перед Ним, прощення прогріхів наших, утвердження нас у благочесному мирному житті, благого християнського кінця і доброї відповіді на Страшному Суді Його для нас просячи. Щоб спасенні Твоїми всесильними до Нього молитвами ми райське блаженство успадкували і з усіма святими прославляли пречесне і величне ім’я достойної поклоніння Тройці — Отця, і Сина, і Святого Духа, і Твоє велике до нас милосердя на віки віків. Амінь.

\section{Молитва до Пресвятої Богородиці на честь Її ікони «Смоленська» або «Одигітрія»}
О, Предивна і Найвеличніша за все творіння Царице Богородице, Небесного Царя Христа Бога нашого Мати, Пресвята Одигітріє Маріє! Почуй нас, грішних та недостойних, що нині перед Твоїм Пречистим Образом припадаємо і зі сльозами та риданням до Тебе молимось і з сокрушенням серця промовляємо: відведи нас, Блага і Одигітріє, від провалля пристрастей, визволи нас від усякої скорботи і смутку, охорони від усякої напасті та злих наклепів і від неправедного ворожого підступу. Бо можеш, Благодатна Мати наша, не лише від усякого зла людей Твоїх охоронити, але й всяким благодіянням збагатити і спасти. Бо іншої заступниці в бідах та ворожих нападах і щирої Молитвениці за нас грішних до Сина твого, Христа Бога нашого, крім Тебе, не маємо. Його ж ублагай, Владичице, спасти нас і Царства Небесного сподобити. Щоб ми, спасенні Тобою, Тебе, як Помічницю в нашому спасінні, в майбутньому віці славили і величали Всесвяте і Величне ім’я Отця, і Сина, і Святого Духа, в Тройці славимого та поклоняємого Бога, на віки віків. Амінь.

\section{Молитва до Пресвятої Богородиці на честь Її ікони «Леньковська» аво «Спасителька потопаючих»}
Щира Заступнице, Мати Господа Вишнього! Ти всім християнам, особливо тим, що у бідах перебувають, — допомога та заступництво. Зглянься нині з висоти святої Твоєї і на нас, що з вірою поклоняємось Пречистому Образу, і яви, молимо Тебе, швидку допомогу Твою тим, що по морю подорожують і від сильних вітрів потерпають. Спонукай усіх православних християн, на спасіння тих, хто у водах потопає, і подавай усім, що вправляються у цьому, багаті милості і щедроти Твої. Нині бо, на образ Твій споглядаючи, Тобі, що з милості Своєї з нами перебуваєш, смиренні наші молитви приносимо. Бо не маємо ні іншої допомоги, ні іншого заступництва, ні утіхи, крім Тебе, Мати всіх скорботних і терплячих напасті. Ти у Бозі наша Надія та Заступниця і тому ми, на Тебе надіючись, самі себе, і один одного, і все життя Тобі віддаємо на віки віків. Амінь.

\section{Молитва до Пресвятої Богородиці перед Її іконою «Пом’якшення злих сердець»}
Пом’якши злі серця наші, Пресвята Богородице, і напади тих, що ненавидять нас, угамуй, і всяку тісноту душі нашої розріши. На Твій святий образ споглядаючи, Твоїм стражданням і милосердям до нас зворушені, рани Твої цілуємо і стріл наших, що зранюють Тебе, жахаємося.

Не дай нам, Мати Благосерда у жорстокосерді нашому й від жорстокосердя ближніх загинути, Ти бо воістину злих сердець пом’якшення.

\emph{(Молитву читають при нападах роздратування чи неприязні до нас ближніх)}
\end{document}