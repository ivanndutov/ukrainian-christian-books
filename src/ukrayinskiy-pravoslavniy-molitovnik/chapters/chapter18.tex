\documentclass[chapters.tex]{subfiles}

\begin{document}
\chapter{Принагідні молитви}
\section{Молитва перед їжею}
Перед їжею читай «Отче наш» або: Очі всіх на Тебе, Господи, уповають з надією, бо Ти даєш поживу всім вчасно, простягаєш Свою щедру руку і всіх насичуєш милістю Своєю. Амінь.

\section{Молитва після їжі}
Дякуємо Тобі, Христе Боже наш, що наситив нас земними Твоїми благами. Не позбав нас і Небесного твого Царства, але, як прийшов Ти посеред учнів Твоїх, Спасе, мир даруючи їм, прийди до нас і спаси нас.

\section{Молитва на Новий рік}
Господи Боже, всього видимого й невидимого творіння Сотворителю, Ти заснував часи і роки: Сам і нині благослови початок нового року, що ми відлічуємо від Твого втілення заради нашого спасіння. Благослови провести цей рік і багато років після нього в мирі та злагоді з ближніми нашими; зміцни й пошир Церкву Православну, що Сам її заснував і спасенною жертвою на хресті освятив. Вітчизну нашу піднеси, охорони і прослав; довгоденство, здоров’я, достаток плодів земних і добре поліття дай нам; мене, грішного раба Твого, всіх рідних і ближніх моїх та всіх православних християн, доглядай, охороняй і на путь спасенну направляй, щоб ми, йдучи нею, після довгочасного й щасливого життя на цьому світі досягли Царства Твого Небесного і удостоїлися вічного блаженства з усіма святими Твоїми. Амінь.

\section{Молитва перед початком будь-якої конкретної справи}
\emph{Коротко:} Господи благослови.

\emph{Або:} Господи Ісусе Христе, Сину Єдинородний Безначального Твого Отця, Ти сказав Пречистими Устами Своїми: без Мене не можете робити нічого. Господи, Боже мій, вірою приймаючи в душу і серце слова Твої, припадаю до Твоєї благості: допоможи мені грішному цю справу, що я починаю, як для Тебе Самого робити, в ім’я Отця і Сина, і Святого Духа. Амінь.

\section{Молитва перед початком будь-якої доброї справи}
Милосердний Боже, Отче і Господи, дай мені жадати всім серцем тільки того, чого потребує від мене Твоя свята воля. У всякому ділі дай мені силу шукати перш за все виконання святих Твоїх заповідей. Настав мене на Твою праведну путь і навчи у всякій справі не сходити з неї. Створи моє життя таким, щоб я міг робити все з волі Твоєї, дай мені силу виконати Твої святі заповіді. На всіх стежках життя мого підтримай мої слабкі сили, оборони від лукавої спокуси і благослови на все добре, чисте і святе, щоб і словами уст моїх і всіма ділами життя мого я прославляв святе ім’я Твоє, Бога Великого, для нас, людей, Милосердного Отця, і Сина, і Святого Духа, нині, і повсякчас, і на віки віків. Амінь.

\section{Молитва після скінчення будь-якої доброї справи}
Всемогутній Боже, Милосердний Отче! Я знаю, що я негідний приносити мою подяку Тобі. Ти не потребуєш нічого від мене, знаючи мою нікчемність, мою бідність перед Тобою. Але Ти дав мені ласкаво Твою заповідь приносити Тобі славословлення і подяку мого вдячного серця.

Великий Боже, Милосердний Отче! Яку, достойну величности Твоєї, подяку можу скласти Тобі я, бідний і нікчемний грішник? Все ж, віруючи у Твою милість до мене, грішного, у Твою всепрощальну любов, я приношу Тобі в ім’я Сина Твого, нашого Спасителя і Визволителя, жертву хваління, жертву славословлення мого відданого Тобі і вдячного серця. Ти благословляєш мене у кожній хвилині життя мого, Ти настановляєш мене на все добре, чисте і святе. Ти підтримуєш мене і ведеш мене по дорозі Твоїй, Ти приводиш мене терпляче до вічного блаженства у Твоїм Небеснім Царстві, прощаючи з любов’ю мої провини.

Нехай же слава, честь і поклоніння Тобі, Богу Вічному, ніколи не замовкнуть у серці моїм, нехай воно славить Тебе, Бога Милосердного, в Тройці Святій Єдиного, Отця, і Сина, і Святого Духа, нині, і повсякчас, і на віки віків. Амінь.

\section{Молитва на початок Великого посту}
Господи Боже наш, надіє християн всіх країв землі і тих, хто сьогодні перебуває далеко від дому. Ти призначив святі дні посту в часи Старого Заповіту через пророків Твоїх, і в Новому — через Апостолів і Євангелистів. Сподоби ж усіх нас у чистоті час посту провести, віру тверду зберегти і Заповіти Твої виконувати на протязі усіх днів життя нашого. Благаємо Тебе, Владико Милосердний: пристав до нас Ангела Твого, щоб охороняв нас немічних у всіх ділах наших і допомагав нам, щоб ми були слухняними й догоджували найперше Тобі, та щоб сподобилися гідно причаститися Святих Твоїх Тайн.

Прийми, Господи, поклони наші і дотримування посту, слуг Твоїх (імена), і всім нам подай благословення через Христа Ісуса, Господа нашого, з яким благословенним є Ти, з Пресвятим, Милосердним і Животворним Твоїм Духом, сьогодні, і повсякчас, і на віки вічні. Амінь.

\section{Молитва про Божу мудрість}
Великий і Всемогутній Боже! Зішли на мене з високих і святих своїх Небес і від Престолу своєї святої Слави — Твою святу Мудрість, яка перебуває праворуч Тебе.

Дай мені Мудрість Твого уподобання, щоб я в житті умів усе те, що Тобі є милим: гаряче бажати, мудро шукати, у правді признавати і досконало все виконувати, на славу й честь Твого Святого Імені, «на хвалу Слави Твоєї Ласки».

Дай мені, Боже, Мудрість мого стану, щоб я все виконував, чого Ти бажаєш.

Дай мені Мудрість розуміти мої обов’язки, дай мені Мудрість моїх обов’язків і дай мені їх виконати так, як потрібно і як належить на Славу Твою і на користь для моєї душі.

Дай мені Мудрість Твоїх доріг і Мудрість ходити стежками Твоєї Святої Волі. Дай мені Мудрість поведінки: щоб я вмів не підніматися в очах одних і не упадати в очах інших.

Дай мені Мудрість радості і Мудрість смутку: нехай я тішуся лише тим, що до Тебе веде, нехай буду засмученим лиш тим, що від Тебе віддалює.

Дай мені Мудрість всього того, що проминає, і всього того, що триває: нехай перше в моїх очах маліє, а друге нехай росте.

Дай мені Мудрість праці та Мудрість відпочинку: нехай мені буде розкішшю — праця для Тебе, а втомою — відпочинок без Тебе.

Дай мені Мудрість щирого і простого наміру: Мудрість простоти, Мудрість щирості. Нехай серце моє звертається до Тебе і в усьому шукає Тебе протягом цілого життя.

Дай мені Мудрість послуху для Твоїх Законів, для Твоєї Церкви.

Дай мені Мудрість бідності, щоб я ніколи не оцінював добро інакше, ніж відповідно до їхньої дійсної вартості.

Дай мені Мудрість чистоти відповідно до мого стану та покликання.

Дай мені Мудрість терпеливості, Мудрість покори, Мудрість веселості і поваги, Мудрість Господнього страху, Мудрість правдомовності і добрих діл. Нехай я буду терпеливим без ніякого нарікання, покірним без найменшого лукавства, веселим без помірного сміху, поважним без суворості, щоб я був правдомовним без тіні двоязичності, нехай мої добрі діла будуть вільними від самозадоволення.

Дай мені Мудрість — ближніх у потребі нагадувати без гордості, дай мені Мудрість будувати словом і ділом без лицемірства.

Дай мені, Боже, Мудрість співчуття, уваги й обережності, нехай мене не зводить на бездоріжжя ніяка порожня думка.

Дай мені Мудрість благородності, нехай мене ніколи не принизить ніяке брудне й непристойне прив’язування.

Дай мені Мудрість правоти, нехай мене ніколи не зводить з дороги моїх обов’язків ніяке егоїстичне намірення.

Дай мені Мудрість відважності й сили, нехай мене не переможе ніяка буря.

Дай мені Мудрість свободи, нехай мене ніколи не поневолить ніяка насильна пристрасть!

\emph{(З цього місця прохання Мудрості відносяться найперше до людей, які посвятили своє життя на служіння Богові, Церкві і народові).}

Дай мені Мудрість богословських і моральних чеснот: віри, надії, любові, шляхетності, побожності, стриманості і відваги.

Дай мені, Боже, Мудрість апостолів, Мудрість мучеників.

Дай мені Мудрість священичу й душепастирську.

Дай мені Мудрість проповідників і вчителів.

Дай мені Мудрість служителів Святих Тайн.

Дай мені Мудрість Євхаристійну і Мудрість молитви та духовного зростання.

А найперше, Господи, дай мені Мудрість сердечного розкаяння і досконалого жалю.

Дай мені Мудрість пізнання себе в моїй немочі і злобі.

Дай мені Мудрість утихомирення тіла і посту.

Дай мені Мудрість відречення і пожертвування собою, дай мені Мудрість пожертви, Мудрість хреста, Мудрість терпіння.

Боже, дай мені ту Мудрість, яка згідно зі Святими Твоїми намірами, веде до з’єднання Церков під одним Вселенським пастирем.

Дай мені, Боже, Мудрість цю ідею святого з’єднання цінувати, любити, для нього й життя своє посвятити.

Дай мені Мудрість нашого східного обряду, його придержуватися, його відновлювати і розвивати.

Дай мені Мудрість Отців Святої Східної Церкви і всіх Великих Церковних Учителів.

Дай мені Мудрість великого Твого апостола Павла, щоб я його Послання добре розумів, пам’ятав та вмів їх пояснювати Твоєму народові.

Дай мені Мудрість великого Твого апостола Петра, щоб я розумів наміри Твого Божого провидіння, які управляють Церквою, дай мені Мудрість послуху їм та Вселенській Церкві. Дай мені Мудрість церковної історії та теології.

Дай мені ту Мудрість, якої мені і моєму народові найбільше не вистачає.

Дай мені Мудрість правдивого задоволення від життя і правдивого щастя. Амінь.

\section{Молитва перед читанням Євангелія}
Засвіти в серцях наших, Чоловіколюбче Владико, чисте світло Твого Богопізнання і очі розуму нашого розкрий на розуміння євангельської Твоєї науки. Дай нам страх Блаженних Заповітів Твоїх, щоб, перемігши всі тілесні пристрасті, ми проводили духовне життя, думали й чинили все так, як подобається Тобі. Бо Ти є світло душ і тіл наших, Христе Боже, і Тобі славу возсилаємо з Предвічним Твоїм Отцем, і Всесвятим, і Благим, і Животворним Твоїм Духом, сьогодні, і повсякчас, і на віки вічні. Амінь.

\section{Молитва при вході до церкви}
Милосердя двері відкрий нам, Благословенна Богородице, щоб ми, на Тебе надіючись, не загинули, але щоб визволилися Тобою від усякого лиха, бо Ти є спасіння роду християнського.

Це входжу я до дому Твого Святого, Господи, щоб поклонитися Твоїй Величності і вознести свою щиру молитву перед святим Твоїм Престолом, Владико світу. Стаючи перед Тобою, Господи, я приношу всі справи життя мого, а Ти щедрою рукою Своєю поблагослови мою щиру працю і мої добрі наміри і веди мене дорогою безпечною, щоб не похитнулися ноги мої. Я дякую Тобі, Отче Небесний, за всі щедрі дари, які подає нам батьківська рука Твоя. Я прославляю Тебе, Творче і Владико Неба і землі, як Радість життя нашого, як Світло, яке життя наше освічує, як Законоположника, що життя наше на праведну дорогу наставляє, як Суддю, який всім віддасть за їхні заслуги, як Доброчинця, який щедрою рукою допомагає тим, хто на Тебе надіється. Нехай же приємною буде Тобі молитва, наче кадило запашне, як щира жертва серця мого для Тебе, Господи. І Тобі я славу возсилаю, Отцю, і Сину, і Святому Духові, сьогодні, і повсякчас, і на віки вічні. Амінь.

\section{Молитва перед початком Святої Літургії}
Господи Ісусе Христе, Ти є Щедрим і Милостивим до всіх, хто до Тебе звертається.

Молюся до Тебе: дай мені зі щирістю, любов’ю, та страхом і всією увагою аж до кінця цієї Служби Божої пробути і з совістю та щирим серцем Тобі молитися, Милосердному Богові. Господи Царю, вислухай мене, що благаю Тебе, і прости мені гріхи мої, бо Ти один є Милосердним, і Тебе славлю з Отцем, і Святим Духом, на віки вічні. Амінь.

\section{Молитва подячна після завершення Святої Літургії}
Дякую Тобі, Боже, Спасителю мій!

Усією душею моєю і тілом, славлю і вихваляю Тебе, що сподобив мене, грішного, взяти участь у принесенні Твоєї Безкровної Жертви за наші гріхи і на спомин Твого земного життя, спасаючих мук, хресної смерті, преславного Воскресіння і Вознесіння на Небо. Молюся до Тебе: всі мої гріхи обмий, очисти і прости, і дай мені, на протязі всіх днів життя мого, творити Волю Твою і з чистою совістю приносити подяку й молитви Тобі з Предвічним Твоїм Отцем і Пресвятим, Благим і Животворним Твоїм Духом сьогодні, і повсякчас, і на віки вічні. Амінь.

\section{Молитва на прийняття просфори і святої води}
Господи Боже мій, нехай буде дар Твій святий: Просфора і Свята Твоя Вода, на прощення гріхів моїх, на просвітлення розуму мого, на здоров’я душі і тіла мого, на приборкання моєї пристрасті і моєї немічності, безмежним Милосердям Твоїм, молитвами Пречистої Твоєї Матері і Всіх Святих Твоїх. Амінь.

\section{Молитва за наших ворогів і недоброзичливців}
Спаси, Господи, і помилуй тих, хто ненавидить і кривдить мене, і спричиняє мені нещастя, і не дай їм загинути через мене грішного. Амінь.

\section{Молитва за навернення заблудлих}
Всевишній Боже, Владико і Сотворителю всього творіння, що все наповняєш Своєю величчю і утримуєш Твоєю силою! Тобі, Всещедрому в помочі Господу нашому, ми, недостойні, подяку приносимо, бо Ти не відвертаєшся від нас через наші беззаконня, але випереджаєш нас Своїми щедротами. Ти для нашого визволення послав Свого Єдинородного Сина і благовістив Твою безмежну милість до людського роду. Бо зі Своєї волі Ти милуєш і очікуєш, щоб ми до Тебе навернулися і спасіння отримали. Ти, зглянувшись над неміччю нашої природи, зміцнюєш нас всесильною благодаттю Святого Твого Духа, заспокоюєш спасенною вірою і досконалою надією на вічні блага і, скеровуючи Твоїх вибраних до горнього Сиону, зберігаєш їх як зіницю ока. Сповідуємо, Господи, Твоє велике та незрівнянне милосердя і чоловіколюбність. Але, бачачи як багато хто похитнувся, щиро молимо Тебе, Всеблагий Господи: зглянься над Твоєю Церквою і побач, як Твоє спасенне благовістя, яке хоча ми й радісно приймаємо, терня суєти і пристрасті робить в деяких мало плідним. У деяких і безплідним, а через примноження гріха одні, опираючись Твоїй євангельській істині єресями, інші розколами відходять від Твого насліддя, відкидають Твою благодать і впроваджують себе до суду Твого Пресвятого Слова. Премилосердний і Всесильний, не до кінця гнівайся, Господи! Будь милостивим до нас, зміцни нас Твоєю силою у правовірності; тим, що схибили, просвіти очі їхнього розуму Твоїм Божественним світлом, щоб вони зрозуміли Твою істину: пом’якши їхню лютість і відкрий їм слух, щоб вони пізнали Твій голос і навернулися до Тебе, нашого Спасителя. Виправи, Господи, тих, що розбрат чинять і життя невідповідно до християнського благочестя проводять. Зроби так, щоб ми всі свято і непорочно пожили, а спасенна віра в наших серцях укоренилася і стала плодоносною. Не відверни лиця Твого від нас, Господи, поверни нам радість спасіння Твого. Пастирям Твоєї Церкви подай, Господи, святу ревність і приправ євангельським духом їхнє піклування про спасіння і навернення заблудлих. Щоб ми всі так керовані потрапили туди, де виконання віри, здійснення надії та істинна любов, і там з ликами чесніших небесних сил прославимо Тебе Господа нашого, Отця, і Сина, і Святого Духа на віки віків Амінь.

\section{Молитва свт. Іоана Золотоустого на всяку годину дня і ночі}
Господи, не позбав мене небесних Твоїх благ. Господи, визволи мене від вічних мук. Господи, розумом чи думкою словом чи ділом згрішив я, прости мені. Господи, визволи мене від усякого незнання, забуття, легкодухості, закам’янілої нечутливості. Господи, визволи мене від усякої спокуси. Господи, просвіти моє серце, затьмарене злою похіттю. Господи, як людина я згрішив. Ти ж, як Бог щедрий, помилуй мене, знаючи неміч моєї душі. Господи, пошли благодать твою на поміч мені, щоб я прославив Ім’я Твоє Святе. Господи, Ісусе Христе, запиши мене, раба Твого, в книзі життя і подай мені кінець благий. Господи Боже мій, хоч нічого доброго не вчинив я перед Тобою, але дай мені з благодаті Твоєї покласти добре починання. Господи, зроси моє серце росою Твоєї благодаті. Господи неба і землі, пом’яни мене, грішного, мерзенного і нечистого раба Твого, у Царстві Твоїм. Господи, прийми мене в покаянні. Господи, не залишай мене. Господи, не дай мені впасти у спокусу. Господи, дай мені благі помисли. Господи, дай мені сльози каяття, пам’ять про смерть та упокоєння. Господи, дай мені твердий намір покаятись у гріхах моїх. Господи, дай мені смирення, чистоту тіла мого та послух. Господи, дай мені терпіння, великодушність і лагідність. Господи, вкорени страх Твій благий в серце моє. Господи, сподоби мене любити Тебе від усієї душі і в усьому виконувати волю Твою. Господи, захисти мене від лихих людей і бісів, і пристрастей, і від усього шкідливого для мене. Господи, твори за Твоїм бажанням все, що Ти хочеш, і нехай буде воля Твоя у мені грішнім, бо Ти Благословенний єси на віки. Амінь.

\section{Покаянне молитовне зітхання із творів прп. Єфрема Сиріна}
Ти, Кому благоугодні ті, що каються, вияви Своє благовоління і до мене, грішника. Насити мене крихтами з великої трапези Твоєї, не попусти, щоб життя моє загинуло у темряві ліворуч Тебе; і правда Твоя нехай не спогляне на страшні нечистоти моєї убогості того великого ранку, коли проголошено буде вічний присуд.

Радість світу цього гірка; горе тому, хто спокушається нею! Наче корабель хвилями, поборюється життя моє бідами; суєтна радість полонить його видовищами задоволень. Ти, Господи, будь керманичем моїм і приведи корабель мій до Своєї пристані того великого ранку Твого, коли проголошено буде вічний присуд.

Бог любить грішника, коли той приступає до покаяння і з повними очима сліз зітхає та, ридаючи, взиває до Нього: Господи наш, визволи мене від вогню! Благаю Тебе: прийми сльози моєї убогості. Добровільно грішив я перед Тобою, але добровільно і каюся.

Отож, сміливо приступи, грішнику, бо двері вже відчинені і готові прийняти тебе. Принеси Господу в жертву сльози і вільно йди до Нього. Не вимагає Він дарів і зважання на особи немає у Нього. Милосердний Він до людей і охоче прощає гріхи грішникам, що каються.

\section{Молитва оптинських старців}
Отче, дай мені з душевним спокоєм зустріти все, що принесе мені цей день. Дай мені цілком віддатися волі Твоїй святій, в усякий час цього дня, в усьому настав і підтримай мене. Які б я не одержав звістки протягом дня, навчи мене сприйняти їх із спокійною душею і твердим упевненням, що на все свята воля Твоя. В усіх словах і справах моїх керуй моїми думками й почуттями. В усіх несподіваних випадках не дай мені забути, що все Ти послав. Навчи мене щиро й розумно обходитися з кожною людиною, нікого не соромлячи й не засмучуючи. Отче, дай мені сили знести втому прийдешнього дня та всі події сьогодення. Керуй моєю волею і навчи мене молитися, вірити, надіятись, терпіти, прощати й любити. Амінь.
\end{document}