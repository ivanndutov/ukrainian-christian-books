\documentclass[chapters.tex]{subfiles}

\begin{document}
\chapter{Природа -- дар Божий}
\section{Загальна молитва за відвернення стихійних нещасть}
\emph(вогню, голоду, пошесті, повені посухи тощо)

О, Всемогутній Творче і Господи, Небесний Владико, який всім світом управляєш і в Своїй правиці всю владу тримаєш, будь до нас милостивим. Прихилися до молитви нашої щирої й охорони нас, Боже, від усілякого зла та нещастя. Охорони, Всеблагий, наші здобутки: і наші будинки, і наше майно, і господарство, і нас самих від вогню всезнищуючого. Охорони наш народ український від голоду страшного й від пошесті та мору великого. Охорони ниви й луки наші, Господи, від граду, бурі несподіваної, від зливи й повені і від посухи всеруйнівної. Дай нам, Господи Милосердний, нехай збережуться від цих стихій, наші будинки й наше майно, і все наше господарство, і самі ми неушкодженими. Нехай не забракне нам хліба святого для людей і корму для худоби, нехай не загинемо смертю страшною в хворобах заразливих. Нехай ниви, городи й луки наші красуються багатим урожаєм у добру весну, спокійне літо й гарну осінь, щоб могли ми в здоров’ї та спокої жити, працювати й плоди своєї праці збирати, Тебе, Небесного Владику, прославляти й Тобі єдиному поклонятися на віки вічні. Амінь.

\section{Молитва захисту від грому і блискавиць}
Господи, Боже наш, що утверджуєш громи й перетворюєш хуртовину і все чиниш на спасіння наше, не погорди творінням рук Твоїх, що до Твого Чоловіколюб’я прибігають, але визволи нас від усілякого смутку і наступаючого лиха. Загримів Ти з Небес як Господь блискавки послав і тим засмутив нас. За зміною погоди до Тебе прибігаємо, благаючи: Пошли нам сонячні щедроти Твої, Господи, і помилуй слуг Твоїх. Бо Ти Благий і Чоловіколюбний і Тобі славу віддаємо: Отцю, і Сину, і Святому Духові, сьогодні і повсякчас, і на віки вічні. Амінь.

\section{Молитва за гарну погоду}
Боже, Всемогутній Отче, Ти в мудрості й любові створив усяке Своє створіння, Ти доручив людині увесь світ, щоб служачи Тобі Самому, вона управляла всіляким створінням. Вислухай наші молитви й відверни від нас негоду, град, повінь, і засуху та все те, що нам шкодить. Дай нам всього того, що потрібно для життя. Нехай нас благословить Всемогутній Бог і пішле нам добру погоду, а наші серця будуть прославляти Його, через Ісуса Христа, Господа нашого. Амінь.

\section{Коли просимо дощу}
Боже, через Тебе ми живемо, рухаємося й існуємо. Ти найкраще знаєш, чого нам потрібно для життя. Дай нам дощу, якого так потребує наша висушена земля.

Хліб наш щоденний дай нам і цього року, щоб ми, радіючи земним життям, і ще більшою довірою прагнули поживи, яка дає життя вічне, через Ісуса Христа, Господа нашого. Амінь.

\section{Молитва у час посухи}
Владико, Господи Боже наш. Ти виконав благання пророка Іллі Фесвітянина заради любови його до Тебе: стримав дощ, що посилався тоді на землю, і знову після його молитви дощ плодоносний землі дарував. Сам, Владико всіх, щиро благаємо, — з милости Твоєї пошли дощ рясний людям Твоїм і, простивши гріхи наші, дай дощу Твого на місця, де просять і потребують його; обрадуй землю заради нещасних людей Твоїх, і дітей, і худоби, і всього іншого живого творіння, бо всі від Тебе очікують, що Ти даси їм їжу в свій час. Бо Ти — Бог наш; Бог, що милує і спасає, і Тобі славу возсилаємо, Отцю, і Сину, і Святому Духові, нині, і повсякчас, і на віки віків. Амінь.

\section{Молитва в час негоди (безперервного дощу)}
Владико, Господи Боже наш. Ти колись виконав молитву люблячого Тебе пророка, ревнителя Твого Іллі і благословив на час спинити дощ. І тепер, Чоловіколюбний Творче і Милостивий Господи, прийми й від нас, смиренних і негідних рабів Твоїх, покірні молитви, що їх приносимо Тобі, і, як щедрий, прости наші провини і з чоловіколюбства Твого вислухай наші благання, пошли ясну погоду людям Твоїм і світло сонячне всім, що потребують і просять милости в Тебе; обрадуй землю і задля нещасних людей Твоїх, дітей, худоби і всього іншого живого творіння, що їх Ти годуєш з милости Твоєї і посилаєш їм їжу в свій час. О, Господи Боже наш, виконай благання наше і не осором надії нашої, а помилуй нас з милости Твоєї й пошли нам щедроти Твої, бо наші дні швидко минають, і в стражданнях життя наше вкорочується. Не згуби нас за беззаконня наші, якими ми заслужили Твій гнів і кару, але вчини з нами по Твоїй любові й великій милості Твоїй, бо з душею стурбованою і духом засмученим до Тебе припадаємо і, як раби негідні, ще більшої кари варті, з каяттям жалібно благаємо Тебе: грішили ми, беззаконня чинили і у всьому гріховно поводилися, порушуючи Твої заповіді, а тому все, що Ти посилаєш на нас, посилаєш по справедливості й правді. Але не віддавай нас на знищення до кінця, голод і загибель, і щоб не затопила нас бурхлива вода, — згадай у гніві про ласку і будь милостивим заради щедрот Твоїх, помилуй з милосердя Твого творіння і діло рук Твоїх і від усякого зла скоро визволи. Бо Тобі одному, Боже наш, належать милість і спасіння наше, і Тобі славу возсилаємо, Отцю, і Сину, і Святому Духові, нині, й повсякчас, і на віки віків. Амінь.

\section{Молитва на нивах}
Господи Боже наш, який на початку Свого Творення створив небо і землю! Небо Ти прикрасив світилами, щоб вони освітлювали землю, і через них пробуджувався подив до Тебе, Єдиного Творця і Владики всього творіння. Землю ж Ти прикрасив рослинами і травою, і розмаїттям полів, засіяних кожне окремим видом насіння. Ти всю її облагородив красою квітів та поблагословив. І сьогодні, Владико, Ти, зі святої оселі Своєї, милостиво поглянь на цю ниву і поблагослови її, і збережи неушкодженою від усякого чаклування і ворожби. Захисти від усілякого зла, від недоброї цікавості і підступів злих людей, дай цій ниві у свій час приносити щедрі плоди, сповнені Благословення Твого. Віджени від неї всякого звіра, гусінь і злих комах, хвороби рослинні, спеку й засуху, та надмірність вітрів, які приносять їй шкоду. Бо святим і прославленим, і величним є ім’я Твоє, Отця, і Сина, і Святого Духа, нині, і повсякчас, і на віки вічні. Амінь.

\section{Молитва на благословення садів і полів}
Всемогутній, предвічний Боже, Отче милосердя! Поглянь, благаємо Тебе, на наші щоденні потреби й для нашого пожитку рясно подай нам добрий урожай. Возвелич над нами Ім’я Твоє святе та на поля і на сади наші вилий прещедре Твоє благословення, щоб усі труди слуг Твоїх, садові дерева і польові посіви наші, приносили щедрі плоди. Відверни, Милосердний, від нашого краю несприятливе поліття, невчасний мороз, руйнівний град, надмірну зливу, затоплення посівів і всілякого лиха, що може пошкодити людському майнові. Бо Ти є Бог наш, ми Тобі славу віддаємо, Отцю, і Сину, і Святому Духові, сьогодні, і повсякчас, і на віки вічні. Амінь.

\section{Молитва на збереження засіяного поля}
Всемогутній Боже, ми звертаємося до Твого милосердя, щоб на це зерно, яке Ти виплекав дощем і повітрям, та якому дозволив так рясно прорости, Ти зіслав Своє благословення. Зроби Владико так, щоб ми завжди Тобі дякували за Твої щедрі дари. Відверни спраглі наші душі від непліддя землі, щоб бідні й заможні, недужі й здорові завжди прославляли ім’я Твоє. Подай нашій землі щедрий урожай, сторицею збільш її плоди, щоб, прийшовши у відповідний час, ми з радістю і подякою зібрали наш урожай і прославляли Тебе, Всемогучого Бога, Отця і Сина, і Святого Духа, сьогодні, і повсякчас, і на віки вічні. Амінь.

\section{Молитва за добрий урожай}
Всемогутній Боже, Сотворителю світу й Владико життя, Ти підтримуєш усе, що існує, Своєю Силою, Ти — наш Отець і знаєш найкраще, чого нам потрібно для життя. Дозволь земним плодам добре зростати й дай нам добрий урожай. Нехай урожайною землю зроблять Твої дощі і сонце. Бережи наші поля й луки, городи й ліси від негоди, надмірних дощів, граду, засухи. Благослови працю наших рук і розуму, щоб ласка Твоя нас духовно збагачувала. Нехай нас не турбують клопоти й тривоги, а наші серця нехай збирають скарби у Бога. Нехай на нас, на нашу працю зійде Благословення Всемогутнього Бога Отця, і Сина, і Святого Духа, і залишається з нами назавжди. Амінь.

\section{Подяка за добрий урожай}
Всемогутній Боже і Найкращий Отче! Ти дозволяєш людям обробляти землю і збирати на ній щороку нові плоди. Ти доручив синам вибраного народу приносити до священиків плоди землі та снопи зібраного збіжжя. Прийми подяку за цьогорічний добрий урожай збіжжя, овочів та фруктів. Господи, зроби так, щоб у цьому році всім вистачило їжі. Допомагай Своєю благодаттю всім тим, хто працював на землі, та навчи нас ділитися Твоїми дарами з усіма потребуючими. А після земного життя щоб з повними оберемками добрих учинків прийшли до Тебе, і заслужили вічного щастя на небі. Амінь.

\section{Молитва на благословення плодів з нового урожаю}
Господи Боже, Сотворителю всесвіту, Ти даєш росу з неба й удобрюєш землю, щоб вона родила щедрий урожай. Ти, даючи добру погоду і животворний дощ, чиниш так, що земля приносить щедрі плоди. Твоїй святій величі складаємо подяку за ці зібрані плоди землі. У Своїй милості поблагослови ці плоди і вчини так, щоб Твої люди завжди складали подяку за отримані від Тебе дари. Через ці дари Своєї ласки Ти сповнив прагнення Твоїх вірних. Вчини так, щоб ми достойно прославляли Твоє милосердя і, використовуючи земні багатства, завжди шукали вічного добра. Нехай убогі й нужденні, споживаючи щедрі плоди урожайної землі, прославляють Твоє ім’я, через Ісуса Христа, Господа нашого. Амінь.

\section{Молитва господаря за худобу}
Преблагий, Боже, який Провидінням Своїм Святим дбаєш про всіх тварин на землі й про найдрібнішу комашку, молюся до Тебе, Господи, дай мені дочекатися достатку з моєї праці, бо вона є годувальницею моїх дітей, і не дай зменшитися йому, ні від хвороби тваринної, ні від злих людей, ні від нещасного випадку. Молю Тебе і низький поклін Тобі, Єдиному возсилаю, Отцю, і Сину, і Святому Духові, нині, і повсякчас, і на віки вічні. Амінь.

\section{Молитва на благословення стада}
Владико, Господи Боже наш, який маєш владу над кожним творінням. Тобі ми молимося і Тебе просимо: як поблагословив Ти і примножив стада патріархові Якову, так поблагослови і цю худобу, стада Твого слуги (ім’я). І розмножуючи стадо це, зроби його численним і сильним, збережи його від влади диявола, від всяких заздрощів, від нападу розкрадачів, від усіляких ворожих підступів, від смертоносного повітря і від згубних хвороб, прибери від нього всяку неміч та захисти його Твоїми святими ангелами. Бо Твоє є Царство, і Сила і Слава, Отця і Сина, і Святого Духа, нині і повсякчас, і на віки вічні. Амінь.

\section{Молитва робітника, господині}
Господи Ісусе Христе, Ти Сам показав нам приклад, через важку працю тесляра, як ми повинні працювати, прийми й освяти мою працю. Амінь.

\section{Молитва подяки за отримані ласки}
Господи Святий, Отче Всемогутній, Предвічний Боже, від якого походить всякий дар і всяке добро! Дякую Тобі за всі Твої добродійства, яких так багато подарував Ти мені, грішному і недостойному (грішній і недостойній), та сердечно молюся до Тебе: так як тепер Ти милостиво вислухав мої молитви і виявив до мене Свою доброту, так і назавжди виконуй мої добрі прохання і виявляй через мене велику милість Твою. Бо Ти щедрий і чоловіколюбний Бог є, і я Тобі славу возсилаю, Отцю, і Сину, і Святому Духові, нині, і повсякчас, і на віки вічні. Амінь.
\end{document}