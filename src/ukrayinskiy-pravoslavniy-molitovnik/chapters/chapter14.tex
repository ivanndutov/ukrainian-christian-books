\documentclass[chapters.tex]{subfiles}

\begin{document}
\chapter{Молитви для тих, хто навчається}
\section{Молитва за дітей, які починають навчатися}
Господи Ісусе Христе! Ти радів, коли до Тебе приносили чи приводили дітей, Ти пригортав їх до Себе і благословляв. Просимо Тебе, прийми й наші щирі молитви, відкрий розум цих дітей до науки і щедро поблагослови початок їхнього навчання. Ти кличеш їх до Себе, то ж оточи їх Своєю Любов’ю та Опікою. Вчини так, щоб вони зростали в мудрості та у Твоїй Благодаті.

Допоможи їхнім батькам та вчителям, поблагослови їхні старання та працю. Нехай над цими дітьми завжди буде Твоє Благословення. Амінь.

\section{Молитва на початок навчального року}
Боже, вислухай наші прохання, які ми з глибокою вірою підносимо до Тебе через найкращого Учителя Ісуса Христа, нашого Господа, який Живе і Царює в Єдності Святого Духа. Нехай Всемогутній Бог благословить учнів і провадить їх у новому навчальному році так, щоб вони пізнали Його Мудрість і старанним навчанням добре готувалися до виконання свого життєвого покликання. Нехай Ісус Христос, Божий Син і Єдиний їхній Учитель, освітлює їх Своїм Словом, щоб вони у новому навчальному році йшли за Ним і зростали в Його Дружбі і Любові, а Дух Святий, який від Хрещення постійно їх провадить, дав їм мужність і витривалість, щоб вони словом і прикладом привертали до Ісуса Христа ще й інших людей.

Нехай усіх їх благословить Всемогутній Бог Отець, і Син, і Святий Дух. Амінь.

\section{Молитва учня (учениці)}
Мій любий Ісусе! З цілого серця дякую Тобі за Ласку, що можу ходити до школи. З вдячності за це обіцяю добре навчатися. Обіцяю завжди бути чемним (чемною) і слухняним (слухняною) перед моїми вчителями, бо знаю, що й Ти, любий Ісусе, зростаючи в Назареті, був пильним, уважним і слухняним Своїй Матері та опікунові, святому Йосифові. Благослови мене, Любий Спасителю! Твоєю Ласкою допоможи мені, щоб я виконав (виконала) мої постанови, щоб я добре навчався (навчалася) і тим приносив (приносила) Тобі більшу славу, батькам потіху, українському народові честь, а собі, як для душі, так і для тіла, користь. Амінь.

\section{Молитва за підлітка, який погано навчається}
Господи Ісусе, Христе Боже наш, Ти, що благодаттю Всесвятого Духа (Який у вигляді вогняних язиків зійшов) істинно вселився у серця дванадцяти апостолів, та відкрив їхні вуста і почав говорити іншими мовами: Сам, Господи Ісусе, Христе Боже наш, пошли того Твого Святого Духа на цього отрока (ім’я) і насади у вухах його серця святі заповіді, які Твоя пречиста рука законодавцю Мойсею накреслила на скрижалях, нині й повсякчас, і на віки віків. Амінь.

\section{Молитва перед навчанням}
Милосердний Господи, пошли нам благодать Духа Твого Святого, що подає розум і зміцнює духовні сили наші, щоб ми, уважно переймаючи науку, виросли Тобі, Творцеві нашому, на славу, батькам нашим на радість, Церкві й Україні на користь. Дякуємо Тобі, Боже, Творцеві нашому, що Ти сподобив нас ласки Твоєї, щоб розуміти навчання. Благослови наших начальників, батьків та вчителів, що ведуть нас до пізнання добра, і пошли нам розум та силу продовжувати науку нашу. Амінь.

\section{Молитва свт. Іоана Золотоустого про дарування розуму}
Дай, Господи, мені недостойному милість розуміння, щоб розпізнавати все те, що є приємним для Тебе, а корисним для мене, і не тільки розпізнавати, але й виконувати, щоб я не захоплювався і не займався непотрібним, щоб співчував стражденним і розумів грішників. Амінь.

\section{Молитва допомоги}
\emph{(Якщо не знаєш як поступити в даній ситуації)}

Господи, я є людиною грішною і не розумію, як зараз потрібно мені поступити, але Ти, Милостивий, підкажи мені. Амінь.

\section{Молитва за вибір стану}
Господи, Творче всіх людей, Ти створив Ангелів і людей, щоб перші в Небі, а другі на землі служили Тобі. Ти установив різні стани життя і хочеш, щоб кожен добровільно вибрав собі такий стан, який найкраще відповідає його фізичним, моральним і духовним здібностям. Всі життєві стани і фахи стоять переді мною, але я не знаю, до котрого Ти хочеш покликати мене. Ласкавий Отче, Боже Непомильного Світла, просвіти мій слабенький розум і покажи, в якому стані бажаєш, щоб я служив (служила) Тобі та спас (спасла) свою безсмертну душу. Як тільки довідаюся про Твою Волю, то без вагання піду за Твоїм Покликом. Господи, промов до мого серця і просвіти мене. Припадаю до стіп Твоїх Господи і чекаю на Твоє Святе і Непомильне вирішення. Амінь.

\section{Молитва за випускників навчальних закладів}
Боже, наш Найкращий Отче, у Тайні Хрещення і Миропомазання Ти зробив їх Своїми дітьми, у Тайні Покаяння Ти формував їхні характери. Ти дозволив їм черпати силу зі Святої Літургії і через Тайну Євхаристії покликав їх прийняти відповідальність за справи Божого Царства. За все це дякуємо Тобі, Господи. Ще просимо Тебе за молодь, яка сьогодні завершує навчання і розпочинає новий етап свого життя, щоб наші діти щоразу відважніше долали диявольські пастки і нахили до поганого. Вислухай, Господи, наші прохання і зміцни Своїм Благословенням добру волю молоді й турботу батьків за них, бо Ти живеш і Царюєш на віки вічні. Амінь.
\end{document}