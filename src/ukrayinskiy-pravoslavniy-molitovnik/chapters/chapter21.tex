\documentclass[chapters.tex]{subfiles}

\begin{document}
\chapter{В дорогу -- з Богом}
\section{Молитва на благословення прочан}
Всемогутній Боже, до тих, хто любить Тебе, Ти завжди виявляєш Своє Милосердя і в жодній країні не перебуваєш далеко від тих, хто Тебе шукає.

Будь зі слугами Своїми, які розпочинають цю прощу, і керуй їхньою дорогою згідно з Твоєю Волею. Нехай удень їх береже Твоя Спасаюча Опіка, а вночі освітлюй їх сяйвом Своєї Благодаті, щоб, зміцнені Твоєю присутністю, вони щасливо дійшли до мети своєї прощі, з радістю повернулися додому, словом і життям проповідуючи всім про Твої чудеса, через Ісуса Христа, Господа нашого. Амінь.

\section{Молитва за того, хто збирається в дорогу}
Господи Боже наш, Ти є Істиною й Живою Дорогою, Ти подорожував зі слугою Твоїм Йосифом, подорожуй, Владико, і зі слугою Твоїм (ім’я). Від усякого буревію й напасті збережи його, мир та Свою могутність влаштуй навколо нього, щоб він завжди поступав правдиво та згідно Заповітів Твоїх. Нехай він наповниться життєвих і Небесних дарів, і з Твоїм благословенням назад повернеться. Бо Твоє є Царство, і Сила, і Слава, Отця, і Сина, і Святого Духа, сьогодні, і повсякчас, і на віки вічні. Амінь.

\section{Молитва благословення початку подорожі}
Всемогутній Вічний Боже! Ти наказав Авраамові вийти з його землі та з рідного дому, Ти оберігав його безпеку на всіх дорогах його мандрування. Ти провів синів Ізраїля через Червоне море так, що вони й ніг не замочили, а під проводом зірки Ти показав дорогу мудрецям, які йшли до Твого Сина. Господи, будь нам допомогою в потребі, товаришем і втіхою в дорозі, захистом у злиднях, щоб ми під Твоїм проводом щасливо досягнули нашої мети і повернулися до наших домівок, а колись причалили до порту вічного спасіння. Нехай Бог обдарує нас усяким благословенням з Неба, нехай щасливо покерує нашими дорогами, щоб ми серед мінливостей цього світу завжди впізнавали Його турботу, через Христа, Господа нашого. Амінь.

\section{Благословення в дорогу}
\emph{(Кондак 6, акафіст до св. Миколая)}

Весь світ вихваляє тебе, преблаженний Миколаю, швидкого заступника при бідах. Бо багато разів по землі мандруючим і по морю плаваючим, попереджуючи, допомагаючи, разом всіх від злого охороняв ти, тому співаємо до Бога: Алилуя.

\section{Молитва подорожуючих}
Коли Апостоли Твої, Ісусе Христе, були на морі під час бурі, Ти прийшов до них, і сказав: «Не бійтеся», — і втихомирив бурю та привів їх спокійно до берега. Так, Господи, і мене бережи в моїй подорожі, будь помічником і захисником для мене перед усіма небезпеками, допомагай мені в труднощах і охороняй від усіляких негараздів, які можуть мені зустрітися в дорозі. Я передаю себе від Твій Всемогутній Покров, а Ти веди мене Своєю Всесильною Рукою і приведи до щасливого завершення моєї дороги. На Твою допомогу я сподіваюся і до Тебе безперестанно буду звертати очі свої. А коли небезпека наблизиться до мене, Ти захисти мене і приведи мене до місця безпечного і землі праведної. Амінь.

\section{Молитва перед виходом з дому}
Відрікаюся від тебе, сатано, від твоєї гордості і служінню тобі, і з’єднуюся з Тобою, Христе, в ім’я Отця, і Сина, і Святого Духа. Амінь.

\emph{(Під час виголошення славослів’я потрібно перехреститися).}

\section{Молитва водія до святителя Миколая}
Боже Всеблагий і Всемилостивий, Який все охороняєш Своєю Милістю і Чоловіколюбством, смиренно молюся до Тебе: через заступництво Пресвятої Богородиці, святителя Миколая і всіх Святих, охорони від усіляких неприємностей і раптової смерті мене, грішного, і довірених мені людей і допоможи неушкодженими доставляти кожного згідно їхньої потреби. Боже Милостивий! Звільни мене від злого духу порушення правил їзди і алкоголю, які викликають нещастя і наглу смерть без покаяння. Спаси і допоможи мені, Господи, з чистою совістю дожити до глибокої старості, без тягара провини за вбитих і покалічених через мою байдужість людей, і нехай прославиться Ім’я Твоє сьогодні і, повсякчас і, на віки вічні. Амінь.

\section{Перша молитва на благословення автомобілів (або інших транспортних засобів)}
Боже, Ти силою Свого Слова створив цілий Всесвіт, дав людині владу над усім створінням, а також силу, щоб вона могла вдосконалювати світ згідно з задумами Твоєї Волі. Вчини так, щоб ці автомобілі (назва транспорту), витвір Твій й людської праці, завдяки Твоєму Благословенню служили для добра людини, для її праці та відпочинку. Як ефіопу, який читав Святе Писання, перебуваючи в колісниці, Твій слуга Филип сповістив Добру Новину, й охрестив його, так і Твоїм вірним слугам покажи дорогу спасіння, щоб, підтримані Твоєю Благодаттю, вони робили добрі вчинки і після всіх трудів земної мандрівки заслужили вічного щастя.

Вислухай наші прохання і поблагослови ці автомобілі (назва транспорту) та їхніх водіїв. Нехай їх супроводжують і бережуть від усіляких небезпек: Пресвята Діва Марія, яка поспішала гірською стежкою до своєї родички Єлизавети, святий Рафаїл і Святі Ангели. Через заступництво Святого Христофора, покровителя водіїв, збережи, Господи, всіх тих, хто їздитиме цими автомобілями (назва транспорту), від усіляких нещасть для душі й тіла, щоб вони могли безпечно доїхати до місця призначення.

Боже невичерпного добра! Поблагослови ці автомобілі (назва транспорту), творені людьми, щоб вони їм служили, і вчини так, щоб усі, хто буде ними користуватися, робили це на славу Тобі, для власної користі та користі своїх близьких, через Ісуса Христа, Господа нашого. Амінь.

\section{Друга молитва на благословення автомобілів (або інших транспортних засобів)}
Всемогутній Предвічний Боже, Ти створив усі сили природи на Свою славу і на користь для людей, Тобі усердно молимося: зішли з Небес Своє Благословення на цей автомобіль (назва транспорту) і на все приладдя його. Твоєю вседоброю запобігливістю збережи його цілим та неушкодженим, допомагай слугам Твоїм, які будуть на ньому подорожувати, і пошли їм Благодать Твою, щоб виконуючи Закони Твої тут на землі і дороги Заповітів Твоїх зберігаючи, змогли щасливо зайти до Небесної батьківщини.

Владико, Господи Боже наш, вислухай молитви наші, які сьогодні Тобі приносимо, і правицею Твоєю Святою благослови автомобіль (назва транспорту) цей, пошли твоїх Ангелів Хоронителів, щоб усі, що схочуть на ньому подорожувати, були збереженими і захищеними від усіляких злих випадків. Бо Твоя є влада, і Твоє є Царство, і Сила, і Слава, і Тобі славу віддаємо, Отцю, і Сину, і Святому Духові, нині і повсякчас, і на віки вічні. Амінь.
\end{document}