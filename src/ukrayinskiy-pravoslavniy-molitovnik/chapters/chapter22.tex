\documentclass[chapters.tex]{subfiles}

\begin{document}
\chapter{Мій дім -- моя фортеця}
\section{Молитва на початок будівництва помешкання}
Боже Вседержителю, Ти премудро створив Небеса й укріпив землю в рівновазі. Боже, Творче і Будівничий всього Всесвіту! Милостиво зверни Свій погляд на слугу Твого (ім’я), який береться в Твоїй могутній державі будувати будинок для життя і здійснює сьогодні закладання його фундаменту. Закріпи цей будинок на основі міцного каменю так, щоб йому, згідно Твого Божого Євангельського слова, не міг би зашкодити ні вітер, ні вода, ні інші стихії чи негаразди. Благослови, Господи, щоб він був швидко доведеним до завершення, а тих, хто має намір жити в ньому, позбав усіляких ворожих підступів. Бо це є Твоя держава і Твоє Царство, і Сила, і Слава, Отця і Сина, і Святого Духа, нині, і повсякчас, і на віки вічні. Амінь.

\section{Молитва на благословення новобудови}
Всемогутній Боже, Отче Милосердя, Ти через Свого Сина все створив і встановив Його фундаментом Свого царства. Ти доручив людині працювати, то підтримай сьогодні наші починання. Ми пам’ятаємо слова з Псалму: «Коли Господь дому не збудує, то даремно працюють його будівничі при ньому». Поблагослови, Господи, цю новобудову, допоможи допровадити її до омріяної форми та оточуй опікою Своєю будівничих і захищай їх від усіляких небезпек. Просимо Тебе, вчини так, щоб, зміцнені даром Твоєї Предвічної Мудрості, ми змогли допровадити до щасливого завершення цю справу, яку сьогодні розпочинаємо, для нашої користі і на славу Твого Імені, через Ісуса Христа, Господа нашого. Амінь.

\section{Молитва на благословення помешкання}
Всемогутній Боже, щиро молимося до Тебе за будинок цей (квартиру цю) і за тих, хто живе в ньому (ній), і за майно, яке дозволиш поблагословити і всілякі блага в ньому примножити. Подай їм, Господи, достаток від роси Небесної та від земних благ — життя, і їхні бажання приведи Твоїм Милосердям до щасливого завершення. З нашим приходом дозволь будинок цей (квартиру) благословити, так як Ти благословив дім Авраама, Ісаака та Якова. Нехай же в стінах цього будинку (квартири) перебувають Ангели Твої і стережуть мешканців його (її). Бо Твоя є Сила і Слава, Отця і Сина, і Святого Духа, сьогодні, і повсякчас, і на віки вічні. Амінь.

\section{Молитва на благословення нового будинку}
Ти, Господи, Боже Всесвіту, поблагослови цей будинок, бо Твій Син — Ісус Христос, заради нас став людиною і залишив нам приклад, як ми повинні виконувати Волю Твою, навчив нас у кожній справі звертатися до Тебе. Поглянь з любов’ю на цей будинок і обдаруй його щедрим Своїм Благословенням. Вислухай молитви, які ми підносимо до Тебе разом з усіма домашніми, дай їм миру й радості у Святому Духові, підтримуй їх в усіх їхніх починаннях, захищай їх від усіляких небезпек, вбережи їх від оманливої надії на земні блага та навчи їх розуміти, що тільки Ти один — наша мета і що тільки Ти даєш вічне життя.

Сьогодні на будинок цей зійшло Благословення. Тож нехай мир і Боже Благословення, увійшовши в нього, наповнять серця всіх його мешканців.

Нехай вам дасть це Всемогутній Бог: Отець, і Син, і Святий Дух, на сьогодні, і на завжди і на віки вічні. Амінь.

\section{Молитва на благословення нової квартири}
Небесний Отче! Ти даєш нам багато доказів Своєї доброти, а тому наповни наші серця вдячністю за все те, що Ти для нас робиш. Поглянь з любов’ю на всіх тих, хто на Тебе покладає свою надію. Поблагослови цю квартиру і збережи її мешканців. Дай їм Твій мир, збережи їх від зла, дай їм усього того, чого потрібно для життя, і відкрий їхні серця для любові та на потреби ближніх. Нехай вони завжди пам’ятають, що наше помешкання на землі не є довговічним, що всі ми покликані до вічного перебування з Тобою в Небі, через Ісуса Христа, Господа нашого. Амінь.

\section{Молитва на благословення кабінету}
Боже, Ти є Джерело Правди, поблагослови цей кабінет і вчини так, щоб ми, працюючи в ньому, з вірою шукаючи правди і справедливості, любили Тебе щоразу більше, через Ісуса Христа, Господа нашого. Амінь.

\section{Молитва на благословення спальні}
Господи, поблагослови цю спальню, щоб усі домашні, які будуть відпочивати тут після праці, набиралися нових сил для виконання своїх обов’язків згідно з Твоєю Волею, через Ісуса Христа, Господа нашого. Амінь.

\section{Молитва на благословення кухні}
Всемогутній і Милосердний Боже! Ти даєш нам поживу для тіла і у всьому нас підтримуєш, тому просимо: вчини так, щоб ми за Твоїм Благословенням ніколи не страждали від голоду і вживали їжу з подякою та щоб були здоровими й досягнули вічного спасіння, через Христа, Господа нашого. Амінь.

\section{Молитва на входження до нового дому}
Боже, Спасителю наш, що зволив у оселю Закхеєву увійти і спасінням йому і дому його стати, Сам і нині тих, що тут жити бажають, і разом з нами, недостойними, прохання і молитви до Тебе підносять, від усякого лиха збережи неушкодженими, благословляючи їхнє житло і життя без осудження зберігаючи. Амінь.

\section{Молитва захисту помешкання від диявольського впливу}
Владико, молимо Тебе, завітай в цей будинок (цю квартиру) і прожени від нього далеко всі диявольські підступи. Нехай Твої Святі Ангели перебувають ньому і стережуть його в мирі, нехай на ньому завжди спочиває Твоє Святе Благословення.

Молимо Тебе, Владико, бережи цей будинок (цю квартиру) і від малого до великого мешканців його перед пожежею, бурею, гибеллю та знищенням, недугами та всілякими лихими і ворожими нападами, щоб це помешкання стало затишним для всіх тих, хто надіється на Тебе. Бо Тебе хвалять всі Сили Небесні і Тебе славлять всі створіння на віки вічні. Амінь.

\section{Молитва заклинання злих духів}
Боже і Володарю над володарями! Творче вогняних чинів! Владико безтілесних сил і Творче всього піднебесного!

Тебе ніхто з людей ніколи не бачив і бачити не спроможний, перед Тобою тремтять всі створіння. Ти скинув з Неба колишнього архистратига з його послідовниками, бо вони, впавши в гордість, переступили Твій Закон. Через свою злобу ангели — відступники стали бісами і Ти передав їх темряві безпросвітній у найдальших глибинах.

Допоможи нам, Господи, щоб це моє заклинання, що його чиню у Твоєму Імені, було грізним йому, володареві лукавства, та всім його помічникам, які спали з ним з висот, — і заміни їхній наступ на втечу.

Накажи йому Господи, відійти звідси, щоб надалі нічого шкідливого для нас не чинив. Але нехай ті, хто отримав Хрещення і люблять Тебе, прийняли від Тебе силу, яку Ти дав слугам Своїм — наступати на зміїв, скорпіонів і на всяку ворожу силу та не бути ушкодженим. Бо всякий дух оспівує, величає і славить зі страхом і тремтінням пресвяте ім’я Твоє: Отця, і Сина, і Святого Духа, нині, і повсякчас, і на віки вічні. Амінь.

\section{Молитва при виході з дому}
\emph{Перед тим, як переступити поріг будинку, при виході з нього, християнин, який боїться впливу нечистих духів, повинен перехреститися, промовляючи такі слова}: «Зрікаюсь тебе, сатано, гордині твоєї і служінню тобі, і з’єднуюся з Тобою, Христе, в ім’я Отця, і Сина, і Святого Духа, нині і повсякчас, і навіки віків. Амінь.»
\end{document}