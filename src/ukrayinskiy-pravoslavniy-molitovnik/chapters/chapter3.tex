\documentclass[chapters.tex]{subfiles}

\begin{document}
\chapter{Молитви зі всенічної}
\section{З вечірні}
Блажен муж, що не йде на раду нечестивих. Алилуя (тричі).

Бо знає Господь путь праведних, і путь нечестивих загине. Алилуя (тричі). Служіть Господеві зі страхом і радійте Йому з трепетом. Алилуя (тричі).

Блаженні всі, що надіються на Нього. Алилуя (тричі).

Воскресни, Господи, спаси мене, Боже мій. Алилуя (тричі).

Господнім є спасіння, і на людей Твоїх благословення Твоє. Алилуя (тричі).

Слава: Алилуя (тричі).

І нині: Алилуя (тричі).

Алилуя, алилуя, алилуя, слава Тобі, Боже.

Господи, взиваю до Тебе, вислухай мене; вислухай мене, Господи.

Господи, взиваю до Тебе, вислухай мене; вислухай, Боже, голос моління мого, коли буду молитись до Тебе. Вислухай мене, Господи.

Нехай піднесеться молитва моя, наче кадило перед Тобою, підношення рук моїх — як жертва вечірняя. Вислухай мене, Господи.

Світе тихий святої слави Безсмертного Отця небесного, Святого, Блаженного, Ісусе Христе, прийшовши на захід сонця, бачивши світло вечірнє, оспівуєм Отця, Сина і Святого Духа, Бога. Достойний Ти повсякчас оспіваним бути голосами преподобними, Сину Божий, що життя даєш, тому світ Тебе славить.

Сподоби, Господи, в вечір цей без гріха зберегтися нам. Благословений Ти, Господи, Боже отців наших, і хвальне і прославлене ім’я Твоє навіки. Амінь. Нехай буде, Господи, милість Твоя на нас, бо ми надіємось на Тебе. Благословенний Ти, Господи, навчи мене оправданням Твоїм. Благословенний Ти, Владико, врозуми мене оправданнями Твоїми. Благословенний Ти, Святий, просвіти мене оправданнями Твоїми. Господи, милість Твоя повік, творіння рук Твоїх не зневаж. Тобі подобає хвала, Тобі подобає оспівування, Тобі слава подобає, Отцю, і Сину, і Святому Духові, нині, і повсякчас, і на віки віків. Амінь.

Нині відпускаєш раба Твого, Владико, за словом Твоїм з миром, бо побачили очі мої спасіння Твоє, що Ти приготував перед лицем всіх людей, світло на одкровення народам, і славу людей Твоїх, Ізраїля.

Богородице Діво, радуйся, Благодатна Маріє, Господь з Тобою. Благословенна Ти між жонами, і благословен плід утроби Твоєї, бо Ти породила Спаса душ наших (один поклін).

Слава Отцю, і Сину, і Святому Духові.

Хрестителю Христів, усіх нас пом’яни, щоб визволитися нам від беззаконня нашого, бо тобі дано благодать молитися за нас (один поклін).

І нині, і повсякчас, і на віки віків. Амінь.

Молітеся за нас, святі апостоли й всі святії, щоб визволитися нам від лиха й скорботи, бо вас, теплих заступників перед Спасом, маємо (один поклін).

Під Твою милість прибігаємо, Богородице Діво, молитов наших в час журби не відкинь, а від бід визволяй нас, єдина чиста і благословенна.

\section{З ранньої Шестипсалміє}
Слава в вишніх Богу і на землі мир, в людях благовоління (тричі).

Господи, відкрий уста мої, і уста мої сповістять хвалу Твою (двічі).

\section{Псалом З}
Господи, як намножилося напасників моїх, многі постають на мене, многі кажуть душі моїй: нема спасіння йому в Бозі його. Ти ж, Господи, Заступник мій єси, слава моя, Ти возносиш голову мою. Голосом моїм до Господа воззвав, і Він почув мене з гори святої Своєї. Я лягаю, і сплю, і встаю, бо Господь заступить мене. Не убоюсь я тьми людей, що звідусіль нападають на мене. Воскресни, Господи, спаси мене, Боже мій, бо Ти уразив усіх, що даремно ворогують зі мною, зуби грішників сокрушив єси.

Господнім є спасіння, і на людей Твоїх благословення Твоє. Я лягаю, і сплю, і встаю, бо Господь заступить мене.

\section{Псалом 37}
Господи, не суди мене у ярості Твоїй і не карай мене у гніві Твоїм. Бо стріли Твої пройняли мене, і утвердив єси на мені руку Твою. Нема зцілення у плоті моїй від лиця гніву Твого, нема в костях спокою моїх від лиця гріхів моїх. Бо беззаконня мої перевищили голову мою і, як тягар великий, пригнітили мене. Засмерділись і загнилися рани мої від лиця безумства мого. Пригноблений я і зовсім поник, весь день сумуючи ходжу. Бо стегна мої сповнилися запаленням, і нема зцілення у плоті моїй. Озлоблений був і смирився до кінця; ридаю від болю серця мого. Господи, перед Тобою всі бажання мої, і зітхання моє від Тебе не втаїться. Серце моє стривожене, покинула мене сила моя, і світло очей моїх — і того вже не стало в мене. Друзі мої і приятелі мої відійшли від мене, і родина моя стала осторонь мене. А ті, що шукають душу мою, поставили сіті, і ті, що бажають мені зла говорять про погибель і весь день замишляють підступи. Я ж, немов глухий, не чую і, як німий, не відкриваю уст своїх. І став я як людина, що не чує і не має в устах своїх виправдання. Бо на Тебе, Господи, уповаю я: Ти почуй, Господи, Боже мій. Тому я сказав: нехай ніколи не потішаються вороги мої; коли спіткнуться ноги мої, вони звеличуються наді мною. Я готовий до ран, і недуги мої повсякчас переді мною. Бо беззаконня мої я визнаю і печалюся через гріх мій. Вороги ж мої живуть і зміцнюються більше за мене, і намножилося тих, що ненавидять мене не по правді. Ті, що відплачують мені злом за добро, обмовляють мене, бо я про благодіяння дбаю. Не покинь мене, Господи, Боже мій, не відступи від мене, і Поспіши на поміч мені, Господи спасіння мого.

Не покинь мене. Господи, Боже мій, не відступи від мене. Поспіши на поміч мені, Господи спасіння мого.

\section{Псалом 62}
Боже, Боже мій, до Тебе зранку лину я, жадає Тебе душа моя, за Тобою знемагає плоть моя. В землі пустельній, непрохідній і безводній — як у святиню явився до Тебе, щоб бачити силу Твою і славу Твою. Бо милість Твоя краща за життя, уста мої прославлятимуть Тебе. Так благословлятиму Тебе в житті моїм, і в ім’я Твоє піднесу руки свої. Як ситтю та єлеєм наповнюється душа моя, і голосом радости вихваляють Тебе уста мої, коли згадую Тебе на постелі моїй і в усі години ночі розмірковую про Тебе, бо Ти єси Помічник мій, і під покровом крил Твоїх возрадуюся. Припала душа, моя до Тебе, і правиця Твоя підтримує мене. Ті ж, що даремно шукають душу мою, впадуть у безодню підземну, уразить їх сила меча, стануть здобиччю лисів. Цар же звеселиться в Бозі: прославиться кожний, хто клянеться Ним, і замовкнуть уста тих, що, говорять неправедно.

В усі години ночі розмірковую про Тебе, бо Ти єси Помічник мій, і під покровом крил Твоїх возрадуюся. Припала душа моя до Тебе, і правиця Твоя підтримує мене.

Слава Отцю, і Сину, і Святому Духові нині, і повсякчас, і на віки віків. Амінь.

Алилуя, алилуя, алилуя, слава Тобі, Боже (тричі без поклонів). Господи, помилуй (тричі).

Слава Отцю, і Сину, і Святому Духові нині, і повсякчас, і на віки віків. Амінь.

\section{Псалом 87}
Господи, Боже спасіння мого, вдень я взиваю і вночі перед Тобою. Нехай дійде до Тебе молитва моя, прихили вухо Твоє до моління мого. Бо стражданнями сповнилася душа моя, і життя моє до пекла наблизилося. Приєднався я до тих, що сходять у могилу: я став як людина без сили, між мертвими кинутий я, — як убиті, що сплять у гробі, про яких Ти вже не згадуєш і які від руки Твоєї відринуті. Ти поклав мене у рові преісподнім, в темряві і тіні смертній На мені утвердилася ярість Твоя, і всі хвилі Твої навів Ти на мене. Віддалив Ти знайомих моїх від мене, зробив мене огидним для них. Я в неволі і не можу вийти. Очі мої знемоглися від горя. Взивав до Тебе, Господи, весь день, підносив до Тебе руки мої. Хіба для мертвих Ти твориш чудеса? Чи мертві воскреснуть і прославлятимуть Тебе? Хіба розповість хто у гробі про милість Твою і про істину Твою — в погибелі? Чи пізнані будуть у темряві чудеса Твої і правда Твоя — в землі забуття? Я ж до Тебе, Господи, взиваю, і вранці молитва моя упередить Тебе. Чому, Господи, відкидаєш душу мою, відвертаєш лице Твоє від мене? Убогий я, і страждаю від юности моєї, смирився і знеміг. По мені пройшов гнів Твій, і страхання Твої сокрушили мене. Оточили мене, як вода, весь день, облягли мене всі вкупі. Віддалив Ти від мене друга і ближнього, знайомих моїх — від моїх пристрастей.

Господи, Боже спасіння мого, вдень я взиваю і вночі перед Тобою. Нехай дійде до Тебе молитва моя, прихили вухо Твоє до моління мого.

\section{Псалом 102}
Благослови, душе моя, Господа і, вся істото моя, ім’я святеє Його. Благослови, душе моя, Господа і не забувай усіх добродійств Його. Він очищає всі беззаконня твої, зціляє всі недуги твої. Він звільняє від тління життя твоє, вінчає тебе милістю і щедротами. Він виконує благі бажання твої: оновиться, подібно орляті, юність твоя. Господь творить справедливість і суд усім покривдженим. Показав путі Свої Мойсеєві, синам Ізраїлевим — хотіння Свої. Щедрий і Милостивий Господь, Довготерпеливий і Многомилостивий. Не до кінця прогнівається, і повік не ворогуватиме. Не за беззаконнями нашими вчинив нам, і не за гріхами нашими воздав нам. Бо як високо небо над землею, так утвердив Господь милість Свою над тими, що бояться Його. Як далеко схід від заходу, так віддалив Він від нас беззаконня наші. Як отець милує дітей, так милує Господь тих, що бояться Його. Бо Він знає сутність нашу, пам’ятає, що ми — порох землі. Людина, як трава, дні її, немов цвіт польовий, цвіте і відцвітає. Повіє вітер над нею, і не стане її: не знайти вже й місця по ній. Милість же Господня від віку й до віку на тих, що бояться Його. І правда Його на синах синів, що бережуть Завіти Його і пам’ятають Заповіді Його, щоб виконувати їх. Господь на небесах уготував Престіл Свій, і Царство Його усім володіє.

Благословіть Господа, всі ангели Його, сильні міцністю, що виконуєте слово Його, слухаючи голосу слів Його. Благословіть Господа, всі Сили Його, слуги Його, що творите волю Його. Благословіть Господа, всі діла Його. На всіх місцях володіння Його благослови, душе моя, Господа!

На всіх місцях володіння Його благослови, душе моя, Господа!

\section{Псалом 142}
Господи, почуй молитву мою, зглянься на моління моє в істині Твоїй, вислухай мене у правді Твоїй. І не входь у суд з рабом Твоїм, бо не виправдається перед Тобою ніхто з живих. Бо ворог переслідує душу мою, втоптав у землю життя моє, посадив мене у темряву, як давно померлих. І впав у мені дух мій, стривожилося у мені серце моє. Я згадую дні давні, розмірковую про усі діла Твої і в творінні рук Твоїх повчаюся. До Тебе простягаю руки мої: душа моя, як земля безводна, перед Тобою. Скоро почуй мене, Господи, згасає дух мій. Не відверни лиця Твого від мене, бо уподібнюся тим, що сходять у могилу. Дай мені зрання відчути милість Твою, бо на Тебе уповаю. Вкажи мені, Господи, путь, якою піду, бо до Тебе підношу душу мою. Забери мене від ворогів моїх, Господи, до Тебе вдаюся, навчи мене творити волю Твою, бо Ти єси Бог мій. Дух Твій Благий наставить мене на землю правди; імені Твого заради, Господи, оживи мене правдою Твоєю. Виведи з печалі душу мою і милістю Твоєю знищ ворогів моїх. І вигуби гнобителів душі моєї, бо я раб Твій є.

Почуй мене, Господи, у правді Твоїй і не входь у суд з рабом Твоїм. Дух Твій Благий наставить мене на землю правди.

Слава Отцю, і Сину, і Святому Духові нині, і повсякчас, і на віки віків. Амінь. Алилуя, алилуя, алилуя, слава Тобі, Боже.

\section{Полієлей}
Хваліте ім’я Господнє, хваліте, раби Господа. Алилуя (тричі).

Благословен Господь від Сиону, що живе в Єрусалимі. Алилуя (тричі).

Сповідайтеся Господеві, бо Він Благий, бо повіки милість Його. Алилуя (тричі).

Сповідайтеся Богу Небесному, бо повіки милість Його. Алилуя (тричі).

\section{Тропарі недільні Глас 5}
Благословенний Ти, Господи, навчи мене оправданням Твоїм.

Ангельський Собор здивувався, побачивши Тебе, до мертвих причисленого, але Ти, Спасе, смертну силу зруйнував, і з Собою Адама воздвигнув, і всіх від пекла визволив.

Благословенний Ти, Господи…

«Чому миро з жалісливими слізьми, учениці, розчиняєте? — Сяючи у гробі, ангел до мироносиць промовив: — Дивіться на гріб і зрозумійте, що Спас воскрес із гробу».

Благословенний Ти, Господи…

Дуже рано мироносиці поспішили до гробу Твого ридаючи, але з’явився їм ангел, кажучи: «Ридання час минув, не плачте, воскресіння ж апостолам звістіте».

Благословенний Ти, Господи…

Мироносиці-жони з миром прийшли до гробу Твого, Спасе, ридаючи, ангел же до них промовив, кажучи: «Чому з мертвими Живого помишляєте? Адже Бог воскрес із гробу».

Слава Отцю, і Сину, і Святому Духові.

Поклонімось Отцю, і Його Сину і Святому Духові, Святій Тройці у Єдиносущному єстві, з серафимами взиваючи: Свят, Свят, Свят єси, Господи.

І нині, і повсякчас, і на віки віків. Амінь.

Життєдавця народивши, від гріха Ти Діво, Адама визволила, і радість Єві замість печалі подала єси. Тих, що втратили життя, до нього направив Той, що з Тебе воплотився, Бог і Чоловік. Алилуя, алилуя, алилуя, слава Тобі, Боже (тричі).

Від юности моєї многі борють мене пристрасті, але Сам мене заступи і спаси, Спасе мій. Ненависники Сиону, осоромитесь від Господа, і, як трава від вогню, посохнете.

Слава…

Святим Духом всяка душа оживляється і чистотою підноситься, світліється Троїчною єдністю священнотайно.

\section{Недільна пісня по Євангелію}
Воскресіння Христове бачивши, поклонімось Святому Господеві Ісусу, Єдиному Безгрішному. Хресту Твоєму поклоняємось, Христе, і Святеє Воскресіння Твоє оспівуємо і славимо, бо Ти єси Бог наш, крім Тебе іншого не знаємо, ім’я Твоє призиваємо. Прийдіть, усі вірні, поклонімось Святому Христовому Воскресінню, бо через Хрест прийшла радість усьому світові. Завжди благословляючи Господа, оспівуємо Воскресіння Його, бо, перетерпівши розп’яття, Він смертю смерть переміг.

\section{Пісня Пресвятої Богородиці}
Величає душа моя Господа, і зрадів дух мій у Бозі, Спасі моїм.

\emph{Після кожного стиха співається:}

Чеснішу…

Бо зглянувся на смирення раби Своєї, ось бо віднині ублажатимуть мене всі роди.

Бо вчинив мені велич Всемогутній, і святе ім’я Його, і милість Його в роди родів на тих, що бояться Його.

Сотворив державу силою Своєю, розсіяв гордих у помислах сердець їхніх.

Скинув сильних з престолів і підніс смиренних, голодних наповнив благами, а багатих відпустив ні з чим.

Сприйняв Ізраїля — отрока Свого, згадавши милості, як говорив отцям нашим, Аврааму і нащадкам його довіку.

Пісня до Пресвятої Богородиці
Преблагословенна Ти, Богородице Діво, бо Той, Хто воплотився від Тебе, пекло полонив. Адама відкликано, прокляття знищено. Єву визволено, смерть умертвлено і нас оживлено. Тому, оспівуючи, взиваємо: благословенний Христос Бог, що так благоволив, слава Тобі.

\section{Велике славослів’я}
Слава в вишніх Богу і на землі мир, в людях благовоління. Хвалимо Тебе, благословимо Тебе, поклоняємось Тобі, прославляємо Тебе, дякуємо Тобі ради великої слави Твоєї.

Господи, Царю Небесний, Боже, Отче Вседержителю, Господи, Сину Єдинородний, Ісусе Христе і Святий Душе, Господи Боже, Агнче Божий, Сину Отця, Ти, що береш гріхи світу, помилуй нас. Ти, що береш гріхи світу, прийми молитву нашу. Ти, що сидиш праворуч Отця, помилуй нас. Бо Ти Єдиний Святий, Ти Єдиний Господь, Ісус Христос, задля слави Бога Отця. Амінь. Повсякчас благословлятиму Тебе і вихвалятиму ім’я Твоє навіки і повік віків.

Сподоби, Господи, в день цей від гріха встерегтися нам. Благословен єси, Господи, Боже отців наших, і хвальне й прославлене ім’я Твоє навіки. Амінь.

Нехай буде, Господи, милість Твоя на нас, бо ми надіємось на Тебе.

Благословенний Ти, Господи, навчи мене оправданням Твоїм (тричі).

Господи, пристановищем став Ти нам з роду в рід. Я сказав: «Господи, помилуй мене і зціли душу мою, бо я согрішив Тобі. Господи, до Тебе прибіг, навчи мене творити волю Твою, бо Ти є Бог мій, бо у Тебе джерело життя, у світлі Твоїм побачимо світло, продовж милість Твою тим, що знають Тебе». Святий Боже, Святий Кріпкий, Святий Безсмертний, помилуй нас (тричі).

Слава Отцю, і Сину, і Святому Духові нині, і повсякчас, і на віки віків. Амінь.

Святий Безсмертний, помилуй нас. Святий Боже, Святий Кріпкий, Святий Безсмертний, помилуй нас.

В день спасіння миру, співаймо Воскреслому із гробу і Начальнику життя нашого, зруйнував бо Він смертю смерть, перемогу дав нам і велику милість (1, 3, 5 і 7 гласи).

Воскрес із гробу і пута розбив пекельні; розірвав і осуд смертний, Ти, Господи, всіх од сітей ворожих визволив, явив Себе апостолам Твоїм, послав їх на проповідь і ними мир Твій дав світові. Єдиний, Многомилостивий (2, 4, 6 і 8 гласи).

Непереможній Воєводі — переможнії ми, звільнившися від бід, вдячні пісні підносимо Тобі, раби Твої, Богородице. Але Ти, що маєш державу непереможну, від всяких нас бід визволи, щоб до Тебе взивати: «Радуйся, Невісто Неневісная».
\end{document}