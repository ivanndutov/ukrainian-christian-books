\documentclass[chapters.tex]{subfiles}

\begin{document}
\chapter{Молитви вечірні}
В ім’я Отця, і Сина, і Святого Духа. Амінь.

Господи Ісусе Христе, Сину Божий, молитвами Пречистої Твоєї Матері і всіх святих помилуй нас. Амінь.

Слава Тобі, Боже наш, слава Тобі.

Царю Небесний, Утішителю, Душе істини, що всюди єси і все наповняєш, Скарбе добра і життя Подателю, прийди і вселися в нас, і очисти нас від усякої скверни, і спаси, Благий, душі наші.

Святий Боже, Святий Кріпкий, Святий Безсмертний, помилуй нас (тричі).

Слава Отцю, і Сину, і Святому Духові нині, і повсякчас, і на віки віків. Амінь.

Пресвята Тройце, помилуй нас; Господи, очисти гріхи наші; Владико, прости беззаконня наші; Святий, зглянься і зціли немочі наші імені Твого ради.

Господи, помилуй (тричі).

Слава Отцю, і Сину, і Святому Духові нині, і повсякчас, і на віки віків. Амінь.

Отче наш, що єси на небесах, нехай святиться ім’я Твоє; нехай прийде Царство Твоє; нехай буде воля Твоя, як на небі, так і на землі. Хліб наш насущний дай нам сьогодні; і прости нам провини наші, як і ми прощаємо винуватцям нашим; і не введи нас у спокусу, але визволи нас від лукавого. Бо Твоє є Царство, і сила, і слава, Отця, і Сина, і Святого Духа, нині, і повсякчас, і на віки віків. Амінь.

\section{Тропарі покаянні}
Помилуй нас, Господи, помилуй нас, бо, жодного виправдання не маючи, ми, грішні, Тобі, як Владиці, цю молитву приносимо: помилуй нас. Слава Отцю, і Сину, і Святому Духові.

Господи, помилуй нас, бо на Тебе ми надіємось; не прогнівайся дуже на нас і не пам’ятай беззаконь наших, але зглянься і нині, як Благоутробний, і визволи нас від ворогів наших. Бо Ти Бог наш, а ми, люди Твої, всі творіння рук Твоїх і ім’я Твоє призиваємо. І нині, і повсякчас, і на віки віків. Амінь.

Милосердя двері відкрий нам, Благословенна Богородице, щоб ми, на Тебе надіючись, не загинули, а від усякого лиха Тобою визволилися. Бо Ти єси спасіння роду християнського.

Господи, помилуй (12 разів).

\section{Молитва 1-а, святого Макарія Великого до Бога Отця}
Боже вічний і Царю всякого створіння, що сподобив мене до цього часу дожити, прости мені гріхи, які вчинив я в цей день ділом, словом і думкою; і очисти, Господи, грішну мою душу від усякої скверни тіла й духу. І дай мені, Господи, в цю ніч спокійний сон, щоб, вставши з мого смиренного ложа, я догоджав пресвятому імені Твоєму в усі дні життя мого й переміг усіх ворогів тілесних і безтілесних. І охорони мене, Господи, від пустих думок і недобрих бажань, що осквернюють мене, і похотей лукавих. Бо Твоє є Царство, і сила, і слава, Отця, і Сина, і Святого Духа, нині, і повсякчас, і на віки віків. Амінь.

\section{Молитва 2-а, св. Антиоха до Господа Ісуса Христа}
Вседержителю, Слово Отчеє, Всесвятий Ісусе Христе. З великого милосердя Твого ніколи не покидай мене, раба Твого, але завжди в мені перебувай. Ісусе, Добрий Пастирю Твоїх овець, не віддай мене на поталу зміїну і не попусти, щоб сатана спокушав мене, бо в мені є насіння зла. Отже, Ти, Господи Боже, поклоніння достойний, Царю Святий, Ісусе Христе, охорони мене, коли я спатиму, тим Світлом, що ніколи не меркне, — Духом Твоїм Святим, що Ним освятив Ти Своїх учеників. Подай, Господи, мені, недостойному рабові Твоєму, спасіння Твоє на ложі моїм, просвіти розум мій світлом зрозуміння Святого Євангелія Твого, душу — любов’ю до Хреста Твого, серце — чистотою Слова Твого, тіло моє — Твоїми страстями безвинними, думку мою Твоїм смиренням охорони і в слушний час підведи мене від сну на славослів’я Твоє, бо Ти препрославлений єси з Безпочатковим Твоїм Отцем і Пресвятим Духом навіки. Амінь.

\section{Молитва 3-я, до Святого Духа}
Господи, Царю Небесний, Утішителю, Душе істини, змилосердься й помилуй мене, грішного раба Твого, і відпусти і прости мені, недостойному, все, чим я згрішив перед Тобою сьогодні як людина, а навіть і не як людина, але гірше скотини; всі мої гріхи вільні і невільні, свідомі і несвідомі, вчинені через молодість або через недобру науку, зухвальство й зневіру. Може, іменем Твоїм я клявся або ганьбив його в своїх думках, може, кому докоряв, або обмовляв кого в гніві моїм, або журився, або від чого-небудь прогнівався, або неправду сказав, або передчасно спав, або захожого до мене вбогого зневажив, або брата мого засмутив, або затівав сварки, чи, може, кого обсудив, або величався, або погордував, або під час молитви мій розум поривався до лукавства світу цього, або про розпусне помислив, або об’їдався чи обпивався, або без глузду сміявся, або зле задумав, або, бачачи доброту ближнього, заздрив йому, або негідне говорив, або посміявся з гріха брата мого, тоді як мої гріхи незліченні, або до молитви був недбайливий чи ще що вчинив злого і забув, бо все оце і ще більше я вчинив. Помилуй мене, Творче мій і Владико, лінивого і негідного раба Твого, і вибач, і відпусти, і прости мені, як Благий і Чоловіколюбний, нехай спокійно ляжу, засну та відпочину я, блудний, грішний і окаянний і поклонюся, і оспіваю, й прославлю пречесне ім’я Твоє з Отцем і Єдинородним Його Сином нині, і повсякчас, і на віки віків. Амінь.

\section{Молитва 4-а, св. Макарія Великого}
Що Тобі принесу? Або чим Тобі віддячу, Великодаровитий, Безсмертний Царю, Щедрий і Чоловіколюбний Господи, за те, що мене, який лінується на Твоє служіння і нічого доброго не зробив, Ти довів до кінця цього минулого дня, направляючи до покірности і спасіння мою душу. Будь же милостивий до мене грішного, бо я не маю ніякого доброго діла. Віднови пропащу мою душу, що осквернилася безліччю гріхів, і відкинь від мене помисел лукавий видимого цього життя. Єдиний Ти Безгрішний, прости провини мої, вчинені перед Тобою в цей день свідомо і несвідомо, словом, ділом і думкою й всіма моїми почуттями. Ти Сам, покриваючи, охорони мене від усякого ворожого нападу, захищаючи Божественною Твоєю владою і невимовним чоловіколюбством і силою. Очисти, Боже, очисти безліч моїх гріхів. Благоволи, Господи, звільнити мене від тенет диявольських, спаси мою стражденну душу і осіни мене світлом лиця Твого, коли прийдеш у славі. І нині дай мені заснути сном неосужденним, і охорони думки раба Твого від мрій і стурбованости. Віджени від мене всю сатанинську дію, просвіти мисленні очі мого серця, щоб не заснути мені смертним сном. Пошли ангела миру, охоронця і наставника душі і тілу моєму, щоб він позбавляв мене від моїх ворогів, щоб, вставши з моєї постелі, я приніс би Тобі подячні молитви. О Господи, почуй мене, грішного й убогого раба Твого. Даруй мені після пробудження з чистим сумлінням навчатися закону Твого, віджени подалі від мене через Твоїх ангелів сатанинську зневіру, щоб благословляти ім’я Твоє святе і прославляти, і славити Пречисту Богородицю Марію, дану нам грішним в захист, і прийми її, що молиться за нас. Бо знаю, що Вона, наслідуючи Твоє чоловіколюбство, не перестає молитися за нас. Її заступництвом, і Чесного Хреста знаменням, і всіх святих Твоїх заради збережи мою убогу душу, Ісусе Христе, Боже наш, бо Ти Святий єси і Найславніший повіки. Амінь.

\section{Молитва 5-а}
Господи Боже наш, все, у чому я згрішив сьогодні, словом, ділом чи думкою, Ти, як Благий і Чоловіколюбний, прости мені, мирний і безтурботний сон подай мені, ангела Твого охоронителя пошли мені, щоб він покривав і охороняв мене від усякого зла. Бо Ти Охоронець душ і тіл наших і Тобі славу возсилаємо, Отцю, і Сину, і Святому Духові, нині, і повсякчас, і на віки віків. Амінь.

\section{Молитва 6-а}
Господи Боже наш, в Якого ми віруємо і Чиє ім’я більше всякого імені призиваємо, подай нам, що відходимо до сну, полегшення для душі і тіла й охорони нас від усякого марева і від темної пристрасті; припини пристрасне бажання; погаси полум’я тілесного збудження; подай нам цнотливо пожити в слові і в ділі, щоб, сприйнявши доброчесне життя, не позбутися нам обіцяного Твого блага, бо Ти благословенний єси навіки. Амінь.

\section{Молитва 7-а, св. Іоана Золотоустого, на всяку годину дня й ночі}
Господи, не позбав мене небесних Твоїх благ. Господи, визволи мене від вічних мук. Господи, розумом чи думкою, словом чи ділом згрішив я, прости мені. Господи, визволи мене від усякого незнання, забуття, легкодухости і закам’янілої нечутливості. Господи, визволи мене від усякої спокуси. Господи, просвіти моє серце, затьмарене злою похіттю. Господи, я як людина згрішив, Ти ж, як Бог щедрий, помилуй мене, знаючи неміч моєї душі. Господи, пошли благодать Твою на поміч мені, щоб я прославив ім’я Твоє святе. Господи, Ісусе Христе, запиши мене, раба Твого, в книзі життя, подай мені кінець благий. Господи Боже мій, хоч я і нічого доброго не вчинив перед Тобою, але дай мені з благодаті Твоєї покласти добре починання. Господи, зроси моє серце росою Твоєї благодаті. Господи неба і землі, пом’яни мене, грішного, мерзенного і нечистого раба Твого, в Царстві Твоїм. Амінь. Господи, прийми мене в покаянні. Господи, не залишай мене. Господи, не заведи мене в напасть. Господи, дай мені благі помисли. Господи, дай мені сльози, пам’ять про смерть і розчулення серця. Господи, дай мені намір сповідання гріхів моїх. Господи, дай мені смирення, цнотливість і послух. Господи, дай мені терпіння, великодушність і лагідність. Господи, вкорени страх Твій благий в серце моє. Господи, сподоби мене любити Тебе від усієї душі моєї і в усьому виконувати волю Твою. Господи, захисти мене від злих людей, і бісів, і пристрастей, і від усього, шкідливого для мене. Господи, твори за Твоїм бажанням все, що Ти хочеш, і нехай буде воля Твоя у мені грішнім, бо Ти благословенний єси навіки. Амінь.

\section{Молитва 8-а, до Господа нашого Ісуса Христа}
Господи Ісусе Христе, Сину Божий, заради Найчеснішої Матері Твоєї, і безплотних Твоїх ангелів, а також пророка Предтечі і Хрестителя Твого, Богоглаголивих апостолів, світлих і добропобідних мучеників, преподобних і богоносних отців і всіх святих молитвами визволи мене від теперішніх нападів бісівських. О Господи мій і Творче, що не бажаєш смерти грішнику, але чекаєш його навернення і життя, дай навернення і мені окаянному і недостойному; забери мене із пащі згубного змія, що хоче пожерти мене і звести до пекла заживо. О Господи мій, Утіхо моя, що мене ради окаянного зодягнувся у тлінну плоть, врятуй мене від нещастя і подай утіху душі моїй окаянній. Прищепи моєму серцю виконувати Твої повеління, й залишити злі діла, і здобути блаженство Твоє, бо на Тебе, Господи, я уповаю, спаси мене. Амінь.

\section{Молитва 9-а, Петра Студита до Пресвятої Богородиці}
До Тебе, Пречистої Божої Матері, я, припадаючи, молюся: Ти знаєш, Царице, як я безперестанно згрішаю і прогнівляю Сина Твого і Бога мого. І хоч багаторазово каюся, але неправедним виявляюся перед Богом, і знову зі страхом каюся, і відразу знову те ж саме роблю: невже Господь уразить мене? Знаючи про те, Владичице моя Богородице, що я цілком гидуюся злими моїми ділами і всіма думками люблю закон Бога мого, але не знаю, Пречиста Царице, чому доброго не роблю, а зле, якого не хочу, роблю, не допускай, Пречиста, виконуватися злій моїй волі, але нехай буде воля Сина Твого і Бога мого, Який мене спасе, врозумить і подасть благодать Святого Духа, щоб я відтепер перестав робити зле і решту часу прожив би по заповідях Сина Твого й Бога мого, Якому належить всяка слава, честь і держава з Безпочатковим Його Отцем і Пресвятим, Благим і Животворчим Духом нині, і повсякчас, і на віки віків. Амінь.

\section{Молитва 10-а, до Пресвятої Богородиці}
Благого Царя Блага Мати, Пречиста і Благословенна Богородице Маріє, милість Сина Твого і Бога нашого вилий на мою грішну душу і Твоїми молитвами настав мене на діяння благі, щоб останок мого життя я прожив без гріха і через Тебе рай знайшов, Богородице Діво, єдина чиста і благословенна. Амінь.

\section{Молитва 11-а, до св. Ангела-Охоронителя}
Ангеле Христів, Охоронителю мій святий, покровителю душі і тіла мого, все мені прости, чим я згрішив сьогодні, і від усякої спокуси мого ворога-супротивника звільни мене, щоб я ніяким гріхом не прогнівив Бога мого, але молися за мене, грішного й недостойного раба, щоб стати мені достойним благости й милости Всесвятої Тройці, і Матері Господа мого Ісуса Христа, і всіх святих. Амінь.

\section{Кондак Богородиці, глас 8}
Непереможній Воєводі — переможнії ми, звільнившися від бід, вдячні пісні підносимо Тобі, раби Твої, Богородице. Але Ти, що маєш державу непереможну, від всяких нас бід визволи, щоб до Тебе взивати: «Радуйся, Невісто Неневісная».

Преславна Приснодіво, Мати Христа Бога, принеси нашу молитву Синові Твоєму і Богові нашому, щоб Він спас через Тебе душі наші.

Все уповання моє на Тебе покладаю, Мати Божа, збережи мене під покровом Твоїм.

Богородице Діво, не погордуй мною грішним, що потребує Твоєї помочі й Твого захисту, бо на Тебе уповає душа моя, і помилуй мене.

Уповання моє — Отець, пристановище моє — Син, захист мій — Дух Святий: Тройце Свята, слава Тобі.

Достойно є, і це є істина, славити Тебе, Богородицю, Присноблаженну і Пренепорочну і Матір Бога нашого. Чеснішу від херувимів і незрівнянно славнішу від серафимів, що без істління Бога Слово породила, сущу Богородицю, Тебе величаємо.

Слава Отцю, і Сину, і Святому Духові нині, повсякчас, і на віки віків. Амінь.

Господи, помилуй (тричі).

Господи, благослови. І відпуст.

Господи Ісусе Христе, Сину Божий, молитвами Пречистої Твоєї Матері і всіх святих помилуй нас. Амінь.

\section{Молитва св. Іоана Дамаскіна}
\emph{Вказуючи на постіль свою, промовляй:}

Владико Чоловіколюбче, невже ця постіль буде мені домовиною чи ще окаянну мою душу освітиш світлом дня? Ось бо домовина стоїть переді мною, ось смерть чекає на мене. Суду Твого, Господи я боюся і нескінченної муки, зло ж робити не перестаю і завжди прогнівляю Тебе, Господа мого, й Пречисту Твою Матір, і всі Небесні Сили, святого Ангела-Охоронителя мого. Знаю, Господи, що я недостойний чоловіколюбства Твого, але достойний осудження і муки. Та, Господи, хочу чи не хочу спаси мене. Якщо праведника спасаєш, то тут немає нічого дивного, і якщо чистого помилуєш, то тут нема нічого дивного, бо вони достойні Твоєї милости, але вияви дивну Твою милість на мені грішному й у ньому яви чоловіколюбство Твоє, і нехай не перевершить моє зло Твою невимовну благість і милосердя, і, як тільки хочеш, так і вчини діло мого спасіння.

\emph{Перед тим, як ляжеш у постіль, промовляй ще:}

Просвіти очі мої, Христе Боже, щоб я не заснув на смерть, щоб не сказав ворог мій: я подолав його.

Слава Отцю, і Сину, і Святому Духові.

Боже, будь заступником душі моєї, бо я ходжу серед багатьох тенет; визволи мене від них і спаси мене, Благий, як Чоловіколюбний.

І нині, і повсякчас, і на віки віків. Амінь.

Преславну Божу Матір, святішу від святих ангелів, невмовкно оспівуємо серцем і устами, визнаючи її Богородицею, що істинно народила нам Бога втіленого й безперестанно молиться за душі наші.

\emph{Також, поцілувавши хрест свій, осіняй себе хрестом і постіль свою від голови до ніг, а також усі чотири сторони світу, читаючи}

\section{Молитву до Чесного Хреста}
Нехай воскресне Бог і розвіються вороги Його, і нехай біжать від лиця Його всі ненависники Його. Як щезає дим, нехай щезнуть, як тане віск від лиця вогню, так нехай загинуть біси від лиця тих, хто любить Бога і хто осіняє себе хресним знаменням і в радості промовляє: радуйся, Пречесний і Животворчий Хресте Господній що проганяєш бісів силою розп’ятого на тобі Господа нашого Ісуса Христа, що до пекла зійшов, й подолав силу диявола, й дарував нам тебе, Хрест Свій Чесний, на прогнання всякого супротивника.

О Пречесний і Животворчий Хресте Господній, допомагай мені зі Святою Дівою Богородицею й зо всіма святими Небесними Силами завжди, нині, і повсякчас, і на віки віків. Амінь.

\section{Молитва}
Захисти мене, Господи, силою Чесного і Животворчого Твого Хреста і ним охорони мене від усякого зла.

\emph{Потім замість прощення:}

Полегши, відпусти, прости, Боже, гріхи наші вільні й невільні, словом чи ділом, свідомі чи несвідомі, вдень чи вночі, чи в думках, чи в помислах заподіяні, — все нам прости, бо Ти Благий і Чоловіколюбний.

\section{Молитва}
Тих, що ненавидять і кривдять нас, прости, Господи Чоловіколюбний. Благочинцям — добро вчини. Братам і рідним нашим даруй, чого вони просять для спасіння й життя вічного. Хворих відвідай і зцілення їм даруй. Кермуй тими, що на морі. Мандрівникам будь супутником, всім православним християнам допомагай у боротьбі. Тим, хто служить нам і милує нас, гріхів відпущення даруй. Тих, що заповіли нам, недостойним, молитися за них, помилуй з великої Твоєї милости. Пом’яни, Господи, раніше спочилих отців, братів і сестер наших і упокой їх там, де сяє світло лиця Твого. Пом’яни, Господи, братів наших поневолених і визволи їх від усякого нещастя. Пом’яни, Господи, і тих, що дари приносять і добро чинять у святих Твоїх церквах, дай їм чого вони просять для спасіння і життя вічне. Пом’яни, Господи, і нас, смиренних, грішних і недостойних рабів Твоїх, і просвіти наш розум світлом розуму Твого, і настанови нас на стежку заповідей Твоїх молитвами Пречистої Владичиці нашої Богородиці і Приснодіви Марії і всіх святих Твоїх, бо Ти благословенний єси на віки віків. Амінь.

\section{Молитва щоденного сповідання гріхів}
Сповідаю Тобі, Господу Богові моєму і Творцю, Єдиному в Святій Тройці, хвальному і поклоняємому Отцю, і Синові, і Святому Духові, всі мої гріхи, вчинені в усі дні життя мого, і у кожну годину, і в нинішній час, і в минулі дні й ночі ділом, словом, помислом: об’їданням, пияцтвом, пустослів’ям, зневір’ям, лінощами, суперечками, неслухняністю, обмовлянням, осудженням, недбайливістю, самолюбством, багатопридбанням, розкраданням, неправдомовністю, зажерливістю, користолюбством, запопадливістю, ревнощами, заздрістю, гнівом, злопам’ятством, ненавистю, лихварством, і всіма моїми почуттями: зором, слухом, нюхом, смаком, дотиком, і інші мої гріхи душевні вкупі й тілесні, чим Тебе, Бога мого й Творця, прогнівив і ближнього мого образив. Про що жалкую, винним з’являюся перед Тобою, Богом моїм, і маю волю каятися, тільки допоможи мені, Господи Боже мій, покірно зі сльозами молю Тебе. Вчинені ж перелічені гріхи мої милосердям Твоїм прости мені й звільни від них, я ж визнав їх перед Тобою, бо Ти Благий і Чоловіколюбний.

\section{Молитва св. Іларіона}
О Владико, Царю і Боже наш, високий і славний Чоловіколюбче, що подаєш за труди славу і честь і що твориш учасниками Свого Царства, пом’яни, як Благий, і нас, немічних Твоїх, бо Твоє ім’я — Чоловіколюбець.

Якщо добрих діл і не маємо, але заради великої Своєї милости спаси нас. Ми ж бо люди Твої і вівці пастви Твоєї, яку Ти пасеш, врятувавши від загибелі ідолослужіння. Пастирю Добрий, що кладеш душу за овець! Не залиш нас, хоч ще й далі блудимо; не відкинь нас, хоч ще й далі грішимо перед Тобою, як ті новопридбані раби, що нічим не догоджають господареві своєму. Незважаючи на те, що ми стадо мале, промов до нас: «Не бійся, мале стадо! Бо Отець ваш благозволив дати вам Царство». Багатий милістю і благий щедротами! Ти, Який обіцяв прийняти тих, що каються, й очікуєш навернення грішних, не пам’ятай багатьох гріхів наших, прийми нас, що звертаємося до Тебе; зітри рукописання провин наших, прикороти гнів, яким ми розгнівали Тебе, Чоловіколюбний! Ти бо єси Господь, Владика і Творець, і в Тобі спочиває влада, чи нам жити, чи вмерти.

Відклади гнів, Милостивий, хоч і достойні ми його за гріхи наші; оминай нас випробуваннями, бо земля ми і порох, і не входь у суд з рабами Своїми. Ми люди Твої, Тебе шукаємо, до Тебе припадаємо, Тебе благаємо. Ми нагрішили і зла натворили; не виконували і не творили того, що Ти заповідав нам. Земними бувши, до земного схилилися і лиха натворили перед лицем слави Твоєї, тілесним похотям віддавшися; поневолили себе гріхами й турботами життєвими; стали втікачами від свого Владики, убогі на добрі діла, окаянні заради злого життя. Каємося, просимо, молимося. Каємося у злих ділах наших; просимо, щоб Ти послав Свій страх у серця наші; молимося, щоб на страшному суді Ти помилував нас. Спаси, зглянься, прости, відвідай, умилосердься, помилуй. Твої-бо ми, Твоє творіння і діло рук Твоїх. Бо якщо на беззаконня будеш зважати, Господи, Господи, хто встоїть? І якщо віддаси кожному за ділами його, то хто спасеться? Бо очищення — тільки від Тебе; бо в Тебе милість і велике визволення. І душі наші в руках Твоїх, і дихання наше у волі Твоїй. Задля того ж умилостивлення Твого до нас ми в добрі перебуваємо. Якщо ж з гнівом споглянеш на нас, зникнемо, як рання роса. Не встоїть-бо порох проти бурі, а ми проти гніву Твого. Але як творіння ми від Творця милости просимо: «Помилуй нас, Боже, з великої милости Твоєї». Все-бо благе для нас — від Тебе, все ж неправедне від нас — до Тебе. Всі бо ми відхилилися і всі разом відкинені були; нема бо між нами жодного, хто дбав би про небесне і змагався б, але всі турбуємося про земне, всі в турботах життєвих. Бо не стало праведного на землі (Пс. 11,1), не Ти залишаєш і забуваєш нас, але ми не шукаємо Тебе і до всього видимого принаджуємося. Тому-то й побоюємося, що з нами вчиниш, як з Єрусалимом, що залишив Тебе і не схотів ходити шляхами Твоїми. Але не вчини нам, як їм, за ділами нашими, і не за гріхами нашими віддай нам. Будь терплячим до нас і ще довго потерпи; спини полум’я гніву Твого, що простягається на нас, рабів Твоїх; Сам настанови нас на істину Твою, навчаючи нас творити волю Твою, бо Ти єси Бог наш, а ми люди Твої, Твоя частка, Твоє насліддя. Не підносимо бо ми руки свої до Бога чужого, не наслідуємо якогось лжепророка і не дотримуємося єретичної науки. Але прикликаємо Тебе, Істинного Бога, і до Тебе, що живеш на небесах, наші очі підносимо; до Тебе руки наші простягаємо і молимось.

Прости нас, як Благий і Чоловіколюбний; помилуй нас, що кличеш грішників до покаяння; і на страшному суді Твоєму не позбав нас стояння праворуч Тебе, але сподоби нас бути причасниками благословення праведників. І доки існуватиме світ, не наводь на нас напастей спокуси; не віддай нас в руки чужинців, щоб не назвалося місто Твоє містом полонених, а паства Твоя захожими не на своїй землі; щоб не говорили народи: «Де ж Бог їх?» Не допускай на нас турбот, ні голоду, ні марної смерти, вогню, потопу, щоб не відпали від віри нетверді вірою. Мало карай нас, а багато милуй; мало насилай на нас рани, а милостиво зціляй нас; менше кривдь нас, а скоріше звеселяй. Бо наше єство не може довго переносити гніву Твого, як стебла — вогню. Але злагіднися до нас і змилосердься, бо Тобі належить милувати і спасати нас. Тим-то продовжи милість Твою на людей Твоїх; війни відвертай, мир утверджуй, країни заспокой, голодних насити, володарів наших учини грізними для інших країн, начальників умудри, міста розшири. Церкву Твою розрости, володіння Своє збережи, мужів, жінок і немовлят спаси; невільних, полонених, засланих, подорожніх, плаваючих, ув’язнених, у голоді, спразі й наготі, — всіх помилуй, усіх втіш, всіх обрадуй, подаючи їм радість тілесну й душевну молитвами й молінням Пречистої Твоєї Матері і святих Небесних Сил, Предтечі Твого й Хрестителя Іоана, апостолів, пророків, мучеників, преподобних і молитвами всіх святих. Змилосердься над нами і помилуй нас, щоб ми, бережені Твоєю милістю в єдності віри, сукупно й радісно прославляли Тебе, Господа нашого Ісуса Христа, з Отцем, з Пресвятим Духом, — Тройцю Нероздільну, Єдинобожественну, Яка царствує на небесах і на землі, над ангелами й людьми, видимим і невидимим творінням, нині, і повсякчас, і на віки віків. Амінь.

\emph{Коли лягаєш спати, промовляй:}

В руки Твої, Господи Ісусе Христе, Боже мій, передаю дух мій. Ти ж мене благослови, Ти мене помилуй і життя вічне даруй мені. Амінь.

\section{Псалом 90}
Хто живе під охороною Всевишнього, той під покровом Бога Небесного оселиться. Каже він до Господа: «Ти Пристановище і Захист мій, Бог мій, і я уповаю на Нього». Він спасе тебе від сіті ловця і від пошести згубної. Плечима Своїми Він захистить тебе, і під тінню крил Його ти надійно спочиватимеш. Обороною тобі буде правда Його. Не побоїшся страху вночі, ані стріли, що летить удень. Ані пошести, що ходить у темряві, ані напасти духа зла опівдні. Впаде біля тебе тисяча, і десять тисяч праворуч тебе, але до тебе не наблизиться. Тільки очима твоїми будеш дивитися і помсту над беззаконними бачити. Бо ти сказав: «Господь — надія моя», і Всевишнього ти обрав за оборонця собі. Отже, не прийде до тебе лихо, і пошесть не наблизиться до оселі твоєї. Бо Він ангелам Своїм звелить, щоб охороняли тебе на всіх путях твоїх. На руках вони понесуть тебе, щоб нога твоя не спіткнулася об камінь. На гаспида й василиска ти наступатимеш і потопчеш лева й змія. Бо каже Господь: «За те, що він поклав надію на Мене, Я визволю його і захищу його, бо він знає ім’я Моє. Буде кликати Мене, Я почую його; буду з ним у скорботі, визволю його і прославлю його. Довгим життям обдарую його і дам йому спасіння Моє».
\end{document}